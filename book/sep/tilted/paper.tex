\lefthead{Shan and Biondi}
\righthead{Plane-wave migration in tilted coordinates}
\footer{SEP--124}
\email{shan@sep.stanford.edu, biondo@sep.stanford.edu}
\title{Plane-wave migration in tilted coordinates}
\author{Guojian Shan and Biondo Biondis}

\maketitle

\begin{abstract}
Plane-wave migration in tilted coordinates is powerful to image steeply dipping reflectors, 
although the one-way wave-equation operator is used. In plane-wave migration, 
the recorded surface data are transferred to plane-wave data by slant-stack processing. 
Both the source plane-wave and the corresponding slant-stacking data
are extrapolated into the subsurface, and images are generated by cross-correlating these two
wavefields. For each plane-wave source, we assign tilted coordinates whose direction depends on
the propagation direction of the plane-wave. For isotropic media, the one-way wave-equation
operator does not change. For vertical transversely isotropic (VTI) media, 
one-way wave-equation operators for tilted transversely isotropic (TI) media are
required because of the rotation of the coordinates. I apply this method to the BP 2004 velocity benchmark,
 to a synthetic dataset for VTI media, and to a real anisotropic dataset. The numerical examples show
that plane-wave migration in tilted coordinates can image steeply dipping reflectors 
and overturned waves, and it is a good tool for salt delineation.
\end{abstract}

\section{Introduction}
Migration methods based on wave-equation extrapolation are gaining popularity in the 
industry. They can potentially better handle multi-pathing caused by complex geological 
structure. In nature, wave propagation has no directional preference: up-going and 
down-going waves propagate simultaneously. However, two-way wave-equation-based methods, 
such as reverse time migration \cite{SEG-1983-S10.1,GEO.48.11.15141524,SEG-2002-12841287}, are still too expensive
 for today's computing resources. As a result, downward continuation \cite{jon1985book}, 
based on the one-way wave equation, is usually used. In downward-continuation methods, 
only down-going waves are allowed in the source wavefield, and only up-going waves are 
allowed in the receiver wavefield. Overturned waves, which travel downward first and 
then upward, are totally eliminated by the one-way wave equation.  
Even for waves only propagating upward or downward, it is difficult for downward-continuation methods to handle waves propagating 
far from the vertical direction in laterally varying media. As a result, downward-continuation methods have 
difficulty imaging steeply dipping reflectors.

\par
A lot of effort has been made to improve the accuracy of one-way wave equation 
extrapolators in laterally varying media \cite{GEO50-10-16341637,GEO59-12-18821893, dehoop.jmathphys.96,SEG-1996-0415}. At the same time, 
people also design coordinates to make the extrapolation direction close to the 
propagation direction of waves, thereby reducing the accuracy requirement of the extrapolator. 
These include tilted coordinates \cite{GEO50-11-17841796, Etgen-SEG02}, 
beam migration \cite{BDEtgen-SEG03}, 
and Riemannian coordinates \cite{rieman}. 
\plot{impulse}{width=6in,height=8in}{Impulse response of two-way wave equation (a), downward continuation (b), and the plane-wave migration in tilted coordinates (c).}
%\plot{twoway}{width=6in,height=3in}{Impulse response of the two-way wave equation.}

\par
In this paper, we review plane-wave migration in tilted coordinates 
\cite{SEG2004sgj}. We present plane-wave migration in tilted coordinates for both isotropic and anisotropic media. Isotropic media are still isotropic after rotating the coordinates, while 
VTI media change to tilted TI media after rotating the coordinates. We use the one-way wave-equation
extrapolation operator for tilted TI media proposed by \longcite{SEG2005sgj3d}.
We apply plane-wave migration in tilted coordinates to the BP synthetic velocity benchmark dataset \cite{bpdata},
a synthetic dataset for VTI media, and an anisotropic real dataset.


\section{Plane-wave migration in tilted coordinates}
Plane-wave (source plane-wave) migration 
\cite{whitmorephd,Rietveldphd,SEG-2001-10331036, SEG-2002-11561159, zhangyudelayshot}
 synthesizes plane-wave source experiments from shot records. The recorded data are 
decomposed into plane source gathers by slant-stack processing:
\begin{equation}
 u(x_r,z=0;p;\omega)=\int U(x_r,z=0;x_s;\omega)e^{i\omega px_s}dx_s,
\end{equation}
where $x_s$ is the source location, $x_r$ is the receiver location, $p$ is the ray parameter,
$U$ is the recorded surface data, and $u$ is the synthesized surface data for the plane-wave source. The corresponding
plane-source is 
\begin{equation}
  d(x_r,z=0;p;\omega)=e^{i\omega p x_s}.
\end{equation}
The plane-wave source $d$ and its corresponding synthesized data $u$ are independently extrapolated into the 
subsurface, and the image can be obtained by cross-correlating these two wavefields. 
Plane-wave migration is potentially more efficient than shot-profile migration. 
It uses the whole seismic survey as the migration aperture, which is helpful for imaging 
steeply dipping reflectors. 
\plot{bpvelall}{width=6in,height=3in}{The velocity model.}

\par
Given the plane-wave source with a ray-parameter $p$, its take-off angle at the surface is $\arcsin(pv)$, where $v$ is the surface velocity. 
we use tilted coordinates $(x^\prime,z^\prime)$  satisfying
\begin{equation}
\left( 
\begin{array}{l}
x^\prime\\
z^\prime
\end{array}
\right)=
\left(
\begin{array}{cc}
\cos(\theta) & \sin(\theta)\\
-\sin(\theta) & \cos(\theta)
\end{array}
\right)
\left(
\begin{array}{l}
x\\
z
\end{array}
\right),
\end{equation}
where $(x,z)$ are Cartesian coordinates and $\theta$ is an angle close to the take-off angle of the plane-wave 
source at the surface. 
In the tilted coordinates, the extrapolation direction  is potentially closer to the propagation direction.  
Therefore, in tilted coordinates, we can extrapolate the wavefield accurately with a less accurate extrapolator. 
Waves that were overturned in Cartesian coordinates are not overturned in tilted coordinates. 
Therefore, we can image them with the one-way wave equation.

\section{Numerical examples}
\subsection{Impulse responses}
Figure \ref{fig:impulse} compares the impulse responses  of the two-way wave equation, 
the downward continuation and  plane-wave migration in tilted coordinates. 
Figure \ref{fig:impulse}(a) is the impulse response of the two-way wave equation, 
Figure \ref{fig:impulse}(b) is the impulse response of downward continuation, and Figure \ref{fig:impulse}(c) is 
the impulse response of plane-wave migration in tilted coordinates.
In Figure \ref{fig:impulse}(b), the small angle part of the wavefront is similar to
that of the two-way wave equation, while the high angle part of the wavefront 
loses energy due to the overturned waves.
In Figure \ref{fig:impulse}(c), there are no reflections or multiples, since
we still use the one-way wave-equation extrapolator, but the wave front of the direct arrival 
mimics that of the two-way wave equation very well, even
at the high angle and with overturned waves.
\plot{bpvelleft}{width=6in,height=4in}{The velocity model of the left salt body.}

%point out multiples
\subsection{BP 2004 velocity benchmark dataset}
The BP synthetic dataset is designed to test velocity estimation. Figure \ref{fig:bpvelall} shows the velocity model of the dataset. 
One of the challenges for velocity analysis of this dataset is the delineation of the two salt bodies. 
The left one is a complex, multi-valued salt body with a greatly rugose top. 
Some parts of its top and flank and the sediments inside the salt are steeply dipping. 
These features are not easily imaged by downward continuation. The right salt body is deeply rooted, and  
its roots are very steep. Waves reflected at the center part of the two roots overturn and travel through 
the complex salt body to the left. Downward continuation loses the energy and cannot connect these two roots. 
Even with the true velocity, it is still challenging to image the complex salt bodies.

\plot{bpleftnotilt}{width=6in,height=4in}{The image of the left salt body in Cartesian coordinates.}
\plot{bplefttilt}{width=6in,height=4in}{The image of the left salt body in tilted coordinates.}

We run plane-wave migration in both Cartesian and tilted coordinates. We generate $200$ plane-wave sources in total,
and the take-off angles at the surface range from $-45^\circ$ to $45^\circ$. Since we do not run multiple
attenuation, the images are contaminated by the multiples.

\par
Figure \ref{fig:bpvelleft} shows the velocity model of the left rugose salt body.
Figures \ref{fig:bpleftnotilt} and \ref{fig:bplefttilt} show the images. 
Figure \ref{fig:bpleftnotilt} is obtained by downward continuation, 
and Figure \ref{fig:bplefttilt} is obtained by plane-wave migration in tilted coordinates. 
In Figure \ref{fig:bpleftnotilt}, the image of the 
top of the salt is not continuous. Some parts of the top are almost vertical and are lost in the downward 
continuation. This image captures only the bottom of the two canyons, and loses their flanks. 
Downward continuation also loses the steep salt flanks in the multi-valued area. These losses make picking 
the salt difficult. In contrast, plane-wave migration in tilted coordinates images the salt body 
very well and makes salt picking easier.

\plot{bpvelright}{width=6in,height=4in}{The velocity model of the right salt body.}

\par
Figure \ref{fig:bpvelright} shows the velocity model of the right salt body.
Figures \ref{fig:bprightnotilt} and \ref{fig:bprighttilt} show the images. 
Figure \ref{fig:bprightnotilt} is obtained by downward continuation, and 
Figure \ref{fig:bprighttilt} is obtained by plane-wave migration in tilted coordinates. Downward continuation loses the flanks 
of the top part of the salt. Due to overturned waves, downward continuation also loses the right leg and the center 
part of the left leg. In contrast, plane-wave migration images most parts of the salt flanks and the two legs. 
Hence, it is much easier to interpret the salt body from Figure \ref{fig:bprighttilt}.


\subsection{A synthetic dataset for VTI media}
Figures \ref{fig:vpani} -- \ref{fig:dltani} show a synthetic model for a VTI medium. Figure \ref{fig:vpani} shows the
velocity model, Figure \ref{fig:epsani} is the anisotropy parameter $\varepsilon$, and Figure \ref{fig:dltani}
is the anisotropy parameter $\delta$. There are 720 shots in total, and the maximum offset for each shot
is 8000 meters. The challenging part of this model is to accurately image the steep fault, salt flank and the
two abnormal sediments  near the right corner of the salt body. I run plane-wave migration, 
using the extrapolation operator suggested by \longcite{SEG2005sgj3d}. 
I generate 80 plane-wave sources, and the take-off angles at the surface range from $-40^\circ$ to $40^\circ$. 

\plot{bprightnotilt}{width=6in,height=4in}{The image of the right salt body in Cartesian coordinates.}
\plot{bprighttilt}{width=6in,height=4in}{The image of the right salt body in tilted coordinates.}
\plot{vpani}{width=6in,height=3.5in}{The vertical velocity model.}

\plot{epsani}{width=6in,height=3.5in}{The anisotropy parameter $\varepsilon$.}
\plot{dltani}{width=6in,height=3.5in}{The anisotropy parameter $\delta$.}
\plot{isohesstilt}{width=6in,height=3.5in}{Isotropic migration in tilted coordinates.}
\plot{anihessnotilt}{width=6in,height=3.5in}{Anisotropic migration in Cartesian coordinates.}
\plot{anihesstilt}{width=6in,height=3.5in}{Anisotropic migration in tilted coordinates.}
\plot{anihesstilteps}{width=6in,height=3.5in}{Anisotropic migration in tilted coordinates overlaid with the model.}

\par
Figure \ref{fig:isohesstilt} is the image obtained by isotropic plane-wave migration in tilted coordinates. Though
we can see the energy of steeply dipping reflectors like the salt flanks and fault, they are not at the correct
positions. Figure \ref{fig:anihessnotilt} is the image obtained by anisotropic plane-wave migration in Cartesian 
coordinates. The reflectors are at the right positions, but some parts of the steeply dipping salt flank are lost,  and
the bottom of the fault is not well focused. Also the bottom abnormal sediment at the right corner of the salt body
loses its energy where it is close to the salt. Figure \ref{fig:anihesstilt} is the image obtained by anisotropic plane-wave
migration in tilted coordinates. In Figure \ref{fig:anihesstilt}, the salt flanks, the fault  and 
abnormal sediments are all well imaged. Figure \ref{fig:anihesstilteps} overlays the image by anisotropic plane-wave
migration in tilted coordinates with the actual model, and shows that the reflectors are imaged at 
the correction positions.


\subsection{A 2D slice of a real dataset from the Gulf of Mexico}
In this 2D slice, there are 1600 shots, and the maximum offset 
is 4000 meters. I run plane-wave migration, using the wavefield-extrapolation operator suggested by \longcite{SEG2005sgj3d}.
I generate 80 plane-wave sources, for which the take-off angles at the surface range from $-40^\circ$ to $40^\circ$.
 Figure \ref{fig:aniexxonnotilt} is the image obtained
by anisotropic plane-wave migration in Cartesian coordinates, and Figure \ref{fig:aniexxontilt} is the image
obtained by  anisotropic plane-wave migration in tilted coordinates. Figure \ref{fig:aniexxontilt} 
images the salt flank much better than Figure \ref{fig:aniexxonnotilt}, and we can see a lot of small, steeply dipping events 
inside the shallow sediments that are quite continuous right to the edge of the salt.


\section{Conclusion}
We applied plane-wave migration in tilted coordinates on isotropic and anisotropic data. 
Plane-wave migration in tilted coordinates is potentially cheaper than shot-profile migration. 
For both isotropic and anisotropic data, plane-wave migration in tilted coordinates can handle overturned waves and
 image steeply dipping salt flanks and faults. 
It also images the rugose top of the salt very well.  
Plane-wave migration is a good tool for complex salt-body delineation.

\plot{aniexxonnotilt}{width=6in,height=3.5in}{Anisotropic plane-wave migration in Cartesian coordinates.}
\plot{aniexxontilt}{width=6in,height=3.5in}{Anisotropic plane-wave migration in Cartesian coordinates.}

\section{Acknowledgments}
We thank Amerada Hess, BP, and ExxonMobil for making the data available. Guojian Shan also
thanks Amerada Hess for the summer internship working on the synthetic data for VTI media.




\vskip 1cm
\bibliographystyle{segnat}
\bibliography{SEP,MISC,GEOTLE,SEG,GEOPHYSICS,paper}
