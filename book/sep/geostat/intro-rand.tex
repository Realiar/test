%In Chapter~\ref{chap:reg} and Appendix~\ref{chap:nonstat} I introduced
%several different methods to characterize the covariance. 
%In each case I presented  a single answer to the interpolation or
%tomography problem.  
\par
When solving a missing data problem, geophysicists and geostatisticians have
very similar strategies.
Each use the known data to characterize the model's covariance.  At SEP we
often characterize the covariance through Prediction Error Filters (PEFs)
\cite[]{gee}.  Geostatisticians build variograms  from the known data
to represent the model's covariance \cite[]{geostat}.  
Once each has some measure of the model covariance they attempt to fill
in the missing data. Here their goals slightly diverge.
The geophysicist solves a global estimation problem and
attempts to create a model whose 
covariance is equivalent to the covariance of the known data.
The geostatistician performs kriging, solving a series of
local estimation problem. Each model estimate
is the linear combination of nearby data points
that best fits their predetermined covariance estimate.
Both of these approaches are in some ways exactly what we want:
given a problem give me  `the answer'.  
\par
The single solution approach
however has a couple significant drawbacks.  First, the solution tends to 
have low spatial frequency. Second, it does not provide information
on model variability or provide error bars on our model estimate.
Geostatisticians  have these
abilities  in their repertoire through
what they refer to as `multiple realizations' or `stochastic simulations'.
They introduce a random component, based on
properties (such as variance) of the data,   to their estimation procedure.  
Each realization's frequency content is more accurate and by
comparing and contrasting  the equiprobable realizations,
model variability can be assessed.
\par
In this paper I present a method to achieve the same goal
using a formulation that better fits into geophysical techniques.  
I modify the model styling goal,
replacing the zero vector with a random vector.  I show  how
the resulting models have a more pleasing texture and can
provide information on variability.
