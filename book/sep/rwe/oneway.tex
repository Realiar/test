\section{One-way wave-equation in 3-D Riemannian spaces}
Equation~\reqo{weqrc.3d} can be used to describe 
two-way propagation of acoustic waves in a semi-orthogonal 
Riemannian space.
For one-way wavefield extrapolation, we need to modify
the acoustic wave \req{weqrc.3d} by selecting a single 
direction of propagation.
\par
In order to simplify the computations,
we introduce the following notation:
\beqa 
\czz &=& \frac{1  }{\AA^2}                                    \;, \nonumber \\
\cxx &=& \frac{\GG}{\JJ^2}                                    \;, \nonumber \\
\cyy &=& \frac{\EE}{\JJ^2}                                    \;, \nonumber \\
\cxy &=& \frac{\FF}{\JJ^2}                                    \;, \nonumber \\
\cz  &=& \frac{1}{\AA\,\JJ} \eone{\lp\frac{\JJ}{\AA}\rp}{\qz} \;, \nonumber \\
\cx  &=& \frac{1}{\AA\,\JJ} 
\lb \eone{\lp\GG\frac{\AA}{\JJ}\rp}{\qx} -
    \eone{\lp\FF\frac{\AA}{\JJ}\rp}{\qy} \rb \;, \nonumber \\
\cy  &=& \frac{1}{\AA\,\JJ} 
\lb \eone{\lp\EE\frac{\AA}{\JJ}\rp}{\qy} -
    \eone{\lp\FF\frac{\AA}{\JJ}\rp}{\qx} \rb \;.
\label{eqn:coefs.3d}
\eeqa
All quantities in \reqs{coefs.3d}
are only function of the chosen coordinate system, and 
do not depend on the extrapolated wavefield.
They can be computed by finite-differences for any choice of 
Riemannian coordinates
which fulfill the orthogonality condition indicated earlier.
In particular, we can use ray coordinates to compute those 
coefficients. With these notations, the acoustic wave-equation
can be written as:
\beq \label{eqn:weqrc.3d.coefs}
\czz \dtwo{\W}{\qz} +
\cxx \dtwo{\W}{\qx} + 
\cyy \dtwo{\W}{\qy} +
\cz  \done{\W}{\qz} +
\cx  \done{\W}{\qx} + 
\cy  \done{\W}{\qy} +
\cxy \mtwo{\W}{\qx}{\qy} = - \frac{\ww^2}{\vv^2} \W \;.
\eeq
For the particular case of Cartesian coordinates
($\cx=\cy=\cz=0, \cxx=\cyy=\czz=1, \cxy=0$),
the Helmholtz \req{weqrc.3d.coefs} reduces to its familiar form
\beq 
    \dtwo{\W}{\qz}
 +  \dtwo{\W}{\qx}
 +  \dtwo{\W}{\qy}
 = -\frac{\ww^2}{\vv^2} \W \;.
\eeq
\par
From \req{weqrc.3d.coefs}, we can directly
deduce the modified form
of the dispersion relation for the wave-equation in a
semi-orthogonal {3-D} Riemannian space:
\beq \label{eqn:disp.3d}
- \czz \kz^2
- \cxx \kx^2
- \cyy \ky^2
+i\cz  \kz
+i\cx  \kx
+i\cy  \ky
- \cxy \kx\ky = - \ww^2\ss^2 \;.
\eeq
For one-way wavefield extrapolation, we need to solve
the quadratic \req{disp.3d} 
%%\beq
%%  \czz \kz^2
%%-i\cz  \kz 
%%+ \lb
%%  \cxx \kx^2 
%%+ \cyy \ky^2
%%-i\cx  \kx   
%%-i\cy  \ky
%%+ \cxy \kx\ky
%%- \ww^2\ss^2 \rb =0 \;,
%%\eeq
for the wavenumber of the extrapolation direction $\kz$,
and select the solution with the appropriate sign to
extrapolate waves in the desired direction:
\beq \label{eqn:oneway.3d}
\kz = i \frac{\cz}{2\czz} \pm
\sqrt{
\frac{\lp\ww\ss\rp^2}{\czz} - \lp\frac{\cz}{2\czz}\rp^2
- \lb \frac{\cxx}{\czz}\kx^2 -i \frac{\cx}{\czz}\kx \rb
- \lb \frac{\cyy}{\czz}\ky^2 -i \frac{\cy}{\czz}\ky \rb
- \frac{\cxy}{\czz} \kx\ky 
}\;.
\eeq
The solution with the positive sign in \req{oneway.3d} corresponds to 
propagation in the positive direction of the extrapolation axis $\qz$.
\par
For the particular case of Cartesian coordinates
($\cx=\cy=\cz=0, \cxx=\cyy=\czz=1, \cxy=0$),
the one-way wavefield extrapolation equation takes the
familiar form
\beq
\kz = \pm\sqrt{\lp\ww\ss\rp^2 -\kx^2 - \ky^2}\;.
\eeq
\par
We can simplify our Riemannian wavefield extrapolation method 
by dropping the first-order terms in \req{oneway.3d}. 
According to the theory of 
characteristics for
second-order hyperbolic equations \cite[]{courant}, 
these terms affect only the amplitude of the propagating waves. 
To preserve the kinematics, it is sufficient to keep only the 
second order terms of \req{oneway.3d}:
\beq \label{eqn:oneway.3d.kinematic}
\kz = \pm
\sqrt{
\frac{\lp\ww\ss\rp^2}{\czz}
- \frac{\cxx}{\czz}\kx^2
- \frac{\cyy}{\czz}\ky^2
- \frac{\cxy}{\czz}\kx\ky 
}\;.
\eeq
We do not address in this paper the meaning of true-amplitude
migration using one-way wavefield extrapolation.
\par
Finally, we note that the coordinate system coefficients for 
Riemannian wavefield extrapolation given by \reqs{coefs.3d} 
have singularities at caustics, 
e.g., when the geometrical spreading term $\JJ$,
defining a cross-sectional area of a ray tube, goes to zero.
In our examples, we have used simple numerical
regularization, by adding a small non-zero quantity to the 
denominators to avoid division by zero. 
