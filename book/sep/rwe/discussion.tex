%%\section{Discussion}
\section{Practical aspects}
\begin{description}
%%
\item[Coordinate system construction:]
The ray coordinate systems do not need to be created using the
same velocity model as the one used for extrapolation.
We can use a smooth velocity model to create the coordinate
system by ray tracing, and then remap the rough velocity
\uline{to the ray coordinates}, 
similar to the method used by \cite{BDEtgen-EAGE03}.
An alternative method of creating ray coordinate systems
is discussed by \cite{Shragge.segab.2004}.
The coordinate system can be initiated from a point source
or an arbitrary surface in 3D (or a line in 2D) which positions 
it optimally relative to our target.
Furthermore, with Riemannian wavefield extrapolation,
we can address a particular target
in the image and thus we do not need to construct a
coordinate system which is appropriate for the entire
image.
%%
\item[Coordinate system regularization:]
The coordinate system coefficients for Riemannian wavefield 
extrapolation given by \reqs{coefs.3d} have singularities at 
caustics, i.e. when the geometrical spreading term $\JJ$,
defining a cross-sectional area of a ray tube, is zero.
We address this problem through a simple numerical
regularization, by adding a small non-zero quantity to the 
denominators to avoid division by zero. This strategy worked
reasonably well for our current examples, although better
strategies are needed.
\par
In principle, it is best if coordinate system 
triplications are avoided.
    However, for velocity models with large
    contrasts (e.g. salt), avoiding such triplications 
    may require large smoothing prior to ray tracing. 
    In these situations,
    there is a strong possibility that the waves do not
    propagate close to our extrapolation axis, thus requiring 
    higher-order terms in the extrapolator 
    at increased cost.
%%
\item[Prestack data:]
Our examples of Riemannian wavefield extrapolation are based on
\req{oneway.3d} which corresponds to the single-square root 
(SSR)
equation of standard Cartesian wavefield extrapolation.
Riemannian wavefield extrapolation can be extended
to prestack data either for shot-profile, plane-wave or 
S-G migration by appropriate definitions of the 
underlying ray coordinate system.
Figure~\ref{fig:spmig} is a schematic representation of 
shot-profile migration in ray coordinates, 
where both source and receivers
are extrapolated in the same ray coordinate system
appropriate for overturning waves.
However, source and receiver wavefields
can be migrated in different coordinate systems,
with the imaging condition applied after interpolation to 
Cartesian coordinates. 
%%
\sideplot{spmig}{width=3.0in}
{Shot-profile migration sketch.
Sources (a) and receivers (b) are both extrapolated in
a ray coordinate system appropriate for 
overturning waves.
}
%%
\item[Interpolation:]
The images created with wavefield extrapolation in Riemannian
coordinates require interpolation to a Cartesian coordinate 
system. This is a shared difficulty of all methods 
operating on non-Cartesian grids. In our current implementation, 
we use simple sinc-type interpolation based on the
explicit mapping of the Cartesian coordinates $(\x,\z)$
function of the ray coordinates $(\t,\g)$ given by ray tracing.
%%
\item[Cost:]
The main cost of an implicit finite-difference solution to the 
one-way equation in Riemannian coordinates involves solving a
tridiagonal system. In this respect, the cost of Riemannian
wavefield extrapolation is identical to the cost of Cartesian
downward continuation for the same number of samples.
However, computing the coefficients of the tridiagonal 
system adds modestly to the cost, since they can be 
precomputed ahead of time.
\par
A second consideration is that
    we are comparing extrapolation 
    in different domains (space for downward continuation
    and shooting angle for Riemannian extrapolation). 
    Since in Riemannian coordinates
    we extrapolate at small angles, we can
    sample the wavefronts less and achieve same or better
    angular accuracy than in Cartesian coordinates at lower
    cost.
%%
\end{description}







