In the previous sections I have shown why mode separation
in two dimensions should always work,
and that mode separation in three dimensions should
always work for {\qP} modes.
Is there any reason to expect mode separation in three dimensions
to fail for {\qS} modes?

\subsection{Defining ``the'' isotropic shear modes}

In isotropic media, it is customary to consider the degenerate
shear wave as a single vector pure mode.
We typically resolve this globally degenerate mode into
``SV'' and ``SH'' scalar modes, but this choice is arbitrary.
Isotropy, with its infinite symmetry,
lies at the intersection of all the various anisotropic
symmetry systems.
Perturb the elastic constants away from isotropy
and the perfect global symmetry is broken; the shear
waves become nondegenerate.

The standard ``\{SV,SH\}'' labels result
if we perturb in the direction of transverse isotropy
with a vertical symmetry axis.
There are many other possibilities,
resulting in other equally valid
definitions of ``the isotropic shear modes''.
If the SV and SH modes as shown in Figure~\ref{Problem-Slowsurf}
are arbitrary, why are they so widely used?
SV and SH are popular because they are easy to
understand in terms of vertical two-dimensional slices,
which is how we usually look at the Earth.
Figure~\ref{Separ3-Funnyshear}
shows an example of the bizarre sorts of modes that usually
result if the elastic constants are perturbed randomly.

\Tactiveplot{Separ3-Funnyshear}{width=5.75in}{\figdir}
{Shear modes resulting from a random perturbation to isotropy.}
{
Shear slowness surfaces for arbitrary isotropic shear modes.
Mathematically these modes are just as valid a decomposition
of the degenerate isotropic shear modes as the
standard method shown in Figure~\protect\ref{Problem-Slowsurf}.
One singularity is labeled, although several are visible.
(The top plot is the slow ({\qS2}) split shear wave;
Figure~\protect\ref{Sing-Funny} shows another view.)
\index{shear singularities | example}
\index{slowness surface | 3D example}
}

Previously we saw that the standard SV and SH modes shown in
Figure~\ref{Problem-Slowsurf} have an artificial
discontinuity for vertically traveling waves.
Such a point on the slowness surface where
the particle-motion direction becomes a discontinuous
function of phase direction is called a ``singularity''.
\index{definition | singularity}
\index{particle motion | singularities in}
They are usually associated with S or {\qS} waves, and so are
often called ``shear singularities''.
\index{shear singularities}
The arbitrary modes shown in Figure~\ref{Separ3-Funnyshear}
likewise display shear singularities in the particle-motion direction
field for waves traveling in several apparently random directions.
Singularities are annoying, because they cause discontinuities
in the vector field $\L$ at the core of our wavetype-separation algorithm.

\subsection{Furry ball theorem}

Finally we are back to our original question
from page~\pageref{Separ3-Question}:
Are singularities avoidable?
Pure S-type motion requires that the particle-motion direction
be perpendicular to the direction of plane-wave propagation.
In Figure~\ref{Problem-Slowsurf}, the particle-motion direction
is represented by ``hairs'' attached to the slowness surface.
For a pure shear wave, each hair must lie flat against
the ball's surface.
Imagine what happens if you try to comb the hairs
on a furry ball flat and regular everywhere.
No matter how you proceed, there must always be
a ``whorl'' or ``cowlick'' somewhere,
where adjacent hairs point in different directions.\footnote{
This is a special case of a well known result from
differentiable manifold theory:
no even-dimensional sphere has any continuous nonvanishing field of
tangent vectors
\refer{Boothby}
{An introduction to differentiable manifolds and Riemannian geometry}
{1975}.
}
We now have the answer: these discontinuous points
are shear singularities, and they cannot be eradicated.
There is no way to define two globally continuous
isotropic shear modes.

Shear-wave singularities must always occur for pure S modes,
but what about {\qS} modes?
Simply apply the previous argument for pure S modes
to the pure S component
of the particle-motion direction at each point (i.e., 
the component of the particle motion perpendicular
to the plane-wave propagation direction).
The only way to escape the singularity is by letting the
pure S component of the particle-motion direction go to zero
where we would have had trouble
(i.e. the ``{\qS} mode'' has a pure P-mode direction there:
the ``hair'' at the center of the whorl sticks straight up).
Since the defining property of a {\qS} mode is that
it does not have pure P-direction particle motion anywhere,
this cannot happen.

\subsection{What do shear singularities represent?}
We know that physical
wavefronts do not support discontinuous particle-motion directions.
How can we reconcile this observation with the ubiquity of singularities?

For any phase direction there are three {\em orthogonal} modes,
so it is impossible for one wavetype to have a discontinuous
particle-motion direction in isolation;
at least one of the other wavetypes
must share in the discontinuous particle-motion directions
at the singular point (although rotated by $90^\circ$).
Precisely at the singular point the particle-motion directions
of these two modes must become indeterminate.
This can only happen if the two modes are degenerate there.

A singularity, then, must correspond to a point where two
otherwise continuous and distinct modes touch and temporarily
lose their unique identities
\refer{Crampin and Yedlin}
{Shear-wave singularities of wave propagation in anisotropic media}
{1981}.
\label{Separ3-Touch}
We can see this illustrated in Figure~\ref{Separ3-Wavedef};
the singularity for vertically traveling S waves (along the $k_z$ axis)
always occurs on two intersecting surfaces. On the axis particle
motions in the $x$ and $y$ directions are symmetrically equivalent.
Figure~\ref{Separ3-Mobius} shows the nature of singular points
even more clearly.
In this orthorhombic example the {\qS} modes are clearly
two separate shells that pinch together at a finite number
of singular points.

\Tactiveplot{Separ3-Mobius}{width=5.75in}{\figdir}
{A canonical orthorhombic slowness surface.}
{
The three-dimensional slowness surfaces
of the same orthorhombic medium shown
in Figure~\protect\ref{Problem-Anisotrick2}.
One octant has been removed to show the true topology:
two nested {\qS} surfaces that just touch at the singularities.
(There are four singularities in all; the other three are symmetrically
equivalent to the labeled one.)
Neither of the surfaces can be adequately
described as ``{\qSV}'' or ``{\qSH}''.
\index{shear singularities | example}
\index{slowness surface | 3D example}
}

A shear singularity does not cause a discontinuous particle motion
in reality because for some range of angles around the
singular phase direction two orthogonal wavetypes
are strongly coupled together.
While individually their particle-motion vectors rapidly twist around,
for any finite frequency their sum remains smooth.
(Singularities can produce other visible effects, however, as
will be demonstrated in section~\ref{Exam3-Chimera}.)

\subsection{{\qS} modes are inseparable}
I have now shown that any {\qS} mode has singularities on it,
and that singularities are places where two modes touch.
If there are two {\qS} modes that are both everywhere slower
than a single {\qP} mode, then the two {\qS} modes must be
touching each other.
Does this automatically mean mode separation of
{\qS} waves is impossible? How does trouble occur?

At step~(2A),
we must expect occasional trouble with nonunique {\qS} solutions
in three dimensions,
because shear-wave singularities represent
degenerate directions.
A typical way around such problems in ray tracing is to perturb
the elastic constants so that the singularity moves off
our current ray direction
\refer{Jech and P\v{s}en\v{c}\'{\i}k}
{First order perturbation method for anisotropic media}
{1989}.
This procedure shouldn't create significant errors since
a tiny perturbation in the elastic constants should only cause a
tiny change in the global wavefield.
(We are still left with the problem of figuring out which
of the barely split modes is the one we want.)

Steps (1)~and~(2B) of our algorithm presuppose we know what the
global {\qS} modes look like. (I haven't even yet proven that there
is such a thing as a global {\qS} mode for general anisotropy!)
The isotropic modes are usually separated into
\{P,SV,SH\}, and the transversely isotropic modes into
\{\qP,\qSV,SH\}, but we have seen these designations are arbitrary
at best.  Is there any sensible way to
separate the connecting {\qS} surfaces into two {\qS} pure modes
for completely general anisotropic media?

The example in Figure~\ref{Separ3-Mobius} suggests one possibility.
We wish to cut the two-sheeted {\qS} surface in this figure
into two pure modes.
We cannot do this arbitrarily.
\index{definition | pure mode}
\index{wave modes | definition}
\index{wave modes | general}
Step~(2B) of our algorithm requires that the three
plane wavetypes at each $(k_x,k_y,k_z)$,
$ \{ {\vv}_1, {\vv}_2, {\vv}_3 \} $,
must be identified as a permutation of
the three global wave modes
(usually \{{\qP},{\qS1},{\qS2}\})
evaluated for that phase direction.
The three resulting pure modes must be
unique single-valued {\em continuous} functions of phase velocity versus
plane-wave propagation direction.
(In fact I would say this must be the definition of a global pure mode.)
As should be clear from Figure~\ref{Separ3-Mobius},
the only possible place to cut is at the shear singularities.

This definition makes the modes very easy to identify
computationally; the modes are simply sorted on their phase velocity
% Reference from chapter 2 where we talk about ordering wavetypes
% by phase velocities.
\label{Separ3-speedorder}
\index{wave modes | ordering}
$(\omega / k )$ into an inner (fastest) {\qP},
\index{definition | {\qS1} and {\qS2} modes}
a middle {\qS1}, and an outer (slowest) {\qS2} mode
\refer{Crampin}
{A review of wave motion in anisotropic and cracked elastic media}
{1981}.
(Of course while this method is the most general, it still may make
sense to use notations such as \{\qP,\qSV,SH\} when appropriate ---
although these symmetric notations are misleading, as will be shown
in section~\ref{Exam3-Chimera}.)

Unfortunately, as can be seen in Figure~\ref{Separ3-Blowup},
sorting the modes by phase velocity
does nothing to ameliorate the particle-motion discontinuity
at the shear singularities. As I will demonstrate in the next section,
this local discontinuity in the Fourier
domain produces a large planar artifact in the space-domain version
of the corresponding wavetype-separation operator, although
it can be suppressed in certain special cases.

We have seen there are troublesome problems at step~(2B).
What about step~(2C)? Figure~\ref{Separ3-Blowup} shows most of the particle-motion
direction vectors as two-sided, since mathematically there is nothing
in equation~(\ref{TI-chris}) to distinguish any eigenvector $\vv$
from its opposite $-\vv$. I have attempted to pick one sign
as ``positive'' for each particle-motion vector along a path
(shown as a dotted line) around the singularity;
these vectors are shown as one-sided rays.
Unfortunately if we pick the vectors to preserve
continuity as we move along the path, we find the signs don't match
when we complete the loop. There is an inescapable branch cut in
step~(2C).\footnote{This is an example of a phenomenon of much
current interest in quantum physics, anholonomy
\refer{Berry}
{Anticipations of the geometric phase}
{1990}.
\index{anholonomy}
The basic idea is that after some parameters of a system have been
perturbed around a closed loop in a continuous way, some other
seemingly unrelated parameters are found to have changed sign or
otherwise shifted in phase.}

\Tactiveplot{Separ3-Blowup}{width=5.75in}{\figdir}
{A close-up view of a slowness surface showing a canonical shear singularity.}
{
A close-up view of the outer surface in Figure~\protect\ref{Separ3-Mobius},
centered on the shear singularity (here indicated by an ``{\bf $S$}'').
Note that the particle-motion directions become discontinuous at
the singularity.
There is a more subtle disturbance at the singularity as well.
Most of the particle-motion direction vectors are drawn two-sided.
For the vectors along the dotted path looping around the singularity,
however, we have attempted to pick a preferred ``positive'' direction and so
these are drawn one-sided.
Once we make a choice of sign at the beginning, we are constrained at each
step to choose the direction consistent with the previous choice.
Unfortunately, the continuity we have carefully maintained going around
the loop is lost at the end; the choices at the beginning and end of
the loop do not agree.
Contrast this situation with the one in two dimensions shown in
Figure~\protect\ref{Separ3-Arrow2}.
\index{shear singularities | example}
\index{slowness surface | 3D example}
}

This branch cut
makes a mess of any attempt to construct a pure-mode scalar field.
If we are satisfied merely to separate the wavetypes, however, we
can avoid the sign-choice problem by using a slightly modified algorithm:
\label{Separ3-ModAlg}
\begin{list}{}{}
\item[($\hbox{\rm 3.}^\prime$)] Fourier transform $\U(x,y,z;t)$ over $x$, $y$, and $z$,
obtaining $\hat \U(k_x,k_y,k_z;t)$.
\item[($\hbox{\rm 4.}^\prime$)] For all $(k_x,k_y,k_z)$:
\item[] \{
$$ \hat \Mode(k_x,k_y,k_z;t) = \L(k_x,k_y,k_z) \Bigl( \L(k_x,k_y,k_z) \joecdot \hat \U(k_x,k_y,k_z;t) \Bigr) $$
\item[] \}
\item[($\hbox{\rm 5.}^\prime$)] Inverse Fourier transform $\hat \Mode(k_x,k_y,k_z;t)$
over $k_x$, $k_y$, and $k_z$, obtaining $\Mode(x,y,z;t)$.
\end{list}
The branch cut does not adversely affect this operator because the
sign of $\vv$ enters the equation twice, and so cancels out.

\Tactiveplot{Separ3-Sfig}{width=5.75in}{\figdir}
{Wavetype separation has trouble for 3D {\qS} modes.}
{Three-dimensional {\qS} mode-separation examples.
The plot and parameters are like those in Figure~\protect\ref{Separ3-Pfig},
except this time the $x$ component of displacement is shown and
a gpow of .8 has been applied to make weak events more easily visible.
Upper left: The original wavefield.
Upper right: The {\qP} mode has been successfully nulled.
Lower left: Trying to pass only the {\qS1} wave causes artifacts.
Lower right: The {\qS2} wave likewise doesn't work by itself.
The sum of the two lower plots gives the upper right plot.
\index{wavetype separation | example}
}
\Tactiveplottop{Separ3-SVfig}{width=5.75in}{\figdir}
{Trying to define and separate a 3D {\qSV} pseudo-mode.}
{Two attempts at defining and separating a three-dimensional {\qSV} mode.
The plot and parameters are like those in Figure~\protect\ref{Separ3-Sfig}.
Left: For each $(k_x,k_y,k_z)$ the {\qS} plane-wave mode with
particle motion more nearly ``SV'' was chosen. This pseudo-mode does
not have a continuous phase-velocity function.
Right: The orthorhombic elastic constants were perturbed to become
transversely isotropic, and the wavetype-separation operator appropriate for
the {\qSV} mode for the perturbed TI medium was applied to the original
orthorhombically anisotropic wavefield. As a result the {\qP} mode is
not completely nulled.
\index{wavetype separation | example}
}

This new wavetype-separation operator does not attempt to
collapse the vector wavefield onto a scalar pure mode $M$. Instead
it nulls the two unwanted wave types while passing the desired
pure-mode component through unchanged.
Since $\Mode$ is a vector wavefield like $\U$, some important
manipulations become easier than they would be with a scalar mode like $M$.
For example, we can construct the difference $\U - \Mode$;
this {\it nulls}\/ a single wavetype instead of passing a single wavetype.

Unfortunately, the fundamental problem of particle-motion discontinuities
at shear singularities remains. There is no solution to this problem;
it is a consequence of the basic inconsistency between
the ``furry ball'' theorem and
the definition of a {\qS} ``pure mode''.

I conclude that in general anisotropic media it is not possible
to cleanly separate the ``{\qS1} and {\qS2} modes''. The particle-motion
discontinuity at the singularities will always cause the two
modes to leak energy into each other.
\index{shear singularities | mode-mode coupling}
\index{wave modes | coupling}
This mode-mode coupling is especially bothersome for elastic ray tracing in
three dimensions.
It is possible to handle the coupling,
but only by propagating the two not-quite-distinct shear modes together
\refer{Chapman and Shearer}
{Ray tracing in azimuthally anisotropic media}{1989}.
While it may often prove useful to differentiate two {\qS} waves
for some set of propagation angles,
globally they are inextricably linked and cannot be cleanly
separated.

\subsection{Examples}
Figure~\ref{Separ3-Sfig} shows how this works in practice.
The upper left plot shows the trial input wavefield.
In the upper right plot the {\qP} wave has been successfully removed.
In the lower two plots I have attempted to pass only the {\qS1} or
{\qS2} pure modes. The Fourier-domain particle-motion discontinuities
at the shear singularities create strong planar artifacts that can
be seen to wrap around several times. The artifacts have opposite
signs on the {\qS1} and {\qS2} plots; when the two {\qS} modes are summed
the artifacts cancel and the ``{\qP} removed'' plot results.

For these models a $z$ source was used.
Since the only particle-motion discontinuity for SV modes is for
vertically traveling shear waves, which are unexcited by a $z$ source,
we might expect an SV mode-separation operator to be more tractable.
There are two possible approaches to use; Figure~\ref{Separ3-SVfig}
shows both of them.

In the left example in Figure~\ref{Separ3-SVfig}
for each $(k_x,k_y,k_z)$ plane-wave direction
I have chosen the mode that most nearly fits the designation ``SV''.
Sure enough, this operator picks out something resembling an SV mode,
but the resulting wavefield is cluttered by the artifacts caused by
both the particle-motion direction {\em and} phase-velocity
discontinuities.

In the right example in Figure~\ref{Separ3-SVfig}
I have perturbed the elastic constants of the
wavetype-separation operator to become transversely isotropic.
(This was easy to do, since the original orthorhombic elastic constants
were generated by numerically ``cracking''
\refer{Schoenberg and Muir}
{A calculus for finely layered anisotropic media}
{1989}
transversely isotropic
Greenhorn Shale
\refer{Jones and Wang}
{Ultrasonic velocities in Cretaceous shales from the Williston basin}
{1981}
in the first place.)
For the perturbed medium there is a true {\qSV} mode, and
so the separation works (given the $z$ source used in this example).
The {\qP} mode does slightly leak through because the elastic constants
don't quite fit the medium, however.
