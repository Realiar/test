We begin with a simple example of how two-dimensional intuition can
be misleading.
Refer to Figure~\ref{Problem-Anisotrick1},
which shows three symmetry-plane slices through a simple
orthorhombic anisotropic slowness surface.
Each slice shows what seem to be pure
P, SV, and SH modes; furthermore, on each slice the P mode and one
of the S modes are circular, while the other S mode is
elliptical. Does this mean this medium supports pure P,
SV, and SH modes?

\inputdir{.}
\plot{problem0}{width=5.75in}{
Symmetry-plane slices through the three-dimensional slowness surfaces
of a simple orthorhombic medium.
The fat bars and dots show particle-motion direction. (Dots represent
motion in and out of the plane.)
Following the numbers from $1$ through
$3$ seems to show the shear surfaces have a mobius-like topology.
To see what's really happening,
look at Figure~\protect\ref{Problem-Anisotrick2}.
\index{Symmetry plane | slices}
\index{slowness surface | 3D slice example}
}

No, as we can see by attempting to follow one shear surface around
through all three plots. Start at ``1'' on the left plot where
the ``SV surface'' intersects the $(k_x / \omega)$ axis. Trace from
there up to ``2''. Now jump to ``2'' on the center plot (remembering
that in three dimensions both ``2''s mark the same point).
Continuing on in the same way, we get to ``3'' and jump to the right plot.
Here there is a surprise:
we are now on the ``SH surface'', and the ``SH surface'' appears not
to reconnect back to ``1'' where we started.
What is going on?

Figure~\ref{Problem-Anisotrick2} shows a more accurate three-dimensional view.
(\cite{musgrave1981elastodynamic}
shows similar pictures for several physical media.)
At ``3'', the outer surface
is ``SV'' in the $k_y$--$k_z$ plane, but ``SH'' in the $k_x$--$k_z$
plane.
This is possible because the designations
``SH'' and ``SV'' are relative concepts that
depend on the orientation of the slice under consideration.
Is the labeled ``crossing'' really a point where a ``{\qSH}'' and
``{\qSV}'' surface intersect? No, as we can see by
closely examining the slices cut
at $10^\circ$ and $45^\circ$ angles to the $k_x$--$k_z$ plane.
Clearly trying to shoehorn shear modes like these into ``{\qSV} and {\qSH}''
designations is not going to work in three dimensions.
(\cite{crampin1981review} also remarks on many of these points.)

\plot{problem1}{width=5.75in}
{
Slices through the three-dimensional slowness surfaces
of a simple orthorhombic medium.
The fat bars show particle-motion direction.
The ``crossing'' is not what it appeared to be in
Figure~\protect\ref{Problem-Anisotrick1}.
For another view of the same surface,
look ahead to Figure~\protect\ref{Separ3-Mobius}.
\index{Symmetry plane | slices}
\index{slowness surface | 3D slice example}
\index{wave modes | orthorhombic}
}

\Tactiveplot{Problem-Slowsurf}{width=5.75in}{\figdir}
{The canonical isotropic ``P, SV, and SH'' slowness surfaces in 3D.}
{
Three-dimensional slowness surfaces for the standard \{P,SV,SH\}
isotropic wave modes. Slowness surfaces are plots of phase slowness
versus plane-wave propagation direction; they can also be considered as
dispersion relation plots.
We show particle-motion direction by the ``sticks'' attached to
the surfaces.
All three surfaces could not be plotted together because the two
shear surfaces are everywhere coincident.
\index{wave modes | isotropic P SV SH}
\index{slowness surface | 3D example}
}

Perhaps this example is atypically perverse.
Do the standard isotropic shear modes harbor any similar misbehaviors?
In Figure~\ref{Problem-Slowsurf},
note that by defining the shear surfaces as SH and SV
we have introduced a nonphysical particle-motion direction
discontinuity for vertically propagating waves.
This is a consequence of the basic ``in-plane'' and ``cross-plane''
definitions of SV and SH;
the same vertically traveling wave with North-South particle motion
is called ``SH'' on an East-West section
but ``SV'' on a North-South one.
