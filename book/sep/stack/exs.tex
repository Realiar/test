\section{EXAMPLES}
%%%%%%%%%%%%%%%% 
In this section, I consider several particular examples of stacking
operators used in seismic data processing and derive their asymptotic
pseudo-unitary versions.

\subsection{Datuming}
%%%%%%%%%%%%%%%%
Let $x$ denote a point on the surface at which the propagating
wavefield is recorded. Let $y$ denote a point on another surface, to
which the wavefield is propagating. Then the summation path of the
stacking operator for the forward wavefield continuation is
\begin{equation}
\theta(x;t,y)  =  t - T(x,y)\;,
\label{eqn:datp}
\end{equation}
where $t$ is the time recorded at the $y$-surface, and $T(x,y)$ is the
traveltime along the ray connecting $x$ and $y$. The backward
propagation reverses the sign in (\ref{eqn:datp}), as follows:
\begin{equation}
\widehat{\theta}(y;z,x)  =  z + T(x,y)\;.
\label{eqn:datm}
\end{equation}
Substituting the summation path formulas (\ref{eqn:datp}) and (\ref{eqn:datm}) into
the general weighting function formulas (\ref{eqn:wp}) and (\ref{eqn:wm}), we
immediately obtain
\begin{equation}
w^{(+)}  =  w^{(-)}  =  {1\over{\left(2\,\pi\right)^{m/2}}} \, 
\left|{{\partial^2 T}\over{\partial x\,\partial y}}\right|^{1/2}\;.
\label{eqn:datw}
\end{equation}
Gritsenko's formula \cite[]{gritsenko,ig1} states that the second mixed
traveltime derivative ${{\partial^2 T}\over{\partial x\,\partial y}}$
is connected with the geometric spreading $R$ along the $x$-$y$ ray by
the equality
\begin{equation}
R(x,y) = {\sqrt{\cos{\alpha(x)}\,\cos{\alpha(y)}}\over v(x)}\,
\left|{{\partial^2 T}\over{\partial x\,\partial y}}\right|^{-1/2}\;,
\label{eqn:gritsenko}
\end{equation}
where $v(x)$ is the velocity at the point $x$, and $\alpha(x)$ and
$\alpha(y)$ are the angles formed by the ray with the $x$ and $y$
surfaces, respectively. In a constant-velocity medium,
\begin{equation}
R(x,y) = v^{m-1}\,T(x,y)^{m/2}\;.
\end{equation}
Gritsenko's formula (\ref{eqn:gritsenko}) allows us to
rewrite equation (\ref{eqn:datw}) in the form \cite[]{goldin}
\begin{eqnarray}
w^{(+)}(x;t,y) & = & {1\over{\left(2\,\pi\right)^{m/2}}} \,
{\sqrt{\cos{\alpha(x)}\,\cos{\alpha(y)}}\over {v(x)\,R(x,y)}}\;,
 \\
w^{(-)}(y;z,x) & = & {1\over{\left(2\,\pi\right)^{m/2}}} \,
{\sqrt{\cos{\alpha(x)}\,\cos{\alpha(y)}}\over {v(y)\,R(y,x)}}\;.
\end{eqnarray}
\par
The weighting functions commonly used in Kirchhoff datuming
\cite[]{GEO44-08-13291344,GEO49-08-12391248,svgdat} are defined as
\begin{eqnarray}
w(x;t,y) & = & {1\over{\left(2\,\pi\right)^{m/2}}} \,
{{\cos{\alpha(x)}}\over {v(x)\,R(x,y)}}\;,
 \\
\widehat{w}(y;z,x) & = & {1\over{\left(2\,\pi\right)^{m/2}}} \,
{{\cos{\alpha(y)}}\over {v(y)\,R(y,x)}}\;.
\label{eqn:datw2}
\end{eqnarray}
These two operators appear to be asymptotically inverse according to
formula (\ref{eqn:whatw}). They coincide with the asymptotic pseudo-unitary
operators if the velocity $v$ is constant ($v(x)=v(y)$), and the two
datum surfaces are parallel ($\alpha(x) = \alpha(y)$).


\subsection{Migration}
%%%%%%%%%%%%%%%%%
\emph{Least-squares} migration, envisioned by \cite{lailly} and 
\cite{taran}, has recently
become a practical method and gained a lot of attention in the
geophysical literature \cite[]{nemeth99,chavent,duquet,resol}. Using the
theory of asymptotic pseudo-unitary operators allows us to reconcile 
this approach with the method of \emph{asymptotic true-amplitude}
migration \cite[]{bleistein}.

As recognized by \cite{tygel}, true-amplitude migration
\cite[]{tam,vor} is the asymptotic inversion of seismic modeling
represented by the Kirchhoff high-frequency approximation. The
Kirchhoff approximation for a reflected wave \cite[]{haddon,norm}
belongs to the class of stacking-type operators (\ref{eqn:operator})
with the summation path
\begin{equation}
\theta(x;t,y)  =  t - T\left(s(y),x\right) - T\left(x,r(y)\right)\;,
\label{eqn:modt}
\end{equation}
the weighting function
\begin{equation}
w(x;t,y)  =  {1\over{\left(2\,\pi\right)^{m/2}}} \,
{{C\left(s(y),x,r(y)\right)} 
\over {R\left(s(y),x\right)\,R\left(x,r(y)\right)}}\;,
\label{eqn:modw}
\end{equation}
and the additional time filter $\left({\partial \over {\partial
z}}\right)^{m/2}$. Here $x$ denotes a point at the reflector surface,
$s$ is the source location, and $r$ is the receiver location at the
observation surface. The parameter $y$ corresponds to the
configuration of observation. That is, $s(y) = s\,,\;r(y) = y$ for the
common-shot configuration, $s(y) = r(y) = y$ for the zero-offset
configuration, and $s(y) = y - h\,,\;r(y) = y + h$ for the
common-offset configuration (where $h$ is the half-offset). The
functions $T$ and $R$ have the same meaning as in the datuming example,
representing the one-way traveltime and the one-way geometric
spreading, respectively. The function $C(s,x,r)$ is known as the {\em
obliquity factor}. Its definition is
\begin{equation}
C(s,x,r) = {1 \over 2}\,
\left({{\cos{\alpha_s(x)}} \over {v_s(x)}} +
      {{\cos{\alpha_r(x)}} \over {v_r(x)}}\right)\;, 
\label{eqn:obliquity}
\end{equation}
where the angles $\alpha_s(x)$ and $\alpha_r(x)$ are formed by the
incident and reflected waves with the normal to the reflector at the
point $x$, and $v_s(x)$ and $v_r(x)$ are the corresponding velocities
in the vicinity of this point. In this paper, I leave the case of
converted (e.g., P-SV) waves outside the scope of consideration and
assume that $v_s(x)$ equals $v_r(x)$ (e.g., in P-P reflection). In this
case, it is important to notice that at the stationary point of the
Kirchhoff integral, $\alpha_s(x) = \alpha_r(x) = \alpha(x)$ (the law of
reflection), and therefore
\begin{equation}
C(s,x,r) = {{\cos{\alpha(x)}} \over {v(x)}}\;.
\label{eqn:statpoint}
\end{equation}
The stationary point of the Kirchhoff integral is the point where the
stacking curve (\ref{eqn:modt}) is tangent to the actual reflection
traveltime curve. When our goal is asymptotic inversion, it is
appropriate to use equation (\ref{eqn:statpoint}) in place of
(\ref{eqn:obliquity}) to construct the inverse operator. The weighted
function (\ref{eqn:modw}) can include other factors affecting the
leading-order (WKBJ) ray amplitude, such as the source signature, 
caustics counter (the KMAH-index), and transmission coefficient
for the interfaces \cite[]{chapman,cerveny}. In the following analysis,
I neglect these factors for simplicity.
\par
The model $M$ implied by the Kirchhoff modeling integral is the
wavefield with the wavelet shape of the incident wave and the
amplitude proportional to the reflector coefficient along the
reflector surface. The goal of true-amplitude migration is to recover
$M$ from the observed seismic data. In order to obtain the image of
the reflectors, the reconstructed model is evaluated at the time $z$
equal to zero. The Kirchhoff modeling integral requires explicit
definition of the reflector surface. However, its inverse doesn't
require explicit specification of the reflector location. For each
point of the subsurface, one can find the normal to the hypothetical
reflector by bisecting the angle between the $s-x$ and $x-r$
rays. Born scattering approximation provides a different physical
model for the reflected waves. According to this approximation, the
recorded waves are viewed as scattered on smooth local inhomogeneities
rather than reflected from sharp reflector surfaces. The inversion of
Born modeling \cite[]{GEO52-07-09430964,GEO52-07-09310942} closely
corresponds with the result of Kirchhoff integral inversion. For an
unknown reflector and the correct macro-velocity model, the asymptotic
inversion reconstructs the signal located at the reflector surface
with the amplitude proportional to the reflector coefficient.
\par
As follows from the form of the summation path (\ref{eqn:modt}), the
integral migration operator must have the summation path
\begin{equation}
\widehat{\theta}(y;z,x)  =  z + T\left(s(y),x\right) + T\left(x,r(y)\right)
\end{equation}
to reconstruct the geometry of the reflector at the migrated section.
According to equation~(\ref{eqn:inverse}), the asymptotic
reconstruction of the wavelet requires, in addition, the derivative
filter $\left(- {\partial \over {\partial t}}\right)^{m/2}$. The
asymptotic reconstruction of the amplitude defines the true-amplitude
weighting function in accordance with equation~(\ref{eqn:whatw}), as
follows:
\begin{equation}
\widehat{w}(y;z,x)  =  
{{v(x)\,R\left(s(y),x\right)\,R\left(x,r(y)\right)}   
\over {\left(2\,\pi\right)^{m/2}\,\cos{\alpha(x)}}}\,
\left|{{\partial^2 T\left(s(y),x\right)} 
\over {\partial x\,\partial y}} +     
      {{\partial^2 T\left(x,r(y)\right)} 
\over {\partial x\,\partial y}}\right|\;.
\label{eqn:migw}
\end{equation}
The weighting function of the asymptotic pseudo-unitary migration is
found analogously to equation~(\ref{eqn:datw}) as
\begin{equation}
w^{(+)}  =  w^{(-)}  =  {1\over{\left(2\,\pi\right)^{m/2}}} \,
\left|{{\partial^2
T\left(s(y),x\right)}
\over {\partial x\,\partial y}} +     
      {{\partial^2 T\left(x,r(y)\right)} 
\over {\partial x\,\partial y}}\right|^{1/2}\;.
\label{eqn:pumigw}
\end{equation}
Unlike true-amplitude migration, this type of migration operator
does not change the dimensionality of the input. Several specific cases exist
for different configurations of the input data.

\subsubsection{1. Common-shot migration}
In the case of common-shot migration, we can simplify equation (\ref{eqn:migw})
with the help of Gritsenko's formula (\ref{eqn:gritsenko}) to the form
\begin{equation}
\widehat{w}_{CS}(r;z,x)  =  {1\over{\left(2\,\pi\right)^{m/2}}} \,
{{\cos{\alpha(r)}} \over {v(x)}}\,
{{R(s,x)} \over {R(x,r)}} = {1\over{\left(2\,\pi\right)^{m/2}}} \,
{{\cos{\alpha(r)}} \over {v(r)}}\,
{{R(s,x)} \over {R(r,x)}} \;,
\label{eqn:csmigw}
\end{equation}
where the angle $\alpha(r)$ is measured between the reflected ray and
the normal to the observation surface at the reflector point $r$.
Formula (\ref{eqn:csmigw}) coincides with the analogous result of
\cite{GEO53-12-15401546}, derived directly from Claerbout's imaging
principle \cite[]{GEO35-03-04070418}. An alternative derivation is
given by \cite{goldin87}.  \cite{GEO56-08-11641169} points out a
remarkable correspondence between this formula and the classic results
of Born scattering inversion \cite[]{GEO52-07-09310942}.

For common-shot migration, pseudo-unitary weighting coincides with the
weighting of datuming and corresponds to the downward continuation of the
receivers.

\subsubsection{2. Zero-offset migration}
In the case of zero-offset mi\-gra\-tion, Grit\-senko's for\-mu\-la
sim\-pli\-fies the true-amp\-li\-tude mi\-gra\-tion weighting function
(\ref{eqn:migw}) to the form
\begin{equation}
\widehat{w}_{ZO}(y;z,x)  =  {{2^m}\over{\left(2\,\pi\right)^{m/2}}} \,
{{\cos{\alpha(y)}} \over {v(y)}}\;.
\label{eqn:zomigw}
\end{equation}
In a constant-velocity medium, one can accomplish the true-amplitude
zero-offset migration by premultiplying the recorded zero-offset
seismic section by the factor $\left(v \over 2 \right)^{m-1}\,\left(t
\over 2\right)^{m/2}$ [which corresponds at the stationary point to the
geometric spreading $R(x,y)$] and downward continuation according to
formula (\ref{eqn:datw2}) with the effective velocity $v/2$ 
\cite[]{goldin87,GEO56-01-00180026}. This conclusion is in
agreement with the analogous result of Born inversion
\cite[]{GEO50-08-12531265}, though derived from a different viewpoint. 

In the zero-offset case, the pseudo-unitary forward operator reduces to
downward pseudo-unitary continuation with a velocity of $v/2$.

\subsubsection{3. Common-offset migration}
In the case of common-offset migration in a general variable-velocity
medium, the weighting function (\ref{eqn:migw}) cannot be simplified to a
different form, and all its components need to be calculated
explicitly by dynamic ray tracing \cite[]{castro}.  In the
constant-velocity case, we can differentiate the explicit expression
for the summation path
\begin{equation}
\widehat{\theta}(y;z,x)  =  z + 
{{\rho_s(x,y) + \rho_r(x,y)} \over v}\;,
\end{equation}
where $\rho_s$ and
$\rho_r$ are the lengths of the incident and reflected rays:
\begin{eqnarray}
\rho_s(y,x) & = & 
\sqrt{x_3^2 + (x_1 - y_1 + h_1)^2 + (x_2 - y_2 + h_2)^2}\;,
 \\
\rho_r(y,x) & = & 
\sqrt{x_3^2 + (x_1 - y_1 - h_1)^ 2+ (x_2 - y_2 - h_2)^2}\;.
\end{eqnarray}
For simplicity, the vertical component of the midpoint $y_3$ is set here to zero. Evaluating the second derivative term in formula
(\ref{eqn:migw}) for the common-offset geometry leads, after some heavy
algebra, to the expression 
\begin{equation}
\left|{{\partial^2 T\left(s(y),x\right)} 
\over {\partial x\,\partial y}} +     
      {{\partial^2 T\left(x,r(y)\right)}
\over {\partial x\,\partial y}}\right| = 
{{x_3\,(\rho_s^2 + \rho_r^2)} \over {v\,(\rho_s\,\rho_r)^2}}\,
\left({{\rho_s + \rho_r} \over {v\,\rho_s\,\rho_r}}\right)^{m-1}
\,\cos{\alpha(x)}\;.
\label{eqn:cobeilk}
\end{equation}
Substituting (\ref{eqn:cobeilk}) into the general formula (\ref{eqn:migw}) yields
the weighting function for the common-offset true-amplitude
constant-velocity migration:
\begin{equation}
\widehat{w}_{CO}(y;z,x)  =  {1\over{\left(2\,\pi\right)^{m/2}}} \,
{{x_3\,(\rho_s + \rho_r)^{m-1}\,(\rho_s^2 + \rho_r^2)} \over 
{v\,(\rho_s\,\rho_r)^{m/2+1}}}\;.
\label{eqn:comigw}
\end{equation}
Equation~(\ref{eqn:comigw}) is similar to the result obtained by
\cite{GEO52-06-07450754}. In the case of zero offset $h=0$,
it reduces to equation~(\ref{eqn:zomigw}). Note that the value
of $m=1$ in (\ref{eqn:comigw}) corresponds to the two-dimensional (cylindric)
waves recorded on the seismic line. A special case is the 2.5-D inversion,
when the waves are assumed to be spherical, while the recording is on a line,
and the medium has cylindric symmetry. In this case, the modeling weighting
function (\ref{eqn:modw}) transforms to
\cite[]{GPR31-02-02930333,GPR34-05-06860703}
\begin{equation}
w(x;t,y)  =  {1\over{\left(2\,\pi\right)^{1/2}}} \,
{\sqrt{v}\,{C\left(s(y),x,r(y)\right)} 
\over {\sqrt{\rho_s\,\rho_r\,(\rho_s + \rho_r)}}}\;,
\end{equation}
and the time filter is $\left({\partial \over {\partial
z}}\right)^{1/2}$. Combining this result with formula (\ref{eqn:cobeilk})
for $m=1$, we obtain the weighting function for the 2.5-D
common-offset migration in a constant velocity medium
\cite[]{GEO52-06-07450754}:
\begin{equation}
\widehat{w}_{CO;2.5D}(y;z,x)  =  {1\over{\left(2\,\pi\right)^{1/2}}} \,
{{x_3\,\sqrt{\rho_s + \rho_r}\,(\rho_s^2 + \rho_r^2)} \over 
{\sqrt{v}\,(\rho_s\,\rho_r)^{3/2}}}\;.
\end{equation}
The corresponding time filter for 2.5-D migration is $\left(-
{\partial \over {\partial t}}\right)^{1/2}$.
\par

In the
common-offset case, the pseudo-unitary weighting is defined from
(\ref{eqn:pumigw}) and (\ref{eqn:cobeilk}) as follows:
\begin{equation}
w^{(-)}_{CO}(y;z,x)  =  {1\over{\left(2\,\pi\,v\right)^{m/2}}} \,
{{\sqrt{x_3\,\cos{\alpha}}\,(\rho_s + \rho_r)^{{m-1} \over 2}\,
\sqrt{\rho_s^2 + \rho_r^2}} \over 
{(\rho_s\,\rho_r)^{{m+1} \over 2}}}\;,
\end{equation}
where
\begin{equation}
\cos{\alpha} = \left({
{(x - y)^2 + \rho_s\,\rho_r - h^2} \over {2\,\rho_s\,\rho_r}}
\right)^{1/2}\;.
\end{equation}

\subsection{Post-Stack Time Migration}
%%%%%%%%%%%%%%%%%%%%%%%%%%%%%%%%%
\inputdir{miginv}

An interesting example of a stacking operator is the hyperbola summation
used for time migration in the post-stack domain. In this case, the
summation path is defined as
\begin{equation}
\widehat{\theta}(y;z,x)  =  \sqrt{z^2+{{(x-y)^2}\over {v^2}}}\;,
\label{eqn:zomt}
\end{equation}
where $z$ denotes the vertical traveltime, $x$ and $y$ are the
horizontal coordinates on the migrated and unmigrated sections
respectively, and $v$ stands for the effectively constant
root-mean-square velocity \cite[]{Claerbout.bei.95}.  The summation path
for the reverse transformation (demigration) is found from solving
equation (\ref{eqn:zomt}) for $z$. It has the well-known elliptic form
\begin{equation}
\theta(x;t,y)  =  \sqrt{t^2-{{(x-y)^2}\over {v^2}}}\;.
\end{equation}
The Jacobian of transforming $z$ to $t$ is
\begin{equation}
\left|\partial \widehat{\theta} \over \partial z\right| = {z \over t}\;.
\label{eqn:zomj}
\end{equation}
If the migration weighting function is defined by conventional
downward continuation \cite[]{GEO43-01-00490076}, it takes the following form,
which is equivalent to equation (\ref{eqn:datw2}):
\begin{equation}
\widehat{w}(y;z,x)  =  {1\over{\left(2\,\pi\right)^{m/2}}} \,
{{\cos{\alpha(y)}}\over {v\,R(y,x)}} = 
{1\over{\left(2\,\pi\right)^{m/2}}} \,
{\cos{\alpha} \over {v^m\,t^{m/2}}}\;.
\label{eqn:zomw}
\end{equation}
The simple trigonometry of the reflected ray suggests that the cosine
factor in formula (\ref{eqn:zomw}) is equal to the simple ratio between the
vertical traveltime $z$ and the zero-offset reflected traveltime $t$:
\begin{equation}
\cos{\alpha} =  {z \over t}\;.
\label{eqn:cos}
\end{equation}
The equivalence of the Jacobian (\ref{eqn:zomj}) and the cosine factor
(\ref{eqn:cos}) has important interpretations in the theory of Stolt
frequency-domain migration
\cite[]{GEO43-01-00230048,GEO46-05-07170733,Levin.sep.48.147}.
According to equation (\ref{eqn:tildew}), the weighting function of the
adjoint operator is the ratio of (\ref{eqn:zomw}) and (\ref{eqn:zomj}):
\begin{equation}
\widetilde{w}(x;t,y) = {1\over{\left(2\,\pi\right)^{m/2}}} \,
{1 \over {v^m\,t^{m/2}}}\;.
\label{eqn:adjzomw}
\end{equation}
We can see that the cosine factor $z/t$ disappears from the adjoint
weighting. This is completely analogous to the known effect of
``dropping the Jacobian'' in Stolt migration
\cite[]{Harlan.sep.35.181,Levin.sep.80.513}.  The product of the
weighting functions for the time migration and its asymptotic inverse
is defined according to formula (\ref{eqn:whatw}) as
\begin{equation}
w\,\widehat{w}={1\over{\left(2\,\pi\right)^m}} \, 
{\sqrt{\left|F\,\widehat{F}\right|\,
\left|\partial \widehat{\theta} \over \partial z\right|^m}} =
{1 \over {(v^2\,t)^m}}\;.
\label{eqn:zomww}
\end{equation}
Thus, the asymptotic inverse of the conventional time migration has
the weighting function determined from equations (\ref{eqn:whatw}) and
(\ref{eqn:zomw}) as
\begin{equation}
w(x;t,y) = {1\over{\left(2\,\pi\right)^{m/2}}} \, {{t/z} \over
{v^m\,t^{m/2}}}\;.
\label{eqn:invzomw}
\end{equation}
The weighting functions of the asymptotic pseudo-unitary operators are
obtained from formulas (\ref{eqn:wp}) and (\ref{eqn:wm}). They have the form
 \begin{eqnarray} 
w^{(+)}(x;t,y) & = &
{1\over{\left(2\,\pi\right)^{m/2}}} \, 
{\sqrt{t/z} \over {v^m\,t^{m/2}}}\;.
\label{eqn:zomwp} \\
w^{(-)}(y;z,x) & = & {1\over{\left(2\,\pi\right)^{m/2}}} \, 
{\sqrt{z/t} \over {v^m\,t^{m/2}}}\;.
\label{eqn:zomwm}
\end{eqnarray}
The square roots of the cosine factor appearing in formulas
(\ref{eqn:zomwp}) and (\ref{eqn:zomwm}) correspond to the analogous
terms in the pseudo-unitary Stolt migration proposed by
\cite{Harlan.sep.48.127}.

Figure~\ref{fig:migiter} shows the output of a simple numerical test.
The synthetic zero-offset section used in this test is shown in the
left plot of Figure~\ref{fig:migcvv}. The data are taken from
\cite{Claerbout.bei.95} and correspond to a synthetic reflectivity
model, which contains several dipping layers, a fault, and an
unconformity. The input zero-offset section is inverted using an
iterative conjugate-gradient method and two different weighting
schemes: the uniform weighting and the asymptotic pseudo-unitary
weighting~(\ref{eqn:zomwp}-\ref{eqn:zomwm}). I compare the iterative
convergence by measuring the least-squares norm of the data residual
error at different iterations.  Figure~\ref{fig:migiter} shows that
the pseudo-unitary weighting provides a significantly faster
convergence.  The result of inversion after 10 conjugate-gradient
iterations is shown in Figures \ref{fig:migcvv} and \ref{fig:migrst}.
The right plot in Figure~\ref{fig:migcvv} shows the output of the
least-squares migration. Figure~\ref{fig:migrst} shows the
corresponding modeled data and the residual error. The latter is very
close to zero. Although this example has only a pedagogical value, it
clearly demonstrates possible advantages of using asymptotic
pseudo-unitary operators in least-squares migration.

\sideplot{migiter}{height=3in}{Comparison of convergence
  of the iterative least-squares migration. The dashed line
  corresponds to the unweighted (uniformly weighted) operator. The
  solid line corresponds to the asymptotic pseudo-unitary operator.
  The latter provides a noticeably faster convergence.}

\plot{migcvv}{width=6in,height=3in}{Input zero-offset
  section (left) and the corresponding least-squares image (right)
  after 10 iterations of iterative inversion.}

\plot{migrst}{width=6in,height=3in}{The modeled
  zero-offset (left) and the residual error (right) plotted at the
  same scale.}

\begin{comment}
\subsection{Post-Stack Residual Migration}
%%%%%%%%%%%%%%%%%%%%%%%%%%
In an earlier article \cite[]{vc}, I found the integral solution of the boundary
problem for the velocity continuation partial differential equation
\cite[]{Claerbout.sep.48.79}
\begin{equation}
{{\partial^2 P}\over{\partial v\,\partial z}} +
v\,t\,{{\partial^2 P}\over{\partial x^2}} = 0
\label{eqn:VCequation}
\end{equation}
with the boundary conditions $\left.P\right|_{v=v_0} = P_0$ and
$\left.P\right|_{z \rightarrow \infty} = 0$. The solution has the form
of the stacking operator (\ref{eqn:operator}), with the model $M$ replaced by
$P_0$, the summation path
\begin{equation}
\widehat{\theta}(y;z,x)  =  \sqrt{z^2+{{(x-y)^2}\over {v^2-v_0^2}}}\;,
\label{eqn:rmt}
\end{equation}
the weighting function 
\begin{equation}
w_{(-)}(y;z,x) = {1\over{\left(2\,\pi\right)^{m/2}}} \,
{1 \over {v^m\,t^{m/2}}}\;,
\label{eqn:rmwp}
\end{equation}
which is coincident with (\ref{eqn:adjzomw}), and the correction filter
$\left(\mbox{sign}(v_0-v)\,{d \over {d t}}\right)^{m/2}$. Comparing equations
(\ref{eqn:rmt}) and (\ref{eqn:zomt}), we can see that this solution is equivalent
kinematically to residual migration with the velocity
$v_r=\sqrt{v^2-v_0^2}$ \cite[]{GEO50-01-01100126}. The reverse operator
is the solution of equation (\ref{eqn:VCequation}) with the boundary
condition on $v$ and has the reciprocal form of the summation path
\begin{equation}
\theta(x;t,y)  =  
\sqrt{t^2+{{(x-y)^2}\over {v_0^2-v^2}}} = 
\sqrt{t^2-{{(x-y)^2}\over {v^2-v_0^2}}}\;,
\end{equation}
the weighting function
\begin{equation}
w_{(+)}(x;t,y) = {1\over{\left(2\,\pi\right)^{m/2}}} \,
{1 \over {v^m\,z^{m/2}}}\;,
\end{equation}
and the correction filter $\left(\mbox{sign}(v-v_0)\,{d \over
{d\,z}}\right)^{m/2}$. The derivative filters are connected by the
simple asymptotic relationship
\begin{equation}
\left(\pm\,{d \over {d\,z}}\right)^{m/2} =  
\left(\pm\,{d \over {d\,t}}\right)^{m/2}\,
\left({{dt} \over {dz}}\right)^{m/2} =
\left(\pm\,{d \over {d\,t}}\right)^{m/2}\,
\left({z\over t}\right)^{m/2}\;,
\end{equation}
which transforms the reversed velocity continuation operator to the
familiar form (\ref{eqn:backward}) with the weighting function equal to
(\ref{eqn:rmwp}). According to formula (\ref{eqn:zomww}), these two operators are
seen to be asymptotically inverse. 
\par
To obtain the velocity continuation operator completely equivalent to
residual migration with the weighting function (\ref{eqn:zomw}), we can
divide the continued wavefield by the time $t$, which is equivalent to
transforming equation (\ref{eqn:VCequation}) to the form
\begin{equation}
{{\partial^2 P}\over{\partial v\,\partial t}} +
v\,t\,{{\partial^2 P}\over{\partial x^2}} +
{1 \over t}\,{\partial P \over {\partial v}} = 0\;.
\label{eqn:VCequation1}
\end{equation}
The reverse continuation in this case has the weighting function
(\ref{eqn:invzomw}).  
\par
Analogously, one can obtain the pseudo-unitary residual migration with the weighting
functions (\ref{eqn:zomwp}) and (\ref{eqn:zomwm}) by dividing the
wavefield by $\sqrt{t}$. This leads to the equation
\begin{equation}
{{\partial^2 P}\over{\partial v\,\partial t}} +
v\,t\,{{\partial^2 P}\over{\partial x^2}} +
{1 \over {2\,t}}\,{\partial P \over {\partial v}} = 0\;.
\end{equation}
\par
It is apparent that the operators of forward and reverse continuation
with equation (\ref{eqn:VCequation}) become adjoint to each other if the
definition of the dot product is changed according to formulas
(\ref{eqn:wsproduct}) and (\ref{eqn:wmproduct}) with the model weight $W_M(z)=z$
and the data weight $W_S(t)=t$. Analogously, the solutions of equation
(\ref{eqn:VCequation1}) are adjoint if $W_M(z)={1 \over z}$ and $W_S(t)={1
\over t}$. This is a simple example of how the arbitrarily chosen definition of
the dot product can affect the basic properties of the
inverted operators.
\end{comment}

\subsection{Velocity Transform}
%%%%%%%%%%%%%%%%%%%%%%%%%% 
\inputdir{velinv}

Velocity transform is another form of hyperbolic stacking with the
summation path
\begin{equation}
\widehat{\theta}(h;t_0,s)  =  \sqrt{t_0^2 + s^2\,h^2}\;,
\label{eqn:vtt}
\end{equation}
where $h$ corresponds to the offset, $s$ is the stacking slowness, and
$t_0$ is the estimated zero-offset traveltime. Hyperbolic stacking is
routinely applied for scanning velocity analysis in common-midpoint
stacking. Velocity transform inversion has proved to be a powerful
tool for data interpolation and amplitude-preserving multiple
suppression \cite[]{Thorson.sepphd.39,Ji.sepphd.90,SEG-1995-1460}.
\par
Solving equation (\ref{eqn:vtt}) for $t_0$, we find that the asymptotic
inverse and adjoint operators have the elliptic summation path
\begin{equation}
\theta(s;t,h)  =  \sqrt{t^2 - s^2\,h^2}\;.
\end{equation}
The weighting functions of the asymptotic pseudo-unitary velocity
transform are found using formulas (\ref{eqn:wp}) and (\ref{eqn:wm})
to have the form
\begin{eqnarray}
  \label{eqn:vtpsun1}
w^{(+)} & = & {1\over{\left(2\,\pi\right)^{1/2}}} \, 
\left|F\,\widehat{F}\right|^{1/4}\,
\left|\partial \widehat{\theta} \over \partial t_0\right|^{- 1/4} =
{1\over{\sqrt{\pi}}}\, 
{{\sqrt{s\,h}\,\sqrt{t/t_0}} \over {\sqrt{t}}}\;.
 \\
 \label{eqn:vtpsun2}
w^{(-)} & = & {1\over{\left(2\,\pi\right)^{1/2}}} \, 
\left|F\,\widehat{F}\right|^{1/4}\,
\left|\partial \widehat{\theta} \over \partial t_0\right|^{3/4} =
{1\over{\sqrt{\pi}}} \, 
{{\sqrt{s\,h}\,\sqrt{t_0/t}} \over {\sqrt{t}}}\;.
\end{eqnarray}
The factor $\sqrt{s\,h}$ for pseudo-unitary velocity transform
weighting has been discovered empirically by \cite{Claerbout.bei.95}.

Figure~\ref{fig:cgiter} shows the output of a numerical test of the
least-squares velocity transform inversion using a CMP gather from the
Mobil AVO dataset \cite[]{SEG-1995-1460}. The input CMP gather (shown in
the left plot of Figure~\ref{fig:dircvv}) is inverted using an
iterative conjugate-gradient method and two different weighting
scheme: the uniform weighting and the asymptotic pseudo-unitary
weights~(\ref{eqn:vtpsun1}-\ref{eqn:vtpsun2}). Analogously to
Figure~\ref{fig:migiter}, the iterative convergence is measured by the
least-squares norm of the data residual error at different iterations.
Figure~\ref{fig:cgiter} shows that the pseudo-unitary weighting
provides a noticeably faster convergence at the first three
iterations. At later iterations, the residual errors of the two
methods are very close to each other. The use of a pseudo-unitary
weighting will be justified in this case if only three iterations are
practically affordable.  The results of inversion after 10
conjugate-gradient iterations are plotted in Figures \ref{fig:dircvv}
and \ref{fig:dirrst}. The right plot in Figure~\ref{fig:dircvv} shows
the output of the velocity transform inversion: an optimized velocity
scan. Figure~\ref{fig:dirrst} shows the corresponding modeled CMP
gather and the residual error. The error is negligible which
indicates a successful inversion.

\sideplot{cgiter}{height=3in}{Comparison of convergence of the
  iterative velocity transform inversion. The dashed line corresponds
  to the unweighted (uniformly weighted) operator. The solid line
  corresponds to the asymptotic pseudo-unitary operator.  The latter
  provides a faster convergence at early iterations.}

\plot{dircvv}{width=6in,height=3in}{Input CMP gather (left) and its
  velocity transform counterpart (right) after 10 iterations of
  iterative least-squares inversion.}

\plot{dirrst}{width=6in,height=3in}{The modeled CMP gather (left) and
  the residual error (right) plotted at the same scale.}


\subsection{Offset Continuation and DMO}
%%%%%%%%%%%%%%%%%%%%%%%%%%%%%%%%%%%
Offset continuation is the operator that transforms seismic reflection
data from one offset to another
\cite[]{GPR30-06-08130828,GPR30-06-08290849}. If the data are continued
from half-offset $h_1$ to a larger offset $h_2$, the summation path of
the post-NMO integral offset continuation has the following form
\cite[]{Biondi.sep.80.125,stovas,Fomel.sepphd.107}:
\begin{equation}
\theta(x;t,y)  =  {t \over h_2}\,\sqrt{{U+V} \over 2}\;,
\label{eqn:ttOC12}
\end{equation}
where $U = h_1^2 + h_2^2 - (x - y)^2$, $V = \sqrt{U^2 -
4\,h_1^2\,h_2^2}$, and $x$ and $y$ are the midpoint coordinates before and
after the continuation. The summation path of the reverse continuation
is found from inverting (\ref{eqn:ttOC12}) to be
\begin{equation}
\widehat{\theta}(y;z,x)  =  {z \,h_2}\,\sqrt{2 \over {U+V}} = 
{z \over h_1}\,\sqrt{{U-V} \over 2}\;.
\label{eqn:ttOC21}
\end{equation}
The Jacobian of the time coordinate transformation in this case is simply
\begin{equation}
\left|\partial \widehat{\theta} \over \partial z\right| = {t \over z}\;.
\label{eqn:OCj}
\end{equation}
Differentiating summation paths (\ref{eqn:ttOC12}) and (\ref{eqn:ttOC21}), we
can define the product of the weighting functions according to formula
(\ref{eqn:whatw}), as follows:
\begin{equation}
w\,\widehat{w}={1\over{2\,\pi}} \, 
{\sqrt{\left|F\,\widehat{F}\right|\,
\left|\partial \widehat{\theta} \over \partial z\right|}} =
{t\over{2\,\pi}} \,{{\left(h_2^2-h_1^2\right)^2 - (x-y)^4} \over 
V^3}\;.
\label{eqn:OCww}
\end{equation}
The weighting functions of the amplitude-preserving offset
continuation have the form \cite[]{Fomel.sepphd.107}
\begin{eqnarray}
w(x;t,y) & = & \sqrt{z \over {2\,\pi}}\;
{{h_2^2-h_1^2-(x-y)^2} \over {V^{3/2}}}\;, 
\label{eqn:wOC12} \\
\widehat{w}(y;z,x) & = & {{t/\sqrt{z}} \over \sqrt{2\,\pi}}\;
{{h_2^2-h_1^2 + (x-y)^2} \over {V^{3/2}}}\;. 
\label{eqn:wOC21} 
\end{eqnarray}
It easy to verify that they satisfy relationship (\ref{eqn:OCww});
therefore, they appear to be asymptotically inverse to each other. 
\par
The weighting functions of the asymptotic pseudo-unitary offset
continuation are defined from formulas (\ref{eqn:wp}) and (\ref{eqn:wm}), as follows:
\begin{eqnarray}
w^{(+)} & = & {1\over{\left(2\,\pi\right)^{1/2}}} \; 
\left|F\,\widehat{F}\right|^{1/4}\,
\left|\partial \widehat{\theta} \over \partial t_0\right|^{- 1/4} =
\sqrt{z \over {2\,\pi}}\,
{{\left(\left(h_2^2-h_1^2\right)^2 - (x-y)^4\right)^{1/2}} 
\over {V^{3/2}}}\;, 
\label{eqn:puwOC12} \\
w^{(-)} & = & {1\over{\left(2\,\pi\right)^{1/2}}} \; 
\left|F\,\widehat{F}\right|^{1/4}\,
\left|\partial \widehat{\theta} \over \partial t_0\right|^{3/4} =
{{t/\sqrt{z}} \over \sqrt{2\,\pi}}\,
{{\left(\left(h_2^2-h_1^2\right)^2 - (x-y)^4\right)^{1/2}} 
\over {V^{3/2}}}\;.
\label{eqn:puwOC21} 
\end{eqnarray}
\par
The most important case of offset continuation is the continuation
to zero offset. This type of continuation is known as {\em dip moveout
(DMO)}. Setting the initial offset $h_1$ equal to zero in the general
offset continuation formulas, we deduce that the inverse and forward
DMO operators have the summation paths
\begin{eqnarray}
\theta(x;t,y)  & = & {t \over h_2}\,\sqrt{h_2^2-(x-y)^2}\;,
\label{eqn:ttDMO12} \\
\widehat{\theta}(y;z,x) &  = & {{z \,h_2} \over \sqrt{h_2^2-(x-y)^2}}\;. 
\label{eqn:ttDMO21}
\end{eqnarray}
The weighting functions of the amplitude-preserving inverse and
forward DMO are
\begin{eqnarray}
w(x;t,y) & = & \sqrt{z \over {2\,\pi}}\;{1 \over h_2}\;,
\label{eqn:wDMO12} \\
\widehat{w}(y;z,x) & = & {{t/\sqrt{z}} \over \sqrt{2\,\pi}}\;
{{h_2\,\left(h_2^2 + (x-y)^2\right)} \over
{\left(h_2^2-(x-y)^2\right)^2}}\;,
\label{eqn:wDMO21} 
\end{eqnarray}
and the weighting functions of the asymptotic pseudo-unitary DMO are
\begin{eqnarray}
w^{(+)} & = & 
\sqrt{z \over {2\,\pi}}\;
{\sqrt{h_2^2 + (x-y)^2} \over {h_2^2-(x-y)^2}}\;,
\label{eqn:puDMO12} \\
w^{(-)} & = & 
{{t/\sqrt{z}} \over \sqrt{2\,\pi}}\;
{\sqrt{h_2^2 + (x-y)^2} \over {h_2^2-(x-y)^2}}\;.
\label{eqn:puDMO21} 
\end{eqnarray}
Equations similar to~(\ref{eqn:wDMO12}) and~(\ref{eqn:wDMO21}) have
been published by \cite{stovas}. Equation~(\ref{eqn:wDMO21}) differs
from the similar result of \cite{black} by a simple time
multiplication factor. This difference corresponds to the difference
in definition of the amplitude preservation criterion.
Equation~(\ref{eqn:wDMO21}) agrees asymptotically with the
frequency-domain Born DMO operators
\cite[]{Born,GEO56-02-01820189,cwp}. Likewise, the stacking operator
with the weighting function (\ref{eqn:wDMO12}) corresponds to Ronen's
inverse DMO \cite[]{GEO52-07-09730984}, as discussed by
\cite{Fomel.sepphd.107}.  Its adjoint, which has the weighting
function
\begin{equation}
\widetilde{w}(x;t,y)  =  {{t/\sqrt{z}} \over {2\,\pi}}\;{1 \over h_2}\;,
\label{eqn:hDMO21}
\end{equation}
corresponds to Hale's DMO \cite[]{GEO49-06-07410757}.

%%% Local Variables: 
%%% mode: latex
%%% TeX-master: t
%%% TeX-master: t
%%% End: 
