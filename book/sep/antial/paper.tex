\footer{SEP--89}
\published{Journal of Seismic Exploration, v. 10, 293-310 (2002)}
\title{Antialiasing of Kirchhoff operators by reciprocal
  parameterization}

\author{Sergey Fomel}
\maketitle

\title{Introduction}
Welcome to a brief introduction to Madagascar.  The purpose of this
document is to teach new users how to use the powerful tools in
Madagascar to: process data, produce documents and build your own
Madagascar programs.

To gradually introduce you to Madagascar, we have created a series of
tutorials that are targeted to distinct audiences and designed to make
you an experienced Madagascar user in a short-time period.  The
tutorials are divided by interest into three main categories:
\begin{description}
    \item[Users] learn about Madagascar, how to use the processing programs, and build scripts.
    \item[Authors] learn how to build reproducible documents using Madagascar.
    \item[Developers] build new Madagascar programs that add additional functionality to Madagascar.
\end{description}

Each tutorial is designed to be completed in a short period of time.
Additionally, each tutorial has hands-on examples that you should be
able to reproduce on your computer as you go along with the tutorials.
Most tutorials will use scripts that you can edit, modify, or play
with to further gain experience and understanding.  By the end of the
tutorial series, you should be able to use all of the tools inside of
Madagascar.  Please note that this tutorial series does not explicitly
show you how to process certain types of data, or how to perform common data
processing operations (e.g. CMP semblance picking,
time migration,etc.).  Additional tutorials on those specific subjects
will be added over time.  The purpose of this document is simply to
familiarize you with the Madagascar framework in a general sense.

Before you go on, here are some notes on notation:
\begin{itemize}
    \item important names, or program names are usually bold in the text.  For example: \textbf{sfwindow}
    \item code snippets are always in the following formatting: \begin{verbatim} sfwindow \end{verbatim}
\end{itemize}


\section{Overview of existing methods}

I start with reviewing the existing approaches to operator
antialiasing and discussing their main principles and limitations. The
two reviewed approaches are temporal filtering, as suggested by
\cite{GPR40-05-05650572} and \cite{SEG-1994-1282}, and Hale's spatial
filtering technique, developed originally for an integral
implementation of the dip moveout operator \cite[]{GEO56-06-07950805}.

\subsection{Temporal filtering}
%%%%%%%%%%%%%%%%%%%%%%%%
The temporal filtering idea follows
from the well-known Nyquist sampling criterion. With application to
integral operators, the Nyquist criterion takes the form
\begin{equation}
\Delta x \leq {{\Delta t} \over {|\partial t / \partial x|}}\;,
\label{eqn:Nyquist}
\end{equation}
where $t(x)$ is the traveltime of the operator impulse response,
$\Delta x$ is the space sampling interval and $\Delta t$ is the time
sampling interval. In the steep parts of the traveltime curve, the
sampling criterion (\ref{eqn:Nyquist}) is not satisfied, which causes
aliasing artifacts in the output data. To overcome this problem, the
method of local triangle filtering
\cite[]{Claerbout.sep.73.371,SEG-1994-1282} suggests convolving the
traces of the generated impulse response with a triangle-shaped filter
of the length
\begin{equation}
\delta t = \Delta x\,|\partial t / \partial x|\;.
\label{eqn:dt}
\end{equation}

\inputdir{XFig}

Cascading operators of causal and anticausal numerical integration is
an efficient way to construct the desired filter shape.

Triangle filters approximate the ideal (sinc) low-pass time filters.
The idea of using low-pass filtering for antialiasing
\cite[]{GPR40-05-05650572} is illustrated in Figure \ref{fig:amolow}.
When a steeply dipping event is included in the operator, its
counterpart in the frequency domain wraps around to produce the
aliasing artifacts. Those are removed by a dip-dependent low-pass
filtering.

\plot{amolow}{width=4.5in,height=1.5in}{Schematic
illustration of low-pass antialiasing (triangle filters). The aliased
events are removed by low-pass filtration on the temporal frequency axis.
The width of the low-pass filter depends on dips of the aliased events.}

\inputdir{flt}

The method of low-pass filtering is less evident in the case of a
three-dimensional integral operator. We can take the length of a
triangle filter proportional to the absolute value of the time
gradient \cite[]{SEG-1994-1282}, the maximum of the gradient
components in the two directions of the operator space
\cite[]{GEO64-06-17831792}, or the sum of these components. The latter
follows from considering the 3-D operator as a double integration in
space. Decoupling the 3-D integral into a cascade of two 2-D operators
suggests convolving two triangle filters designed with respect to two
coordinates of the operator. In this case, the length of the resultant
filter is approximately equal to
\begin{equation}
\delta t =      \Delta x\,|\partial t / \partial x| + 
                \Delta y\,|\partial t / \partial y|\;,
\label{eqn:dt3}
\end{equation}
and its shape is smoother than that of a triangle filter (Figure
\ref{fig:amoflt}).

\plot{amoflt}{height=2.5in}{Building the smoothed filter for
3-D antialiasing by successive integration of a five-point wavelet. C
denotes the operator of causal integration, C' denotes its adjoint (the
anticausal integration). The result is equivalent to the convolution
of two equal triangle filters.}

The temporal filtering method was proven to be an efficient tool in
the design of stacking operators of different types. However, when the
operator introduces rapid changes in the length and direction of the
traveltime gradient, it leads to an inexact estimation of the filter
cutoff (triangle length for the method of triangle filtering) at the
curved parts of the operator. Consequently, the high-frequency part of
the output can be distorted, causing a loss in the image resolution.

\subsection{Hale's method}
%%%%%%%%%%%%%%%%%%%%

Considering the case of integral dip moveout, \cite{GEO56-06-07950805}
points out that the steep parts of the operator, while aliased in the
space (midpoint) coordinate, are not aliased with respect to the time
coordinate. He suggests replacing the conventional $t(x)$
parameterization of the DMO impulse response by $x(t)$
parameterization. Conventionally, the integral operators are
implemented by shifting the input traces in space and transforming
them in time.  According to Hale's method, the traces are shifted in
time and transformed along the $x(t)$ trajectories in space.
Interpolation in time, required in the conventional approach, is
replaced by interpolation in space. The idea of Hale's method is
related to the idea of the ``pixel-precise velocity transform''
\cite[]{Claerbout.blackwell.92}.

The steep parts of the operator satisfy the criterion
\begin{equation}
\Delta t \leq {{\Delta x} \over {|\partial x / \partial t|}}\;,
\label{eqn:Nyquist2}
\end{equation}
which is the the obvious reverse of inequality (\ref{eqn:Nyquist}).
Therefore, they are not aliased if defined on the time grid. In these
parts, one can implement the operator by constant time shifts equal to
the time sampling interval $\Delta t$. In the parts where the
criterion (\ref{eqn:Nyquist2}) is not valid (the flat part of the DMO
operator), Hale suggests reducing the length of the time shifts
according to equality (\ref{eqn:dt}), where $\delta t$ becomes less
than $\Delta t$. He formulates the following principle of operator
antialiasing:
\begin{quote}
  To eliminate spatial aliasing, simply never allow successive time
shifts applied to the input trace to differ by more than one time
sampling interval. Further restrict the difference between time shifts
so that the spacing between the corresponding output trajectories
never exceeds the CMP sampling interval.
\end{quote}

\inputdir{XFig}

The idea of Hale's method is illustrated in Figure \ref{fig:amosft}.
Increasing the density of spatial sampling by small successive time
shifts implies increasing the Nyquist boundaries of the spatial
wavenumber. Further interpolation is a low-pass spatial filtering that
removes the parts of the spectrum beyond the Nyquist frequency of the
output. If the dip of the operator does not vary between neighboring
traces (the operator is a straight line as in the slant stack case),
Hale's approach will produce essentially the same result as that of
temporal filtering. Triangle filters in this case approximately
correspond to linear interpolation in space between adjacent traces
\cite[]{Nichols.sep.77.283}. The difference between the two approaches
occurs if the local dip varies in space as in the case of a curved
operator, such as DMO. In this case, Hale's approach provides a more
accurate space interpolation of the operator and preserves the
high-frequency part of its spectrum from distortion.

\plot{amosft}{width=4.125in,height=1.5in}{Schematic
illustration of Hale's antialiasing. The aliased events are removed by
spatial interpolation. In the frequency domain, the interpolation
consists of widening and low-passing on the wavenumber axis. The
low-pass spatial filtering does not depend on dip.} 

Hale's method has proven to preserve the amplitude of flat reflectors
from aliasing distortions, which is the simplest antialiasing test on
a DMO operator. The most valuable advantage of this method in the fact
that the implied low-pass spatial filtering (interpolation) does not
depend on the operator dip and is controlled by the Nyquist boundary
of the spectrum only (compare Figures \ref{fig:amolow} and
\ref{fig:amosft}). This is especially important, when the local dip of
the operator changes rapidly and therefore cannot be estimated
precisely by finite-difference approximation at spatially separated
traces. Such a situation is common in dip moveout and azimuth moveout
integral operators, as well as in prestack Kirchhoff migration.

A weakness of the method is the necessity to switch from
interpolation in space to two-dimensional interpolation in both the
time and the space variables, when trying to construct the flat part
of the operator. 
%In the case of AMO, the 2-D spatial interpolation
%arises as a result of building the operator in the transformed
%coordinate system. 
In the next section, I show how to avoid the expense of the additional
time interpolation required by Hale's method of antialiasing.

\section{Proposed technique}
%%%%%%%%%%%%%%%%%%%%%%%%%%
We can use the reciprocity of the time parameterization and the space
parameterization of integral operators, discovered by Hale, to arrive
at the following antialiasing technique.

For simplicity, let us consider the two-dimensional case first.  The
linearity of a two-dimensional integral operator allows us to
decompose this operator into two parts. The first part corresponds to
the steep part of the travel-time function, which satisfies the
time-sampling criterion (\ref{eqn:Nyquist2}). The second term
corresponds to the flat part of the traveltime, which satisfies the
space-sampling criterion (\ref{eqn:Nyquist}). The first part is not
aliased with respect to the time sampling interval, while the second
one is not aliased with respect to the space sampling. We can apply
interpolation in time to construct the flat part.  Reciprocally,
interpolation in space is applied to construct the steep part of the
operator in the fashion of Hale's time-shifting method. 
%Linear
%interpolation in this case is a cheap substitution for the accurate
%but computationally expensive sinc interpolation. 
The amplitude
difference between the two integrals is simply the Jacobian term

\begin{equation}
{\mbox{amp}_t \over \mbox{amp}_x}=
\left|{\partial x \over \partial t}\right|\,{\Delta t \over \Delta x}=
{\Delta t \over \delta t} \leq 1\;.
\label{eqn:Jacobian}
\end{equation}

According to the proposed modification, Hale's antialiasing principle
is reformulated, as follows:
\begin{quote}
{\em In the steep part of an integral operator, never allow successive
time shifts applied to the input trace to differ by more than one time
sampling interval. In the flat part of the operator, never allow
successive space shifts to differ by more than one space sampling
interval.}
\end{quote}

\inputdir{flt}

Figure \ref{fig:amotra}, borrowed from \cite{Claerbout.bei.95},
illustrates the basic idea of the proposed technique. It clearly shows
the difference between the flat and steep parts of migration
hyperbolas. To observe the reciprocity, rotate the figure by 90
degrees.

\plot{amotra}{height=2in}{Figure borrowed from
  \cite{Claerbout.bei.95} to illustrate the reciprocity
  antialiasing. The flat parts of the hyperbolas require interpolation
  in time. The steep parts of the hyperbolas require interpolation in
  space.}

\inputdir{mod}

To compare the proposed antialiasing method with the temporal
filtering method, I test the antialiased migration program on simple
2-D synthetic tests. Figure \ref{fig:amomod} shows a simple model and
the modeling results from modeling without antialiasing, with temporal
filtering, and with the proposed reciprocity method.  The modeling
results were migrated with the corresponding migration operators to
obtain the image of the model in Figure \ref{fig:amomig}.  Both the
temporal filtering and the proposed method succeed in removing the
major aliasing artifacts. However, the reciprocity method demonstrates
a higher resolution and a better preservation of the frequency
content.

\plot{amomod}{width=4.5in}{Top left is a synthetic model. Top
right is modeling without antialiasing. Bottom left is modeling with
reciprocity antialiasing (the proposed method). Bottom right is modeling
with antialiasing by temporal filtering.}

\plot{amomig}{width=4.5in}{Top left plot is the synthetic
model. The other plots are migrations of the corresponding data shown
in the previous figure . Top right is a migration without
antialiasing. Bottom left is a migration with reciprocity antialiasing
(the proposed method). Bottom right is a migration with triangle
filter antialiasing.}

\inputdir{sig}

These properties are examined more closely in the next synthetic
example. Figure \ref{fig:amosmo} shows a more sophisticated synthetic
model that contains a fault, an unconformity and layered structures
\cite[]{Claerbout.bei.95}. For better displaying, I extract the central
part of the model and compare it with the migration results of
different methods in Figure \ref{fig:amosmi}. Comparing the plots
shows that the reciprocity method successfully removes the aliasing
artifacts (round-off errors) of the aliased (nearest neighbor
interpolation) migration.  At the same time, it is less harmful to the
high-frequency components of the data than triangle filtering. This
conclusion finds an additional support in Figure \ref{fig:amospe} that
displays the average spectrum of the image traces for different
methods. Both of the antialiasing methods remove the high-frequency
artifacts of the nearest neighbor modeling and migration. The
reciprocity method performs it in a gentler way, preserving the
high-frequency components of the model.

\plot{amosmo}{height=2.5in}{Synthetic model used to test
the antialiased migration program.}

\plot{amosmi}{width=6in}{Top left plot is a
zoomed portion of the synthetic model. The other plots are migrated
images. Top right is a migration without antialiasing. Bottom left is
a migration with reciprocity antialiasing (the proposed method). Bottom
right is a migration with triangle filter antialiasing.}

\plot{amospe}{height=2.5in}{Top is the spectrum of the
model. The other plots are the spectra of the migrated images. The
second plot corresponds to the modeling/migration without account for
antialiasing. The third plot is modeling/migration with the
reciprocity antialiasing. The bottom plot is modeling/migration with
triangle antialiasing.}  

The algorithm sequence of the antialiased migration is illustrated in
Figures \ref{fig:amormo} and \ref{fig:amormi}. The two plots in Figure
\ref{fig:amormo} show the steep-dip and flat-dip modeling respectively. The
superposition of these two terms is the resultant antialiased data
shown in the left plot of Figure \ref{fig:amormm}. The right plot of
Figure \ref{fig:amormm} shows the migrated image obtained by adding the
flat-dip (left of Figure \ref{fig:amormi}) and steep-dip (right of Figure
\ref{fig:amormi}) migrations.

\plot{amormo}{width=6in,height=2.25in}{Antialiased modeling. Left
corresponds to the flat-dip term. Right corresponds to the steep-dip
term.}
\plot{amormi}{width=6in,height=2.25in}{Antialiased
migration. Left corresponds to the flat-dip term. Right corresponds to
the steep-dip term.}
\plot{amormm}{width=6in,height=2.25in}{Antialiased
modeling and migration. Left is the superposition of the flat-dip and
steep-dip modeling. Right is superposition of the flat-dip and
steep-dip migration.}

\begin{comment}
The efficiency of the antialiased migration is compared in the CPU
time chart in Figure \ref{fig:amochp}.  The test data set included 500
by 250 data points with $\Delta t=0.004$ sec, and $\Delta x = 25$ m.
The figure shows that the performance of the reciprocity antialiasing
increases with increase of the migration velocity. This behavior can
be explained by the fact that high-velocity migration hyperbolas
require a smaller number of expensive computations in the steep
(aliased) parts. It allows us to expect a high performance of the
method in application to the curvilinear operators with limited
aperture (dip moveout, azimuth moveout, shot continuation).

%\plot{amochp}{height=1.5in}{CPU time of migration programs
%on HP 9000-735 versus the constant migration velocity used in the
%experiment.}
\end{comment}

\subsection{3-D antialiasing}
\inputdir{imp}

The proposed method of antialiasing is easily generalized to the case
of a three-dimensional integral operator. In this case, we need to
consider three different parameterizations: $t(x,y)$, $x(t,y)$, and
$y(t,x)$ and switch from one of them to another according to the rule:
\begin{itemize}
\item if $\Delta t \geq {{\Delta x} \, {|\partial t / \partial x|}}$
and      $\Delta t \geq {{\Delta y} \, {|\partial t / \partial y|}}$,
use $t(x,y)$,
\item if $\Delta x \geq {{\Delta t} \, {|\partial x / \partial t|}}$
and      $\Delta x \geq {{\Delta y} \, {|\partial x / \partial y|}}$,
use $x(t,y)$,
\item if $\Delta y \geq {{\Delta t} \, {|\partial y / \partial t|}}$
and      $\Delta y \geq {{\Delta x} \, {|\partial y / \partial x|}}$,
use $y(t,x)$.
\end{itemize}

Following \cite{GEO66-02-06540666}, I illustrate 3-D
antialiasing by applying prestack time migration on a single input
trace. The results are shown in Figures~\ref{fig:imp-noa},
\ref{fig:imp-aal} and \ref{fig:imp-all}. The result without any
antialiasing protection (Figure~\ref{fig:imp-noa}) contains clearly
visible aliasing artifacts caused by the steeply dipping parts of the
operator. Antialiasing by temporal filtering
(Figure~\ref{fig:imp-aal}) removes the artifacts but also attenuates
the steeply dipping events. Antialiasing by the proposed reciprocal
parameterization (Figure~\ref{fig:imp-all}) removes the aliasing
artifacts while preserving the steeply dipping events and the image
resolution.

\plot{imp-noa}{width=6in}{Prestack 3-D time migration of a single
  input trace. Top: time slice at 1 s. Bottom: vertical slice. No
  antialiasing protection has been applied. As a result, aliasing
  artifacts are clearly visible in the image.}

\plot{imp-aal}{width=6in}{Prestack 3-D time migration of a single
  input trace. Top: time slice at 1 s. Bottom: vertical slice.
  Antialiasing by temporal filtering has been applied. Aliasing artifacts
  are removed, steeply dipping events are attenuated.}

\plot{imp-all}{width=6in}{Prestack 3-D time migration of a single
  input trace. Top: time slice at 1 s. Bottom: vertical slice.  The
  proposed reciprocal antialiasing has been applied. Aliasing
  artifacts are removed, steeply dipping events are preserved.}

\section{Conclusions}
%%%%
I have introduced a new method of antialiasing integral operators,
modified from Hale's approach to antialiased dip moveout. The method
compares favorably with the popular temporal filtering technique. The
main advantages are:
\begin{enumerate}
\item Accurate handling of variable operator dips.
\item Consequent preservation of the high-frequency part of the data
spectrum, leading to a higher resolution.
\item Easy control of operator amplitudes.
\item Easy generalization to 3-D.
\end{enumerate}
The method possesses a sufficient numerical efficiency in practical
implementations. Its most appropriate usage is for antialiasing
operators with analytically computed summation paths, such as prestack
time migration, dip moveout, azimuth moveout, and shot continuation.

\section{Acknowledgments}

I thank Biondo Biondi for many helpful discussions. The financial
support for this work was provided by the sponsors of the Stanford
Exploration Project.

\bibliographystyle{seg}
\bibliography{antial,SEG,SEP2}

