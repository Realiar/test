
\begin{abstract}
I propose a method for antialiasing Kirchhoff operators, which
  switches between interpolation in time and interpolation in space
  depending on the operator dips. The method is a generalization of
  Hale's technique for dip moveout antialiasing. It is applicable to a
  wide variety of integral operators and compares favorably with the
  popular temporal filtering technique. Simple synthetic examples
  demonstrate the performance and applicability of the proposed
  method.
\end{abstract}

\section{Introduction}

Integral (Kirchhoff-type) operators are widely used in seismic imaging
and data processing for such tasks as migration, dip moveout
\cite[]{GEO56-06-07950805}, azimuth moveout \cite[]{GEO63-02-05740588},
and shot continuation \cite[]{GEO61-06-18461858}. In theory, the
operators correspond to continuous integrals. In practice, the
integration is replaced by summation and becomes prone to sampling
errors. A common problem with practical implementation of integral
operators is the operator aliasing, caused by spatial undersampling of
the summation path \cite[]{SEG-1994-1282}. When the integration path is
parameterized in the spatial coordinate, as it is commonly done in
practice, the steeper part of the summation path becomes undersampled.

The operator aliasing problem, as opposed to the data aliasing and
image aliasing problems, is discussed in detail by
\cite{SEG-1994-1282} and \cite{GEO66-02-06540666}. It arises when
the slope of the operator traveltime exceeds the limit, defined by the
time and space sampling of the data - the Nyquist frequencies
\cite[]{Claerbout.sep.73.371}. Even if the input data are not aliased,
operator aliasing can cause severe distortions in the output. Several
successful techniques have been proposed in the literature to overcome
the operator aliasing problem.  Different versions of the temporal
filtering method were suggested by \cite{GPR40-05-05650572} and
\cite{SEG-1994-1282} and further enhanced by \cite{GEO64-06-17831792}
and \cite{GEO66-02-06540666}. This method reduces the aliasing error
by limiting the rate of change in the integrand (the input data) with
temporal filtering. Unfortunately, this approach is suboptimal in the
case of rapid changes in the summation path gradient. A different
approach to antialiasing was suggested by \cite{GEO56-06-07950805} for
the integral dip moveout. Hale's approach provides accurate results by
parameterizing the operator in the time coordinate rather than the
space coordinate.  Unfortunately, this approach requires an additional
expense of interpolation in both space and time coordinates for
computing the flat part of the operator.

In this paper, I propose a new antialiasing method derived from the
time-slice technique, developed by \cite{GEO56-06-07950805}. The
method switches between interpolation in time and interpolation in
space depending on the local operator dips. It is particularly
attractive for computing 3-D operators with rapidly varying dips and
limited aperture \cite[]{Fomel.sep.89.89}.  Synthetic examples show the
superior performance of the new method in comparison the temporal
filtering approach.



