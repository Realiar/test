\published{Geophysical Prospecting, 59, 244-255 (2011)}

\title{Stacking angle-domain common-image gathers for normalization of illumination}
\author{Guochang Liu$^{*}$, China University of Petroleum-Beijing and 
The University of Texas at Austin; Sergey Fomel,
The University of Texas at Austin; Xiaohong Chen, 
China University of Petroleum-Beijing
}

\lefthead{Liu et al.}
\righthead{Stacking for normalization of illumination}

\maketitle
\sloppy

\begin{abstract}
Unequal illumination of the subsurface highly impacts the quality of seismic imaging. 
Different image points of the media have different folds of reflection-angle illumination, 
which can be caused by irregular acquisition or by wave propagation in complex media. 
To address this problem, we present a method of stacking angle-domain common-image gathers (ADCIGs), 
in which we use local similarity with soft thresholding to decide the folds of local illumination. 
Normalization by local similarity regularizes local illumination of reflection angles 
for each image point of the subsurface model. This approach can restore good fidelity of amplitude
by selective stacking in the image space, whatever the cause of acquisition or propagation irregularities.
We use two synthetic examples to demonstrate that our method can normalize migration amplitudes and 
effectively suppress migration artifacts.

\end{abstract}

\section{Introduction}

The image of geology produced by Kirchhoff migration or wave-equation migration often suffers from artifacts, 
especially in the case of irregular acquisition or complex wave propagation \cite[]{Qin,Tang}. 
Artifacts sometimes bury seismic reflectors, especially in subsalt areas, and make interpretation and model 
building difficult. Developments in acquisition technology \cite[]{Howard,Michell} 
have provided richer coverage and better illumination of the subsurface; however, areas with poor illumination 
still exist in complex geological environments, such as subsalt zones. 

\cite{Prucha,Kuhl} used preconditioned inversion in the reflection angle domain to 
improve imaging in complex media. \cite{Xu,Brandsberg} proposed focusing on dip and 
AVA compensation in common-image gathers to suppress artifacts and improve signal-noise ratio in Kirchhoff migration. 
To compensate irregular illumination at image points, \cite{Kessinger} presented illumination-angle compensation in 
Kirchhoff migration, in which inclusion of an amplitude-weighting term can dramatically improve the migrated image. 
\cite{Bloor,Audebert} proposed a method of regularization of illumination in the multiangle 
domain for Kirchhoff migration, in which a hit-count technique is applied to suppress migration artifacts and balance 
migration amplitudes. \cite{Qin} presented interactive dip-gather stacking for attenuating artifacts of Kirchhoff 
migration. \cite{Tang} analytically demonstrated artifacts in angle-domain common-image gathers (ADCIGs) by sparsely-sampled 
wavefields and proposed selective stacking on the basis of local smoothing of the envelope function in shot-profile wave-equation 
migration. \cite{Manning} proposed the MAZ-stack method for weighting signal in areas of poor illumination for multiazimuth 
seismic data, in which weights of stacking were chosen as binary (zero or one). 

Local similarity is a local attribute measured between two signals \cite[]{Fomel2007b} that has been applied to multicomponent 
seismic image registration \cite[]{Fomel2005,Fomel2007b} and time-lapse image registration \cite[]{Fomel2009}. 
In an earlier work \cite[]{Liu}, we also applied local similarity as a weight for stacking common-midpoint gathers 
in order to improve signal-noise ratio of seismic data. 

In this paper, we present another application of local similarity in stacking ADCIGs in order to normalize illumination. 
Our method applies local similarity between initial image and ADCIGs using a soft threshold to normalize stacking. It 
can attenuate migration artifacts and restore migration amplitudes. This method can be regarded as true-amplitude 
illumination compensation arising from the so-called Beylkin determinant \cite[]{Beylkin,Albertin,Audebert}. 

\section{ANGLE-DOMAIN COMMON-IMAGE GATHERS}

ADCIGs can be produced both by Kirchhoff methods \cite[]{Xu,Brandsberg} and wave-equation methods \cite[]{Prucha0,Mosher,Xie2002,Sava2003,Biondi2004}. In wave-equation migration methods, angle gathers can be produced either in the data space during imaging or in the image space after imaging \cite[]{Fomel2004}. These methods can be also applied in source-receiver migration \cite[]{Sava2003}, shot-profile migration \cite{Rickett}, and reverse-time migration \cite[]{Biondi2002,Zhang2007}.

To extract ADCIGs, in wave equation migration we first compute subsurface offset-domain common-image gathers (ODCIGs) using crosscorrelation imaging condition in space and time, followed by extraction at zero time \cite{Sava2006}:
\begin{equation}
  \label{eq:e1}
  I(\mathbf{m} ,\mathbf{h}) = \sum_\omega U_s(\mathbf{m} - \mathbf{h} ,\omega)
  U_r^\ast (\mathbf{m} + \mathbf{h} , \omega) \;.
\end{equation}
Here, \mbox{$I(\mathbf{m} , \mathbf{h})$} is the subsurface-offset dependent image, \mbox{$\mathbf{m} = \left[m_x,m_y,m_z\right]$} is a vector of locations of image points, $\mathbf{h} = \left[h_x,h_y,h_z\right]$ is a vector of local source-receiver separation in the image space, and $\ast$ means the conjugate.

In two dimensions, angle-domain common-image gathers $I(\mathbf{m} ,\theta)$ can be obtained by a simple slant-stack operation on ODCIGs $I(\mathbf{m} , \mathbf{h})$ after migration \cite[]{Stolt,Weglein,Sava2003}. For three dimensions, we can extract ADCIGs after applying imaging condition by transforming local offset gathers in the depth domain \cite[]{Fomel2004,Sava2005}, in which local structural dips need to be estimated. In examples herein, we extract ADCIGs in the image space after applying imaging conditions and test our proposed method on a 2D case.

\section{MEASUREMENT OF LOCAL SIMILARITY}

\cite{Fomel2007b} defined the local similarity attribute using shaping regularization \cite[]{Fomel2007a}. The global correlation coefficient between two discrete signals $a_t$ and $b_t$ is defined as
\begin{equation}
  \label{eq:e2}
  \gamma = \frac{\sum_{i=1}^{N} \mathbf{a}_t \mathbf{b}_t}{\sqrt{\sum_{i=1}^{N} 
  {\mathbf{a}_t}^{2} \sum_{i=1}^{N} {\mathbf{b}_t}^2}}\;.
\end{equation}
To locally measure correlation between two signals according to the definition of local similarity \cite[]{Fomel2007b, Fomel2009}, local similarity $\gamma_t$ can be represented as the product of two least-squares inverses:
\begin{equation}
  \label{eq:e3}
  {\gamma_t}^2 = p_t q_t\;,
\end{equation}
\begin{equation}
  \label{eq:e4}
  p_t = \arg \,\,\,\, \min_{p_t} \left(\sum_{t}\left(a_t-p_tb_t\right)^2+R\left[p_t\right]\right)\;,
\end{equation}
\begin{equation}
  \label{eq:e5}
  q_t = \arg \,\,\,\, \min_{q_t} \left(\sum_{t}\left(a_t-q_tb_t\right)^2+R\left[q_t\right]\right)\;,
\end{equation}
where $R$ is a regularization operator designed to constrain the solution in a desired behavior, such as smoothness. Shaping regularization \cite[]{Fomel2007a} can conveniently be applied in solving inverse problems~(\ref{eq:e4}) and (\ref{eq:e5}) iteratively. The local similarity can smoothly measure the correlation between two signals locally. It is an estimate of waveform similarity of two seismic signals.

\section{STACKING USING LOCAL SIMILARITY FOR NORMALIZATION OF ILLUMINATION}

Once we obtain ADCIGs by migration, the final image of subsurface can be obtained by stacking over the angle axis in ADCIGs. Assuming reflectivity is angle independent, we can write equal-weight stacking as
\begin{equation}
  \label{eq:e6}
  \widehat{I} (\mathbf{m}) = \frac{1}{N} \sum_{\theta=1}^{N}\mathbf{I}(\mathbf{m},\mathbf{\theta})\;.
\end{equation}
where $N$ is the number of samples in the reflection angle. To get rid of the artifacts caused by poor sampling of the sources or receiver wavefield or complex structure, \cite{Tang} suggested stacking only angles with good illumination and dense sampling and applied selective stacking using envelopes as weights of stacking. 

In this paper, we use local similarity with soft thresholding as weights to normalize the stacking: 
\begin{equation}
  \label{eq:e7}
  \widetilde{I} (\mathbf{m}) = \frac{1}{KN} \sum_{\theta=1}^{N}w(\mathbf{m} , \theta)\mathbf{I}(\mathbf{m},\mathbf{\theta})\;,
\end{equation}
\begin{equation}
  \label{eq:e8}
  w(\mathbf{m} , \theta) = \left\{ \begin{array}{ll} \gamma(\mathbf{m} , \theta) - \alpha & \textrm{if $s(\mathbf{m} , \theta) > \alpha$} \\
  0 & \textrm{if $s(\mathbf{m} , \theta) \leq \alpha$} \end{array}\right.\;.
\end{equation}
Here, $\gamma (\mathbf{m},\theta)$ is local similarity between initial imagecomputed $\widehat{I}$ by equation~ref{eq:e6} and ADCIGs $I(\mathbf{m},\theta)$. Local similarity can be computed by equations~\ref{eq:e3} through~ref{eq:e5}. Equation~\ref{eq:e8} is soft thresholding \cite[]{Donoho}, which can define the fold of local illumination for each image point, $\mathbf{m}$. Normalizationis $K$ the sum of weights. Normalization $N$ in equation~\ref{eq:e7}, the number of samples with $w(\mathbf{m},\theta)\ne 0$, can restore migration amplitude by discarding some samples of angle, which are likely artifacts or noise. We assume that the discarded samples in ADCIGs are entirely artifacts or noise, probably caused by irregular acquisition or complex propagation in the complex subsurface model. 

\inputdir{simple}

In the following, we use a simple example to illustrate our method. Figure~\ref{fig:compare1}a is a simple synthetic ADCIG with 30 samples of reflection angle, and we add Gaussian random noise to the ADCIG. Assume that the reflection coefficients are equal and angle independent. Note that the third and fourth reflectors are not entirely illuminated. The image created by directly stacking this ADCIG is shown in Figure~\ref{fig:compare1}d. Third and fourth reflectors are not restored in equal-weight stacking because of poor illumination. The local similarity between the direct equal-weight stacking image (Figure~\ref{fig:compare1}c) and ADCIG (Figure~\ref{fig:compare1}a) computed by equations~\ref{eq:e3} through~\ref{eq:e5} is shown in Figure~\ref{fig:compare1}b. Note that the local similarity can approximate local illumination of the reflection angle. \cite{Xie2006} presented a method of computing illumination distribution as a function of reflection angle on the basis of one-way wave equation. We apply soft thresholding in local similarity to select angles with good illumination to contribute to the stacking. And then we use the number of samples of reflection angles with good illumination to normalize the image. In comparing our method with the equal-weight stacking method, note that the third and the fourth reflection coefficients are balanced well and the signal-noise ratio is also improved by our method (Figure~\ref{fig:compare1}e). Amplitudes in Figure~\ref{fig:compare1}e are almost the same as those of the ideal image (Figure~\ref{fig:compare1}c). 

\plot{compare1}{width=0.9\textwidth}{A simple example. (a) Synthetic ADCIGs with different folds; (b) local similarity; (c) ideal reflectivity; (d) initial equal-weight stacking; (e) stacking with normalization using local similarity.}

In Kirchhoff integral migration, Beylkin's theory \cite[]{Beylkin} shows how to take into account irregularities of illumination caused by acquisition irregularities or wave propagation in complex media \cite[]{Audebert}. The Beylkin determinant estimates the local smooth illumination density function by computation of traveltime between surface locations and subsurface image point. In this paper, the local fold of illumination in the denominator of equation~\ref{eq:e7} is estimated by local similarity with soft thresholding. Because this estimation of the fold of local illumination is computed after imaging, the method can be thought of as postprocessing after migration. 

When reflection coefficients are angle dependent (known as amplitude versus angle [AVA]), the proposed method can be regarded as an average effect of all angle-dependent reflection coefficients. It can enhance the image and restore relative amplitudes. When considering the AVA effect in stacking processing, one should use the accurate AVA inversion to obtain the reflection coefficient of the same reflection angle for all image points \cite[]{Kuhl}, such as zero-reflection-angle reflectivity, or retrieve the map of relative slowness perturbations from multishot seismic data \cite[]{Kiyashchenko}.

\section{Examples}

Two synthetic examples illustrate the method of stacking ADCIGs for illumination normalization.

The first example involves a simple constant-velocity model with three reflectors. The reflecting interfaces are caused by density contrasts, which are chosen in such a way that all reflection coefficients for all reflectors and reflection angles are identical. A similar model was applied by \cite{Zhang2003} and \cite{Schleicher}. The synthetic dataset is generated using Kirchhoff modeling. It includes 25 shots between 3,000 and 5,000 m, and range of offset is -3,000 to 3,000 m. PSPI wave equation migration is performed for generating ADCIGs (Figure~\ref{fig:ppangtr-h}). The three reflectors can be found to have different folds caused by acquisition geometry. The parts that are not illuminated in ADCIGs do not all have zero values; some are occupied by noise or artifacts. Therefore, appropriate normalization when stacking all reflection coefficients of different reflection angles together must be ensured. The local similarity cube (Figure~\ref{fig:similtr}) can provide normalization for stacking. By comparing equal-weight stacking and our method (Figure~\ref{fig:stack,simistack}), we find that image amplitudes are improved, especially on the sides of the model, which are not entirely illuminated because of acquisition.

\inputdir{flat}

\multiplot{2}{ppangtr-h,similtr}{width=0.45\textwidth}{(a) ADCIGs for the simple synthetic example; (b) local similarity cube.} 

\multiplot{2}{stack,simistack}{width=0.45\textwidth}{(a) Equal-weight stack; (b) stack normalized by local similarity.} 

The second example involves the Sigsbee2A model \cite[]{Paffenholz}. This dataset is known to have illumination problems below the salt. Figure~\ref{fig:angs} shows the ADCIGs of Sigsbee2A model. Figures~\ref{fig:ang1} and~\ref{fig:ang2}, respectively, show images of $0\,^{\circ}$ and $20\,^{\circ}$  reflection angles. Figures~\ref{fig:simil1} and~\ref{fig:simil2}, respectively, show the local similarity cube of $0\,^{\circ}$ and $20\,^{\circ}$ reflection angles. Comparing local similarity cubes of different reflection angles, we can find that folds of local illumination are not equal for all image points. The folds are closely uniform in areas with simple structures, but differences can be found in complex media, such as in subsalt areas (Figures~\ref{fig:simil1} and~\ref{fig:simil2}). Figure~\ref{fig:sigstack} shows the result of equal-weight stacking, and Figure~\ref{fig:sigsimistack} shows the result of our method. With normalization, reflection coefficients are restored, such as below the salt (Figure~\ref{fig:sigsimistack}). Our method's image beneath the salt is clearer than that of the conventional stacking method, and amplitude of the base line, with strong reflectivity at the bottom of the Sigsbee2A model, is more balanced. Note that fault and diffraction points below the salt become clearer after normalization using local similarity.

\inputdir{sigsbee}

\plot{angs}{width=0.9\textwidth}{ADCIGs of the Sigsbee2A Model.}

\multiplot{2}{ang1,simil1}{width=0.45\textwidth}{$0\,^{\circ}$ angle image and (a) its local similarity (b) of the Sigsbee2A model.} 
\multiplot{2}{ang2,simil2}{width=0.45\textwidth}{$20\,^{\circ}$ angle image and (a) its local similarity (b) of the Sigsbee2A model.} 
\plot{sigstack}{width=0.9\textwidth}{The image of Sigsbee2A using initial equal-weight stacking.}
\plot{sigsimistack}{width=0.9\textwidth}{The image of Sigsbee2A using stacking with normalization by local similarity.}

\section{Conclusions}

We have presented a method for stacking angle-domain common-image gathers for illumination normalization in which we use a local similarity cube between initial image and ADCIGs as the approximate fold of local illumination for each image point of the media. Normalization by local fold can restore migration amplitude and attenuate artifacts possibly caused by irregular acquisition geometry or by wave propagation in complex media. Our method is easy to implement and can be seen as a postprocessing step after migration. Two synthetic examples show that this method can be used to enhance migrated images and to restore migration amplitudes.

\section{Acknowledgments}

The first author would like to thank Long Jin for inspiring discussions and the China Scholarship Council (grant 2007U44003) for partial support of this work. We would like to thank Paul Sava for help with wave-equation migration code. 

\bibliographystyle{seg}
\bibliography{paper}

%%% Local Variables: 
%%% mode: latex
%%% TeX-master: t
%%% End: 

