\published{Geophysical Prospecting, 63, no. 2, 315-337, (2015)}

\title{A robust approach to time-to-depth conversion and interval velocity 
estimation from time migration in the presence of lateral velocity variations}

\author{Siwei Li and Sergey Fomel}

%\ms{GEO-2013}

\address{Bureau of Economic Geology \\
John A. and Katherine G. Jackson School of Geosciences \\
The University of Texas at Austin \\
University Station, Box X \\
Austin, TX 78713-8924}

\lefthead{Li \& Fomel}
\righthead{Time-to-depth conversion}
\footer{TCCS-8}

\maketitle

\begin{abstract}

The problem of conversion from time-migration velocity to an interval velocity in depth in 
the presence of lateral velocity variations can be reduced to solving a system 
of partial differential equations. In this paper, we formulate the problem as a nonlinear 
least-squares optimization for seismic interval velocity and seek its solution 
iteratively. The input for inversion is the Dix velocity which also serves as an initial guess. The inversion 
gradually updates the interval velocity in order to account for lateral velocity 
variations that are neglected in the Dix inversion. The algorithm has a moderate cost thanks 
to regularization that speeds up convergence while ensuring a smooth output. The proposed method 
\old{is}\new{should be} numerically robust compared to the previous approaches, which amount to extrapolation 
in depth monotonically. For a successful time-to-depth conversion, image-ray caustics should be either 
nonexistent or excluded from the computational domain. The resulting velocity can be used in subsequent 
depth-imaging model building. Both synthetic and field data examples demonstrate the applicability 
of the proposed approach.

\end{abstract}

\section{Introduction}

Time-domain seismic processing has been a popular and effective tool in areas with mild lateral velocity 
variations \cite[]{yilmaz,robein,bartel}. Time migration produces images in the time 
coordinate as opposed to the usual depth coordinate. The time coordinate, along with time-migration velocities, is 
determined during time migration by optimizing image qualities. However, it is highly sensitive to lateral 
velocity changes \cite[]{black,bevc2}. Therefore, the time-migrated image is usually 
distorted \cite[]{hubral,cameron1,cameron2,cameron3,iversen}. For this reason, in many cases time migrations are 
inadequate for accurate geological interpretation of subsurface structures. On the other hand, time-migration 
velocities can be conveniently and efficiently estimated either by repeated migrations \cite[]{yilmaz2} or by 
velocity continuation \cite[]{fomel1}. Depth migration can handle general media and output 
images in regular Cartesian depth coordinates. But it requires an accurate interval 
velocity model construction, which is in practice both challenging and time-consuming. An iterative process of 
tomographic updates is often employed \cite[]{bishop,stork}, where a good initial interval velocity model is 
crucial for achieving the global minimum. It is thus of great interest to convert the time-migration velocity 
to the depth domain in order to unravel inherent distortions in time-domain images and to provide a reasonable 
starting model for building depth-imaging velocities.

The relationship between time and depth coordinates was first explained by \cite{hubral} through the concept of 
\textit{image rays}. An interval velocity model can be converted to one in the time domain by 
tracing image rays that dive into earth with slowness vector normal to the surface. The time 
coordinates are then defined by the traveltime along image rays and its surface location 
\cite[]{larner,robein}. However this process is not trivially revertible and it does not reveal directly the 
connection between time-migration velocity and interval velocity. 
According to \cite{dix}, in a layered medium where $v = v(z)$, the image rays run straightly 
downward and the time-migration velocities are the root-mean-square (RMS) velocities appearing in a truncated 
Taylor approximation for traveltimes. The Dix inversion formula is exact in a $v(z)$ medium, where the conversion 
between time- and depth-domain attributes is theoretically straightforward.

Even a mild lateral velocity variation can cause image rays to bend and invalidate Dix inversion. 
\cite{cameron1,cameron2} studied and established theoretical relations between the time-migration velocity and the 
seismic velocity in depth for general media using the paraxial ray-tracing theory. They showed 
that\old{,} the conventional Dix velocity is equal to the ratio of the interval velocity and the 
geometrical spreading of image rays. This is consistent with Dix formula because when $v = v(z)$ the geometrical 
spreading equals to one everywhere. In order to carry out the time-to-depth conversion in the presence of lateral 
velocity variations, one can solve a nonlinear partial differential equation 
(PDE) of elliptic type with boundary conditions on the surface \cite[]{cameron3}. The problem is mathematically 
ill-posed. \cite{cameron3} adopt a two-step numerical procedure for the time-to-depth conversion. The first step 
is a Lax-Friedrichs-like finite-difference method or a spectral 
Chebyshev method to solve for geometrical spreading in the time coordinate. The next step is a Dijkstra-like solver 
motivated by the fast marching method \cite[]{sethian} for velocity conversion and coordinate mapping. In this 
approach, it is crucial to yield the development of low harmonics and damp the high harmonics during the first 
stage. \cite{iversen} discussed an extension of the time-to-depth conversion problem along 2-D profiles in 3-D. 
An essential part of the algorithm is the time-stepping (integration) along image-rays. The robustness of 
these methods may not be satisfactory in practice because of stability issues that arise from the ill-posed 
nature of the problem.

In this paper, we start with the theoretical relations derived by \cite{cameron1,cameron2} but cast the 
original problem in a nonlinear iterative optimization framework. This idea is motivated by the observation that, 
for arbitrary depth velocity model, two of the PDEs can be always satisfied, while the remaining one associated 
with image-ray geometrical spreading can be rewritten in the form of a cost functional that 
indicates errors in interval velocity. A key benefit in our formulation is that each 
\old{linearization update on the interval velocity will contain information from the 
whole computational domain}\new{iteration will give a global update of the whole velocity model}. 
In contrast, the previous approaches only consider information locally at each time step. Another advantage 
is that we are able to include regularization in the optimization framework in order to deal with 
the ill-posedness issue.

The paper is organized as follows. We first show theoretical derivations of all 
necessary components involved in the proposed approach. Next, we develop a numerical implementation 
for 2-D time-to-depth conversion. Finally, we test the algorithm on synthetic and field data examples. 
\old{Some discussion on the proposed method is given in concluding the paper.}\new{We conclude the 
paper by giving some discussions on the proposed method.}

\section{Theory}

For completeness, we will first review some basic concepts related to time-to-depth conversion. Then we show 
the theoretical derivation for an optimization formulation of the problem and a corresponding inversion 
procedure. For simplicity, we restrict our analysis below to the 2-D case.

\subsection{Connection between time- and depth-domain attributes}
\inputdir{.}

In Figure~\ref{fig:imageray}, we illustrate image rays in 2-D and a \textit{forward} mapping from depth coordinate 
$(z,x)$ to time coordinate $(t_0,x_0)$ \cite[]{hubral}. \new{Throughout this paper $t_0$ stands for one-way time, 
while time migration usually produces images in two-way time \cite[]{yilmaz}, i.e. $(2 t_0,x_0)$.} Under the 
assumption of no caustics, for each subsurface location $(z,x)$ we consider the image ray from $(z,x)$ to the 
surface, where it emerges at point $(0,x_0)$, with slowness vector normal to the surface. Here $x_0$ is the 
location of the image ray at the earth surface and is a scalar. $t_0$ is the traveltime along this image ray between 
$(z,x)$ and $(0,x_0)$. The forward mapping $t_0 (z,x)$ and $x_0 (z,x)$ can be done with a knowledge of interval 
velocity $v (z,x)$. A unique \textit{inverse} mapping $z (t_0,x_0)$ and $x (t_0,x_0)$ also exists that enables 
us to directly map the time-migrated image to depth.
\plot{imageray}{width=0.95\textwidth}{Image ray in (left) the depth-domain 
can be traced with a source at location $x_0$ with slowness vector normal 
to the earth surface. Each depth coordinate $(z,x)$ along this image ray 
is then mapped into (right) the time coordinate $(t_0,x_0)$ by using its 
corresponding traveltime $t_0$ and source location $x_0$.}

The counterpart for $v (z,x)$ in the time-domain is the time-migration velocity $v_m (t_0,x_0)$, which is 
commonly estimated in prestack Kirchhoff time migration \cite[]{yilmaz,fomel1}. In a $v (z)$ medium, $v_m$ 
corresponds to the root-mean-square (RMS) velocity:
\begin{equation}
\label{eq:rms}
v_m (t_0) = \sqrt{\frac{1}{t_0} \int_0^{t_0} v^2 (z(t))\,dt}\;.
\end{equation}
A time-to-depth velocity conversion can be done by first applying the Dix inversion formula \cite[]{dix}, which 
is theoretically exact in this case:
\begin{equation}
\label{eq:dix}
v_d (t) = \sqrt{\frac{d}{d t_0} (t_0 v_m^2 (t_0))}\;,
\end{equation}
followed by performing a simple conversion from $v_d (t)$ to $v (z)$ according to 
$z = \frac{1}{2} \int_0^t v_d (t) dt$. 

In equations \ref{eq:rms} and \ref{eq:dix}, there is no dependency on $x_0$ or $x$, because image rays are 
all vertical. For an arbitrary medium, image rays will bend as they travel through the medium \cite[]{larner}. 
Therefore, in general, $v_m$ is a function of both $t_0$ and 
$x_0$ and no longer satisfies the simple expression \ref{eq:rms}, which limits the applicability of the Dix 
formula. \cite{cameron1} proved that the seismic velocity and the Dix velocity in this case are connected through 
geometrical spreading $Q$ of image rays:
\begin{equation}
\label{eq:cameron}
v_d (t_0,x_0) \equiv \sqrt{\frac{\partial}{\partial t_0} (t_0 v_m^2 (t_0,x_0))} 
= \frac{v(z(t_0,x_0),x(t_0,x_0))}{Q(t_0,x_0)}\;.
\end{equation}
In equation \ref{eq:cameron}, the generalized Dix velocity is defined by \cite{cameron1} in a way similar 
to equation \ref{eq:dix} with a change from $d / dt_0$ to a partial differentiation with respect to $t_0$. The 
quantity $Q$ is related to $x_0 (z,x)$ using its definition \cite[]{popov,cameron1}, as follows:
\begin{equation}
\label{eq:Q}
| \nabla x_0 |^2 = \frac{1}{Q^2}\;.
\end{equation}

Combining equations \ref{eq:cameron} and \ref{eq:Q} results in
\begin{equation}
\label{eq:x0}
| \nabla x_0 (z,x) |^2 = \frac{v_d^2 (t_0 (z,x),x_0 (z,x))}{v^2 (z,x)}\;.
\end{equation}
The traveltimes along image rays obey the eikonal equation \cite[]{hubral,chapman}, thus
\begin{equation}
\label{eq:t0}
| \nabla t_0 (z,x) |^2 = \frac{1}{v^2 (z,x)}\;.
\end{equation}
Finally, since $x_0$ remains constant along each image ray, there is an orthogonality condition between 
gradients of $t_0$ and $x_0$ \cite[]{cameron1}:
\begin{equation}
\label{eq:ortho}
\nabla t_0 (z,x) \cdot \nabla x_0 (z,x) = 0\;.
\end{equation}

Equations \ref{eq:x0}, \ref{eq:t0} and \ref{eq:ortho} form a system of nonlinear PDEs for $t_0 (z,x)$ and 
$x_0 (z,x)$. The input is $v_d (t_0,x_0)$, estimated from $v_m (t_0,x_0)$ by equation \ref{eq:cameron}. Solving a 
boundary-value problem for the PDEs should provide $v (z,x)$, as well as $t_0 (z,x)$ and $x_0 (z,x)$. 
Because seismic acquisitions are 
limited to the earth surface, we can only use boundary conditions at the surface. For a rectangular Cartesian 
domain with $z =0$ being the surface, the boundary conditions are
\begin{equation}
\label{eq:boundary}
\left\{ \begin{array}{lcl}
t_0 (0, x) & = & 0\;, \\
x_0 (0, x) & = & x\;.
\end{array} \right.
\end{equation}

In Appendix A, we show that the time-to-depth conversion is an ill-posed problem because it requires solving 
a Cauchy-type problem for an elliptic PDE. The missing boundary conditions on sides of the computational domain 
other than those in equation \ref{eq:boundary} can induce numerical instability when extrapolating in $t_0$ 
(or, equivalently, $z$). Instead, we consider an alternative formulation of the problem in the following section.

\subsection{The optimization formulation}

Given boundary conditions \ref{eq:boundary}, equation \ref{eq:t0} describes the traveltime $t_0$ 
of a velocity model with a plane-wave source at the surface. For a given $t_0$, equation 
\ref{eq:ortho} is a first-order linear PDE on $x_0$ and thus computation of $x_0$ is 
straightforward. Our idea is inspired by a natural logic: if the resulting $x_0$ does not satisfy equation 
\ref{eq:x0}, we need to modify $v$ in a way such that the misfit decreases, and repeat the 
process until convergence.

Mathematically, we define a cost function $f (z,x)$ based on equation \ref{eq:x0}:
\begin{equation}
\label{eq:costf}
f (z,x) = \nabla x_0 \cdot \nabla x_0 - v_d^2\,w\;,
\end{equation}
where for convenience we use slowness-squared $w (z,x) = v^{-2} (z,x)$ instead of $v$. Note that $f$ is 
dimensionless. The discretized form of equation \ref{eq:costf} reads
\begin{equation}
\label{eq:cost}
\mathbf{f} = (\nabla \mathbf{x_0} \cdot \nabla)\,\mathbf{x_0} 
- \mbox{diag}(\mathbf{v_d} \star \mathbf{v_d})\,\mathbf{w} 
\equiv \mathbf{L}_{x_0}\,\mathbf{x_0} 
- \mbox{diag}(\mathbf{v_d} \star \mathbf{v_d})\,\mathbf{w}\;.
\end{equation}
In equation \ref{eq:cost}, $\mathbf{f}, \mathbf{x_0}, \mathbf{v_d}$ and $\mathbf{w}$ are all column vectors after 
discretizing the computational domain $(z,x)$. For example, $\mathbf{x_0}$ is the discretized column vector of 
$x_0 (z,x)$. The vector $\mathbf{v_d}$ may require interpolation because it is in $(t_0,x_0)$ while 
the discretization is in $(z,x)$. The interpolation can be done after forward mapping from $(z,x)$ to $(t_0,x_0)$ 
at current velocity model. We denote an operator which is a matrix 
$\mathbf{L}_{x_0} = \nabla \mathbf{x_0} \cdot \nabla$. The other operator $\mbox{diag}()$ expands a vector 
into a diagonal matrix. Finally, the symbol $\star$ stands for an element-wise vector-vector multiplication.

As is common in many optimization problems, we seek to minimize the least-squares norm of $\mathbf{f}$:
\begin{equation}
\label{eq:objective}
E [\mathbf{w}] = \frac{1}{2} \|\mathbf{f}\|^2 = \frac{1}{2} \mathbf{f}^T \mathbf{f}\;,
\end{equation}
where the superscript $T$ stands for transpose. The Gauss-Newton method in optimization 
requires linearizing the cost function in equation \ref{eq:cost}:
\begin{equation}
\label{eq:lincost}
\frac{\partial f}{\partial w} = 
2\,(\nabla x_0 \cdot \nabla)\,\frac{\partial x_0}{\partial w} -
2\,v_d\,w\,\frac{\partial v_d}{\partial w} - v_d^2\;.
\end{equation}
The Fr\'{e}chet derivative matrix $\mathbf{J}$ required by inversion is the discretized 
form of equation \ref{eq:lincost}, i.e., $\mathbf{J} = \partial \mathbf{f} / \partial \mathbf{w}$. In Appendix 
B we find that $\mathbf{J}$ is a cascade and summation of several parts. An update 
$\delta \mathbf{w}$ at current $\mathbf{w}$ is found by solving the following normal equation arising from 
the Gauss-Newton approach \cite[]{bjorck}:
\begin{equation}
\label{eq:normal}
\delta \mathbf{w} = \left[ \mathbf{J}^T \mathbf{J} \right]^{-1} \mathbf{J}^T (- \mathbf{f})\;.
\end{equation}

Equations \ref{eq:objective} and \ref{eq:normal} together suggest a nonlinear inversion 
procedure for solving the original system of PDEs \ref{eq:x0}, \ref{eq:t0} and \ref{eq:ortho}. The inversion 
is analogous to traveltime tomography but with more complexity. The cost \ref{eq:costf} can be interpreted as 
difference between modeled and observed geometrical spreadings. However, both of them depend on the model $v$, 
while in traveltime tomography the observed arrival times are independent of $v$. The forward modeling in our case 
involves two steps, which construct a curvilinear coordinate system that is sensitive to lateral velocity 
variations. On the other hand, the forward modeling in traveltime tomography consists of only one step\old{, 
i.e., solving equation \ref{eq:t0} with a point-source boundary condition}. Last but not least, unlike 
traveltime tomography, we have observations everywhere in the computational domain, except for \old{some null 
space created during}\new{areas excluded due to instabilities of the} numerical implementation, as we will 
discuss later.\old{ Note that the inversion is applied directly for slowness-squared $\mathbf{w}$ (and thus 
velocity), and the update $\delta \mathbf{w}$ will incorporate dependency of velocities throughout the domain, 
as physically honored by image rays. Previous methods \mbox{\cite[]{cameron1,cameron2,iversen}} that rely on 
time-stepping in $t_0$ account for such dependencies only locally.}

Before introducing a numerical implementation, we would like to point out several important facts 
and assumptions that make a successful time-to-depth conversion possible by the proposed method:
\begin{itemize}
\item Caustics must be excluded from the computational domain. In regions where caustics develop, the gradient 
$\nabla x_0$ goes to infinity and the cost function is not well-defined. For all numerical examples in this 
paper, we do not encounter this issue. In the Discussion section, we provide a possible strategy to cope with 
this limitation.
\item According to derivations in Appendix B, the calculation of $\delta \mathbf{w}$ depends on 
values of $\partial \mathbf{v_d} / \partial \mathbf{t_0}$ and $\partial \mathbf{v_d} / \partial \mathbf{x_0}$. 
Thus the input $v_d$ should be differentiable. This requirement can be satisfied during $v_m$ estimation by 
using regularization \cite[]{fomel1}.
\item Similarly to all nonlinear inversions, the proposed method requires a 
prior model that is sufficiently close to desired model at the global minimum $E = 0$. Meanwhile, to 
guarantee stability and a smooth output, some form of regularization should be imposed during 
inversion \cite[]{engl,zhdanov}.
\item Our formulation does not change the ill-posed nature of the original problem. One 
assumption is that condition \ref{eq:boundary} describes all \textit{in-flow} domain boundaries of $t_0$ and 
$x_0$. In other words, the image rays are only allowed to be either parallel to or exiting (\textit{out-flow}) 
all other boundaries \new{of the computational domain} except the surface.
\end{itemize}

For the prior model, we adopt the Dix-inverted model. In other words, we seek to refine the interval 
velocity given by equation \ref{eq:dix} by taking the geometrical spreading of image rays into consideration 
according to equation \ref{eq:cameron}.

\section{Numerical Implementation}

Below we outline the steps involved in computing one linearization update:

\begin{enumerate}
\item Given current $v (z,x)$, solve equation \ref{eq:t0} for $t_0 (z,x)$ with $t_0 (0,x) = 0$;
\item Given $t_0 (z,x)$ and $x_0 (0,x) = x$, solve equation \ref{eq:ortho} for $x_0 (z,x)$;
\item Given $t_0 (z,x)$ and $x_0 (z,x)$, interpolate $v_d (z,x)$ from $v_d (t_0,x_0)$ and compute $f (z,x)$ 
based on equation \ref{eq:costf};
\item Assemble linear operator \ref{eq:final} and solve equation \ref{eq:normal} for $\delta w (z,x)$.
\end{enumerate}

First, we apply the fast-marching method \cite[]{sethian,sethian2} to solve the eikonal equation 
\ref{eq:t0} by initializing a plane-wave source at $z = 0$. Computation for $x_0$ can be incorporated into $t_0$ 
by adopting the upwind finite-differences of $t_0$ for equation \ref{eq:ortho}. In Figure~\ref{fig:fmm}, consider 
a currently updated grid point $C$ during forward modeling of $t_0$. If it has only one upwind neighbor $A$ that 
is inside the wave-front, $t_0 (C) > t_0 (A)$, then the image ray must be aligned with grid segment $AC$ and
therefore $x_0 (C) = x_0 (A)$. \new{We refer to this scenario as one-sided.} If $C$ has two upwind neighbors $A$ 
and $B$, $t_0 (C) > \mbox{max}\{t_0 (A), t_0 (B)\}$, and they are both inside the wave-front, then the image 
ray must intersect the simplex $ABC$ from an angle. In this case, we compute $x_0 (C)$ from
\begin{equation}
\label{eq:two}
\frac{(t_0 (C) - t_0 (A))(x_0 (C) - x_0 (A))}{h_z^2} + 
\frac{(t_0 (C) - t_0 (B))(x_0 (C) - x_0 (B))}{h_x^2} = 0\;.
\end{equation}
\plot{fmm}{width=0.55\textwidth}{A modified fast-marching method for 
forward modeling. Black dots represent region that has 
been swept by the wave-front, gray dots are the expanding 
wave-front and grid points being updated, and white dots are region 
yet to be reached.}

Because $\mathbf{x_0}$ at certain grid points is calculated by one-sided scenario, $\mathbf{L}_{x_0}$ there 
contains all zeros. Consequently, an evaluation of the cost $\mathbf{f}$ at these locations with 
$\mathbf{L}_{x_0}\,\mathbf{x_0}$ becomes inaccurate. We exclude these regions from $\mathbf{f}$ and expect 
inversion to fill \old{the null space}\new{them}.

Next, we apply simple bilinear interpolation for $v_d (z,x)$ and estimate $\delta \mathbf{w}$ by 
solving equation \ref{eq:normal} using shaping regularization \cite[]{fomel2}. We use a 
triangular smoother with adjustable size as the shaping operator. We find in numerical tests that shaping 
significantly improves convergence speed compared to that of the traditional Tikhonov 
regularization \cite[]{tikhonov} with gradient operators. We also observe that without regularization the 
model update can be undesirably oscillatory. We believe this phenomenon is related to the ill-posedness of 
the PDEs.

Finally, we reduce computational cost by adopting the method of 
conjugate gradients \cite[]{hestenes} and an efficient implementation of $\mathbf{J}$, as well as its 
adjoint, according to the equations derived in Appendix B. For this purpose, we choose the upwind 
finite-difference scheme \cite[]{franklin,li1} based on 
$\mathbf{t_0}$ for both $\mathbf{L}_{t_0}$ and $\mathbf{L}_{x_0}$. As shown by \cite{li3}, applying 
$\mathbf{J}$ and its transpose involves only sparse triangularized matrix-vector multiplications and 
is therefore inexpensive. \new{For example, at each grid point $\mathbf{L}_{t_0}^{-1}$ relies on only its 
upwind neighbors. The computational complexity of $\mathbf{J}$ and $\mathbf{J}^T$ is $O (N)$, where 
$N$ is the total number of grid points.}

\section{Examples}

\subsection{Constant velocity gradient model}
\inputdir{vgrad}

In a medium with a constant velocity gradient
\begin{equation}
\label{eq:constg0}
v (z,x) = v_0 + g_z z + g_x x\;,
\end{equation}
image rays are segments of circles parallel to each other, and all attributes involved in time-to-depth 
conversion have analytical solutions. In Appendix C, we derive the following 
relationships:
\begin{equation}
\label{eq:analx}
x_0 (z,x) = x + \frac{\sqrt{(v_0+g_x x)^2 + g_x^2 z^2} - (v_0 + g_x x)}{g_x}\;,
\end{equation}
\begin{equation}
\label{eq:analt}
t_0 (z,x) = \frac{1}{g} \mathrm{arccosh} \left[ \frac{g^2 \left( \sqrt{(v_0+g_x x)^2 + g_x^2 z^2}
+ g_z z \right) - v g_z^2}{v g_x^2} \right]\;,
\end{equation}
where $g = \sqrt{g_z^2 + g_x^2}$. The migration velocity $v_m$ and Dix velocity $v_d$ take the 
following expressions:
\begin{equation}
\label{eq:analvm}
v_m (t_0,x_0) = \frac{(v_0 + g_x x_0)^2}{t_0 \left( g \coth (g t_0) - g_z \right)}\;,
\end{equation}
\begin{equation}
\label{eq:analvd}
v_d (t_0,x_0) = \frac{(v_0 + g_x x_0) g}{g \cosh (g t_0) - g_z \sinh (g t_0)}\;.
\end{equation}

It is easy to verify from equations \ref{eq:analx} and \ref{eq:analt} that 
$|\nabla x_0| = 1$, $|\nabla t_0| = 1/v$ and $\nabla x_0 \cdot \nabla t_0 = 0$. Because there is no geometrical 
spreading of image rays in this case, the Dix velocity will be equal to the interval velocity according to 
equation \ref{eq:cameron}. However, a Dix-inverted model will still be distorted if $g_x \not= 0$ because of the 
lateral shift of image rays.

Figures~\ref{fig:vgrad} and \ref{fig:analy} show a velocity model with $v (z,x) = 1.5 + 0.75 z + 0.5 x$ km/s and 
the corresponding analytical $v_d (t_0,x_0)$, $x_0 (z,x)$ and $t_0 (z,x)$. Clearly, the right domain boundary is 
of in-flow type that violates our assumption. To address this challenge, we include Dix velocity 
in regions beyond the original left and right boundaries during inversion, but mask out the cost in these regions. 
It means that the time-to-depth conversion is performed in a sub-domain of time-domain attributes, such that 
information on the in-flow boundary is available. Afterwards, we apply Dix inversion to the expanded model and 
use the result as the prior model. We use the exact Dix velocity in equation \ref{eq:analvd} 
for evaluating the right-hand side of \ref{eq:x0}. Then, in total three linearization updates are carried 
out, which decreases $E$ to relative $0.6\%$. \new{The radiuses of triangular smoother in shaping are $8$ m 
vertically and $30$ m horizontally ($8$ m $\times$ $30$ m).} At last, we cut the computational domain back to its 
original size. Figures \ref{fig:cost} and \ref{fig:error} compare the cost and model misfit before and after 
inversion.

We also synthesize data with Kirchhoff modeling \cite[]{haddon} for several horizontal reflectors using the exact 
model, and examine the subsurface scattering-angle common-image-gathers from Kirchhoff prestack depth 
migration \cite[]{xu} as an evidence of interval velocity improvements. In 
Figure~\ref{fig:cigv}, the shallower events do not improve significantly because the image rays have not yet bent 
considerably. Deeper events become noticeably 
flatter after applying the proposed method.
\plot{vgrad}{width=0.95\textwidth}{(Top) a constant velocity gradient 
model and (bottom) the analytical Dix velocity $v_d$. A curved image 
ray is mapped to the time domain as a straight line.}
\plot{analy}{width=0.95\textwidth}{Analytical values of (top) $t_0$ 
and (bottom) $x_0$ of the model in Figure~\ref{fig:vgrad}. Both figures 
are overlaid with contour lines that, according to equation 
\ref{eq:ortho}, are perpendicular to each other. Each contour line of 
$x_0$ is an image ray, while the contours of $t_0$ illustrate the 
propagation of a plane-wave.}
\plot{cost}{width=0.95\textwidth}{The cost defined by equation 
\ref{eq:costf} (top) before and (bottom) after inversion. The 
least-squares norm of cost $E$ is decreased from $9.855$ to $0.057$.}
\plot{error}{width=0.95\textwidth}{The difference between exact model 
and (top) initial model and (bottom) inverted model. The least-squares 
norm of model misfit is decreased from $15.6\mbox{ km}^2/\mbox{s}^2$ 
to $2.7\mbox{ km}^2/\mbox{s}^2$.}
\plot{cigv}{width=0.95\textwidth}{Comparison of the subsurface 
scattering-angle common-image-gathers at $x = 1.5$ km of (left) 
exact model, (middle) prior model and (right) inverted model.}

\subsection{Constant horizontal slowness-squared gradient model}
\inputdir{hs2grad}

Another medium that provides analytical time-to-depth conversion formulas is
\begin{equation}
\label{eq:consths20}
w (z,x) = w_0 - 2 q_x x\;,
\end{equation}
i.e., the slowness-squared changes linearly in the horizontal direction. In Appendix D, we show that in this case 
the coordinate mapping follows
\begin{equation}
\label{eq:s2analx0}
x_0 (z,x) = \frac{2 w_0 x + q_x z^2}{w_0 + 2 q_x x + \sqrt{(w_0 - 2 q_x x)^2 - 4 q_x^2 z^2}}\;,
\end{equation}
\begin{equation}
\label{eq:s2analt0}
t_0 (z,x) = \frac{\sqrt{2} z \left[ 2 w_0 - 4 q_x x + \sqrt{(w_0 - 2 q_x x)^2 - 4 q_x^2 z^2} 
\right]}{3 \left[ w_0 - 2 q_x x + \sqrt{(w_0 - 2 q_x x)^2 - 4 q_x^2 z^2} \right]^{\frac{1}{2}}}\;.
\end{equation}
The analytical expression for $v_d (t_0,x_0)$ is more complex and can be found in equations \ref{eq:rootdepcubic} 
and \ref{eq:s2analvd}.

In Figure~\ref{fig:hs2analy} we illustrate $x_0 (z,x)$ and $t_0 (z,x)$ in the model $w (z,x) = 1 - 0.052 x$ 
$\mbox{s}^2 / \mbox{km}^2$. To deal with the in-flow boundary issue, we apply the method described previously for 
the constant velocity gradient example. Unlike equation \ref{eq:analx}, equation \ref{eq:s2analx0} indicates 
varying geometrical spreadings in the domain. Figure~\ref{fig:hs2grad} shows the corresponding analytical 
$Q^2 (z,x)$ and $v_d (t_0,x_0)$. The geometrical spreading is significant at the lower-right corner of the domain, 
which translates to the cost at approximately the same location in Figure~\ref{fig:hs2cost}. We use analytical 
Dix velocity as the input in the inversion. Starting from the Dix-inverted model and after three linearization 
updates, $E$ decreases to relative $0.0045\%$. \new{The size of triangular smoother is $8$ m $\times$ $20$ m.} 
The model misfit, as demonstrated in Figure~\ref{fig:hs2error}, is also improved.
\plot{hs2analy}{width=0.95\textwidth}{Analytical values of (top) 
$t_0$ and (bottom) $x_0$, overlaid with contour lines.}
\plot{hs2grad}{width=0.95\textwidth}{The (top) geometrical 
spreading and (bottom) Dix velocity associated with the model used in 
Figure~\ref{fig:hs2analy}.}
\plot{hs2cost}{width=0.95\textwidth}{The cost (top) before and 
(bottom) after inversion. The least-squares norm of cost $E$ is 
decreased from $10.431$ to $0.047$.}
\plot{hs2error}{width=0.95\textwidth}{The difference between exact 
model and (top) initial model and (bottom) inverted model. The 
least-squares norm of model misfit is decreased from 
$5.0\mbox{ km}^2/\mbox{s}^2$ to $0.5\mbox{ km}^2/\mbox{s}^2$.}

\subsection{Spiral model}
\inputdir{synth}
\plot{vz0}{width=0.95\textwidth}{(Top) a synthetic model and 
(middle) Dix velocity converted to depth. Both overlaid with image 
rays. (Bottom) the model perturbation for testing linearization.}

Figure~\ref{fig:vz0} shows a synthetic model borrowed from \cite{cameron2}. The Dix inversion recovers the 
shallow part of the model but deteriorates quickly as geometrical spreading of image rays grows in the deeper 
section.

As a simple verification for the linearization process, we add a small positive velocity perturbation at location 
$(1,3)$ km to the synthetic model. Comparisons between the exact and linearly predicted attributes are 
illustrated in Figures~\ref{fig:pdt}, \ref{fig:pdx} and \ref{fig:diffcost}. In accordance with forward 
modeling, where we solve firstly $t_0$, then $x_0$, and finally $f$, the linearization (see Appendix B) is carried 
out following the same sequence. First, Figure~\ref{fig:pdt} justifies our upwind finite-differences 
implementation of the linearized eikonal equation. The positive perturbation in $v$ in Figure~\ref{fig:vz0} causes 
$t_0$ to decrease in a narrow downwind region. Next, the area affected by the perturbation in Figure~\ref{fig:pdx} 
is wider than that in Figure~\ref{fig:pdt}. It also has both positive and negative amplitudes. These phenomenon 
are physical because image rays should bend in opposite directions in response to the perturbation. Finally, 
effects in cost $f$ in Figure~\ref{fig:diffcost} show alternating polarities and are broader in width compared to 
that of $d t_0$ and $d x_0$. They indicate a complicated dependency of $f$ on $w$. Note the good agreements in 
both shape and magnitude between exact and linearly predicted quantities in all three steps.

Because there is no analytical formula for Dix velocity in this model, we compute $v_d$ by tracing image 
rays numerically in the exact model $v$. Also, based on Figure~\ref{fig:vz0}, there is no in-flow boundary other 
than $z = 0$. Therefore, we do not need to extend the domain as in the preceding examples. We 
use the Dix-inverted model as the prior model and run the inversion. It turns out that the first linearization 
update is sufficient for achieving the desired global minimum as shown in Figure~\ref{fig:bgrad}.
\plot{pdt}{width=0.95\textwidth}{(Top) exact $d t_0$ and (bottom) 
linearly predicted $d t_0$ by equation \ref{eq:dt0}.}
\plot{pdx}{width=0.95\textwidth}{(Top) exact $d x_0$ and (bottom) 
linearly predicted $d x_0$ by equation \ref{eq:dortho}.}
\plot{diffcost}{width=0.95\textwidth}{(Top) exact $d f$ and (bottom) 
linearly predicted $d f$ by equation \ref{eq:final}.}
\plot{bgrad}{width=0.95\textwidth}{(Top) the exact $\delta w$ 
and (middle) the computed $\delta w$ of the first linearization step. 
(Bottom) the inverted interval velocity model. Compare with 
Figure~\ref{fig:vz0}.}

\subsection{Field data example}
\inputdir{beinew}

The field data shown in Figure~\ref{fig:vdix} is from a section of Gulf of Mexico dataset \cite[]{claerbout}. We 
estimate $v_m$ using the method of velocity continuation \cite[]{fomel1} and convert it to $v_d$. Similar to 
the spiral model, no domain extension is needed. In Figure~\ref{fig:init}, the Dix-inverted prior model highly 
resembles the Dix velocity, because the Dix formula only scales the vertical axis from time to depth regardless 
of horizontal $v_d$ variations. Figure~\ref{fig:inv} compares the cost before and after five linearization 
updates\old{.}\new{, with a $125$ m $\times$ $335$ m triangular smoother.} In Figure~\ref{fig:hist}, the 
$l_2$ norm of the cost, $E$, has a rapid decrease to relative $1.3$\%. Figure~\ref{fig:dinv} illustrates the 
inverted model and interval velocity update.
\plot{vdix}{width=0.95\textwidth}{(Top) the estimated 
time-migration velocity of a section of Gulf of Mexico dataset and 
(bottom) the corresponding time-migrated image.}
\plot{init}{width=0.95\textwidth}{(Top) the Dix velocity 
converted from $v_m$ in Figure~\ref{fig:vdix} and (bottom) the 
Dix-inverted prior model for inversion, overlaid with image rays.}
\plot{inv}{width=0.95\textwidth}{The cost of (top) prior model 
($E = 140.25$) and (bottom) inverted model ($E = 1.81$).}
\sideplot{hist}{width=0.55\textwidth}{Convergence history of 
the proposed optimization-based time-to-depth conversion.}
\plot{dinv}{width=0.95\textwidth}{(Top) the inverted model, 
overlaid with image rays, and (bottom) its difference from the prior 
model in Figure~\ref{fig:init}.}

Next, we map the time-migrated image to depth using $t_0$ and $x_0$ generated during inversion. Spline 
interpolation \cite[]{press} is used during the coordinate mapping. We also migrate the 
prestack data by Kirchhoff depth migration \cite[]{li2} (PSDM). Figure~\ref{fig:dmig} compares the 
time-mapped image and PSDM image of the inverted model. A good agreement between these two images justifies that 
time-to-depth conversion has effectively unravelled the distorted time coordinate. Figure~\ref{fig:ddmig0} compares 
PSDM images of the prior and inverted models. The velocity update in Figure~\ref{fig:dinv} results in not only 
changes in structural dips (for example at $(3,12)$ km) but also improved reflector continuity (for example at 
$(3.7,11)$ km). Moreover, the Kirchhoff migration outputs surface offset common-image 
gathers. We choose two midpoint locations, $x = 11$ km and $x = 12$ km, 
and show their common-image gathers in Figure~\ref{fig:cig0}. In deeper sections, flat dashed lines are 
overlaid as references for the 
flatness of gathers. The two common-image gathers of prior model appear curved in opposite directions. After 
time-to-depth conversion, both gathers get flattened across the whole offset range, 
verifying a correct velocity update.
\plot{dmig}{width=0.95\textwidth}{(Top) the time-migrated image 
in Figure~\ref{fig:vdix} is mapped to depth using products of the 
time-to-depth conversion. (Bottom) PSDM image using inverted model 
in Figure~\ref{fig:dinv}.}
\plot{ddmig0}{width=0.92\textwidth}{PSDM images of (top) the 
prior model and (bottom) the inverted model. Both images are plotted 
for the same central deep part.}
\plot{cig0}{width=0.95\textwidth}{The surface offset 
common-image gathers of prior and inverted models.}

\section{Discussion}

\new{Note that in our method the inversion is applied directly for slowness-squared $\mathbf{w}$ (and thus 
velocity), and the update $\delta \mathbf{w}$ will incorporate dependency of velocities throughout 
the domain, as physically honored by image rays. Previous methods \cite[]{cameron1,cameron2,iversen} 
that rely on time-stepping in $t_0$ account for such dependencies only locally. Moreover, regularization 
provides a convenient and effective way of handling ill-posedness of the underlying PDEs. For these 
reasons, the proposed method should be numerically robust compared to the previous approaches.}

A 3-D extension of the proposed method is straightforward. Instead of a scalar $x_0$ we 
need to handle both inline and crossline coordinates $\mathbf{x_0} = (x_0, y_0)$. Consequently, the geometrical 
spreading in 3-D becomes a matrix $\mathbf{Q}$, whose determinant 
is related to the generalized Dix velocity $v_d (t_0,\mathbf{x_0})$ and interval velocity $v (z,\mathbf{x})$ 
\cite[]{cameron1}. Starting from this relationship, an iterative time-to-depth conversion can be established by 
following procedures similar to the ones described in this paper.

The main limitation of our approach is the failure of underlying theory at caustics, which in turn limits either 
the depth or the extent of lateral velocity variation of the model. A possible solution is to divide 
the original domain into several depth intervals, then apply velocity estimation and redatuming 
from one interval to another recursively \cite[]{bevc1,li2}. By doing so, image-ray crossing may 
not develop within each interval thanks to a limited depth.

Another issue is the handling of in-flow boundaries other than the earth surface. In the synthetic examples, 
we avoided this problem by limiting the interval velocity model within image ray coverage. Alternatively, we could 
pad the input $v_d$ laterally. The padding should simply replicate the original boundaries. By doing so, image 
rays at the new left and right boundaries must run straightly downward as the media there are of $v (z)$ type. 
Although we could not expect the inversion to fix the in-flow boundary, the resulting errors should be local 
around that boundary.

Finally, in our approach time-migration velocity not only determines the prior model but also 
drives the inversion. Consequently, errors in time-migration velocity have a direct influence on the accuracy of 
inverted model. Note that usually the picking of time-migration velocity is carried out with 
certain smoothing. Combined with regularization in the time-to-depth conversion, the resulting interval velocity 
model may contain limited fine-scale features and high velocity contrasts. In this regard, we suggest our method 
as an efficient estimation of an initial guess for subsequent depth-imaging velocity model refinements.

\section{Conclusions}

We have introduced a novel nonlinear inversion formulation for time-to-depth velocity conversion and 
coordinate mapping, which involves iterative least-squares optimization constrained by a system of partial differential 
equations. We turn one of the governing equations that relates seismic velocity to geometrical spreading of image 
rays into a cost function, and linearize the other two equations: the eikonal equation and the orthogonality 
condition. Regularization provides extra constraints during inversion\old{ and helps fill the null space}. The 
proposed method appears to be robust and the inversion in practice converges fast.

\section{Acknowledgments}

We are grateful to Tijmen Jan Moser, Alexey Stovas, Anton Duchkov, and two anonymous reviewers for their 
valuable comments and suggestions. We thank BGP International for partial financial support of this research. We 
also thank Alexander Klokov, Alexander Vladimirsky and Changlong Wang for helpful discussions.

\appendix
\section{Appendix A: Ill-posedness of the time-to-depth conversion problem}

Let us consider the problem of solving for $z (t_0,x_0)$ and $x (t_0,x_0)$ instead of 
$t_0 (z,x)$ and $x_0 (z,x)$ by recasting 
equations \ref{eq:x0}, \ref{eq:t0} and \ref{eq:ortho} in the time coordinates:
\begin{equation}
\label{eq:igradx}
\left(\frac{\partial x}{\partial x_0}\right)^2 + 
\left(\frac{\partial z}{\partial x_0}\right)^2 
= \frac{v^2 (z (t_0,x_0), x (t_0,x_0))}{v_d^2 (t_0,x_0)}\;,
\end{equation}
\begin{equation}
\label{eq:igradt}
\left(\frac{\partial x}{\partial t_0}\right)^2 + 
\left(\frac{\partial z}{\partial t_0}\right)^2 
= v^2 (z (t_0,x_0), x (t_0,x_0))\;, 
\end{equation}
\begin{equation}
\label{eq:idotp}
\frac{\partial x}{\partial x_0}\,\frac{\partial x}{\partial t_0} + 
\frac{\partial z}{\partial x_0}\,\frac{\partial z}{\partial t_0} 
= 0\;.
\end{equation}
The corresponding boundary conditions are
\begin{equation}
\label{eq:ixb}
\left\{ \begin{array}{lcl}
z (0, x_0) & = & 0\;, \\
x (0, x_0) & = & x_0\;.
\end{array} \right.
\end{equation}

From equation \ref{eq:idotp}
\begin{equation}
\label{eq:xt}
\frac{\partial x}{\partial t_0} = - \frac{\partial z}{\partial x_0}\,\frac{\partial z}{\partial t_0}\,
\left/ \frac{\partial x}{\partial x_0} \right.\;.
\end{equation}
Substituting equation \ref{eq:xt} into equation \ref{eq:igradt} and combining with equation \ref{eq:igradx} 
produces
\begin{equation}
\label{eq:zt2}
\frac{\partial z}{\partial t_0} = v_d \frac{\partial x}{\partial x_0}\;;
\end{equation}
\begin{equation}
\label{eq:xt2}
\frac{\partial x}{\partial t_0} = - v_d \frac{\partial z}{\partial x_0}\;,
\end{equation}
where we assume that both $\partial z/\partial t_0$ and $\partial x/\partial x_0$ remain positive and there is no 
caustics.

On the first glance, equations \ref{eq:zt2} and \ref{eq:xt2} seem suitable for numerically extrapolating 
$x (t_0,x_0)$ and $z (t_0,x_0)$ in $t_0$ direction using the boundary conditions \ref{eq:ixb}. 
After such an extrapolation, one would be able to reconstruct $v (t_0,x_0)$ from equation \ref{eq:igradt} and 
thus solve the original problem. However, by further decoupling the system using the equivalence of the 
second-order mixed derivatives, we discover that the underlying PDEs are elliptic. For instance, applying 
$\partial / \partial x_0$ to both sides of equation \ref{eq:xt2} results in
\begin{equation}
\label{eq:xt3}
\frac{\partial^2 x}{\partial t_0 \partial x_0} =
- \frac{\partial}{\partial x_0} \left(v_d \frac{\partial z}{\partial x_0}\right)\;.
\end{equation}
Meanwhile, dividing by $v_d$ and applying $\partial / \partial t_0$ to both sides of equation \ref{eq:zt2} 
leads to
\begin{equation}
\label{eq:zt3}
\frac{\partial}{\partial t_0} \left(\frac{1}{v_d} \frac{\partial z}{\partial t_0}\right) =
\frac{\partial^2 x}{\partial x_0 \partial t_0}\;.
\end{equation}
Comparing equations \ref{eq:xt3} and \ref{eq:zt3}, we find
\begin{equation}
\label{eq:zpde}
\frac{\partial}{\partial x_0}\,
\left(v_d \frac{\partial z}{\partial x_0}\right)
+ \frac{\partial}{\partial t_0}\,
\left(\frac{1}{v_d} \frac{\partial z}{\partial t_0}\right) = 0\;.
\end{equation}
Analogously,
\begin{equation}
\label{eq:xpde}
\frac{\partial}{\partial x_0}\,
\left(v_d \frac{\partial x}{\partial x_0}\right)
+ \frac{\partial}{\partial t_0}\,
\left(\frac{1}{v_d} \frac{\partial x}{\partial t_0}\right) = 0\;.
\end{equation}

Solving elliptic equations \ref{eq:zpde} and \ref{eq:xpde} with the Cauchy-type boundary conditions 
\ref{eq:ixb} is an ill-posed problem \cite[]{evans}. A different formulation, leading to a non-linear 
elliptic PDE, was previously discussed by \cite{cameron3}.

\appendix
\section{Appendix B: The Fr\'{e}chet derivative operator}

We continue the derivation of the Fr\'{e}chet derivative matrix from equation 
\ref{eq:lincost}. Using the same notations as introduced in equation \ref{eq:cost}, after discretization equation 
\ref{eq:lincost} becomes 
\begin{equation}
\label{eq:gradient}
\mathbf{J} \equiv \frac{\partial \mathbf{f}}{\partial \mathbf{w}} 
= 2\,\mathbf{L}_{x_0}\,\frac{\partial \mathbf{x_0}}{\partial \mathbf{w}} 
- 2\,\mbox{diag}(\mathbf{v_d} \star \mathbf{w})\,\frac{\partial \mathbf{v_d}}{\partial \mathbf{w}} 
- \mbox{diag}(\mathbf{v_d} \star \mathbf{v_d})\;.
\end{equation}

Because $v_d$ is in time-domain $(t_0,x_0)$, we need to apply the chain-rule for its derivative with respect to 
$w$, i.e.,
\begin{equation}
\label{eq:dvdix}
\frac{\partial \mathbf{v_d}}{\partial \mathbf{w}} = \frac{\partial \mathbf{v_d}}{\partial \mathbf{t_0}} 
\,\frac{\partial \mathbf{t_0}}{\partial \mathbf{w}} + \frac{\partial \mathbf{v_d}}{\partial \mathbf{x_0}} 
\,\frac{\partial \mathbf{x_0}}{\partial \mathbf{w}}\;,
\end{equation}
where $\mathbf{t_0}$ is the discretized column vector of $t_0 (z,x)$. According to equation \ref{eq:ortho} and 
after denoting another matrix operator $\mathbf{L}_{t_0} = \nabla \mathbf{t_0} \cdot \nabla$, we find
\begin{equation}
\label{eq:dortho}
\frac{\partial \mathbf{x_0}}{\partial \mathbf{w}} = 
- (\nabla \mathbf{t_0} \cdot \nabla)^{-1}(\nabla \mathbf{x_0} \cdot \nabla)\,
\frac{\partial \mathbf{t_0}}{\partial \mathbf{w}} \equiv 
- \mathbf{L}_{t_0}^{-1} \mathbf{L}_{x_0}\,\frac{\partial \mathbf{t_0}}{\partial \mathbf{w}}\;.
\end{equation}
Another differentiation of equation \ref{eq:t0} leads to
\begin{equation}
\label{eq:dt0}
\frac{\partial \mathbf{t_0}}{\partial \mathbf{w}} = \frac{1}{2} (\nabla \mathbf{t_0} \cdot \nabla)^{-1} 
\equiv \frac{1}{2} \mathbf{L}_{t_0}^{-1}\;.
\end{equation}

Finally, by inserting equations \ref{eq:dvdix} through \ref{eq:dt0} into \ref{eq:gradient}, we complete the 
derivation of the Fr\'{e}chet derivative matrix:
\begin{eqnarray}
\label{eq:final}
\mathbf{J} & = & - \mathbf{L}_{x_0} \mathbf{L}_{t_0}^{-1} \mathbf{L}_{x_0} \mathbf{L}_{t_0}^{-1} 
- \mbox{diag}(\mathbf{v_d} \star \mathbf{w}) \frac{\partial \mathbf{v_d}}{\partial \mathbf{t_0}} \mathbf{L}_{t_0}^{-1} \nonumber \\
& + & \mbox{diag}(\mathbf{v_d} \star \mathbf{w}) \frac{\partial \mathbf{v_d}}{\partial \mathbf{x_0}} 
\mathbf{L}_{t_0}^{-1} \mathbf{L}_{x_0} \mathbf{L}_{t_0}^{-1} - \mbox{diag}(\mathbf{v_d} \star \mathbf{v_d})\;.
\end{eqnarray}

\appendix
\section{Appendix C: Analytical expressions for the constant velocity gradient medium}
In order to derive the time-to-depth conversion analytically, we first trace image rays in the depth 
coordinate for $z (t_0,x_0)$ and $x (t_0,x_0)$. Then we carry out a direct inversion to find $t_0 (z,x)$ and 
$x_0 (z,x)$. The Dix velocity can be obtained at last following equations \ref{eq:cameron} and \ref{eq:Q}.

Continuing from equation \ref{eq:constg0}, we write the velocity in a coordinate relative to the image ray 
\begin{equation}
\label{eq:constg}
v (z,x) = v_0 + g_z z + g_x x = \tilde{v}_0 + \mathbf{g} \cdot (\mathbf{x} - \mathbf{x_0})\;,
\end{equation}
where $\mathbf{g} = [g_z,g_x]^T$ and $\tilde{v}_0 = v_0 + g_x x_0$. At the starting point, image ray satisfies
\begin{equation}
\label{eq:viray}
\left\{ \begin{array}{lcl}
\mathbf{x_0} & = & [0, x_0]^T\;, \\
\mathbf{p_0} & = & [\tilde{v}_0^{-1}, 0]^T\;, \\
t_0 & = & t\;.
\end{array} \right.
\end{equation}
Here we denote ray parameter $\mathbf{p} = \nabla t$ and $\mathbf{p_0}$ is the ray parameter at source. The 
Hamiltonian for ray tracing reads $H (\mathbf{x},\mathbf{p}) = \mathbf{p} \cdot \mathbf{p} - v^{-2} \equiv 0$. 
The corresponding ray tracing system is \cite[]{cerveny}:
\begin{equation}
\label{eq:vraysys}
\left\{ \begin{array}{lcl}
d \mathbf{x} / d \xi & = & \mathbf{p} v^3\;, \\
d \mathbf{p} / d \xi & = & -\nabla v\;, \\
d t / d \xi & = & \nabla t \cdot d \mathbf{x} / d \xi = \mathbf{p} \cdot \mathbf{p} v^3 = v\;.
\end{array} \right.
\end{equation}

Equation \ref{eq:constg} indicates $\nabla v = \mathbf{g}$, which means $d \mathbf{p} / d \xi$ can be 
integrated analytically and provides
\begin{equation}
\label{eq:vrayint1}
\mathbf{p} = \mathbf{p_0} - \mathbf{g} \xi\;.
\end{equation}
From the eikonal equation and considering 
$\mathbf{p_0} \cdot \mathbf{p_0} = \tilde{v}_0^{-2}$ and $g = |\mathbf{g}| = \sqrt{g_z^2 + g_x^2}$, we have
\begin{equation}
\label{eq:vrayint2}
v = \frac{1}{\sqrt{\mathbf{p} \cdot \mathbf{p}}} = 
\left( \tilde{v}_0^{-2} - 2 \mathbf{p_0} \cdot \mathbf{g} \xi + g^2 \xi^2 \right)^{-\frac{1}{2}}\;.
\end{equation}
Integrating equation \ref{eq:vrayint2} over $\xi$ gives
\begin{equation}
\label{eq:vrayint3}
t = \frac{1}{g} \mathrm{arccosh} 
\left( 1 + \frac{g^2 \xi^2}{\tilde{v}_0^{-2} + v^{-1} \tilde{v}_0^{-1} 
- \mathbf{p_0} \cdot \mathbf{g} \xi} \right)\;.
\end{equation}
Meanwhile, combining equations \ref{eq:constg} and \ref{eq:vraysys}, we find 
$d \mathbf{p} / d \xi \cdot (\mathbf{x} - \mathbf{x_0}) + d \mathbf{x} / d \xi \cdot \mathbf{p} = \tilde{v}_0$, 
i.e., $\mathbf{p} \cdot (\mathbf{x} - \mathbf{x_0}) = \tilde{v}_0 \xi$. Suppose
\begin{equation}
\label{eq:alphabeta1}
\mathbf{x} - \mathbf{x_0} = \alpha \mathbf{p_0} + \beta \mathbf{g}
\end{equation}
then
\begin{equation}
\label{eq:alphabeta2}
\left\{ \begin{array}{lcl}
\mathbf{g} \cdot (\mathbf{x} - \mathbf{x_0}) & = & \alpha \mathbf{p_0} \cdot \mathbf{g} + \beta g^2 = v - \tilde{v}_0\;, \\
\mathbf{p_0} \cdot (\mathbf{x} - \mathbf{x_0}) & = & \alpha \tilde{v}_0^{-2} + \beta \mathbf{p_0} \cdot \mathbf{g}
= \tilde{v}_0 \xi + (v - \tilde{v}_0) \xi = v \xi\;.
\end{array} \right.
\end{equation}
Solving equation \ref{eq:alphabeta2} provides $\alpha (\xi,v)$ and $\beta (\xi,v)$, which after substituting into 
equation \ref{eq:alphabeta1} leads to
\begin{equation}
\label{eq:vrayint4}
\mathbf{x} = \mathbf{x_0} + \frac{(v - \tilde{v}_0) \left[\mathbf{g} 
- (\mathbf{p_0} \cdot \mathbf{g}) \tilde{v}_0^2 \mathbf{p_0} \right]
+ v \tilde{v}_0^2 \left[ g^2 \mathbf{p_0} - (\mathbf{p_0} \cdot \mathbf{g}) \mathbf{g} \right] \xi}
{g^2 - (\mathbf{p_0} \cdot \mathbf{g})^2 \tilde{v}_0^2}\;.
\end{equation}
Note equation \ref{eq:viray} states $\mathbf{p_0} \cdot \mathbf{g} = g_z \tilde{v}_0^{-1}$ and thus equations 
\ref{eq:vrayint1}, \ref{eq:vrayint3} and \ref{eq:vrayint4} can be further simplified.

To connect depth- and time-domain attributes, we first invert equation \ref{eq:vrayint3} such that $\xi$ is 
expressed by $t_0$ and $x_0$
\begin{equation}
\label{eq:viray2}
\xi (t_0,x_0) = \frac{g_z (1 - \cosh (|g t_0|))
+ g \sinh (g t_0)}{g^2 \tilde{v}_0^2}\;.
\end{equation}
Next, we insert equations \ref{eq:vrayint2} and \ref{eq:viray2} into \ref{eq:vrayint4} in order to change its 
parameterization from $(\xi,v)$ to $(t_0,x_0)$. The result is written for the $z$ and $x$ components of 
$\mathbf{x}$ separately, as follows:
\begin{equation}
\label{eq:viray3}
x (t_0,x_0) = x_0 + \frac{\tilde{v}_0 g_x (1 - \cosh (g t_0))}{g (g \cosh (g t_0) - g_z \sinh (g t_0))}\;,
\end{equation}
\begin{equation}
\label{eq:viray4}
z (t_0,x_0) = \frac{\tilde{v}_0 \left[ g_z (1 - \cosh (g t_0)) + g \sinh (g t_0) \right]}
{g (g \cosh (g t_0) - g_z \sinh (g t_0))}\;.
\end{equation}
Inverting equations \ref{eq:viray3} and \ref{eq:viray4} results in
\begin{equation}
\label{eq:vanalx}
x_0 (z,x) = x + \frac{\sqrt{(v_0 + g_x x)^2 + g_x^2 z^2} - (v_0 + g_x x)}{g_x}\;,
\end{equation}
\begin{equation}
\label{eq:vanalt}
t_0 (z,x) = \frac{1}{g} \mathrm{arccosh} \left\{ \frac{g^2 \left[ \sqrt{(v_0 + g_x x)^2 + g_x^2 z^2}
+ g_z z \right] - v g_z^2}{v g_x^2} \right\}\;.
\end{equation}

In the last step, we derive the analytical formula for the Dix velocity. Note that from equation 
\ref{eq:vanalx} $|\nabla x_0|^2 = 1$, i.e., there is no geometrical spreading. The image rays are circles parallel 
to each other. Therefore according to equation \ref{eq:cameron} $v_d = v$ and is found by combining equations 
\ref{eq:vrayint2} and \ref{eq:viray2}
\begin{equation}
\label{eq:vanalvd}
v_d (t_0,x_0) = \frac{(v_0 + g_x x_0) g}{g \cosh (g t_0) - g_z \sinh (g t_0)}\;.
\end{equation}
The time-migration velocity $v_m$, on the other hand, is
\begin{equation}
\label{eq:vanalvm}
v_m (t_0,x_0) = \frac{(v_0 + g_x x_0)^2}{t_0 ( g \coth (g t_0) - g_z )}\;.
\end{equation}

\appendix
\section{Appendix D: Analytical expressions for the constant horizontal slowness-squared gradient medium}
In this appendix, we study the following medium:
\begin{equation}
\label{eq:s2constg}
w (z,x) = w_0 - 2 q_x x = w_0 - 2 \mathbf{q} \cdot \mathbf{x}\;,
\end{equation}
where $\mathbf{q} = [0,q_x]^T$.

Similarly to the constant velocity gradient medium, it is convenient to write down the ray-tracing system in 
the form \cite[]{cerveny}
\begin{equation}
\label{eq:s2raysys}
\left\{ \begin{array}{lcl}
d \mathbf{x} / d \sigma & = & \mathbf{p}\;, \\
d \mathbf{p} / d \sigma & = & \nabla w / 2\;, \\
d t / d \sigma & = & \nabla t \cdot d \mathbf{x} / d \sigma = \mathbf{p} \cdot \mathbf{p}\;.
\end{array} \right.
\end{equation}
Given equation \ref{eq:s2constg}, $\nabla w = -2 \mathbf{q}$ and thus $d \mathbf{p} / d \sigma = - \mathbf{q}$. 
After integration over $\sigma$, equation \ref{eq:s2raysys} becomes
\begin{equation}
\label{eq:s2rayint}
\left\{ \begin{array}{lcl}
\mathbf{x} & = & \mathbf{x_0} + \mathbf{p_0} \sigma - \mathbf{q} \sigma^2/2\;, \\
\mathbf{p} & = & \mathbf{p_0} - \mathbf{q} \sigma\;, \\
t & = & | \mathbf{p_0} |^2 \sigma - \mathbf{p_0} \cdot \mathbf{q} \sigma^2 + | \mathbf{q} |^2 \sigma^3/3\;.
\end{array} \right.
\end{equation}

For a particular image ray
\begin{equation}
\label{eq:s2iray}
\left\{ \begin{array}{lcl}
\mathbf{x_0} & = & [0, x_0]^T\;, \\
\mathbf{p_0} & = & [\sqrt{w_0 - 2 q_x x_0}, 0]^T\;, \\
t_0 & = & t\;,
\end{array} \right.
\end{equation}
the equation for $\mathbf{x}$ in \ref{eq:s2rayint} simplifies to
\begin{equation}
\label{eq:s2iray2}
\left\{ \begin{array}{lcl}
x = x_0 - q_x \sigma^2/2\;, \\
z = \sigma \sqrt{w_0 - 2 q_x x_0}\;.
\end{array} \right.
\end{equation}
Solving equation \ref{eq:s2iray2} for $\sigma$ as a function of $z$ and $x$
\begin{equation}
\label{eq:s2analsig}
\sigma (z,x) = \left[ \frac{(w_0 - 2 q_x x) 
- \sqrt{(w_0 - 2 q_x x)^2 - 4 q_x^2 z^2}}{2 q_x^2} 
\right]^{\frac{1}{2}}\;.
\end{equation}
Combining equations \ref{eq:s2rayint} through \ref{eq:s2analsig}, we find
\begin{eqnarray}
\label{eq:s2analx}
x_0 (z,x) & = & x + \frac{1}{2} q_x \sigma^2 \nonumber \\
& = & \frac{2 w_0 x + q_x z^2}{w_0 + 2 q_x x + \sqrt{(w_0 - 2 q_x x)^2 - 4 q_x^2 z^2}}\;,
\end{eqnarray}
\begin{eqnarray}
\label{eq:s2analt}
t_0 (z,x) & = & (w_0 - 2 q_x x_0) \sigma + \frac{1}{3} q_x^2 \sigma^3 \nonumber \\
& = & \frac{\sqrt{2} z \left[ 2 w_0 - 4 q_x x + \sqrt{(w_0 - 2 q_x x)^2 - 4 q_x^2 z^2} 
\right]}{3 \left[ w_0 - 2 q_x x + \sqrt{(w_0 - 2 q_x x)^2 - 4 q_x^2 z^2} \right]^{\frac{1}{2}}}\;.
\end{eqnarray}

According to equation \ref{eq:Q}, \ref{eq:s2analx} can give rise to the geometrical spreading:
\begin{equation}
\label{eq:s2analQ}
Q^2 (z,x) =
\frac{2 [(w_0 - 2 q_x x)^2 - 4 q_x^2 z^2]}
{(w_0 - 2 q_x x) \left[ w_0 - 2 q_x x + \sqrt{(w_0 - 2 q_x x)^2 - 4 q_x^2 z^2} \right]}\;.
\end{equation}
It is more convenient to express equations \ref{eq:s2constg} and \ref{eq:s2analQ} in $\sigma$ and $x_0$ instead 
of directly in $t_0$ and $x_0$:
\begin{equation}
\label{eq:s2constg2}
w (\sigma,x_0) = w_0 - 2 q_x x_0 + q_x^2 \sigma^2\;,
\end{equation}
\begin{equation}
\label{eq:s2analQ2}
Q^2 (\sigma,x_0) = 1 +
q_x^2 \sigma^2 \left( \frac{1}{w_0 - 2 q_x x_0} - \frac{4}{w_0 - 2 q_x x_0 + q_x^2 \sigma^2} \right)\;,
\end{equation}
where we must resolve $\sigma = \sigma (t_0,x_0)$. This is done by revisiting equation \ref{eq:s2analt}. For 
given $t_0$ and $x_0$, $\sigma$ is the root of a depressed cubic function of the following form:
\begin{equation}
\label{eq:depcubic}
\sigma^3 + \frac{3 (w_0 - 2 q_x x_0)}{q_x^2} \sigma - \frac{3}{q_x^2} t_0 = 0\;.
\end{equation}
After some algebraic manipulations, we find
\begin{eqnarray}
\label{eq:rootdepcubic}
\sigma (t_0,x_0) & = & 
\left[ \frac{3 t_0 + \sqrt{9 t_0^2 + 4 (w_0 - 2 q_x x_0)^3 / q_x^2}}{2 q_x^2} \right]^{\frac{1}{3}} \nonumber \\
& - & \frac{w_0 - 2 q_x x_0}{q_x} 
\left[ \frac{2}{3 q_x t_0 + \sqrt{9 q_x^2 t_0^2 + 4 (w_0 - 2 q_x x_0)^3}} \right]^{\frac{1}{3}}\;.
\end{eqnarray}

Finally, inserting equations \ref{eq:s2constg2} and \ref{eq:s2analQ2} into equation \ref{eq:cameron} results 
in the Dix velocity:
\begin{equation}
\label{eq:s2analvd}
v_d (t_0,x_0) = \frac{\sqrt{w_0 - 2 q_x x_0}}{w_0 - 2 q_x x_0 - q_x^2 \sigma^2}\;.
\end{equation}
\bibliographystyle{seg}
\bibliography{t2df}
