\published{Geophysics, 75, no. 3, V25-V38, (2010)}
\title{Seislet transform and seislet frame}
\author{Sergey Fomel and Yang Liu}

\ms{GEO-2009}

\lefthead{Fomel and Liu}
\righthead{Seislet transform and frame}

\address{Bureau of Economic Geology \\
John A. and Katherine G. Jackson School of Geosciences \\
The University of Texas at Austin \\
University Station, Box X \\
Austin, TX 78713-8924}

\maketitle

\begin{abstract}
  We introduce a digital wavelet-like transform, which is tailored
  specifically for representing seismic data. The transform provides a
  multiscale orthogonal basis with basis functions aligned along
  seismic events in the input data. It is defined with the help of the
  wavelet lifting scheme combined with local plane-wave
  destruction. In the 1-D case, the seislet transform is designed to
  follow locally sinusoidal components. In the 2-D case, it is
  designed to follow local plane wave components with smoothly
  variable slopes. If more than one component is present, the
  transform turns into an overcomplete representation or a {tight}
  frame.  In these terms, the classic digital wavelet transform is
  simply a seislet transform for a zero frequency (in 1-D) or zero
  slope (in 2-D).

  The main objective of the new transform is an {effective} seismic
  data compression for designing efficient data analysis
  algorithms. Traditional signal processing tasks such as noise
  {attenuation} and trace interpolation \old{are} \new{become} simply
  defined in the seislet domain. When applied in the offset direction
  on common midpoint or common image point gathers, the seislet
  transform finds an additional application in optimal stacking of
  seismic records.
\end{abstract}

\section{Introduction}

Wavelet transforms have found many applications in science and engineering 
\cite[]{mallat}, including geophysics \cite[]{SEG-1994-1465,dessing,wapenaar,hesam}. 
The power of wavelet transforms, in comparison with the classic Fourier 
transform, lies in their ability to represent non-stationary signals.  As a 
result, wavelets can provide a compact basis for non-stationary 
data decomposition. Having a compact basis is useful both for data compression 
and for designing efficient numerical algorithms.

A number of wavelet-like transforms that explore directional characteristics 
of images have been proposed in the image analysis literature \cite[]{beyond}. Among 
those transforms are bandelets \cite[]{bandelet}, % beamlets,
contourlets \cite[]{contourlet}, curvelets \cite[]{curvelet}, directionlets 
\cite[]{directionlet}, shearlets \cite[]{shearlet}, etc. Unlike isotropic 
wavelets, di\-rec\-ti\-onal transforms attempt to design basis
functions so that they appear elongated anisotropically along 2-D
curves or 3-D surfaces, which might be characteristic for an
image. Therefore, these transforms achieve better accuracy and better
data compression in representing non-stationary images with curved
edges.  Curvelets are particularly appropriate for seismic data
because they provide {a} provably optimal decomposition of
wave-propagation operators \cite[]{candes}.  Application of the
curvelet transform to seismic imaging and seismic data analysis
\old{is} \new{has been} an area of active research
\cite[]{herrmann,douma,chauris,herrmann08}.

Although {the} wavelet theory originated in seismic data analysis
\cite[]{SEG-1981-S15.1}, none of the known wavelet-like transforms were 
designed specifically for seismic data. Even though some of the
transforms are applicable for representing seismic data, their
original design was motivated by different kinds of data, such as
piecewise-smooth images. In this paper, we investigate the possibility
{of designing} a transform tailored specifically
for seismic data. In analogy with the previous naming games, we call
such a transform \emph{the seislet transform} \cite[]{seislet}.

In seismic data analysis, it is common to represent signals as sums of
sinusoids (in 1-D) or plane waves (in 2-D) with the help of the
digital Fourier transform (DFT). \old{A class of} \new{Certain}
methods for seismic data regularization, such as the anti-leakage
Fourier transform \cite[]{antileak}, the Fourier reconstruction method
\cite[]{frsi1,frsi2,frsi3}, {or POCS \cite[]{pocs}} rely on the ability to represent signals
sparsely in the transform domain. The digital wavelet transform (DWT)
is often preferred to the Fourier transform for characterizing digital
images, because of its ability to localize events in both time and
frequency domains \cite[]{ripples,mallat}.  However, DWT may not be
optimal for describing data that consist of individual sinusoids or
plane waves. It is for {those} kinds of data that the seislet
transform attempts to achieve an optimally compact representation.

The approach taken in this paper follows the general recipe for
digital wavelet transform construction known as the \emph{lifting
scheme} \cite[]{lifting}. The lifting scheme provides a convenient and
efficient construction for digital wavelet transforms of different
kinds. The key ingredients of this scheme are a prediction operator
and an update operator defined at different digital scales. The goal
of the prediction operator is to predict regular parts of the input
data so that they could be subtracted from the analysis. The goal of
the update operator is to carry essential parts of the input data to
the next analysis scale. Conventional wavelet transforms use
prediction and update operators designed for characterizing locally
smooth images.  In this paper, we show how designing prediction and
update tailored for seismic data can improve the effectiveness of the
transform in seismic applications. In 1-D, our prediction and update
operators focus on predicting sinusoidal signals with chosen
frequencies.  In 2-D and 3-D, we use predictions along locally
dominant event slopes found by the method of plane-wave destruction
\cite[]{pvi,GEO67-06-19461960}. \new{One can extend the idea of the seislet transform 
further by changing the definition of prediction and update operators
in the lifting scheme \cite[]{seg2009}.}

The seislet transform decomposes a seismic image into an orthogonal
basis which is analogous to the wavelet basis but aligned along
dominant seismic events. In 1-D, the classic wavelet transform is
equivalent to the seislet transform with a zero frequency. In 2-D,
{the} wavelet transform in the horizontal direction is equivalent to the
seislet transform with a zero slope. When more than one frequency or
more than one slope field are employed for analysis, the seislet
transform turns into an overcomplete representation or a {tight}
frame.

The paper is organized as follows. We start by reviewing the digital
wavelet transform and the lifting scheme. Next, we modify the lifting
scheme to define 1-D and 2-D seislet
transforms. Finally, we generalize the transform construction to a
frame. We illustrate applications of both the seislet transform and
the seislet frame with synthetic and field data examples.

%Other geophysical predictions may
%lead to other kinds of wavelet-like transforms. 

%I show that
%traditional signal processing operations such as noise removal and
%trace interpolation become simply defined in the seislet domain.  The
%seislet transform introduced in this paper is particularly meanigful
%when applied in the offset direction on common midpoint or common
%image point gathers. In this case, the first wavelet coefficient is
%simply the seismic stack but computed in the most effective manner by
%recursive partial stacking of neighboring traces. The seislet stack
%avoids usual problems with wavelet stretching and nonhyperbolic
%moveouts. All other gather attributes including multiple reflections
%and and amplitude variation with offset appear in the higher order
%seislet coefficients. I use a North Sea field data example to
%demonstrate this effect.

\section{Lifting scheme for wavelet transforms}


In order to define the new transform, we follow the general recipe for
digital wavelet transforms provided by \cite{athome}. In the most
general terms, the lifting scheme \cite[]{lifting} \old{is} \new{can
be} defined as follows:
\begin{enumerate}
\item Organize the input data as a sequence of records.
\item Break the data into even and odd components $\mathbf{e}$ and
  $\mathbf{o}$.
\item Find a residual difference $\mathbf{r}$ between the odd
  component and its prediction from the even component:
  \begin{equation}
    \label{eq:c}
    \mathbf{r}  = \mathbf{o} - \mathbf{P[e]}\;,
  \end{equation}
  where $\mathbf{P}$ is a \emph{prediction} operator.
\item Find a coarse approximation $\mathbf{c}$ of the data by updating 
      the even component
\begin{equation}
    \label{eq:r}
    \mathbf{c}  = \mathbf{e} + \mathbf{U[r]}\;,
  \end{equation}
  where $\mathbf{U}$ is an \emph{update} operator.
\item The coarse approximation $\mathbf{c}$ becomes the new data, 
and the sequence of steps is repeated at the next scale	level.
\end{enumerate}
A digital wavelet transform consists of data approximation at the
coarsest level and residuals from all the levels. The key in designing
an effective transform is making sure that the prediction
operator~$\mathbf{P}$ leaves small residuals while the update
operator~$\mathbf{U}$ preserves essential features of the original
data while promoting them to the next scale level.  For example, one
can obtain the classic Haar wavelet by defining the prediction
operator as a simple shift from the nearest sample:
\begin{equation}
  \label{eq:p1}
  \mathbf{P[e]}_k = \mathbf{e}_{k}\;,
\end{equation}
with the update operator designed to preserve the DC (zero frequency) component
of the signal. Alternatively, the $(2,2)$ Cohen-Daubechies-Feauveau
biorthogonal wavelets \cite[]{Cohen92} are constructed by making
the prediction operator a linear interpolation between two neighboring
samples,
\begin{equation}
  \label{eq:p}
  \mathbf{P[e]}_k = \left(\mathbf{e}_{k-1} + \mathbf{e}_{k}\right)/2\;,
\end{equation}
and by constructing the update operator to preserve the running
average of the signal \cite[]{athome}, as follows:
\begin{equation}
  \label{eq:u}
  \mathbf{U[r]}_k = \left(\mathbf{r}_{k-1} + \mathbf{r}_{k}\right)/4\;.
\end{equation}

The digital wavelet transform is an efficient operation. Assuming that
the prediction and update operation take a constant cost per record,
the number of operations at the finest scale is proportional to the
total number of records $N$, the next scale computation takes
$O(N/2)$, etc. so that the total number of operations is proportional
to $N+N/2+N/4+\ldots + 2 =2\,(N-1)$, which is smaller than the
$O(N\,\log{N})$ cost of the Fast Fourier Transform.

The digital wavelet transform is also easily invertible. Reversing the lifting 
scheme operations provides the inverse transform algorithm, as follows:
\begin{enumerate}
\item Start with the coarsest scale data representation $\mathbf{c}$
      and the coarsest scale residual $\mathbf{r}$.
\item Reconstruct the even component $\mathbf{e}$ by reversing the 
      operation in equation~\ref{eq:r}, as follows:
\begin{equation}
    \label{eq:e}
    \mathbf{e} = \mathbf{c} - \mathbf{U[r]}\;,
  \end{equation}
\item Reconstruct the odd component 
  $\mathbf{o}$ by reversing the operation in equation~\ref{eq:c}, as follows:
  \begin{equation}
    \label{eq:o}
    \mathbf{o} = \mathbf{r}  + \mathbf{P[e]}\;,
  \end{equation} 
\item Combine the odd and even components to generate the data at 
      the previous scale level and repeat the sequence of steps.
\end{enumerate}

Figure~\ref{fig:lena,linear2} shows a classic benchmark image from the
image analysis literature and its digital wavelet transform using 2-D
biorthogonal wavelets. Thanks to the general smoothness of the
``Lena'' image, the residual differences from equation~\ref{eq:r}
(stored as wavelet coefficients at different scales) have a small
dynamic range, which enables an effective compression of the
\old{input} image \new{in the transform domain}. Wavelet compression
algorithms are widely used in practice for compression of natural
images. As for compression of seismic data, the classic DWT may not be
optimal, because it does not take into account the specific nature of
seismic data patterns. In the next section, we turn the wavelet
transform into the seislet transform, which is tailored for
representing seismic data.

\inputdir{lena}

\multiplot{2}{lena,linear2}{width=0.5\columnwidth}{Benchmark ``Lena'' 
image from image analysis literature (a) and its 2-D digital wavelet 
transform using bi-orthogonal wavelets (b).}


%Figure~\ref{fig:wrec1,wrec5} shows
%reconstructions of the original image using only the top 1\% and 5\%
%of significant wavelet coefficients and soft thresholding
%\cite[]{donoho}. The most essential features of the original image are
%preserved in the reconstruction.
%
%\multiplot{2}{wrec1,wrec5}{width=0.6\columnwidth}{Inverse digital
%  wavelet transform using only 1\% (a) and 5\% (b) of significant
%  wavelet coefficients from Figure~\ref{fig:linear2}.}
%
%By spanning different scales, the basis functions generated by the
%wavelet transform provide signal localization in both initial and
%Fourier transform domains. The power of multiscale characterization
%combined with the divide-and-conquer algorithmic strategy make the
%digital wavelet transform an effective image analysis tool.
%
%Figure~\ref{fig:ilinear,flinear} shows
%individual wavelets and their Fourier spectra. Spanning a set of
%scales in the initial domain allows the wavelet transform to also span
%a set of spectral scales.

%\nopagebreak
%
%\multiplot{2}{ilinear,flinear}{width=0.6\columnwidth}{Wavelet basis
%  functions for linear bi-orthogonal wavelets (a) and their Fourier
%  spectra (b).}
%
%\nopagebreak

\section{From wavelets to seislets}

We adopt the general idea of the lifting scheme to transforming
seismic data. The key idea of the seislet transform \cite[]{seislet}
is recognizing that
\begin{itemize}
\item seismic data can be organized as collections of traces or
  records;
\item prediction of one seismic trace or record from the other and
  update of records on the next scale should follow features
  characteristic for seismic data.
\end{itemize}

\subsection{1-D seislet transform}

The prediction and update operators employed in the lifting scheme can
be understood as digital filters. In the $Z$-transform notation, the
Haar prediction filter from equation~\ref{eq:p1} is
\begin{equation}
  \label{eq:f1}
  P(Z) = Z
\end{equation}
(shifting each sample by one), and the linear interpolation filter
from equation~\ref{eq:p} is
\begin{equation}
  \label{eq:f}
  P(Z) = 1/2\,(1/Z+Z)\;.
\end{equation}
These predictions are appropriate for smooth signals but may not be
optimal for a sinusoidal signal. \old{On the other hand} \new{In
comparison}, the prediction
\begin{equation}
  \label{eq:s1}
  P(Z) = Z/Z_0\;,
\end{equation}
where $Z_0 = e^{i\,\omega_0 \Delta t}$, perfectly characterizes a
sinusoid with $\omega_0$ circular frequency sampled on a $\Delta t$
grid. In other words, if a constant signal ($\omega_0=0$) is perfectly
predicted by shifting each trace to its neighbor, a sinusoidal signal
($\omega_0\ne0$) requires the shift to be modulated by an appropriate
frequency. 
%$1-P(Z)$ is the corresponding prediction-error
%filter. 
Likewise, the linear interpolation in equation~\ref{eq:f} needs to be
replaced by a filter tuned to a particular frequency in order to
predict a sinusoidal signal with that frequency perfectly:
\begin{equation}
  \label{eq:s}
  P(Z) = 1/2\,(Z_0/Z+Z/Z_0)\;.
\end{equation}
The analysis easily extends to higher-order filters.

\subsection{2-D seislet transform}

If we view seismic data as collections of traces, we can predict one
trace from the other by following local {slopes of} seismic
events. Such a prediction is a key operation in the method of
plane-wave destruction
\cite[]{GEO67-06-19461960}. In fact, it is the minimization of
prediction error that provides a criterion for estimating local slopes
\cite[]{pvi}. {For completeness, we include a review of plane-wave
destruction in the appendix.}

The prediction and update operators for a simple seislet transform are
defined by modifying the biorthogonal wavelet construction in
equations~\ref{eq:p}-\ref{eq:u} as follows:
\begin{eqnarray}
  \label{eq:sp}
  \mathbf{P[e]}_k & = & \left(\mathbf{S}_k^{(+)}[\mathbf{e}_{k-1}] + 
  \mathbf{S}_k^{(-)}[\mathbf{e}_{k}]\right)/2 \\
  \label{eq:su}
  \mathbf{U[r]}_k & = & \left(\mathbf{S}_k^{(+)}[\mathbf{r}_{k-1}] + 
    \mathbf{S}_k^{(-)}[\mathbf{r}_{k}]\right)/4\;,
\end{eqnarray}
where $\mathbf{S}_k^{(+)}$ and $\mathbf{S}_k^{(-)}$ are operators that
predict a trace from its left and right neighbors correspondingly by
shifting seismic events according to their local slopes. The
predictions need to operate at different scales, which, in this case,
mean different separation distances between the traces.
Equations~\ref{eq:sp}-\ref{eq:su}, in combination with the forward and
inverse lifting schemes~\ref{eq:c}-\ref{eq:r}
and~\ref{eq:e}-\ref{eq:o}, provide a complete definition of the 2-D
seislet transform.

\inputdir{seis}

\multiplot{2}{sigmoid,sigmoiddip}{width=0.6\columnwidth}{Synthetic
  seismic image (a) and local slopes estimated by plane-wave
  destruction (b).}
\multiplot{2}{sigmoidwvlt,sigmoidseis}{width=0.6\columnwidth}{Wavelet
  transform (a) and seislet transform (b) of the synthetic image from
  Figure~\ref{fig:sigmoid,sigmoiddip}.}
\plot{sigmoidcoef}{width=\textwidth}{Transform coefficients sorted
  from large to small, normalized, and plotted on a decibel
  scale. Solid line: seislet transform. Dashed line: wavelet
  transform.}
\multiplot{2}{sigmoidimpw,sigmoidimps}{width=0.6\columnwidth}{Randomly
  selected representative basis functions for wavelet transform (a)
  and seislet transform (b).}
\multiplot{2}{sigmoidwrec1,sigmoidsrec1}{width=0.6\columnwidth}{Synthetic image
  reconstruction using only 1\% of significant coefficients (a) by
  inverse wavelet transform (b) by inverse seislet transform. Compare with 
  Figure~\ref{fig:sigmoid}.}
%\multiplot{2}{sigmoidwrec3,sigmoidsrec3}{width=0.6\columnwidth}{Image
%  reconstruction using only 3\% of significant coefficients (a) by
%  inverse wavelet transform (b) by inverse seislet transform.}

Figure~\ref{fig:sigmoid} shows a synthetic seismic image from
\cite{bei}. After estimating local slopes from the image by plane-wave
destruction (Figure~\ref{fig:sigmoiddip}), we applied the 2-D seislet
transform described above. The transform is shown in
Figure~\ref{fig:sigmoidseis} and should be compared with the
corresponding wavelet transform in Figure~\ref{fig:sigmoidwvlt}.
Apart from the fault and unconformity regions, where the image is not
predictable by continuous local slopes, the 2-D seislet transform
coefficients are small, which enables an effective compression. In
contrast, the wavelet transform has small residual coefficients at
fine scales but develops large coefficients at coarser scales.
Figure~\ref{fig:sigmoidcoef} shows a comparison between the decay of
coefficients (sorted from large to small) between the wavelet
transform and the seislet transform. A significantly faster decay of
the seislet coefficients is evident.

Effectively, the wavelet transform in this case is equivalent to the
2-D seislet transform with the erroneous zero slope.
Figure~\ref{fig:sigmoidimpw,sigmoidimps} shows example basis functions
for the wavelet and 2-D seislet transform used in this example. If
using only a small number of the most significant coefficients, the
wavelet transform fails to reconstruct the most important features of
the original image, while the 2-D seislet transform achieves an
excellent reconstruction (Figure~\ref{fig:sigmoidwrec1,sigmoidsrec1}).
We use the method of soft thresholding \cite[]{donoho} for selecting
the most significant coefficients.

%\section{Example applications of 2-D seislet transform}
\subsection{Example applications of 2-D seislet transform}

In this section, we discuss some example applications of the 2-D seislet 
transform.

\subsubsection{Denoising and trace interpolation}

\inputdir{gath}

Figure~\ref{fig:gath} shows a common-midpoint gather from a North Sea
dataset. Plane-wave destruction estimates local slopes shown in
Figure~\ref{fig:gathdip} and enables the 2-D seislet transform shown
in Figure~\ref{fig:gathseis}. Small dynamic range of seislet
coefficients implies a good compression
ratio. Figure~\ref{fig:gathsrec5} shows data reconstruction using only
5\% of the significant seislet coefficients.

If we choose the significant coefficients at the coarse scale and
zero out difference coefficients at the finer scales, the inverse
transform will effectively remove incoherent noise from the gather
(Figure~\ref{fig:gathseis,gathsrec5}). Thus, denoising is a naturally
defined operation in the 2-D seislet domain
(Figure~\ref{fig:sign,gathslet16}).

If we extend the seislet \new {transform} domain and interpolate the
smooth local slope to a finer grid, the inverse seislet transform will
accomplish trace interpolation of the input gather
(Figure~\ref{fig:seis4,gath4}). {We extend not simply with zeros but
with small random noise to account for the fact that realistic noise
is unpredictable and therefore exists on different scale levels.}  In
this example, the number of traces is increased by four. Thus, trace
interpolation also turns out to be a natural operation when viewed
from the 2-D seislet domain.

\multiplot{2}{gath,gathdip}{width=0.6\columnwidth}{Common-midpoint
  gather (a) and local slopes estimated by plane-wave destruction
  (b).}
\multiplot{2}{gathseis,gathsrec5}{width=0.6\columnwidth}{Seislet
  transform of the input gather (a) and (a) and data reconstruction
  using only 5\% of significant seislet coefficients (b).}
\multiplot{2}{sign,gathslet16}{width=0.6\columnwidth}{Zeroing seislet
  difference coefficients at fine scales (a) enables effective
  denoising of the reconstructed data (b).}
\multiplot{2}{seis4,gath4}{width=0.6\columnwidth}{Extending seislet
  transform with random noise (a) enables trace interpolation in the
  reconstructed data. The interpolated section (b) has 4 times more
  traces than the original shown in Figure~\ref{fig:gath}.}

\subsubsection{Seislet stack}

The seislet transform acquires a special meaning when applied in the
offset direction on common midpoint or common image point
gathers. According to the lifting construction, the zero-order seislet
coefficient is nothing more than seismic stack computed in a recursive
manner by successive partial stacking of neighboring traces. As a
consequence, seislet stack avoids the problem of ``NMO stretch''
associated with usual stacking \cite[]{nmo} as well as the problem of
nonhyperbolic moveouts \cite{grechka}. All other gather attributes
including multiple reflections and amplitude variation with offset
appear in the higher order seislet coefficients.
Figure~\ref{fig:sstack} shows a comparison between the conventional
normal moveout stack and the seislet stack. The higher resolution of
the seislet stack is clearly visible. Figure~\ref{fig:nmo,snmo}
compares the common-midpoint gather after conventional normal moveout
correction and an effective seislet moveout computed by separating
contributions from individual traces to the seislet stack.
  
\inputdir{gath} 
\plot{sstack}{width=\columnwidth}{Left: conventional normal-moveout
 stack. Right: seislet stack.}
\multiplot{2}{nmo,snmo}{width=0.6\columnwidth}{Input gather after 
 normal moveout correction (a) and effective seislet moveout (b).}

\section{From transform to frame}

The 1-D and 2-D transforms, defined in the previous sections, are appropriate 
for analyzing signals, which have a single dominant sinusoid or plane-wave 
component. In practice, it is common to analyze signals composed of multiple 
sinusoids (in 1-D) or plane waves (in 2-D). If a range of frequencies or 
plane-wave slopes is chosen, and the appropriate transform is constructed for 
each of them, all the transform domains taken together will constitute an 
overcomplete representation or a frame \cite[]{mallat}.

{Mathematically, if $\mathbf{F}_n$ is the orthonormal seislet transform for $n$-th frequency or plane wave, 
then, for any data vector $\mathbf{d}$,
\begin{equation}
\label{eq:frame}
\sum\limits_{n=1}^N \|\mathbf{F}_n\,\mathbf{d}\|^2 = \sum\limits_{n=1}^N \mathbf{d}^T\,\mathbf{F}_n^T\,\mathbf{F}_n\,\mathbf{d} =
\sum\limits_{n=1}^N \|\mathbf{d}\|^2 = N\,\|\mathbf{d}\|^2\;,
\end{equation}
which means that all transforms taken together constitute a \emph{tight frame} with constant $N$.}

For example, in the 1-D case, one can find appropriate frequencies by
autoregressive spectral analysis
\cite[]{Burg75,marple}. We define the algorithm for the \emph{1-D
  seislet frame} as follows:
\begin{enumerate}
\item Select a range of coefficients $Z_1, Z_2, \ldots, Z_k$. When
  using autoregressive spectral analysis, these coefficients are
  simply the roots of the prediction-error filter. Alternatively, they
  can be defined from an appropriate range of frequencies 
  $\omega_1, \omega_2, \ldots, \omega_k$.
\item For each of the coefficients, perform the 1-D
  seislet transform. 
\end{enumerate}

Because of its over-completeness, a frame representation for a given
signal is not unique. In order to assure that different frequency
components do not leak into other parts of the frame, it is
advantageous to employ sparseness-promoting inversion. We adopt a
nonlinear shaping regularization scheme \cite[]{inv}, analogous to the
sparse inversion method of \cite{daube}, and define sparse
decomposition as  an iterative process 
\begin{eqnarray}
  \label{eq:shape4}
  \widehat{\mathbf{f}}_{k+1} & = & \mathbf{S}[\mathbf{F}\,\mathbf{d}+(\mathbf{I}-\mathbf{F}\,
   \mathbf{F}')\,\widehat{\mathbf{f}}_{k}]\;, \\
   \mathbf{f}_{k+1} & = & \mathbf{f}_k + \mathbf{F}\,\mathbf{d}- \mathbf{F}\,\mathbf{F}' \widehat{\mathbf{f}}_{k+1}\;,
   \label{eq:bregman} 
\end{eqnarray}
where $\mathbf{f}_k$ is the seislet frame at $k$-th iteration,
\new{$\widehat{\mathbf{f}}_k$ is an auxiliary quantity,}
$\mathbf{d}$ is input data, 
$\mathbf{I}$ is the identity operator,
$\mathbf{F}$ and $\mathbf{F}'$ are frame construction and deconstruction 
operators 
\begin{eqnarray*}
  \mathbf{F} & \equiv & 
\left[\begin{array}{cccc}\mathbf{F}_1 &\mathbf{F}_2 & 
    \cdots &\mathbf{F}_k\end{array}\right]^T\;, \\
\mathbf{F}' & \equiv &  
\left[\begin{array}{cccc}\mathbf{F}_1^{-1} &\mathbf{F}_2^{-1} & 
    \cdots &\mathbf{F}_k^{-1}\end{array}\right]\;,
\end{eqnarray*}
where $\mathbf{F}_j$ is the seislet transform for an individual frequency, and 
$\mathbf{S}$ is a nonlinear shaping operator, such as soft thresholding 
\cite[]{donoho}. The iteration~\ref{eq:shape4}-\ref{eq:bregman} starts 
with $\mathbf{f}_0=\mathbf{0}$ and \old{$\mathbf{d}_0=\mathbf{d}$}
\new{$\widehat{\mathbf{f}}_0=\mathbf{F}\,\mathbf{d}$} and is related
to the linearized Bregman iteration \cite[]{Osher05,Yin08}.  We find
that a small number of iterations is usually sufficient for
convergence and achieving both model sparseness and data recovery.

\subsection{1-D data analysis with 1-D seislet frame}

We use a simple synthetic test to verify the compression
 effectiveness of 1-D seislet frame. A test signal mixing two
 sinusoids with different frequencies and some random noise is
 displayed in Figure~\ref{fig:tatrace}. We use a prediction-error
 filter to detect the signal frequencies and to design the
 corresponding seislet frame.  
%The lifting scheme with the CDF 9/7
% biorthogonal wavelets is used as the basis. 
The result is shown in
 Figure~\ref{fig:taft}. The 1-D seislet frame algorithm with shaping
 regularization compresses the sinusoidal signal into two nearly
 perfect impulses with some dispersive random noise. For comparison,
 we also apply DFT and DWT to transform the signal
 (Figures~\ref{fig:fourier} and ~\ref{fig:tdwt}). In the Fourier
 transform domain, the signal appears as two impulses corresponding to
 the chosen frequency components. The resolution is not perfect
 because of spectral leakage caused by non-periodic input data.
%of the random noise and because the frequencies do not fall
% exactly on the Fourier grid. 
In the wavelet domain, the transform coefficients are not compressed
 well. For further comparison, we plot the coefficients in the three
 different transform domains, sorted from large to small, on a decibel
 scale (Figure~\ref{fig:tlog}). The significantly faster rate of
 coefficient decay shows the superiority of {the} 1-D seislet
 frame in compressing sinusoidal signals.

\inputdir{sin2}
\multiplot{4}{tatrace,taft,tdwt,fourier}{width=0.45\columnwidth}{Mixed 
  sinusoidal signal (a), 1-D seislet frame (b), 1-D wavelet transform (c), 
  and 1-D Fourier transform (d).}  
\plot{tlog}{width=\columnwidth}{Compression comparison between digital Fourier 
  transform, digital wavelet transform, and 1-D seislet frame. Transform 
  coefficients are sorted from large to small, normalized, and plotted on a 
  decibel scale.}


 \subsection{2-D data analysis with 1-D seislet frame} 
 
 To analyze 2-D data, one can apply 1-D seislet frame in the distance
 direction after the Fourier transform in time (the $F$-$X$
 domain). In this case, different frame frequencies correspond to
 different plane-wave slopes \cite[]{Canales84}. We use a simple
 plane-wave synthetic model to verify this observation
 (Figure~\ref{fig:plane}). The $F$-$X$ plane is shown in
 Figure~\ref{fig:fft}. We find a prediction-error-filter (PEF) in each
 frequency slice and detect its roots to select appropriate spatial
 frequencies. We use Burg's algorithm for PEF estimation
 \cite[]{Burg75,Claerbout76} and an eigenvalue-based algorithm for
 root finding \cite[]{roots}. The seislet coefficients and the
 corresponding recovered plane-wave components are shown in
 Figure~\ref{fig:plane1,plane2,plane3}. Similarly to the 1-D example,
 information from different plane-waves gets \new{strongly} compressed
 in the transform domain.
\inputdir{plane}
\multiplot{2}{plane,fft}{width=0.6\columnwidth}{Synthetic plane-wave data
  (a) and corresponding Fourier transform along the time direction (b).}
\multiplot{3}{plane1,plane2,plane3}{width=0.7\columnwidth}{Seislet
  coefficients (left) and corresponding recovered plane-wave
  components (right) for three different parts of the 1-D
  seislet frame in the $F$-$X$ domain.}

\subsection{2-D data analysis with 2-D seislet frame} 

\begin{comment}
After adding some random noise into Figure~\ref{fig:plane}, we tested
2-D seislet frame for seismic data analysis 
(Figure~\ref{fig:nplane}). A direct application is 
$\tau$-$p$ transform. Figure~\ref{fig:cdipimps} shows 
example basis functions for 2-D seislet frame, the tendency of frame basis 
follows different plane-wave directions. A seismic slope field is shown in 
Figure~\ref{fig:cdips}. We use the slope field to scan
the data, 2-D seislet transform can compress seismic events corresponding to 
each slope. Sharpening operator enforces large seislet 
coefficients into a small dynamic range  
(Figure~\ref{fig:cdiplet}) and Bregman iteration 
guarantee 2-D inverse seislet frame return data back to original domain. 
Thus, 2-D seislet frame can be used as $\tau$-$p$ 
transform operator.

%\inputdir{diplet}
%\multiplot{4}{nplane,cdipimps,cdips,cdiplet}{width=0.45\textwidth}{Plane-wave
%  model (a), randomly selected representative basis functions for 2-D seislet
%  frame (b), dip field (c), and frame coefficients (d).}


\section{Example applications of seislet frame}

%\begin{comment}
\subsection{Spectral decomposition with 1-D seislet frame}

%\inputdir{boon}
%\multiplot{2}{boon,bspec}{width=0.45\columnwidth}{A section of Boonsville 
%  3-D dataset used for testing spectral decomposition (a) and average 
%  spectrum of data (b).}
%\multiplot{3}{boon3,flet3,spec3}{width=0.45\textwidth}{First
%  spectral component from the seislet frame.  Reconstructed data (a),
%  seislet coefficients (b), and average spectrum (c).}
%\multiplot{3}{boon2,flet2,spec2}{width=0.45\textwidth}{Second
%  spectral component from seislet frame.  Reconstructed data (a),
%  seislet coefficients (b), and average spectrum (c).}

The example is spectral decomposition of a 2-D seismic section,
extracted from the Boonsville 3-D dataset \cite[]{boon} and shown in
Figure~\ref{fig:boon}.  
We use a multi-channel PEF to detect the dominant
frequency components in all traces and then apply 1-D seislet frame
to separate different data components. Two of the components are
displayed in Figures~\ref{fig:boon3,flet3,spec3} and
\ref{fig:boon2,flet2,spec2}.  The first component corresponds to
relatively low frequencies, the second component corresponds to
relatively high frequencies (Figures~\ref{fig:boon3} and
\ref{fig:boon2}). The two components exhibit a high compression ratio
in the transform domain (Figures~\ref{fig:flet3} and \ref{fig:flet2})
and correspond to different parts of the spectrum
(Figures~\ref{fig:spec3} and \ref{fig:spec2}, compare with
Figure~\ref{fig:bspec}). Spectral decomposition
has important applications in seismic interpretation
\cite[]{spectral}.  By increasing the PEF size and, correspondingly,
the number of frame frequencies, one could obtain a more detailed
decomposition of the original data.

\subsection{Hyperbolic radon transform with 2-D seislet frame}
\end{comment}

To show an example of 2-D data analysis with 2-D seislet frames, we
use the CMP gather from Figure ~\ref{fig:gath}.  We try two different
choices for selecting a set of dip fields for the frame construction.

First, we define dip fields by scanning different constant dips
(Figure~\ref{fig:cdips}). In this case, the zero-scale coefficients
out of the 2-D seislet frame correspond to the slant-stack (Radon transform) gather
(Figure~\ref{fig:cdiplet}). Figure~\ref{fig:cdipimps} shows
randomly selected example frame functions for the 2-D seislet frame using constant dips

Our second choice is a set of dip fields defined by the hyperbolic shape
of seismic events on the CMP gather:
\begin{equation}
t(x)=\sqrt{t^2_0+\frac{x^2}{v^2}}\;,
\label{eq:hyper}
\end{equation} 
where $t(x)$ is traveltime for reflection at offset $x$,
$t_0$ is the zero-offset traveltime, and $v$ is the root-mean-square
velocity. For a range of constant velocities, the direct relationship between dip and
velocity is given by 
\begin{equation}
p =\frac{dt}{dx}=\frac{x}{v^2 t}\;.
\label{eq:hyperdip}
\end{equation} 
The dip field $p(x,t,v)$ is shown in
Figure~\ref{fig:rrdips}. Analogously to the case of constant dips, the
frame coefficients at the zero scale correspond to the hyperbolic
Radon transform \cite[]{GEO50-12-27272741}, with the primary and
multiple reflections distributed in different velocity ranges
(Figure~\ref{fig:rrdiplet}).  Figure~\ref{fig:rrdipimps} shows
randomly selected frame functions for the 2-D seislet frame with
varying dip fields defined by a range of constant velocities.

%2-D seislet frame can be treated as $\tau$-$p$ transform, we could further 
%use hyperbolic shape ($t=\sqrt{t^2_0+\frac{x^2}{v^2}}$) of seismic events to 
%define time-space-varying dip, where $t$ is travel time for all reflection, 
%$t_0$ is travel time at zero offset, $x$ is offset, and $v$ is 
%medium velocity. We found the direct relationship between dip and velocity 
%as $p=\frac{dt}{dx}=\frac{x}{v^2 t}$, the dip field depend on both time and 
%space distance. A real data is shown in 
%Figure~\ref{fig:gath}, which is a multiple-infested 
%CMP gather from the Mobil AVO dataset \cite[]{CSI00-00-00010213}. 
%Figure~\ref{fig:rrdipimps} shows basis functions for 
%2-D seislet frame, the tendency of frame basis follows hyperbolic directions.
%After setting up a time-space-varying dip field 
%(Figure~\ref{fig:rrdips}), 2-D seislet frame with
%sharpening operator can perfectly concentrate seislet coefficients in a small
%scale. Especially, the frame coefficients at zero scale equals to hyperbolic 
%radon transform, the primary and multiple reflections distribute in the 
%different velocity range (Figure~\ref{fig:rrdiplet}).

\inputdir{diplet}
\multiplot{2}{cdips,rrdips}{width=0.6\textwidth}{Constant dip field (a) and  time and space varying dip field (b).}
\multiplot{2}{cdiplet,rrdiplet}{width=0.6\textwidth}{2-D seislet frame coefficients with constant dip field (a) and with
  varying dip field (b).}
\multiplot{2}{cdipimps,rrdipimps}{width=0.6\textwidth}{Randomly
selected representative frame functions for 2-D seislet frame with
constant dip field (a) and varying dip field (b).}

\section{Discussion}

\tabl{cpu}{CPU times (in seconds) for different transforms.}
{
    \begin{center}
     \begin{tabular}{|c|c|c|c|c|}
     \hline Data size & 1-D FFT & 1-D DWT & 1-D seislet & 2-D seislet \\
     \hline $1024 \times 1024$ & 0.06 & 0.03 & 0.17 & 1.03 \\
     \hline $512 \times 512$   & 0.02 & 0.01 & 0.04 & 0.22 \\
	\hline 
	\end{tabular} 
   \end{center}
}  

How efficient are the proposed algorithms? \new{The CPU times, 
in our implementation, are shown in Table~\ref{tbl:cpu}.}
\old{This shows} \new{They confirm} that, while the seislet transform and frame can be more
expensive than FFT or DWT, they are still comfortably efficient in
practice. In applications of the 2-D seislet transform, the main cost
may not be in the transform itself but in iterative estimation of the
slope fields.  In practical large-scale applications, it is
advantageous to break the input data in parts and process them in
parallel.

How effective are the seislet transform and frame in compressing
seismic data? \new{In the case of the 2-D seislet transform that
requires a slope field}, it appears that \old{, in the 2-D seislet
transform that requires a slope field,} one would need to store this
field in addition to the compressed data. However, since we force the
estimated slopes to be smooth, the slope field can be easily
compressed with one of the classic compression algorithms. Consider
the example in Figure~\ref{fig:sigmoid,sigmoiddip}. Suppose that we
apply lossy compression and require 99\% of the energy to be
preserved. The seislet transform compression ratio in this case is
less that 1\% while the corresponding wavelet transform ratio is
26\%. Applied to the smooth slope field from
Figure~\ref{fig:sigmoiddip}, the wavelet transform compresses it to
about 0.1\%. This example shows that compressing seismic data with the
seislet transform and the corresponding slope field with the wavelet
transform can be significantly more effective that trying to compress
seismic data with the wavelet transform.

\section{Conclusions}

We have introduced a new digital transform named \emph{seislet
transform} because of its ability to characterize and compress seismic
data in the manner similar to that of digital wavelet transforms. We
define the seislet transform by combining the wavelet lifting scheme
with local plane-wave destruction.  In 1-D, the seislet transform
follows sinusoidal components. In 2-D, it follows locally plane
events. When more than one sinusoid or more than one local slope are
\old{used} \new{applied} for the analysis, the transform turns into an overcomplete
representation or a frame. The seislet transform and seislet frame can
achieve a better compression ratio than either the digital Fourier
transform (DFT) or the digital wavelet transform (DWT).

The seislet transform provides a convenient orthogonal basis with the basis 
functions spanning different scales analogously to those of the digital wavelet
transform but aligned along the dominant seismic events. Traditional signal 
analysis operations such as denoising and trace interpolation become simply 
defined in the seislet domain and allow for efficient algorithms. 
Seismic stacking also has a simple meaning of the zeroth-order seislet 
coefficient computed in an optimally efficient manner by recursive partial 
stacking and thus avoiding the usual problems with wavelet stretch and 
nonhyperbolic moveouts. \old{One can extend the idea of the seislet transform 
further by changing the definition of prediction and update operators in the 
lifting scheme.}

\section{Acknowledgments}

We thank BGP Americas for a partial financial support of this work. The first
author is grateful to Huub Douma for inspiring discussions and for suggesting 
the name ``seislet''. This publication is authorized by the Director, Bureau 
of Economic Geology, The University of Texas at Austin.

%We also thank Hongliu Zeng for useful 
%discussions.

\append{Review of plane-wave destruction}

{This appendix reviews the basic theory of plane-wave destruction 
\cite[]{GEO67-06-19461960}.}

{Following the physical model of local plane waves, we \old{can}
define the mathematical basis of plane-wave destruction filters via
the local plane differential equation \cite[]{pvi}
\begin{equation}
  {\frac{\partial P}{\partial x}} +
  {\sigma\,\frac{\partial P}{\partial t}} = 0\;,
  \label{eqn:pde}
\end{equation}
where $P(t,x)$ is the wave field, and $\sigma$ is the local slope, which may
also depend on $t$ and $x$. In the case of a constant slope,
equation~\ref{eqn:pde} has the simple general solution
\begin{equation}
  \label{eqn:plane}
  P(t,x) = f(t - \sigma x)\;,
\end{equation}
where $f(t)$ is an arbitrary waveform. Equation~\ref{eqn:plane} is
nothing more than a mathematical description of a plane wave.}

{If we assume that the slope $\sigma$ does not depend on $t$, we can
transform equation~\ref{eqn:pde} to the frequency domain, where it
takes the form of the ordinary differential equation
\begin{equation}
  {\frac{d \hat{P}}{d x}} +
  i \omega\,\sigma\, \hat{P} = 0
  \label{eqn:ode}
\end{equation}
and has the general solution
\begin{equation}
  \label{eqn:px}
  \hat{P} (x) = \hat{P} (0)\,e^{i \omega\,\sigma x}\;,
\end{equation}
where $\hat{P}$ is the Fourier transform of $P$. The complex
exponential term in equation~\ref{eqn:px} simply represents a shift
of a $t$-trace according to the slope $\sigma$ and the trace separation
$x$.}

In the frequency domain, the operator for transforming the trace \old{at
position} $x-1$ to the neighboring trace \old{[For simplicity, it is
  assumed that $x$ takes integer values that correspond to trace
  numbering.] and at position} $x$ is a multiplication by $e^{i
  \omega\,\sigma}$. In other words, a plane wave can be perfectly
predicted by a two-term prediction-error filter in the $F$-$X$ domain:
\begin{equation}
  \label{eqn:pef}
  a_0 \, \hat{P} (x) + a_1\, \hat{P} (x-1) = 0\;,
\end{equation}
where $a_0 = 1$ and $a_1 = - e^{i \omega\,\sigma}$. The goal of
predicting several plane waves can be accomplished by cascading
several two-term filters. In fact, any $F$-$X$ prediction-error
filter represented in the $Z$-transform notation as
\begin{equation}
  \label{eqn:pef2}
  A(Z_x) = 1 + a_1 Z_x + a_2 Z_x^2 + \cdots + a_N Z_x^N
\end{equation}
can be factored into a product of two-term filters:
\begin{equation}
  \label{eqn:pef3}
  A(Z_x) = \left(1 - \frac{Z_x}{Z_1}\right)\left(1 - \frac{Z_x}{Z_2}\right)
  \cdots\left(1 - \frac{Z_x}{Z_N}\right)\;,
\end{equation}
where $Z_1,Z_2,\ldots,Z_N$ are the zeroes of
polynomial~\ref{eqn:pef2}. According to equation~\ref{eqn:pef},
the phase of each zero corresponds to the slope of a local plane wave
multiplied by the frequency. Zeroes that are not on the unit circle
carry an additional amplitude gain not included in
equation~\ref{eqn:ode}.

In order to incorporate time-varying slopes, we need to return to
the time domain and look for an appropriate analog of the phase-shift
operator~\ref{eqn:px} and the plane-prediction
filter~\ref{eqn:pef}. An important property of plane-wave
propagation across different traces is that the total energy of the
propagating wave stays invariant throughout the process: the energy of 
the wave at one trace is completely transmitted to the next trace.
 This property
is assured in the frequency-domain solution~\ref{eqn:px} by the fact
that the spectrum of the complex exponential $e^{i \omega\,\sigma}$ is
equal to one.  In the time domain, we can reach an equivalent effect
by using an all-pass digital filter. In the $Z$-transform notation,
convolution with an all-pass filter takes the form
\begin{equation}
\label{eqn:allpass}
\hat{P}_{x+1}(Z_t) = \hat{P}_{x} (Z_t) \frac{B(Z_t)}{B(1/Z_t)}\;,
\end{equation}
where $\hat{P}_x (Z_t)$ denotes the $Z$-transform of the corresponding
trace, and the ratio $B(Z_t)/B(1/Z_t)$ is an all-pass digital filter
approximating the time-shift operator~$e^{i \omega \sigma}$. In
finite-difference terms, equation~\ref{eqn:allpass} represents an
implicit finite-difference scheme for solving equation~\ref{eqn:pde}
with the initial conditions at a constant $x$.  The coefficients of
filter $B(Z_t)$ can be determined, for example, by fitting the filter
frequency response at low frequencies to the response of the
phase-shift operator. This leads to a version of Thiran's
maximally-flat all-pass fractional-delay filters \cite[]{thiran,review}.

{Taking both dimensions into consideration,
equation~\ref{eqn:allpass} transforms to the prediction equation
analogous to~\ref{eqn:pef} with the 2-D prediction filter
\begin{equation}
  \label{eqn:2dpef}
  A(Z_t,Z_x) = 1 - Z_x \frac{B(Z_t)}{B(1/Z_t)}\;.
\end{equation}
In order to characterize several plane waves, we can cascade several
filters of the form~\ref{eqn:2dpef} in a manner similar to that of
equation~\ref{eqn:pef3}. A modified version of the filter
$A(Z_t,Z_x)$, namely the filter
\begin{equation}
  \label{eqn:2dpef2}
  C(Z_t,Z_x) = A(Z_t,Z_x) B(1/Z_t) = B(1/Z_t) - Z_x B(Z_t)\;,
\end{equation}
avoids the need for polynomial division. In case of \old{the} \new{a}
3-point filter $B(Z_t)$, the 2-D filter~\ref{eqn:2dpef2} has exactly
six coefficients. It consists of two columns, each column having three
coefficients and the second column being a reversed copy of the first
one.}


\bibliographystyle{seg}
\bibliography{SEG,seislet}
