\title{Building good starting models for full waveform inversion 
using adaptive matching filtering misfit}
\author{Hejun Zhu\footnotemark[1] and Sergey Fomel\footnotemark[2]}

\address{
\footnotemark[1] Department of Geosciences,
The University of Texas at Dallas, 
Richardson, TX, 75080.
\footnotemark[2] Bureau of Economic Geology,
Jackson School of Geosciences,
The University of Texas at Austin,
Austin, TX 78713.}

\lefthead{Zhu \& Fomel}
\righthead{AMF for FWI}
\footer{TCCS-11}

\maketitle

\begin{abstract}  
We propose a misfit function based on adaptive matching filtering to tackle
challenges associated with cycle skipping and local minima in full waveform inversion (FWI).  
This adaptive matching filter is designed to measure time varying phase 
differences between observations and predications. Compared to classical 
least-squares waveform differences, this misfit function behaves as a smooth, quadratic 
function with a broad basin of attraction. These characters are very important 
because local gradient-based optimization approaches used in most FWI can not 
guarantee convergence towards true models if misfit functions include 
local minima and starting model is far away from the global minimum. 
1D and 2D synthetic experiments illustrate the advantages of the proposed misfit function 
compared to the classical least-squares waveform misfit. Furthermore, we derive 
adjoint sources associated with the proposed misfit function and apply them 
in several 2D time-domain acoustic FWI experiments. 
Numerical results show that the proposed misfit function helps us to build 
good starting models for FWI, particularly when low frequency signals are absent in recorded data. 
\end{abstract} 

\section{Introduction} 
Full waveform inversion (FWI) has became an important model building technique in both exploration and global seismology 
over the past ten years~\citep{fwi2009, Fichtner2009, Tape2009, Zhuetal2012, Zhuetal2013_att,
Zhuetal2013_aniso}. Although its basic theory was proposed in the 1980s~\citep{Lailly1983, 
Tarantola1984, Gauthier_Tarantola1986}, it has not been widely applied 
until late 1990s~\citep{Pratt1998} because of limitations on computing power, 
numerical techniques and data quality. Compared to classical ray-based traveltime tomography, 
FWI enables us to obtain higher-resolution subsurface images~\citep{Sirgue2010}. 
2D and 3D results have shown its capabilities to image structures in 
complicated subsurface environments~\citep{Virieux2009, fwi3D}. 
Meanwhile, multi-parameter FWI has been developed to constrain different 
seismic parameters, including velocity, density, anisotropy 
and attenuation~\citep{attenfwi, multifwi, Warner2013}. 

Although recently FWI has been successfully applied in many synthetic and field data 
experiments, there are still a number of theoretical and practical 
challenges. Among these difficulties, cycle skipping and local minima are 
significant challenges because they directly determine whether FWI can converge 
towards a true solution. One way to tackle these challenges is to design 
well-behaved misfit functions. The following criteria should be taken into account 
when we design new misfit functions. First, they should behave as quadratic functions
around the global minimum. Second, they should be smooth functions, 
avoiding abrupt jumps or discontinuities so that local gradient-based optimization 
approaches can converge robustly. Third, they should involve broad basins of attraction 
so that we are able to obtain robust inversion results. 

Over the past several decades, a variety of misfit functions have been proposed for FWI. 
The most commonly used misfit is a least-squares (L2) waveform difference between 
observations and predictions. These waveform differences can be quantified in the time domain
\citep{Tarantola1984}, frequency domain~\citep{Pratt1999} or Laplace domain~\citep{Shin2008}. 
The second type of misfit function separates traveltimes and amplitudes. 
The amplitudes are always difficult to work with because they are subject to a variety 
of effects besides the variations of subsurface model parameters, 
such as source magnitudes, radiation patterns, local site effects, etc. 
This type of misfit function ignores amplitude discrepancies 
and focuses on traveltime or phase differences between observations and predictions 
~\citep{Luo1991, Leeuwen2010}. These phase differences can be measured by cross correlation or
multitaper techniques~\citep{ZhouY2004, Zhuetal2015_gji}. It has been successfully 
applied in transmission-type tomography. Instead of designing misfits in either time 
or frequency domains, some studies proposed misfit functions in time-frequency 
domains, using instantaneous phases~\citep{Fichtner2008} or envelope differences~\citep{Wu2014}.  
Recently, several approaches used in signal processing were applied 
in FWI, such as dynamic warping~\citep{dynamicwarping}, registration-guided 
waveform differences~\citep{Baek2013} and wavelet analysis~\citep{Yuan2014}. They have shown advantages by comparing with 
L2 waveform misfits. Adaptive waveform inversion~\citep{Warner2014} used matching filters, 
such as the Wiener filter, to quantify discrepancies between observations and predictions. 
By minimizing the filter coefficients, they were able to iteratively update 
velocity structure. Significant improvements have been shown to avoid cycle skipping and 
local minima, even in the cases without good starting models and low frequency signals. 
Wasserstein metric has been proposed to measure transports between two 
time series~\citep{Engquist2014}. 
It is a very promising approach to tackle cycle skipping problem in FWI. In addition, 
instead of quantifying misfit functions with L2 norms, some L1 and L1/L2 mixed norms, 
such as the Huber norm~\citep{huber2009}, were proposed in the waveform inversion. 
 
In this paper, we propose a misfit function based on adaptive matching filtering (AMF). 
We first introduce the definition of AMF, which allows us to measure 
time varying phase differences between two geophysical signals. 
Next, we propose a misfit function based on AMF. Then, we derive expressions for adjoint sources 
associated with the proposed misfit function. Several 1D and 2D synthetic experiments are used 
to illustrate the advantages of the proposed misfit function compared to L2 waveform misfits. 
Finally, we use some 2D numerical examples to show the potential of 
the proposed misfit function for building good starting models for FWI when low frequency signals 
are absent in recorded data.

\section{Theory} 
\subsection{Adaptive Matching Filtering} \label{sec:amf}
AMF is a method to measure nonstationary phase differences between two geophysical signals, 
which has been successfully used for adaptive subtraction of multiple reflections~\citep{nonstationary2009}. 
Given two time series: prediction ${\bf p}(t)$ 
and observation ${\bf d}(t)$,
a stationary matching filter ${\bf f}(\gamma)$ may be defined as the solution to
the following least-squares inverse problem: 

\begin{equation}
\label{eqn:statmf}
{\rm min} ||\sum_{\gamma} {\bf d}(\gamma t) {\bf f}(\gamma)-{\bf p}(t)||^2 
\quad ,
\end{equation}

For the whole traces, it only gives us a single measurement ${\bf f}(\gamma)$ for different 
stretching variable $\gamma$. However, most geophysical signals are nonstationary. 
Our goal here is to extract a time varying matching filter for different values of the stretching variable $\gamma$. 
We can modify the stationary matching filter defined in Equation~\ref{eqn:statmf} to 
a nonstationary AMF, denoted by ${\bf f}(\gamma,t)$. It is defined as the solution to the following 
modified least-squares inverse problem

\begin{equation}
\label{eqn:nonstatmf}
{\rm min} ||\sum_{\gamma} {\bf d}(\gamma t) {\bf f}(\gamma,t)-{\bf p}(t)||^2
\quad ,
\end{equation}

Equation~\ref{eqn:nonstatmf} is an underdetermined inverse problem. 
We have to add additional constraints, i.e., regularization, to solve this problem. 
Shaping regularization~\citep{shaping2007} is one type of regularization approches for solving ill-posed 
inversion problems. In this study, we use it to solve Equation~\ref{eqn:nonstatmf}:

\begin{equation}
\label{eqn:amfsolution}
{\bf f}=[{\bf S}({\bf D}^T{\bf D}-\lambda^2{\bf I})+\lambda^2{\bf I}]^{-1}{\bf S}{\bf D}^T{\bf p}
\quad ,
\end{equation}
where ${\bf D}$ is the data matrix composed of ${\bf d}(\gamma t)$.
$\bf S$ denotes the shaping operator. We choose a smoothing operator in this case to impose 
the smoothness of filter coefficients. \new{For all examples in this paper, we use a triangle smoothing operator as the shaping regularization operator. In addition, the AMF coefficients are solved trace by trace.} $\lambda$ is chosen based on the L2 norm 
of observations, i.e., $\lambda=||{\bf d}||^2$. 
Following \cite{shaping2007}, we can rewrite Equation~\ref{eqn:amfsolution} as 

\begin{equation}
\label{eqn:amfsolution1}
{\bf f}={\bf H}[{\bf H}^T({\bf D}^T{\bf D}-\lambda^2{\bf I}){\bf H}+\lambda^2{\bf I}]^{-1}{\bf H}^T{\bf D}^T{\bf p}
\quad ,
\end{equation}
where the smoothing operator ${\bf S}={\bf H}{\bf H}^T$.
Conjugate gradient method~\citep{Fletcher1964} 
is used to iteratively solve Equation~\ref{eqn:amfsolution1}.
The method converges in practice in a small number of iterations.
\new{Comparing to forward and adjoint calculations for solving wave equations, computational costs 
for calculating AMF are relatively cheap and they can be easily parallelized.}

\subsection{Misfit function based on AMF} 
Given two seismograms, AMF gives us an image, 
which is a function of time $t$ and stretching variable $\gamma$. 
For instance, in Figure~\ref{fig:misfit_data_syn}, we have two synthetic seismograms and each of them involves 
three events. These two seismograms have identical phases for the first event. However, 
for the second and third events, they have phase delays and advances, respectively.  
By solving the least-squares inverse problem in Equation~\ref{eqn:nonstatmf}, we are able to calculate 
AMF as shown in Figure~\ref{fig:misfit_filt}. There are three energy patches in the AMF image. 
Energies concentrating around the stretching variable of $\gamma=1$ correspond to the first events 
with identical phases in the seismograms. While energies deviating away from $\gamma=1$ correspond 
to the second and third events with nonstationary phases differences, i.e., phase advances and delays. 

\inputdir{misfitexample}
\plot{data-syn}{width=0.9\columnwidth}
{\label{fig:misfit_data_syn} 
Comparisons of two seismograms with three events. 
For the first event, these two seismograms have identical phase.
There are phase delay and advance for the second and third events. 
}

\inputdir{misfitexample}
\plot{filt}{width=0.9\columnwidth}
{\label{fig:misfit_filt} 
Magnitude of AMF for seismograms in 
Figure~\ref{fig:misfit_data_syn}. The first energy patch concentrating around $\gamma$=1 
corresponds to the first event where these two seismograms have identical phase. 
The second and third energy spots spreading around $\gamma$=1 correspond to the 
second and third events where these two sesimograms have phase advance and delay.}

With these properties, we propose a misfit function based on AMF. Our approach is inspired by 
the differential semblance optimization principle~\citep{Symes1991, PShen2008} and adaptive 
waveform inversion~\citep{Warner2014}. In the proposed misfit function, we use a penalty 
function $\gamma-1$ to emphasize semblance energies deviating away from $\gamma=1$ 

\begin{equation}
\label{eqn:misfit}
J=\frac{1}{2} \frac{\int \int [{\bf f}(\gamma,t)(\gamma-1)]^2 {\rm d}\gamma {\rm d} t}{\int \int [{\bf f}(\gamma,t)]^2 {\rm d}\gamma {\rm d}t }
\quad , 
\end{equation}

By minimizing the misfit function in Equation~\ref{eqn:misfit}, 
we are able to reduce semblance energies 
deviating away from $\gamma=1$, which then aligns phases between the two seismograms. 

Equation~\ref{eqn:misfit} can be rewritten in a discretized form as 

\begin{equation}
\label{eqn:misfit_disc}
J=\frac{1}{2} \frac{||{\bf w f}||^2}{||{\bf f}||^2}
\quad ,
\end{equation}
where ${\bf f}$ contains the coefficients of AMF and ${\bf w}$ represents the 
weight function $\gamma-1$.

Compared to cross-correlation technique for quantifying phase discrepancies, 
this method allows us to measure phase differences without breaking the data into local windows. 
Therefore, we are able to avoid several non-intuitive 
input parameters, such as window sizes, overlaps between windows, taper functions, etc. 
In addition, the stretching and squeezing 
alignments employed in AMF (Equation~\ref{eqn:nonstatmf}) can better represent phase advances
or delays due to velocity changes. This misfit function resembles the Wiener filter technique proposed 
by~\cite{Warner2014}. However, AMF allows us to measure nonstationary phase differences between 
two seismograms, while the Wiener filter used by~\cite{Warner2014} is a long time invariant filter with 
lengths greater than the input seismograms. In addition, we use the stretch parameter $\gamma$ instead of 
time shifts that are used as penalty functions in the misfit function based on the Wiener filter.

\subsection{Adjoint sources for the proposed misfit function}
A key step in FWI is the derivations of adjoint sources.
In the adjoint methods,  the adjoint sources can be calculated based on the partial derivative 
of misfit functions with respect to predictions~\citep{Plessix2006}, 
i.e., $\frac{\partial J}{\partial {\bf p}}$. For the misfit function $J$ 
defined in Equation~\ref{eqn:misfit}
and Equation~\ref{eqn:misfit_disc}, the adjoint sources can be calculated as

\begin{equation}
\label{eqn:adjsrc}
\frac{\partial J}{\partial {\bf p}}=(\frac{\partial {\bf f}}{\partial \bf p})^T\frac{{\bf w}^T{\bf w}{\bf f}-2 J {\bf f}}{{\bf f}^T{\bf f}} 
\quad ,
\end{equation}
where $(\frac{\partial {\bf f}}{\partial {\bf p}})^T$ can be computed based on Equation~\ref{eqn:amfsolution1}

\begin{equation}
\label{eqn:deriw}
(\frac{\partial {\bf f}}{\partial {\bf p}})^T={\bf D}{\bf H}[{\bf H}^T({\bf D}^T{\bf D}-\lambda^2{\bf I}){\bf H}+\lambda^2{\bf I}]^{-1}{\bf H}^T 
\quad ,
\end{equation}

We use the conjugate gradient method to iteratively solve Equation~\ref{eqn:adjsrc}.
\new{The derivations of Equation~\ref{eqn:adjsrc} and \ref{eqn:deriw} are similar to \cite{Warner2014}
except shaping regularization is used to calculate nonstationary filter coefficients and adjoint sources.}
Once we have expressions for the adjoint sources of the proposed misfit function, we can use them to 
drive adjoint wavefields. The correlations of forward and adjoint wavefields generates misfit gradients, 
which can be used to iteratively update model parameters by local gradient-based optimization approaches,
such as the nonlinear conjugate gradient method~\citep{Fletcher1964} or L-BFGS method~\citep{Nocedal1980}.

\section{Numerical examples}
\subsection{Behavior of misfit functions}
In this section, we use 1D and 2D numerical examples to compare the behavior of misfit functions 
based on L2 waveform differences and AMF. 
First, we use a simple 1D analytical signal as an example 

\begin{equation}
\label{eqn:analytical_signal}
S(t)=(1.0+dA) {\rm exp}[-2(t-3.5+dt)^2]{\rm cos}[2\pi f_0(t-3.5+dt)]
\quad ,
\end{equation}

Equation~\ref{eqn:analytical_signal} describes a cosine function modulated by a Gaussian envolope (Figure~\ref{fig:data-analy}).
Its amplitude and phase are controlled by two parameters $dA$ and $dt$. 
True model is set with $dA=0$ and $dt=0$. Central frequency is chosen as $f_0$=3~Hz. 

\inputdir{onedanalysignal}
\plot{data-analy}{width=0.9\columnwidth}
{\label{fig:1d_data-analy}
1D analytical signal used for comparing the behaviors of misfit functions. 
Blue solid line is original signal (Equation~\ref{eqn:analytical_signal}).
Red dashed line is a modified signal with $dt$=0.2 s and $dA$=0.2. 
In this case, time shift is greater than half period of the original signal.}

We change both amplitude and phase parameters and compute misfit functions based on
L2 waveform differences and AMF. In Figure~\ref{fig:ZMISFIT-analy}, 
we observe that the L2 waveform misfit involves numerous local minima due to cycle 
skipping. This behavior would cause convergency troubles with local optimization approches if 
starting model is far away from the global minimum. However, the proposed misfit function
based on AMF behaves 
as a smooth, quadratic function with a broad basin of attraction. Therefore it would be much easier for local 
gradient-based optmization methods to work with the proposed misfit function.

\inputdir{onedanalysignal}
\plot{ZMISFIT-analy}{width=0.9\columnwidth}
{\label{fig:1d_misfit-analy}
Behaviors of misfit functions for the analytical signal in Figure~\ref{fig:data-analy}. 
From top to bottom are misfit functions based on L2 waveform differences and AMF.}

A simple 2D three-layer model (Figure~\ref{fig:model}) is used in the second example. 
One source and 151 receivers are used in this experiment. Velocities in each layer 
are 2, 3 and 4~km/s in the true model. The depths of each layer are 1.3 and 2.6~km. 
Two cases are demonstrated in this example. In case 1, two model parameters are velocity 
for the second layer and depth for the first layer. In case 2, the model parameters are 
the depths for the first and second layers. 
\new{In all numerical examples in the following sections, we use a constant density, acoustic wave equation to 
propagate wavefields, which is numerically solved by finite difference methods.}
\old{By changing these two model parameters 
and computing the misfits}\new{By changing these two model parameters and computing the 
misfits based on the comparisons of seismic data generated from the base line velocity 
in Figure~\ref{fig:model} and perturbed models}, we are able to compare the behavior of the misfit functions 
based on L2 waveform differences and AMF. 

\inputdir{threelayers-velocity}
\plot{model}{width=\columnwidth}
{\label{fig:3layers_model}
Three-layer model used for the comparisons of misfit functions. Shot 
is located at (4, 0) and receivers are equally distributed 
at Earth surface.}
 
In case 1, we observe that the L2 waveform misfit involves a narrow and long valley.
Away from the valley, there are ``mountains'' and ``plateaus'', which 
lead to local minima (Figure~\ref{fig:3layers_misfit_vel}a). 
If the starting model is far away from the global minimum, 
local gradient-based optimization approaches may converge towards local minima 
instead of the global minimum. In contrast, the proposed misfit function
behaves as a smooth, quadratic function with 
one global minimum. We expect that the local gradient-based 
optimization approaches should work efficiently on the proposed misfit function. 

\inputdir{threelayers-velocity}
\multiplot{3}{ZMISFIT-2D,ZMISFIT-dep,ZMISFIT-vel}{width=0.3\columnwidth}
{\label{fig:3layers_misfit_vel}
Comparisons of misfits functions for the three-layer model in Figure~\ref{fig:3layers_model}. 
Model parameters are the depth of the first layer $d$ and the velocity of 
the second layer $v$. From top to bottom are misfit functions based on L2 
waveform differences and AMF. (a) shows 2D misfit functions. 
(b) shows a cross sections of misfits with $v$=3~km/s. 
(c) shows a cross sections of misfits with $d$=1.3~km. 
}

In case 2, two model parameters are the depths of the first two layers (Figure~\ref{fig:3layers_misfit_depth}). 
We can draw a similar conclusion as in case 1. L2 waveform misfit involves local minima  
while the proposed misfit function exhibits a much better behavior. 

\inputdir{threelayers-depth}
\multiplot{3}{ZMISFIT-2D-depth,ZMISFIT-dep2-depth,ZMISFIT-dep1-depth}{width=0.3\columnwidth}
{\label{fig:3layers_misfit_depth}
The same setting as Figure~\ref{fig:3layers_misfit_vel} except here model parameters are the depths of 
the first and second layers. 
}

\subsection{Behavior of adjoint sources and misfit gradients}
In this section, we compare the behavior of adjoint sources and gradients based on different misfit 
functions. The true velocity model is shown 
in Figure~\ref{fig:grad_comp_model}. The starting model is homogeneous with velocity equal 
to 4~km/s. We perform two experiments with different true models.
The true model for the first case involves a Gaussian anomaly with a small amplitude 
perturbation (maximum 2.5\%) in the middle of the model. For the second case, 
we put a Gaussian anomaly with a large amplitude perturbation (maximum 40\%) in the middle 
of the model. Only one shot and one receiver are used in both cases. 
A Ricker wavelet with central frequency of 15~Hz is used.

Comparisons between data and synthetics are shown in Figure~\ref{fig:grad_comp_adj}. 
For the first case with small velocity perturbations, differences between data and prediction are 
relatively small (within half period of signal). The adjoint sources for both misfit functions only 
involve a single event (Figure~\ref{fig:grad_comp_adj}a). However, 
for the experiment with large velocity perturbations, the delay of data with respect to prediction 
is noticeably greater than half period (Figure~\ref{fig:grad_comp_adj}b). 
In this case, the adjoint source based on L2 waveform difference contains two events (one from data 
and one from prediction), while the adjoint source based on AMF has still only one event,
which shows behavior similar to that in the first experiment with small velocity perturbations. 

\inputdir{gradcomp}
\plot{true-model-small}{width=0.8\columnwidth}
{\label{fig:grad_comp_model}
Velocity model used to compare misfit gradients based on different misfit functions. 
This is the case with small Gaussian perturbation (maximum 2.5\%). Shot is located at 
(2, 0.5) and receiver is located at (1, 3.5).}

We can back propagate these adjoint sources to generate adjoint wavefields which are then 
correlated with forward wavefields to construct misfit gradients. 
Figure \ref{fig:grad_comp_grad} compares misfit gradients based on 
different misfit functions for different cases. For the case with small velocity 
perturbations (Figure~\ref{fig:grad_comp_grad}a), both misfit gradients 
exhibit a similar behavior. The first Fresnel zone involves correct sensitivity, needed to 
reduce velocity. However, for the case with large velocity perturbations, 
the first Fresnel zone of the misfit gradient based on L2 waveform differences 
has a wrong sensitivity (Figure~\ref{fig:grad_comp_grad}b), which does not carry any useful velocity information. 
On the other hand, the 
misfit function based on AMF still contains correct velocity update 
information. These comparisons confirm the robustness of the proposed misfit function 
compared to L2 waveform differences.

\inputdir{gradcomp}
\multiplot{2}{compadj-small,compadj-large}{width=0.45\columnwidth}
{\label{fig:grad_comp_adj}
Comparisons of data and predictions as well as adjoint sources for different misfit functions. 
Left panel (a) for the case with small Gaussian perturbation and right panel (b) for the case with 
large Gaussian perturbation. From top to bottom are comparisons between data (blue) and predictions (red), 
adjoint sources based on L2 waveform differences and AMF, respectively.}

\inputdir{gradcomp}
\multiplot{2}{compgrad-small,compgrad-large}{width=0.45\columnwidth}
{\label{fig:grad_comp_grad}
Comparisons of misfit gradients based on different misfit functions.
Left panel (a) for the case with small Gaussian perturbation and right panel (b) for the 
case with large Gaussian perturbation. From top to bottom are misfit gradients based 
on L2 waveform differences and AMF, respectively.}

\subsection{Transimission waveform tomography experiments}
Once we have misfit gradients, we can integrate them with gradient-based optimization 
approaches to build velocity models. In this section, we consider two transimission 
tomography experiments.

The first experiment involves four Gaussian anomalies with alternative signs in 
the middle of the model. The starting model is homogeneous with velocity equal to 4~km/s. 
The true model involves Gaussian anomalies with maximum velocity perturbations equal 
to 12.5\% (Figure~\ref{fig:gauss_l2_comp}a). This is a case with large velocity perturbations that cause data and 
predictions to be out of phase. We use a Ricker wavelet with central frequency of 15 Hz. 
11 shots are equally distributed 
at z=0.5~km and 151 receivers are distributed at z=3.5~km. No multiscale 
strategy~\citep{Bunks1995, SirguePratt2004} is used in this experiment. 
We performed 10 conjugate gradient iterations for misfit functions based on 
L2 waveform differences and AMF. 

For the case with L2 waveform differences, we observe that the inversion converges towards 
a wrong velocity model. Maximum velocity perturbation in the final model reach up to 65\% (Figure~\ref{fig:gauss_l2_comp}b). 
Evidently, in this case, the inversion suffers from cycle skipping and local minima problems. 
In contrast, for the misfit function based on AMF, the inversion converges towards the global 
minimum. We are able to recover both the shapes and amplitudes of these four Gaussian 
anomalies (Figure~\ref{fig:gauss_l2_comp1}). The elongation of the recovered velocity perturbations along vertical direction is due to 
the limited illumination of survey. 

\inputdir{gauss-l2-diverge}
\multiplot{2}{true-model,model-15-10}{width=0.6\columnwidth}
{\label{fig:gauss_l2_comp}
Inversion results based on L2 misfit function for Gaussian model. (a) true velocity model 
with four Gaussian anomalies. Maximum velocity perturbation is 12.5\%. 
(b) is recovered model based on L2 waveform differences.} 

\inputdir{gauss-lpf-diverge}
\plot{model-15-5}{width=\columnwidth}
{\label{fig:gauss_l2_comp1}
Inversion result based on AMF misfit function for Gaussian model.}

For the second experiment, we test the proposed misfit function for a crosswell 
tomography problem. The starting velocity model is laterally homogeneous. A sequence of true 
models are generated by adding a relatively complicated velocity perturbations to the 
starting model (Figure~\ref{fig:crosswell_model}). 
We tried two cases in this experiment. For the first case, scaled small 
perturbations (maximum 0.2~km/s) are added to the V(z) 
starting model. For the second case, relatively large perturbations (maximum 0.4~km/s) 
are added to the starting model. For this crosswell experiment, we use a Ricker wavelet 
with the central frequency of 10~Hz. 10 shots are equally distributed along the left side of 
the model and 31 receivers are equally distributed along the right side of the model. 
Only 10 conjugate-gradient iterations are performed in both cases. 

\inputdir{crosswell-l2-diverge}
\multiplot{3}{model-m00,true-model,pert}{width=0.3\columnwidth}
{\label{fig:crosswell_model}
True and starting models for crosswell tomography. (a) is starting V(z) model. 
(b) is true velocity model and (c) is perturbation between the true and starting models.
For the experiment with small veloicty perturbations, we add 50\% maximum perturbations of (c) on the starting 
model (a) to generate the true model (see Figure~\ref{fig:crosswell_l2_converge}).}

\inputdir{crosswell-l2-converge}
\multiplot{2}{pertinv,misfit}{width=0.8\columnwidth}
{\label{fig:crosswell_l2_converge}
Inversion result for small velocity perturbations with L2 waveform differences. (a) is 
the recovered model after 10 iterations. (b) shows the evolution of data (left) and model (right) misfits.}

\inputdir{crosswell-l2-diverge}
\multiplot{2}{pertinv,misfit}{width=0.8\columnwidth}
{\label{fig:crosswell_l2_diverge}
The same setting as Figure~\ref{fig:crosswell_l2_converge}, except here large velocity perturbations 
are used. The inversion diverges 
towards local minima.}

\inputdir{crosswell-lpf-diverge}
\multiplot{2}{pertinv,misfit}{width=0.8\columnwidth}
{\label{fig:crosswell-lpf-diverge}
The same setting as Figure~\ref{fig:crosswell_l2_diverge}, except here we use 
AMF misfit function. The inversion 
converges towards correct velocity model (see Figure~\ref{fig:crosswell_l2_converge}).}

For the first case with small velocity perturbations, the inversion based on L2 waveform 
differences converges towards the true model. We are able to recover true velocity 
perturbations. Both data and model misfits are smoothly reduced (Figure~\ref{fig:crosswell_l2_converge}). 
However, for the case 
with large velocity perturbations, the inversion based on L2 waveform differences does not converge to the right answer. 
Although data misfits are reduced to less than 0.2, model 
misfits diverge (Figure~\ref{fig:crosswell_l2_diverge}). 
In contrast, for the misfit function based on AMF, the 
inversion still converges towards the true model even with large velocity perturbations. 
Both data and model misfits are gradually reduced (Figure~\ref{fig:crosswell-lpf-diverge}). 
Thus, in these transmission waveform tomography experiments, 
the proposed AMF misfit function appears more robust for cases with poor starting models 
and high frequency signals. 

\subsection{Modified Marmousi model}
In this section, we use a modified Marmousi model as our final experiment. 
Figure~\ref{fig:marmousi_true} shows the starting V(z) model and true model used in this 
experiment. 12 shots and 181 
receivers are used here and each shot is recorded by all recievers. 
\new{Both shots and receivers are located at the depth of 5~m. The spacing between shots and receivers are 0.64~km and 0.048~km, respectively.}
Maximum velocity perturbation
in this experiment is greater than 30\%. A Ricker wavelet with 10~Hz central frequency is used.

First, we run 10 iterations based on  L2 waveform misfit function. 
Figure~\ref{fig:marmousi_l2}a shows the recovered velocity
perturbations. Obviously, the inversion diverges away from the global minimum and reaches a local minimum. Both data and
model misfits increase significantly, indicating cycle skipping and local minima issues.

\inputdir{reflection-marmousi-l2-diverge}
\multiplot{3}{model-m00,true-model,pert}{width=0.8\columnwidth}
{\label{fig:marmousi_true} 
Starting model (a), true model (b) and perturbations between 
the starting and true models (c) for the modified Marmousi experiment.}

\inputdir{reflection-marmousi-l2-diverge}
\multiplot{2}{pertinv,misfit}{width=0.8\columnwidth}
{\label{fig:marmousi_l2} 
Recovered perturbation (a) and the evolutions of data (left) and model (right) misfits (b) 
based on L2 waveform misfit function for the modified Marmousi experiment.}

For the second case, we first perform 6 iterations with AMF misfit function. 
\new{For the calculations of AMF, the stretching variable $\gamma$ is allowed to vary from 0.9 to 1.1.
We use a triangle operator as the shaping operator and smoothing lengths are 5 grid points in both directions.}
The recovered 
perturbation and model are shown in Figure~\ref{fig:marmousi_lpf}a and b. We observe that 
the AMF misfit function allows us to recover long-wavelength velocity structures even 
without low frequency signals. We then use this updated model as the starting model 
and perform additional 10 iterations with L2 waveform misfit. The final recovered 
perturbations and model are displayed in Figure~\ref{fig:marmousi_lpf}c and d. 
The inversion appears to converge towards the true model (compared with Figure~\ref{fig:marmousi_true}
and \ref{fig:marmousi_l2}). 

\inputdir{reflection-marmousi-lpf-diverge}
\multiplot{4}{pert-10-6,model-10-6,pertinv,model-10-15}{width=0.45\columnwidth}
{\label{fig:marmousi_lpf} 
Recoverd perturbation (a) and model (b) after 6 iterations based on 
AMF misifit. This recovered model is used as starting model for FWI based on L2 waveform differences.
Recovered perturbation (c) and model (d) after additional 10 iterations based on L2 waveform differences.}

Figure~\ref{fig:marmousi_misfit} shows that both data and model misfits are steadily reduced 
for this inversion strategy. Because of the broad basins of the AMF misfit function, the reduction 
of data misfit is relatively slow for the first 6 iterations. For the final 10 iterations, due to 
the high resolution of L2 waveform misfit, the reductions of both data and model misfits are significant. 

\inputdir{reflection-marmousi-lpf-diverge}
\plot{misfit}{width=0.9\columnwidth}
{\label{fig:marmousi_misfit} Evolution of data (left) and model (right) misfits
for the modified Marmousi experiment. The first 6 iterations use AMF misfit and the final 10 
iterations use L2 waveform misfit.}

Finally, we compare predictions and data for two different shots in Figure~\ref{fig:marmousi_data} and 
\ref{fig:marmousi_data1}. 
Because of a poor starting 
model, the predictions and data are obviously out of phase at the begining. 
After 6 iterations with AMF misfits, 
we observe that the refraction and diving waves are corrected to fit the input data. 
After 10 additional iterations with L2 waveform misfit, we are able to fit both diving waves and 
reflected energies. These comparisons again confirm the robustness of the proposed inversion strategy. 

\inputdir{reflection-marmousi-lpf-diverge}
\plot{compdata1}{width=1\columnwidth}
{\label{fig:marmousi_data} Comparisons between predictions and data for the modified Marmousi experiment. 
(a)-(d) are common shot gathers for a shot at 2~km. (a) predictions 
for the starting model, (b) predictions after 6 iterations with AMF misfit function, 
(c) predictions based on final recovered model, (d) original data.}

\inputdir{reflection-marmousi-lpf-diverge}
\plot{compdata2}{width=1\columnwidth}
{\label{fig:marmousi_data1} 
The same setting as Figure~\ref{fig:marmousi_data} except for a shot at 4~km.} 

\section{Discussion}
Choosing a well-behaved misfit function is crucial in FWI because of the following reasons.
FWI is an ill-posed inverse problem due to inaccurate, insufficient 
and inconsistent waveform observations~\citep{JacksonDD1972}. It is also 
a highly nonlinear inverse problem because of strong nonlinear relations between 
waveform observations and seismic parameters. Additionally, seismograms are oscillatory 
time series, which are subject to cycle skipping when phase differences 
between observations and predictions are greater than half period of signals~\citep{fwi2009}. 
This cycle skipping problem usually leads to local minima in the misfit functions. 
Furthermore, most current FWI studies depend on local 
gradient-based optimization approaches, such as conjugate gradients~\citep{Fletcher1964} 
or L-BFGS~\citep{Nocedal1980}. If a misfit function involves numerous 
local minima, and the starting model is far away from the true model, 
these local gradient based optimization approaches are doomed to fail. Because of these issues, 
FWI based on L2 waveform differences can not always guarantee 
convergency towards the global minimum.

This problem becomes more serious when velocity perturbations are large, 
for instance exceeding 10\%, which are ubiquitous at exploration scales. These 
difficulties might be mitigated by a multi-scale inversion strategy~\citep{Bunks1995, SirguePratt2004}, 
which starts with low frequency signals to constrain long-wavelength structure and gradually incorporates high 
frequency signals to constrain short-wavelength features. However, field data 
usually do not have enough low frequency signals, making this multi-scale strategy 
infeasible in practice. 

In this paper, we propose a misfit function to tackle cycle skipping and 
local minima problems. This misfit function is based on AMF, which 
measures time varying phase differences between observations and predictions. 
Since no amplitude information is considered in this measurements, it is more 
robust than direct waveform differences. Based on numerical comparisons 
with L2 waveform misfit, we observe that the proposed misfit function behaves 
as a smooth, quadratic function. Since the proposed misfit function involves broad basins 
of attraction, it can work nicely when the starting model is far away from the true model. 
However, it may have slower convergency rates compared to more high-resolution L2 waveform misfit. 
Our stategy therefore is to use AMF misfit function to build a good starting model in cases 
we have relatively poor knowledge about the model. Once the iterations with AMF help us reach 
vallys near the global minimum, L2 waveform misfit can be employed to speed up the inversion 
towards the global minimum. In this stategy, our initial model can be far away from 
the true model and we do not need low frequency signals in the data to constrain 
long-wavelength structures. \new{At current stage, velocity updates in our inversion mainly rely on 
diving/transmitted waves. However, the AMF misfit function can be extended for reflection FWI.} 

\section{Conclusions}
The goal of this study is to design well behaved misfit functions for FWI in order to tackle 
challenges associated with cycle skipping and local minima. We propose a misfit function based on 
adaptive matching filtering, which measures time varying phase differences between observations 
and predictions. Based on numerical experiements, the proposed misfit function behaves as 
a smooth, quadratic function with a broad basin of attraction. It can increase probabilities 
for local gradient-based optimization approaches to reach the global minimum. 
We have applied the proposed misfit function in several 2D acoustic time-domain FWI experiments. 
Numerical results have indicated that the proposed misfit function can help us build good 
starting models for FWI when low-frequency signals are absent in the recorded data.

\section{Acknowledgments}
This work was supported by the Jackson School Distinguished Postdoctoral Fellowship and the 
Texas Consortium for Computational Seismology. 
We appreciate discussions with Hyoungsu Baek, Bj\"{o}rn Engquist, Brittany Froese, Michael Warner and Sven Treitel. 
We thank the associate editor Kris Innanen and three anonymous reviewers for comments and suggestions, which helped 
to improve the manuscript.

\bibliographystyle{seg}
\bibliography{misfit}

