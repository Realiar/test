\published{Geophysics, 81, no. 2, C39-C52, (2016)}

\title{Simulating propagation of separated wave modes in general anisotropic media,
Part II: qS-wave propagators}

\author{Jiubing Cheng\footnotemark[1] and Wei Kang\footnotemark[2]}

\address{
\footnotemark[1] State Key Laboratory of Marine Geology,\\
Tongji University, Shanghai, China. E-mail: cjb1206@tongji.edu.cn\\
\footnotemark[2] Formerly Tongji University, Shanghai, China;\\
presently Schlumberger, Houston, Texas, USA. E-mail: wkang@slb.com\\
}

\lefthead{Cheng \& Kang}
\righthead{Propagate separated qS-wave modes}

\maketitle

\newpage
%\begin{abstract}
Riemannian wavefield extrapolation (RWE) is a technique for one-way
extrapolation of acoustic waves. RWE generalizes wavefield
extrapolation by downward continuation by considering coordinate
systems different from conventional Cartesian. Coordinate systems can
conform with the extrapolated wavefield, with the velocity model or
with the acquisition geometry.  
%
\par When coordinate systems conform with the propagated wavefield,
extrapolation can be done accurately using low-order kernels. However,
in complex media or in cases the coordinate systems do not conform
with the propagating wavefields, low order kernels are not accurate
enough and need to be replaced by more accurate, higher order
kernels. Since RWE is based on factorization of an acoustic
wave-equation, higher order kernels can be constructed using methods
analogous with the one employed for factorization of the acoustic
wave-equation in Cartesian coordinates. Thus, we can construct
space-domain finite-differences as well as mixed-domain techniques for
extrapolation.  
%
\par High-order RWE kernels improve the accuracy of extrapolation,
particularly when the Riemannian coordinate systems does not match
closely the general direction of wave propagation.
\end{abstract}

\begin{abstract}
Shear waves, especially converted modes in multicomponent seismic data, 
provide significant information that allows better delineation of
geological structures and characterization of petroleum reservoirs.
Seismic imaging and inversion based upon the elastic wave equation involve high computational
cost and many challenges in decoupling the wave modes and estimating so many model parameters.
For transversely isotropic media, shear waves can be designated as pure SH and quasi-SV modes.
Through two different similarity transformations to the Christoffel equation aiming to project the
vector displacement wavefields onto the isotropic references of the polarization directions,
we derive simplified second-order systems (i.e., pseudo-pure-mode wave equations)
for SH- and qSV-waves, respectively.
The first system propagates a vector wavefield with two horizontal components, of which the summation
produces pure-mode scalar SH-wave data,
while the second propagates a vector wavefield with a summed horizontal component and a vertical component,
of which the final summation produces a scalar field dominated by qSV-waves in energy.
The simulated SH- or qSV-wave has the same kinematics as its counterpart in the elastic wavefield.
As explained in our previous paper (part I), we can obtain completely separated 
scalar qSV-wave fields after spatial filtering the pseudo-pure-mode qSV-wave fields.
Synthetic examples demonstrate that these wave propagators provide efficient and flexible tools 
for qS-wave extrapolation in general transversely isotropic media.
\end{abstract}

%\section{Introduction}

%% 
 % the need for velocity estimation
 %%

Accurate wave-equation depth imaging requires accurate knowledge of a
velocity model. The velocity model is used in the process of wavefield
reconstruction, irrespective of the method used to solve the acoustic
wave-equation, e.g. by integral methods (Kirchhoff migration), or by
differential/spectral methods (migration by wavefield extrapolation
and reverse-time migration).

%% 
 % velocity estimation strategies (tomography vs. MVA)
 % what is the difference between tomography and MVA?
 %%

Generally-speaking, there are two possible strategies for velocity
estimation from surface seismic data in the context of wavefield depth
migration. The two strategies differ by the domain in which the
information used to update the velocity model is collected. The first
strategy is formulated in the \textit{data space}, prior to migration,
and it involves matching of recorded and simulated data using an
approximate (background) velocity model. Techniques in this category
are known by the name of tomography (or inversion).  
%\onecolumn 
The
second strategy is formulated in the \textit{image space}, after
migration, and it involves measuring and correcting image features
that indicate model inaccuracies. Techniques in this category are
known as migration velocity analysis (MVA), since they involve
migrated images and not the recorded data directly.

%% 
 % implementation strategies for MVA and tomography?
 % (rays vs. waves)
 %%

Tomography and migration velocity analysis can be implemented in
various ways that can be classified based on the carrier of
information from the data or image to the velocity model. Thus, we can
distinguish between \textit{ray-based} methods and \textit{wave-based}
methods. This terminology is applicable to both tomography and
migration velocity analysis.
% rays
For ray-based methods, the carrier of information are wide-band rays
traced using a background velocity model from picked events in the
data (or image). Methods in this category are known as
\textit{traveltime tomography} \cite[]{GEO50-06-09030923} and
\textit{traveltime MVA}, sometimes described as image-space traveltime
tomography
\cite[]{GEO57-05-06800692,AlYahya.thesis,Fowler.thesis,Etgen.thesis,GEO60-02-04760490,GEO66-03-08450860,GEO67-04-12021212,GEO67-04-12131224,GEO68-03-10081021,GPR52-06-06710681,GEO69-02-05330546}.
% waves
For wave-based methods, the carrier of information are band-limited
wavefields constructed using a background velocity model. Methods in
this category are known as \textit{wave-equation tomography}
\cite[]{GEO51-07-13871403,tarantola.elsevier,GEO54-12-15751586,GEO57-01-00150026,GEO64-03-08880901,2004_Pratt_BPsalt},
and \textit{wave-equation MVA}
\cite[]{BiondiSava.segab.1999,SavaBiondi.gp.wemva1,SavaBiondi.gp.wemva2,ShenSymes.segab.2005,Albertin.eageab.2006,MaharramovAlbertin.segab.2007}.
This paper concentrates on methods from the latter category.

%% 
 % why are waves better than rays?
 %%

The volume of information used for model updates using wave-based
methods is at least one order of magnitude larger than the volume of
information used for ray-based methods. Thus, a fundamental
question we should ask is what is gained by using wave-based
methods over ray-based methods. First, modern imaging
applications using wave-based methods (downward continuation or
reverse-time extrapolation) require consistent velocity estimation
methods which interact with model in the same frequency band as the
migration methods. Second, wave-based methods are robust
(i.e. stable) for models with large and sharp velocity variations
(e.g. salt). Third, wave-based methods describe naturally all
propagation paths, thus they can easily handle multi-pathing
characterizing wave propagation in media with large velocity
variations.

%% 
 % WE tomography vs. WE MVA
 % match data or match images
 %%

Wave-equation tomography and wave-equation MVA have both similarities
and differences.
%
Wave-equation tomography uses the advantage that the residual used for
velocity updating is obtained by a direct comparison between recorded
and measured data. In contrast, wave-equation MVA uses the property
that the residual used for velocity updating is obtained by a
comparison between a reference image and an improved version of it.
%
On the other hand, wave-equation tomography has the disadvantage that
the kinematics of events in the data domain are more complex than in
the image domain. In addition, the dimensionality of the space in
which to evaluate the misfit between recorded and simulated data is
higher too, potentially making a comparison more complex.
%
Also, wave-equation MVA optimizes directly the desired end product,
i.e. the migrated image, which potentially makes this technique more
``interpretive'' and less of a computational ``black-box''.

%% 
 % where does MVA information come from?
 % (moveout vs. focusing)
 % (offset-gathers vs. angle-gathers)
 %%

One important component of MVA methods is what type of measurement on
migrated images is used to evaluate its deficiencies. Although
strictly related to one-another, we can describe two kinds of
information available to quantify image quality. First is
\textit{focusing analysis}, which evaluates whether point-like events,
e.g. fault truncations, are focused in migrated images at their
correct position. Image enhancement for wave-equation MVA can be
formulated purely based on this type of information, which makes the
techniques fast since it only operates with zero-offset data
\cite[]{Sava.geo.fva}. Second is \textit{moveout analysis}, which is
the case for all conventional MVA techniques, whether using rays or
waves. In this case, we can formulate wave-equation MVA based on
analysis of moveout of common-image gathers using velocity scans
\cite[]{BiondiSava.segab.1999,SavaBiondi.gp.wemva1,SavaBiondi.gp.wemva2,SoubarasGratacos.2007}
or based on analysis of differential semblance of nearby traces in
similar common-image gathers \cite[]{ShenSymes.segab.2005}.

%% 
 % can also use amplitude information! (4D - Albertin)
 %%

Moveout analysis requires construction of common-image gathers (CIGs)
characterizing the dependence of reflectivity function of various
parameters used to parametrize multiple experiments used for
imaging. There are two main alternatives for common-image gather
construction. First, we can construct \textit{offset-domain} CIGs
\cite[]{GEO49-10-16641674}, when reflectivity depends on
source-receiver offset on the acquisition surface, which is a data
parameter. Second, we can construct \textit{angle-domain} CIGs
\cite[]{GEO55-09-12231234, SEG-1999-08240827, SEG-2000-08300833,
GEO67-03-08830889, XieWu.adcig, GEO68-03-10651074, SEG-2003-08890892,
Fomel.seg.3dadcig, GEO69-05-12831298}, when reflectivity depends on
the angles of incidence at the reflection point, which is an image
parameter. For wave-equation migration, offset-domain CIGs are not a
practical solution, since the information from multiple offsets (or
multiple seismic experiments) mixes in the process of
migration. Furthermore, angle-domain CIGs suffer from fewer artifacts
than offset-domain CIGs, due to the fact that reflectivity
parametrization for angle-gathers occurs after wavefield
reconstruction in the subsurface, as oppose to offset-gathers when
reflectivity parametrization is related to data parameters
\cite[]{GEO69-02-05620575}.

%% 
 % what is the problem with wave-based methods?
 %%

Both wave-equation tomography and wave-equation MVA methods are based
on a fundamental ``small perturbation'' assumption, which requires a
reasonably-good starting model. This requirement represents a drawback
which is responsible for the main difficulty of methods in both
categories. For wave-equation tomography or inversion, we can update
models based on differences between recorded and simulated data. If
the starting background model is not accurate enough, we run the risk
of subtracting wavefields corresponding to different events. Likewise,
for wave-equation MVA, we update the model based on differences
between two images, one simulated in the background model and an
enhanced version of this image. If the enhanced version of the image
goes too far from the reference image, we run the risk of subtracting
events corresponding to different reflectors. This phenomenon is
usually referred-to in the literature as ``cycle skipping'' and
various strategies have been designed to ameliorate this problem,
e.g. by optimal selection of frequencies used for velocity analysis
\cite[]{GEO69-01-02310248,AlbertinFreq}. However, alternative methods
used to evaluate image accuracy, e.g. differential semblance
\cite[]{GEO56-05-06540663,SEG-2003-21322135}, have the best potential
to ameliorate this situation. In this case wave-equation MVA analyzes
difference between image traces within common-image gathers which are
likely to be similar enough from one-another such as to avoid the
cycle-skipping problem. Even in this case, the assumption made is that
the nearby traces in a gather are sampled well-enough, i.e. the
seismic events differ by only a fraction of the wavelet, which is a
function of image sampling and frequency content. In practice, there
is no absolute guarantee that nearby events are closely related to one
another, although this is more likely to be true for DSO than it is
for direct image differencing.

%% 
 % what are we discussing about in this paper?
 %%

In this paper, we concentrate on the implementation of the
wave-equation migration velocity analysis operators for various
wave-equation migration configurations. The main objective of the
paper is to derive the linearized operators linking perturbations of
the slowness model to the corresponding perturbations of the seismic
wavefield and image. All our theoretical development is formulated
under the single scattering (Born) approximation applied to acoustic
waves. We begin by describing the MVA operators corresponding to
zero-offset, survey-sinking and shot-record wave-equation migration
frameworks.  We describe the theoretical background for each operator
and emphasize the similarities and differences between the different
operators.  Throughout the paper, we use pseudo-code to illustrate
implementation strategies and data flow for the various wave-equation
MVA operators. Finally, we illustrate the wave-equation MVA operators
with impulse responses corresponding to simple and complex models. We
leave outside the scope of this paper the procedure in which the
discussed operators are used for migration velocity analysis.

\section{Introduction}
Ultrasonic laboratory studies as well as seismic field investigations have shown that many geological materials
and subsurface structures are elastically anisotropic.
It is well known that a shear wave passing through an anisotropic material splits into two mutually
 orthogonal waves, which propagate at different velocities.
Therefore, seismic waves propagate through the earth as a superposition of three body wave modes, one P-wave and two S-waves.
Generally they are polarized neither parallel to nor perpendicular to the direction of wave travel, 
thus are called quasi-P (qP) and quasi-S (qS) waves, with quasi- means similar to but not exactly.
P- and S-waves were originally named for their arrival times,
with P for the first (primary) and S for the second. Today, the indicators P and S are often connected with polarization,
 i.e., P with compressional (or longitudinal) and S with shear (or transverse), with the specification SH and SV for waves
 with transverse displacements in the horizontal and vertical planes, respectively \cite[]{winterstein:1990}.
For vertical transversely isotropic (VTI) media, one often uses this terminology with qP, qSV, and SH, since the first two of
these waves are generally not purely longitudinal and transverse, respectively.

Because seismic anisotropy by nature is an elastic phenomenon, the full elastic wave equation 
is usually more accurate for wavefield extrapolation than the acoustic equation.
However, seismic imaging
using the elastic wave equation involve high computational cost and many
challenges in decoupling wave modes to get
 physically interpretable images of the subsurface \cite[]{dellinger.etgen:1990,zhang.mcmechan:2010,yan.sava:2011,cheng.fomel:2014}.
For real-sized applications of seismic imaging and 
inversion, it is necessary to resort to a simplified description of wave
propagation in anisotropic media.

Pseudoacoustic wave equations are the most common approximations made
to mono-component (mainly pressure) seismic data. They are derived by setting the
qS-wave phase velocity along the symmetry axis to zero  
for VTI or orthorhombic media \cite[]{alkhalifah:2000,
alkhalifah:2003,duveneck:2011}.
Pseudoacoustic wave equations describe the kinematic signatures of qP-waves
   with sufficient accuracy and are simpler than their elastic
   counterparts, which leads to computational savings in practice
   \cite[]{zhou:2006eage,fletcher:2009,zhang.zhang:2011}.
They also have fewer parameters, which is important for inversion.
However, we note several limitations of the acoustic anisotropic wave equation.
First, acoustic approximation does not prevent the propagation of qS-waves in directions other than
   the symmetry axis \cite[]{grechka:2004,zhang:2005}; the residual qS-waves are
   regarded as artifacts in the framework of acoustic modeling, reverse-time migration
   (RTM) and full waveform inversion (FWI)
   \cite[]{alkhalifah:2000,zhang:2009,operto:2009}.
Second, stability analysis based on requiring the stiffness tensor to remain positive definite
\cite[]{helbig:1994} shows that wavefield extrapolation in a pseudo-acoustic TI or orthorhombic medium
can become unstable \cite[]{alkhalifah:2000,grechka:2004,fowler.king:2011}.
Alternatively, qP- and qS-wave propagation can be formally decoupled in the wavenumber domain
to yield pure-mode pseudodifferential equations \cite[]{liu:2009,du:2014}.
Unfortunately, these equations in time-space domain cannot be solved with traditional numerical schemes.
Through factorizing and approximating the qP-qSV dispersion relations or phase velocities,
many authors proposed to simulate propagation of scalar pure-mode waves
using mixed-domain recursive integral operators
\cite[]{etgen:2009,liu:2009,zhang.zhang:2009,crawley:2010,
fowler:2010,chu:2011,pestana:2011,zhan:2012,fomel:2013}
or novel finite-difference operators \cite[]{song:2013}.
To avoid solving the pseudodifferential equation, \cite{xu.zhou:2014}
proposed a nonlinear wave equation for a pseudo-acoustic qP-wave with an auxiliary
scalar operator depending on the material parameters and the phase direction of the
propagation at each spatial location.

 Multicomponent seismic data are increasingly acquired on land and at the ocean
 bottom to better delineate geological structures and to characterize oil and gas
 reservoirs \cite[]{li:1997,thomsen:1999,cary:2001,stewart:2002,hardage:2011}. 
 The development of unconventional reservoirs and the microseismic monitoring of hydraulic fracturing
 has led to more interest in shear waves because microseismic sources emit strong shear energy that is routinely 
 recorded by three-component geophones \cite[]{maxwell} and is widely recognized as
 being useful for locating microseismic events and estimating their focal mechanisms
\cite[]{baig.urbancic:2010,grechka.yaskevich:2014}.
 In fact, anisotropic phenomena are especially noticeable in shear and mode-converted wavefields.
 Therefore, modeling of anisotropic shear waves
may be important both theoretically and practically.
As we know, the pseudoacoustic approximation is not appropriate for qS-waves.
In addition to amplitude errors, the kinematic accuracy of qS-waves is reduced if we use the
existing numerical schemes based on factorizing and approximating the dispersion relations or phase velocities.

In kinematics, there are various forms equivalent to the original elastic wave equations.
In our previous paper (part I), we derived the pseudo-pure-mode wave equation for qP-waves
by applying a particular similarity transformation to the Christoffel equation and demonstrated its features in
describing wave propagation for TI and orthorhombic media.
Except for its application to scalar qP-wave RTM \cite[]{cheng.kang:2014}, the pseudo-pure-mode wave equation
provides new insight into developing approaches for multicomponent qP-wave inversion
\cite[]{djebbi.alkhalifah:2014}.
The same theoretical framework described in part I is applied to qS-waves in this paper.
First we derive the pseudo-pure-mode wave equations for qS-waves in TI media through
new similarity transformations to the original Christoffel equation.
Numerical examples demonstrate the features of the proposed qS-wave propagators in
 2D and 3D TI media.
Then we discuss the dynamic features of the pseudo-pure-mode qS-wave equations and the challenges to
extending them to anisotropic media with lower symmetry.

%\section{Elastic wave mode separation}
Using the Helmholtz decomposition theory \cite[]{morse.feshbach:1953,aki.richards:1980}, 
a vector wavefield $\mathbf{U}=\{U_{x},U_{y},U_{z}\}$ can be decomposed into
a curl-free P-wavefield and a divergence-free S-wavefield:
$\mathbf{U} = \mathbf{U}^{P} + \mathbf{U}^{S}$\old{, in which P-waves}\new{. The P- and S-waves satisfy, respectively,}
\begin{equation}
\label{eq:helmP}
\nabla\times\mathbf{U}^{P} = 0,\qquad \mbox{and} \qquad \nabla\cdot\mathbf{U} = \nabla\cdot\mathbf{U}^{P},
\end{equation}
and \old{S-waves satisfy,}
\begin{equation}
\label{eq:helmS}
\nabla\cdot\mathbf{U}^{S} = 0,\qquad \mbox{and} \qquad \nabla\times\mathbf{U} = \nabla\times\mathbf{U}^{S}.
\end{equation}
These equations imply that \new{the} divergence and curl operations pass P- and S-wave modes respectively.
In the Fourier-domain, equivalent operations are expressed as follows:
\begin{equation}
\label{eq:helmPK}
\tilde{P}(\mathbf{k}) = i\mathbf{k}\cdot\tilde{\mathbf{U}}(\mathbf{k}), \qquad \mbox{and} \qquad \widetilde{S}(\mathbf{k}) = i\mathbf{k}\times\tilde{\mathbf{U}}(\mathbf{k}),
\end{equation}
where $\mathbf{k}=\{k_{x}, k_{y}, k_{z}\}$ represents the wave vector and $\tilde{U}(k_{x}, k_{y}, k_{z})$ is the
 3D wavefield in the wavenumber domain. 
These operations essentially project the elastic wavefield onto the wave vector or its orthogonal directions,
thus separate P- and S-waves successfully.
In anisotropic media, however, qP- and qS-waves are not generally polarized parallel and perpendicular to the wave vector.
\cite {dellinger.etgen:1990} extended wave mode separation to anisotropic media with the following divergence-like and
curl-like operators in the wavenumber-domain,
\begin{equation}
\label{eq:div.curl.like}
\widetilde{qP}(\mathbf{k}) = i\mathbf{a}_{p}(\mathbf{k})\cdot\tilde{\mathbf{U}}(\mathbf{k}), \qquad \mbox{and} \qquad \widetilde{qS}(\mathbf{k}) = i\mathbf{a}_{p}(\mathbf{k})\times\tilde{\mathbf{U}}(\mathbf{k}),
\end{equation}
where $\mathbf{a}_{p}(\mathbf{k})$ stands for the normalized polarization vector of qP wave \new{in the wavenumber domain, calculated 
	from Christoffel equation.}
Note that the second equation of equation~\ref{eq:div.curl.like} separates only the \old{non-qP}\new{shear} part of the
elastic wavefields\new{, which contain the fast and low S-waves, i.e., $qS_{1}$- and $qS_{2}$ modes. 
Unlike the well-behaved qP mode, the two qS modes do not consistently polarize as a function of the propagation direction (or wavenumber) and
thus cannot be designated as SV and SH waves, except in isotropic and TI media \cite[]{winterstein,crampin:1991,dellinger.thesis,zhang.mcmechan:2010}.
In this paper, the approaches to separate and decompose qS-waves are restricted to TI anisotropy.}
\old{Alternatively, we can separate scalar qSV and SH waves in TI media by projecting the elastic wavefield onto their polarization directions using:}

\new{
For TI media, one can separate scalar qSV and SH waves by projecting the 
elastic wavefield onto their polarization directions using}
\begin{equation}
\label{eq:curllikeSV}
\widetilde{qSV}(\mathbf{k}) = i\mathbf{a}_{sv}(\mathbf{k})\cdot\tilde{\mathbf{U}}(\mathbf{k}), \qquad \mbox{and} \qquad \widetilde{SH}(\mathbf{k}) = i\mathbf{a}_{sh}(\mathbf{k})\cdot\tilde{\mathbf{U}}(\mathbf{k}),
\end{equation}
where $\mathbf{a}_{sv}(\mathbf{k})$ and $\mathbf{a}_{sh}(\mathbf{k})$ represent normalized polarization vectors of the qSV and SH waves, respectively.
For heterogeneous TI media, these operations can be expressed as nonstationary filtering in the space domain \cite[]{yan.sava:2009}. 
In fact, the cost may become prohibitive in 3D because it is proportional to the number of grids in the model and the size of each filter \new{ \cite[]{yan.sava:2011}.}

In general, we can determine polarization \new{vectors} by solving the Christoffel equation:
\begin{equation}
\label{eq:christoffel}
(\widetilde{\mathbf{G}}-\rho{V_{n}}^2\mathbf{I})\mathbf{a}_{n}=0,
\end{equation}
where $\widetilde{\mathbf{G}}$ represents the Christoffel tensor in the Voigt notation with $\widetilde{G}_{ij}=c_{ijkl}n_{j}n_{l},
 c_{ijkl}$ as the stiffness tensor, and
$n_{j}$ and $n_{l}$ are the normalized wave vector components in $j$ and $l$ directions, with $i,j,k,l=1,2,3$. The
parameter $V_{n} (n=qP, qS_{1}, qS_{2})$ represents phase velocities of $qP$-, $qS_{1}$- and $qS_{2}$-wave modes. 
 The Christoffel equation poses a standard $3\times3$ eigenvalue problem, the three eigenvalues of which correspond to phase
velocities of the three wave modes and the corresponding eigenvector $\mathbf{a}_{n}$ represents polarization direction of the given mode.
When shear singularities appear, \old{local discontinuity of the polarization} \new{the coincidence of the longitudinal and
transverse polarizations} prevents us from
constructing 3D global operators to separate qSV and SH waves on the base of the Christoffel solution\new{,
and the polarization discontinuity will cause the two modes to leak energy into each other
\cite[]{dellinger.thesis,yan:2009,zhang.mcmechan:2010,yan.sava:2011}.}
Following \cite {yan:2009,yan.sava:2011}, we mitigate the kiss singularity \new{at $k_{z}=\pm1$} in \new{3D} TI media by using relative qP-qSV-SH mode 
polarization orthogonality and scaling the polarizations of the qSV- and SH-waves by $\sin{\phi}$,
 with $\phi$ being the polar angle.

\section{Elastic wave vector decomposition}
Wavefield decomposition aims achieving mode separation and vector decomposition simultan\old{u}\new{e}ously.
On the base of the Helmholtz theory and the theory of anisotropic wave mode separation via the Christoffell equation,
\cite{zhang.mcmechan:2010} develop a new solution to the problem of decomposing an elastic wavefield into P- and S-waves for
isotropic and VTI media. We summarize here only the\new{ir} results used \old{in}\new{for} this study.

For isotropic media, the Helmholtz equations for the P-wave are transformed into the wavenumber-domain as,
\begin{equation}
\label{eq:helmP1K}
\mathbf{k}\times\tilde{\mathbf{U}}^P = \mathbf{0}, \qquad \mbox{and} \qquad \mathbf{k}\cdot\tilde{\mathbf{U}} = \mathbf{k}\cdot\tilde{\mathbf{U}}^P.
\end{equation}
From these \new{equations}, the vector decomposition equation of the separated P-wave is given by:
\begin{equation}
\label{eq:decomPK}
\tilde{\mathbf{U}}^P(\mathbf{k}) = \bar{\mathbf{k}}[\bar{\mathbf{k}}\cdot\tilde{\mathbf{U}(\mathbf{k})}].
\end{equation}
where $\bar{\mathbf{k}}$ represents the normalized wave vector.

In a TI medium, equation~\ref{eq:decomPK} is extended to separate and decompose qP-wave by substituting $\mathbf{a}_{p}$ for $\bar{\mathbf{k}}$,
\begin{equation}
\label{eq:decomPKTI}
\tilde{\mathbf{U}}^{qP}(\mathbf{k}) = \mathbf{a}_{p}(\mathbf{k})[\mathbf{a}_{p}(\mathbf{k})\cdot\tilde{\mathbf{U}}(\mathbf{k})].
\end{equation}
Similar\old{ly,} \new{equations}
\begin{equation}
\label{eq:decomSVKTI}
\tilde{\mathbf{U}}^{qSV}(\mathbf{k}) = \mathbf{a}_{sv}(\mathbf{k})[\mathbf{a}_{sv}(\mathbf{k})\cdot\tilde{\mathbf{U}}(\mathbf{k})],
\end{equation}
and
\begin{equation}
\label{eq:decomSHKTI}
\tilde{\mathbf{U}}^{SH}(\mathbf{k}) = \mathbf{a}_{sh}(\mathbf{k})[\mathbf{a}_{sh}(\mathbf{k})\cdot\tilde{\mathbf{U}}(\mathbf{k})],
\end{equation}
\new{are proposed to }decompose qSV and SH waves using their \old{owns}\new{respective} polarization vectors\old{, respectively}.
Note that vector decomposition satisfies the linear superposition relation
 $\tilde{\mathbf{U}} = \tilde{\mathbf{U}}^{qP} + \tilde{\mathbf{U}}^{SH} + \tilde{\mathbf{U}}^{qSV}$, 
and the separated wavefields are orthogonal to one another and have the same amplitude, phase, and physical units as the input wavefields.

\section{Phase velocity and polarization characteristics}

Following \cite{carcione:2007},
we denote the spatial variables $x$, $y$ and $z$ of a Cartesian system by
the indices $i, j,$ \ldots $=1$, $2$ and $3$, respectively, the position vector by
$\mathbf{x}$, a partial derivative with respect to a variable $x_i$ with $\partial_i$,
and the first and second time derivatives with $\partial_t$ and $\partial_{tt}$.
Matrix transposition is denoted by the superscript $"\top"$. We also denote
the scalar and matrix products by the symbol $"\cdot"$, and the gradient operator by $\bigtriangledown$.
%The Einstein convention of repeated indices is
%assumed unless otherwise specified.

The wave equation in a general heterogeneous anisotropic medium can be expressed as
\begin{equation}
\label{eq:elastic}
\rho{\partial_{tt}\mathbf{u}} = [{\bigtriangledown}{\mathbf{C}{\bigtriangledown}^{\top}}]\mathbf{u} + \mathbf{f},
\end{equation}
where $\mathbf{u}=(u_x,u_y,u_z)^{\top}$ is the particle displacement vector,
$\mathbf{f}=(f_x,f_y,f_z)^{\top}$ represents
 the force term, $\rho$ the density, $\mathbf{C}$ the matrix representing the stiffness tensor in a
 two-index notation called the “Voigt recipe”. The gradient operator has
 the following matrix representation:
\begin{equation}
\label{eq:grad}
\tensor{\bigtriangledown} =
\begin{pmatrix}{\partial}_x &0 &0 &0& {\partial}_z & {\partial}_y \cr
0& {\partial}_y &0 & {\partial}_z  &0 & {\partial}_x \cr
0& 0& {\partial}_z & {\partial}_y & {\partial}_x &0 \end{pmatrix}.
\end{equation} 
Neglecting the source term, a plane-wave analysis of the elastic wave equation yields the 
Christoffel equation,
\begin{equation}
\label{eq:chris1}
\widetilde{\mathbf{\Gamma}}\widetilde{\mathbf{u}} = \rho{\omega}^2\widetilde{\mathbf{u}},
\end{equation}
where $\omega$ is the angular frequency and
$\widetilde{\mathbf{u}}=(\widetilde{u}_x,\widetilde{u}_y,\widetilde{u}_z)^{\top}$
is the wavefield in Fourier domain; the wavenumber-domain counterpart of the gradient
operator is written as
\begin{equation}
\label{eq:wavenumber}
\widetilde{\mathbf{L}}=
\begin{pmatrix}k_x & 0 &0 &0 & k_z & k_y \cr
         0 & k_y & 0 & k_z &0 & k_x \cr
         0 & 0 & k_z & k_y & k_x &0\end{pmatrix},
\end{equation}
in which the propagation direction is specified by the wave vector
$\mathbf{k}=(k_x,k_y,k_z)^{\top}$, 
and the symmetric Christoffel matrix $\widetilde{\Gamma} =
\widetilde{\mathbf{L}}\mathbf{C}\widetilde{\mathbf{L}}^{\top}$ satisfies:
\begin{equation}
\label{eq:gamma}
\widetilde{\Gamma} = 
 \begin{pmatrix} C_{11}{k_x}^2 + C_{66}{k_y}^2 + C_{55}{k_z}^2  &
 (C_{12}+C_{66}){k_x}{k_y}    &(C_{13}+C_{55}){k_x}{k_z} \cr
 (C_{12}+C_{66}){k_x}{k_y} & C_{66}{k_x}^2+C_{22}{k_y}^2+C_{44}{k_z}^2
 &(C_{23}+       C_{44}){k_y}{k_z} \cr
 (C_{13}+C_{55}){k_x}{k_z} & (C_{23}+C_{44}){k_y}{k_z} &
 C_{55}{k_x}^2+C_{44}{k_y}^2+  C_{33}{k_z}^2\end{pmatrix}.
\end{equation}
The squared phase (or normal) velocities ${V^2}_q (q=1,2,3)$ are eigenvalues of the 
Christoffel matrix. The inequalitities
\begin{equation}
\label{eq:v1v2v3}
{V_1(\mathbf{k})}\geq{V_2(\mathbf{k})}\geq{V_2(\mathbf{k})}
\end{equation}
establish the types of waves.
Except for anomalous cases of elastic anisotropy, which are of little interest in geophysics,
the qP-wave ($q=1$) usually is faster than qS-waves ($q=2,3$)
and the equation
\begin{equation}
\label{eq:v1v2}
{V_1(\mathbf{k})}={V_2(\mathbf{k})}
\end{equation}
is a great rarity \cite[]{yu:1993}. However, the equation
\begin{equation}
\label{eq:v2v3}
{V_2(\mathbf{k})}={V_3(\mathbf{k})}
\end{equation}
is a common event because the phase velocity surfaces (corresponding to qS1 and qS2 waves) can touch or intersect each other \cite[]{musgrave}.
Directions of wave normals along which the two phase velocities are equal to each other are called acoustic axes or singularity directions. 
For the shear singularities, the Christoffel matrix is degenerate.
\cite{crampin:1991} distinguished three kinds of singularity: point, kiss and line.
In point and kiss singularities, the phase velocity surfaces touch at a single point, while in line singularities                      
they intersect.
Generally, inserting a nondegenerate eigenvalue back into the Christoffel equation gives ratios of
the components of $\mathbf{u}$, 
which specify polarization along a given phase direction for a given wave mode.
The polarization consists of the geometrical properties of the particle motion, including trajectory shape and spatial orientation,
but excludes magnitudes of the motion.
Polarization of an isolated body wave in a noise-free perfectly elastic medium is linear \mbox{\cite[]{winterstein:1990}}.
In isotropic media, polarizations of such body waves are either parallel to the direction of wave travel, for P-waves,
or perpendicular to it, for S-waves. The polarization vectors of the S-wave may take an arbitrary orientation in the 
plane orthogonal to the P-wave polarization vector.
In anisotropic media, however, the polarizations are often neither parallel nor
perpendicular to the direction of wave propagation.
Whether the medium is anisotropic or not, the polarizations of the three wave modes are always mutually orthogonal for a given propagation direction.
So we may separate the elastic wavefield into single-mode scalar fields using the polarization-based projection:
\begin{equation}
\label{eq:sep}
\widetilde{w}=i\mathbf{a}_w\cdot\widetilde{\mathbf{u}},
\end{equation}
with $\mathbf{a}_w$ representing the normalized polarization vector of the given mode
$w=\{qP, qS_1, qS_2\}$ \cite[]{dellinger.thesis}.

In practice, horizontally polarized (or SH) and vertically polarized (or SV) are
likely to be the most useful S-wave modes when consideration is restricted to isotropic or TI media,
in which all rays lie in symmetry planes.
For isotropic media, the above polarization-based projection is material-independent,
because the following vectors related to the wave vector $\mathbf{k}$, i.e.,
\begin{equation}
\label{eq:e123}
\mathbf{e}_1=(k_x,k_y,k_z)^{\top},\qquad\mathbf{e}_2=(-k_y,k_x,0)^{\top}\qquad\mbox{and}\qquad\mathbf{e}_3 =(k_xk_z,k_yk_z,-(k^2_x+k^2_y))^{\top},
\end{equation}
indicate the polarization direction of pure P-, SH- and SV-wave, respectively. 
On the contrary, for anisotropic media, the polarization-based projection depends on local material parameters,
because the polarizations are generally specified by the eigenvectors of the original Christoffel equation.
For a VTI medium, the stiffness coefficients satisfy: $C_{12}=C_{11}-2C_{66}$, $C_{22}=C_{11}$, $C_{23}=C_{13}$ and $C_{55}=C_{44}$.
In this case, people still prefer to designate the qS-waves as SH-like and SV-like modes due to the following fact:
The use of qS1 and qS2 distinguished by phase velocities does not always give continuous polarization surfaces,
while the use of SV and SH distinguished by polarization does, except at the kiss singularity at $k_x=k_y=0$
\cite[]{crampin.yedlin:1981,zhang.mcmechan:2010}.
In this case, the SH-waves polarize perpendicular to the symmetry plane and are pure,
and the SV-waves polarize in symmetry planes and are usually quasi-shear.
As shown in Figure~\ref{fig:P_last,SV_last,SH_last}, for both qP and qSV modes, the polarization directions in a VTI material
deviate from those in the corresponding isotropic (reference) medium in most propagation directions,
but no deviation exists for the SH mode in any direction for either medium.
To our interest, the polarization deviations of qP- or qSV-waves between an ordinary VTI
medium and its isotropic reference are usually very small, although exceptions are possible \cite[]{thomsen:1986,tsvankin.chesnokov:1990}.
In addition, taking vectors $\mathbf{e}_1$, $\mathbf{e}_2$ and $\mathbf{e}_3$ as the three mutually perpendicular polarization vectors
in the unperturbed isotropic medium, approximate formulas for the qP- and qS-wave polarizations 
in an arbitrary anisotropic medium can be developed using perturbation theory \cite[]{cerveny.jech:1982,psencik:1998,farra:2001}.

\inputdir{.}
\multiplot{3}{P_last,SV_last,SH_last}{width=0.72\textwidth}
{
Polarization vectors in a 3D VTI material with $v_{p0}=3.0$ km/s,
$v_{s0}=1.5$ km/s, $\epsilon=0.25$, and $\delta=-0.29$. Its isotropic reference
medium is determined by setting $\epsilon=0$ and $\delta=0$.
One can observe polarization deviations between VTI (red) and its isotropic reference (blue) media for (a) P- and (b) S-waves in
most propagation directions, but no deviation for (c) SH-waves in any direction.}

To provide for more possibilities and flexibility in describing single-mode wave propagation in anisotropic media, \mbox{\cite{cheng.kang:2014}} suggest splitting
the one-step polarization-based projection into two steps, of which the first step implicitly implements partial wave-mode separation
during wavefield extrapolation with a transformed wave equation, while the second step is designed to correct the
projection deviation due to the approximation of polarization directions.
The transformed wave equation (that they called a pseudo-pure-mode qP-wave equation) was derived from the original Christoffel equation through
a similarity transformation aiming to project the displacement wavefield onto the isotropic (reference) polarization direction indicated by $\mathbf{e}_1$.  
In this paper, taking another two orthogonal vectors in equation~\ref{eq:e123}, namely $\mathbf{e}_2$ and $\mathbf{e}_3$,
as the reference polarization directions for SH- and qSV-waves,
we apply the same strategy to derive simplified wave equations for these qS-wave modes in general TI media.

%\input{pseudopure.ti.sh}
\section{Pure-mode SH-wave equation}
Pure SH-waves horizontally polarize in the planes perpendicular to the symmetry axis of VTI media with
$u_z\equiv0$, so we introduce a similarity transformation to the Christoffel matrix 
ignoring the vertical component, i.e.,
\begin{equation}
\label{eq:tansChrisMsh}
\widetilde{\overline{\mathbf{\Gamma}}}_\mathbf{m} =
\mathbf{M}\widetilde{\mathbf{\Gamma}}_2\mathbf{M}^{-1},
\end{equation}
with a generally invertible $2\times2$ matrix $\mathbf{M}$ related to the reference polarization direction $\mathbf{e}_2$:
\begin{equation}
\label{eq:tansMsh}
\mathbf{M}=
\begin{pmatrix}-{k_y} & 0 \cr
	0 & {k_x} \end{pmatrix},
\end{equation}
and
\begin{equation}
\label{eq:gamma2}
\widetilde{\Gamma}_2=\begin{pmatrix}C_{11}{k_x}^2+C_{66}{k_y}^2+C_{44}{k_z}^2&
 (C_{12}+C_{66}){k_x}{k_y} \cr
 (C_{12}+C_{66}){k_x}{k_y} & C_{66}{k_x}^2+C_{22}{k_y}^2+C_{44}{k_z}^2\end{pmatrix}.
\end{equation}
Accordingly, we derive a transformed Christoffel equation,
\begin{equation}
\label{eq:tansChrissh}
\widetilde{\overline{\mathbf{\Gamma}}}_\mathbf{m}\widetilde{\overline{\mathbf{u}}}=
\rho{\omega}^2\widetilde{\overline{\mathbf{u}}},
\end{equation}
for the SH-wave mode:
\begin{equation}
\label{eq:similarTsh}
\widetilde{\overline{\mathbf{u}}}=\mathbf{M}\widetilde{\mathbf{u}}_2,
\end{equation}
in which
$\widetilde{\mathbf{u}}_2=(\widetilde{u}_x,\widetilde{u}_y)^{\top}$ represents the horizontal
components of the original elastic wavefields, and
$\widetilde{\overline{\mathbf{u}}}=(\widetilde{\overline{u}}_x,\widetilde{\overline{u}}_y)^{\top}$
represents the horizontal components of the transformed wavefields.
Note that the matrix $\mathbf{M}$ will be not invertible when $k_x=0$ or/and $k_y=0$.
These special directions don't affect the derivation of the pseudo-pure-mode wave equation for the following reasons:
First, we don't directly project the elastic wavefield into the wavenumber-domain, but instead apply the similarity transformation to the Christoffel equation
and eventually inverse the transformed Christoffell equation back into the time-space-domain.
Second, the original Christofell matrix $\widetilde{\Gamma}_2$ automatically becomes a diagonal matrix in these directions,
so the similarity transformation is not actually needed for the corresponding wavenumber components.

Note the similarity transformation does not change the eigenvalue of the Christoffel
matrix corresponding to the SH-wave and, thus, introduces no kinematic error for this wave mode.
We also can obtain a kinematically equivalent Christoffel equation if $\mathbf{M}$ is constructed using 
the normalized form of $\mathbf{e}_2$ to ensure all spatial frequencies are uniformly scaled.
For a locally smooth medium, applying an inverse Fourier transform to
equation~\ref{eq:tansChrissh}, we obtain a linear second-order system in the time-space domain:
\begin{equation}
\label{eq:tansElasticsh}
\rho\partial_{tt}\overline{\mathbf{u}} =
\overline{\mathbf{\Gamma}}_\mathbf{m}\overline{\mathbf{u}},
\end{equation}
or in its extended form:
\begin{equation}
\label{eq:pseudosh}
\begin{split}
	\rho\partial_{tt}\overline{u}_x &= C_{11}\partial_{xx}{\overline{u}_x}
	+ C_{66}\partial_{yy}{\overline{u}_x}+
	C_{44}\partial_{zz}{\overline{u}_x}
	-(C_{11}-C_{66})\partial_{yy}{\overline{u}_y}, \\
	\rho\partial_{tt}\overline{u}_y &= C_{66}\partial_{xx}{\overline{u}_y}
	+ C_{11}\partial_{yy}{\overline{u}_y} +
	C_{44}\partial_{zz}{\overline{u}_y}
	- (C_{11}-C_{66})\partial_{xx}{\overline{u}_x},
\end{split}
\end{equation}
where $\overline{\mathbf{u}}=(\overline{u}_x, \overline{u}_y)^{\top}$
represents the horizontal components of SH-wave in time-space domain, and
$\overline{\mathbf{\Gamma}}_\mathbf{m}$
 represents the Christoffel differential-operator matrix after the similarity transformation.

Due to the cylindrical symmetry of a TI material, the two equations in equation~\ref{eq:pseudosh} may be summed to 
	produce a scalar wave equation in terms of $\overline{u}$:
\begin{equation}
\label{eq:pseudosh1}
\rho\partial_{tt}\overline{u} =
C_{66}(\partial_{xx}+\partial_{yy}){\overline{u}}+C_{44}\partial_{zz}{\overline{u}},
\end{equation}
with $\overline{u}=\overline{u}_{x}+\overline{u}_{y}$ representing the total horizontal components of the transformed SH-wave fields.
This is consistent with the fact that only $C_{44}$ and $C_{66}$ affect the kinematic signatures of the
SH-wave in VTI media \cite[]{tsvankin:2001}. In addition,  the derived equation naturally reduces to the acoustic wave equation if we apply the isotropic assumption by setting $C_{44}=C_{66}=\rho{V_s}^2$ with $V_s$ representing the velocity of the isotropic shear wave.

%\input{pseudopure.ti.sv}
\section{Pseudo-pure-mode qSV-wave equation}

\subsection{Derivation of pseudo-pure-mode qSV-wave equation}
For the qSV-wave, we should essentially build a projection from the elastic wavefields
$\widetilde{\mathbf{u}}=(\widetilde{u}_x,\widetilde{u}_y, \widetilde{u}_z)^{\top}$ to a pseudo-pure-mode wavefield
$\widetilde{\overline{\mathbf{u}}}=(\widetilde{\overline{u}}_x,\widetilde{\overline{u}}_y,
\widetilde{\overline{u}}_z)^{\top}$.
Naturally, we may introduce the following similarity transformation to the Christoffel
matrix, i.e.,
\begin{equation}
\label{eq:tansChrisMsv}
\widetilde{\overline{\mathbf{\Gamma}}}_\mathbf{n}=\mathbf{N}\widetilde{\mathbf{\Gamma}}\mathbf{N}^{-1},
\end{equation}
with the intuitive projection matrix $\mathbf{N}$ defined by the reference polarization direction $\mathbf{e}_3$:
\begin{equation}
\label{eq:tansMsv0}
\mathbf{N}=
\begin{pmatrix}{k_xk_z} & 0 &0 \cr
         0 & {k_yk_z} &0 \cr
         0 & 0 & {-(k^2_x+k^2_y)}\end{pmatrix},
\end{equation}
or its normalized form.
However, the resulting pseudo-pure-mode wave equation is very complicated and contains mixed
derivatives of time and space. To keep them simple, an intermediate
wavefield
${\widetilde{\overline{\mathbf{u}}}}'=(\widetilde{u}_x,\widetilde{u}_y,{\widetilde{u}}'_z)^{\top}$
is defined by
\begin{equation}
\label{eq:zcomp}
\qquad \widetilde{\overline{u}}_z=(k^2_x+k^2_y){\widetilde{\overline{u}}}'_z.
\end{equation}
So we project the vector displacement wavefields using:
\begin{equation}
\label{eq:similarTsv}
{\widetilde{\overline{\mathbf{u}}}}' =
{\mathbf{N}}'\widetilde{\mathbf{u}},
\end{equation}
with an intermediate projection matrix:
\begin{equation}
\label{eq:tansMsv}
{\mathbf{N}}'=
\begin{pmatrix}{k_xk_z} & 0 &0 \cr
         0 & {k_yk_z} &0 \cr
         0 & 0 & -1\end{pmatrix}.
\end{equation}
Accordingly, we apply the similarity transformation using ${\mathbf{N}}'$ to
equation \ref{eq:chris1} and finally get an equivalent Christoffel equation:
\begin{equation}
\label{eq:tansChrissv}
{\widetilde{\overline{\mathbf{\Gamma}}}}'_\mathbf{n}{\widetilde{\overline{\mathbf{u}}}}' =
\rho{\omega}^2{\widetilde{\overline{\mathbf{u}}}}'.
\end{equation}
with
${\widetilde{\overline{\mathbf{\Gamma}}}}'_\mathbf{n}={\mathbf{N}}'\widetilde{\mathbf{\Gamma}}(\mathbf{N}')^{-1}$.

For a locally smooth medium, applying an inverse Fourier transformation to
equation~\ref{eq:tansChrissv}, we obtain another coupled
forth-order linear system:
\begin{equation}
\label{eq:tansElasticsv}
\rho\partial_{tt}{\overline{\mathbf{u}}}'={\overline{\mathbf{\Gamma}}}'_\mathbf{n}{\overline{\mathbf{u}}}',
\end{equation}
or in its extended form:
\begin{equation}
\label{eq:pseudosv}
\begin{split}
	\rho\partial_{tt}\overline{u}_x&=(C_{11}\partial_{xx}+C_{66}\partial_{yy}+ C_{44}\partial_{zz}){\overline{u}_x}
	+(C_{11}-C_{66})\partial_{xx}{\overline{u}_y} 
	+(C_{13}+C_{44})\partial_{xx}\partial_{zz}{{\overline{u}}'_z}, \\
	\rho\partial_{tt}\overline{u}_y&=(C_{11}-C_{66})\partial_{yy}{\overline{u}_x}
	+ (C_{66}\partial_{xx}+C_{11}\partial_{yy}+C_{44}\partial_{zz}){\overline{u}_y}
	+(C_{13}+C_{44})\partial_{yy}\partial_{zz}{{\overline{u}}'_z}, \\
	\rho\partial_{tt}{\overline{u}}'_z&=(C_{13}+C_{44})\overline{u}_x
	+(C_{13}+C_{44})\overline{u}_y
	+C_{44}(\partial_{xx}+\partial_{yy})\overline{u}'_z+C_{33}\partial_{zz}{\overline{u}}'_z.
\end{split}
\end{equation}
where
${\overline{\mathbf{u}}}'=(\overline{u}_x,\overline{u}_y,{\overline{u}}'_z)^{\top}$
is an intermediate wavefield in the time-space domain, and
${\overline{\mathbf{\Gamma}}}'_\mathbf{n}$
 represents the corresponding Christoffel differential-operator matrix after the similarity
 transformation.
 The intermediate wavefield has the same horizontal components but a different vertical
 component of the pseudo-pure-mode wavefield $\overline{\mathbf{u}}=(u_x, u_y,
 u_z)^{\top}$.
 Equation \ref{eq:zcomp} indicates that the vertical component satisfies:
\begin{equation}
\label{eq:zcomp1}
\overline{u}_z=-(\partial_{xx}+\partial_{yy}){\overline{u}}'_z.
\end{equation}

Due to the symmetry property of a VTI material, we may sum the horizontal components and replace the vertical component
with the relation given in equation~\ref{eq:zcomp1}, and finally obtain a simpler second-order system that honors the kinematics of both qP- and qSV-waves:
\begin{equation}
\label{eq:pseudosvxy}
\begin{split}
\rho\partial_{tt}\overline{u}_{xy}&=C_{11}(\partial_{xx}+\partial_{yy})\overline{u}_{xy}+C_{44}\partial_{zz}{\overline{u}}_{xy}-(C_{13}+C_{44})\partial_{zz}{\overline{u}_z},\\
\rho\partial_{tt}\overline{u}_z&=-(C_{13}+C_{44})(\partial_{xx}+\partial_{yy})\overline{u}_{xy}+C_{44}(\partial_{xx}+\partial_{yy})\overline{u}_z+C_{33}\partial_{zz}\overline{u}_z,
\end{split}
\end{equation}
with $\overline{u}_{xy}=\overline{u}_x+\overline{u}_y$.
Note that pure SH-waves always polarize in the planes perpendicular to the symmetry axis with the polarization direction indicated by $\mathbf{e}_2$,
which implies $(k_xk_z)\widetilde{u}_{x}+(k_yk_z)\widetilde{u}_{y}\equiv0$, i.e., $\overline{u}_{xy}\equiv0$, for the SH-wave.
Therefore, the partial summation (after the similarity transformation) automatically removes the SH component from the transformed wavefields.
As a result, there are no terms related to $C_{66}$ any more in equation~\ref{eq:pseudosvxy}.
In order to produce a pseudo-pure-mode scalar qSV-wave field, we sum all components of the transformed wavefields, namely
\begin{equation}
\label{eq:scalalr}
\overline{u}=\overline{u}_{xy}+\overline{u}_{z}.
\end{equation}

For a 2-D VTI medium, equation ~\ref{eq:pseudosvxy} reduces to the following form:
\begin{equation}
\label{eq:pseudosvx}
\begin{split}
\rho\partial_{tt}\overline{u}_x&=C_{11}\partial_{xx}\overline{u}_x+C_{44}\partial_{zz}{\overline{u}}_x
-(C_{13}+C_{44})\partial_{zz}{\overline{u}_z},\\
\rho\partial_{tt}\overline{u}_z&=-(C_{13}+C_{44})\partial_{xx}\overline{u}_x
+C_{44}\partial_{xx}\overline{u}_z+C_{33}\partial_{zz}\overline{u}_z.
\end{split}
\end{equation}
In fact, we can derive the same pseudo-pure-mode wave equation for a 2-D qSV-wave by
projecting the 2-D Christoffel matrix onto a reference vector
$\mathbf{e}'_3=(k_z, -k_x)^{\top}$.
Similarly, a 2D pseudo-pure-mode scalar qSV-wave field is given by the summation: $\overline{u}=\overline{u}_{x}+\overline{u}_{z}$.

If we apply the isotropic assumption by setting $C_{11}=C_{33}$ and $C_{13}+C_{44}=C_{33}-C_{44}$,
and sum the two equations in equation~\ref{eq:pseudosvxy}, we get the scalar wave equation:
\begin{equation}
\rho{\partial_{tt}\overline{u}} = 
C_{44}(\partial_{xx}+\partial_{yy}+\partial_{zz}){\overline{u}},
\end{equation}
with $\overline{u}=\overline{u}_{xy}+\overline{u}_{z}$ representing a shear wave field, and $C_{44}=\rho{V_s}^2$ with $V_s$
representing the propagation velocity of the isotropic shear wave.

The derived pseudo-pure-mode qSV-wave equations have some interesting and valuable features.
First, the projection using matrix $\mathbf{N}$ yields wave-mode separation to some
extent, because the chosen projection direction, $\mathbf{e}_3$, represents the polarization direction of
the SV-wave in an isotropic medium.
As investigated by \mbox{\cite{tsvankin.chesnokov:1990}} and \mbox{\cite{psencik:1998}}, and also demonstrated in Figure 1b,
the difference in polarization directions between isotropic and VTI media is
generally quite small in most propagation directions for SV-waves.
Therefore, considering the mode separator (namely equation~\ref{eq:sep}) and the small projection deviation,
summing all the pseudo-pure-mode wavefield components in equation \ref{eq:pseudosvxy} or \ref{eq:pseudosvx}
partially achieves wave-mode separation and produces a scalar wavefield dominated by
the energy of qSV-waves. This will be demonstrated in the numerical examples.
Second, the pseudo-pure-mode wave equations are easier to calculate
than the original elastic wave equation because they have no terms of mixed partial
derivatives. More importantly, the summation of the horizontal components further simplifies the
wave equations and reduces the number of parameters needed for scalar qSV-wave extrapolation. 
These features are undoubtedly useful for performing multicomponent seismic imaging and
inversion that mainly use wavefield kinematics when it is necessary to include anisotropy.

\subsection{Removing of the residual qP-waves}

The pseudo-pure-mode qSV-wave equations are derived by using a similarity transformation
that projects the vector displacement wavefield onto the isotropic reference of
the qSV-wave's polarization direction. 
As demonstrated in Figure 1b, even for a very strong VTI medium, the difference bewteen the two directions
is generally quite small in most propagation directions.
However, this difference does result in some qP-wave energy remaining in the pseudo-pure-mode
scalar qSV-wave fields. To remove the residual qP-waves, we have to correct the projection deviations before summing the pseudo-pure-mode wavefield components.
 For heterogeneous VTI media, this can be implemented through nonstationary spatial
 filtering defined by the projection deviations \cite[]{cheng.kang:2014}.

The filters can be constructed once the qSV-wave polarization directions
are determined by solving the Christoffel equation based on local medium properties for every grid point.
However, this operation is computationally expensive, especially in 3D heterogeneous TI media.
We may further reduce the computational cost using a mixed-domain integral algorithm
using a low-rank approximation \cite[]{cheng.fomel:2014}.
We shall observe in the examples that the residual qP-waves in the pseudo-pure-mode qSV-wave fields
 are quite weak, even if the anisotropy becomes strong.
As explained in \cite{cheng.kang:2014}, it is not necessary to apply the filtering
 at every time step for many applications, such as RTM.

In the case of transverse isotropy with a tilted symmetry axis (TTI), the elastic tensor loses its simple form
and the terminology ``in-plane polarization" and ``cross-plane polarization" is to be preferred for qSV- and qSH-waves.
The generalization of pseudo-pure-mode wave equation to a TTI medium
involves no additional physics but greatly complicates the algebra. One strategy for deriving the wave
 equations is to locally rotate the coordinate system so that its third axis coincides with
 the symmetry axis; and to make use of the simple form of the wave equation in VTI media (see
 \mbox{\cite{cheng.kang:2014}}).
 Alternatively, we may use some new strategies to derive more numerically stable pseudo-pure-mode wave
 equations for TTI media with strong variations of parameters
 \mbox{\cite[]{zhang:2011,bube:2012}}. Moreover, the filter to correct the projection deviation can
also be constructed with the coordinate rotation. 

%\def\COSlabel{Velocity map and Riemannian coordinate system}
\def\RWElabel{
Migration impulse responses in Riemannian coordinates.
(a) Extrapolation with the $15^\circ$ finite-differences equation.
(c) Extrapolation with the $60^\circ$ finite-differences equation.
(b) Extrapolation with the pseudo-screen (PSC) equation.
(d) Extrapolation with the Fourier finite-differences (FFD) equation.
}
\def\CARlabel{
Migration impulse responses in Riemannian coordinates after mapping to Cartesian coordinates.
(a) Extrapolation with the $15^\circ$ finite-differences equation.
(c) Extrapolation with the $60^\circ$ finite-differences equation.
(b) Extrapolation with the pseudo-screen (PSC) equation.
(d) Extrapolation with the Fourier finite-differences (FFD) equation.
}

\def\ABlabel{
Coordinate system coefficients defined in \reqs{rwea} and \ren{rweb}.
(a) Parameter $a=s \aa$  in ray coordinates.
(b) Parameter $b=\aa/\jj$ in ray coordinates.
}

\def\COMPlabel{
Comparison of extrapolation in Cartesian and Riemannian coordinates.
(a) Split-step Fourier extrapolation in Cartesian coordinates.
(b) Split-step Fourier extrapolation in Riemannian coordinates.
}

% ------------------------------------------------------------

\section{Examples}
We illustrate the higher-order RWE extrapolators with impulse
responses for two synthetic models.
\par
The first example is based on the Marmousi model
\cite[]{TLE13-09-09270936}. We construct the coordinate system by ray
tracing from a point source at the surface in a smooth version of the
real velocity model. \rFg{M-cos} shows the velocity model with the
coordinate system overlaid, and \rFgs{M-abmRCa}-\rfn{M-abmRCb} show
the coordinate system coefficients $a$ and $b$ defined in \reqs{rwea}
and \ren{rweb}.
%%
\inputdir{marm}
\plot*{M-cos}{width=0.80\textwidth}
{\COSlabel~for the Marmousi example.}
%%
\multiplot{2}{M-abmRCa,M-abmRCb}
{angle=0,width=0.45\textwidth}
{\ABlabel}
%%
\par
The goal of this test model is to illustrate the higher-order
extrapolation kernels in a fairly complex model using a simple
coordinate system. In this way, the coordinate system and the real
direction of wave propagation depart from one-another, thus accurate
extrapolation requires higher order kernels. The coordinate system is
constructed from a point at the location of the wave source. This
setting is similar to the case of extrapolation from a point source in
Cartesian coordinates, where high-angle
\footnote{
If the extrapolation axis is time, the meaning of higher angle
accuracy is not well defined. We can use this terminology to associate
the mathematical meaning of the approximation for the square-root by
analogy with the Cartesian equivalents.  } propagation requires
high-order kernels.
\par
\rFgs{M-migRC-F15}-\rfn{M-migRC-FFD} show impulse responses for a 
point source computed with various extrapolators in ray coordinates
($\qt$ and $\qg$). Panels (a) and (c) show extrapolation with the
$15^\circ$ and $60^\circ$, respectively. Panels (b) and (d) show
extrapolation with the pseudo-screen (PSC) equation, and the Fourier
finite-differences (FFD) equation, respectively. All plots are
displayed in ray coordinates. We can observe that the angular accuracy
of the extrapolator improves for the more accurate extrapolators. The
finite-differences solutions (panels a and c) show the typical
behavior of such solutions for the $15^\circ$ and $60^\circ$ equations
(e.g.  the cardioid for $60^\circ$), but in the more general setting
of Riemannian extrapolation. The mixed-domain extrapolators (panels b
and d) are more accurate the finite-differences extrapolators. The
main differences occur at the highest propagation angles. As for the
case of Cartesian extrapolation, the most accurate kernel of those
compared is the equivalent of Fourier finite-differences.
\par
\rFgs{M-migCC-F15}-\rfn{M-migCC-FFD} show the corresponding plots 
in \rFgs{M-migRC-F15}-\rfn{M-migRC-FFD} mapped in the physical
coordinates. The overlay is an outline of the extrapolation coordinate
system. After re-mapping to the physical space, the comparison of
high-angle accuracy for the various extrapolators is more apparent,
since it now has physical meaning.
\par
\rFgs{M-imgCC}-\rfn{M-migCC-SSF} show a side-by-side comparison of 
equivalent extrapolators in Riemannian and Cartesian coordinates. The
impulse response in \rFg{M-imgCC} shows the limits of Cartesian
extrapolation in propagating waves correctly up to $90^\circ$. The
Riemannian extrapolator in \rFg{M-migCC-SSF} handles much better waves
propagating at high angles, including energy that is propagating
upward relative to the physical coordinates.
%%
\multiplot{4}{M-migRC-F15,M-migRC-PSC,M-migRC-F60,M-migRC-FFD}
{angle=0,width=0.45\textwidth}{\RWElabel}
%%
\multiplot{4}{M-migCC-F15,M-migCC-SSF,M-migCC-F60,M-migCC-FFD}
{angle=0,width=0.45\textwidth}{\CARlabel}
%%
\multiplot{2}{M-imgCC,M-migCC-SSF}
{angle=0,width=0.80\textwidth}{\COMPlabel}
% ------------------------------------------------------------

% ------------------------------------------------------------
The second example is based on a model with a large lateral gradient
which makes an incident plane wave overturn. A small Gaussian anomaly,
not used in the construction of the coordinate system, forces the
propagating wave to triplicate and move at high angles relative to the
extrapolation direction. \rFg{D-cos} shows the velocity model with the
coordinate system overlaid. \rFgs{D-abmRCa}-\rfn{D-abmRCb} show the
coordinate system coefficients, $a$ and $b$ defined in \reqs{rwea} and
\ren{rweb}.
%%
\inputdir{diving}
\plot{D-cos}{width=0.80\textwidth}
{\COSlabel~for the large-gradient model experiment.}
%%
\multiplot{2}{D-abmRCa,D-abmRCb}
{angle=0,width=0.45\textwidth}{\ABlabel}
%%
\par
The goal of this model is to illustrate Riemannian wavefield
extrapolation in a situation which cannot be handled correctly by
Cartesian extrapolation, no matter how accurate an extrapolator we
use. In this example, an incident plane wave is overturning, thus
becoming evanescent for the solution constructed in Cartesian
coordinates. Furthermore, the Gaussian anomaly shown in \rFg{D-abmRCa}
causes wavefield triplication, thus requiring high-order kernels for
the Riemannian extrapolator.
\par
\rFgs{D-migRC-F15}-\rfn{D-migRC-FFD} show impulse responses for an 
incident plane wave computed with various extrapolators in ray
coordinates ($\qt$ and $\qg$). Panels (a) and (c) show extrapolation
with the $15^\circ$ and $60^\circ$ finite-differences equations,
respectively. Panel (b) and (d) show extrapolation with the
pseudo-screen (PSC) equation and the Fourier finite-differences (FFD)
equation, respectively. All plots are displayed in ray coordinates. As
for the preceding example, we observe higher angular accuracy as we
increase the order of the extrapolator. The equivalent FFD
extrapolator shows the highest accuracy of all tested extrapolators.
\par
As in the preceding example, \rFgs{D-migCC-F15}-\rfn{D-migCC-FFD} show
the corresponding plots in \rFgs{D-migRC-F15}-\rfn{D-migRC-FFD} mapped
in the physical coordinates. The overlay is an outline of the
extrapolation coordinate system.
\par
Finally, \rfgs{D-imgCC-bp} and \rfn{D-migCC-SSF} show a side-by-side
comparison of equivalent extrapolators in Riemannian and Cartesian
coordinates. The impulse response in \rFg{D-imgCC-bp} clearly shows
the failure of the Cartesian extrapolator in propagating waves
correctly even up to $90^\circ$. The Riemannian extrapolator in
\rFg{D-migCC-SSF} handles much better overturning waves, including
energy that is propagating upward relative to the vertical direction.
%%
\multiplot{4}{D-migRC-F15,D-migRC-PSC,D-migRC-F60,D-migRC-FFD}
{angle=0,width=0.45\textwidth}{\RWElabel}
%%
\multiplot{4}{D-migCC-F15,D-migCC-PSC,D-migCC-F60,D-migCC-FFD}
{angle=0,width=0.45\textwidth}{\CARlabel}
%%
\multiplot{2}{D-imgCC-bp,D-migCC-SSF}
{angle=0,width=0.80\textwidth}{\COMPlabel}

\section{EXAMPLES}

The first example compares the synthetic elastic displacement and pseudo-pure-mode qSV-wave field
in 2D homogeneous VTI media with different degrees of anisotropy.
Then we demonstrate the computation of 3D synthetic pseudo-pure-mode qSV-wave and pure-mode SH-wave fields for a two-layer VTI model.
Finally, we investigate the performance of the pseudo-pure-mode qSV-wave propagator on the 2D BP TTI model.
We sue a tenth-order explicit finite-difference scheme on regular grids to solve the involved wave equations.
Point sources are located at the centers of the models.

\subsection{2D homogeneous VTI models}

For comparison, we first apply the original elastic wave equation
to synthesize wavefields in a homogeneous VTI medium with weak anisotropy, in which
$v_{p0}=3000 m/s$, $v_{s0}=1500 m/s$, $\epsilon=0.1$, and $\delta=0.05$. 
Figures 2a and 2b display the horizontal and vertical components of the displacement
wavefields at 0.3 s.
Then we try to simulate the propagation of a single-mode qSV-wave using the pseudo-pure-mode qSV-wave equation
(namely equation \ref{eq:pseudosvx}).
 Figure 2c and 2d display the two components of the pseudo-pure-mode qSV-wave fields,
 and Figure 2e displays their summation, i.e., the pseudo-pure-mode scalar qSV-wave
 fields with weak residual qP-wave energy.
We observe that, in most propagation directions, the polarities are almost reversed for the wavefronts of qP-waves
in the two components of the pseudo-pure-mode qSV-wave fields.
This contributes to suppressing the qP-wave energy through summation of the two components. 
Compared with the theoretical wavefront curves (see Figure 2f) calculated
using group velocities
 and angles, pseudo-pure-mode scalar qSV-wave fields have correct kinematics for
  both qP- and qSV-waves.
 We finally remove residual qP-waves by applying the filtering to correct the projection deviation
 and get completely separated scalar qSV-wave fields (Figure 2g).

\inputdir{homovti.eta0.05}
\multiplot{7}{Elasticx,Elasticz,PseudoPureSVx,PseudoPureSVz,PseudoPureSV,WF,PseudoPureSepSV}{width=0.2\textwidth}
{
Synthesized wavefields in a VTI medium with weak anisotropy: (a) x- and
(b) z-components synthesized by original elastic wave equation; (c) x- and
(d) z-components synthesized by pseudo-pure-mode qSV-wave equation; (e) pseudo-pure-mode scalar qSV-wave fields; 
(f) kinematics of qV- and qSV-waves and (g) separated scalar qSV-wave fields.
}

\inputdir{homovti.eta0.5}

Next, we consider wavefield modeling in a homogeneous VTI medium with strong
anisotropy,
in which $v_{p0}=3000 m/s$, $v_{s0}=1500 m/s$, $\epsilon=0.25$, and $\delta=-0.25$.
Figure 3 displays the wavefield snapshots at 0.3 s synthesized both by using the original
elastic wave equation
and the pseudo-pure-mode qSV-wave equation. Note that the pseudo-pure-mode
  qSV-wave equation still accurately
  represents the qP- and qSV-waves' kinematics. Although the residual qP-wave energy
  becomes stronger when
  the degree of anisotropy increases, the filtering step still removes them effectively.
It takes CPU times of 0.15 and 0.06 seconds to extrapolate the elastic and pseudo-pure-mode qSV-wave fields for one time-step, respectively.
But it takes about 8.2 seconds to remove the residual qP-wave from the pseudo-pure-mode qSV-wave fields using wavenumber-domain filtering.
In fact, separating the pure-mode wavefield from the elastic and pseudo-pure-mode wavefields has almost the same computational cost for TI media \cite[]{cheng.kang:2014}.

  \multiplot{7}{Elasticx,Elasticz,PseudoPureSVx,PseudoPureSVz,PseudoPureSV,WF,PseudoPureSepSV}{width=0.2\textwidth}
  {
  The same plots as Figure 2 but for a VTI medium with stronger anisotropy.
   }

%%%%%%%%%%%%%%%%%%%%%%%%%%%%%%%%%%%%%%%%%%%%%%%%%%%%%%%%%%%%%%%%%%%%%%%%%%%%%%%%%%%%%%%%%%%%%%%%%%%%%%
\subsection{3D two-layer VTI model}

\inputdir{twolayer3dvti}

Figure 4 shows an example of simulating the propagation of pseudo-pure-mode qSV-wave fields in a 3D two-layer
 VTI  model (see Figure 4a), with
$v_{p0}=2500 m/s$, $v_{s0}=1200 m/s$, $\epsilon=0.25$, $\delta=-0.25$ and $\gamma=0.3$ in the first layer, and
$v_{p0}=3600 m/s$, $v_{s0}=1800 m/s$, $\epsilon=0.2$, $\delta=0.1$ and $\gamma=0.05$ in the second layer.
We propagate the 3D pseudo-pure-mode qSV-wave fields using equation~\ref{eq:pseudosvxy}.
Figure 4d displays the pseudo-pure-mode scalar qSV-wave fields resulting from the summation of 
the horizontal (Figure 4b) and vertical (Figure 4c) components, namely $\overline{u}_{xy}$ and $\overline{u}_z$.
We see that the qS-waves dominate the scalar wavefields in energy. 
As shown in Figure 5, we also obtain pure-mode scalar SH-wave fields either using the summation of the horizontal components
synthesized by using the pseudo-pure-mode wave equation~\ref{eq:pseudosh} or directly using the scalar wave equation, i.e., equation~\ref{eq:pseudosh1}. 

\multiplot{4}{vp0Interf,PseudoPureSVxyInterf,PseudoPureSVzInterf,PseudoPureSVInterf}{width=0.45\textwidth}
{
Synthesized wavefield snapshots in a 3D two-layer VTI model using equation~\ref{eq:pseudosvxy} : (a) vertical velocity of qSV-wave,
(b) horizontal component $\overline{u}_{xy}$ and (c) vertical component $\overline{u}_z$ of the pseudo-pure-mode qSV-wave fields,
(d) pseudo-pure-mode scalar qSV-wave fields. The dash line indicates the interface.
}

\multiplot{3}{SHxInterf,SHyInterf,SHInterf}{width=0.45\textwidth}
%\plot{SH}{width=0.45\textwidth}
{
Synthesized wavefield snapshots in a 3D two-layer VTI model using equation~\ref{eq:pseudosh}: (a) x- and (b) y-components
of the pseudo-pure-mode wavefields, (c) pure-mode scalar SH-wave fields calculated as the summation of the two horizontal 
components of the pseudo-pure-mode wavefields. Note that the same scalar wavefields are obtained if we directly
use the scalar wave equation for SH-waves, namely equation~\ref{eq:pseudosh1}.
}

\subsection{BP 2007 TTI model}

Finally, we demonstrate pseudo-pure-mode qSV-wave propagation in the 2D BP TTI model (see Figure 6).
The space grid size is 12.5 m and the time step is 1 ms for
high-order finite-difference operators. Here the vertical velocities for the qSV-wave are set to
 half of the qP-wave velocities.
Figure 7 displays snapshots of wavefield components at the time of 1.4s
synthesized by using the original elastic wave equation and the pseudo-pure-mode qSV-wave equation.
In the elastic wavefields, we observe strong scattered and mode-converted energy in the region with a rapidly varying 
anisotropic symmetry axis direction.
For comparison, in Figures 7c and 7d, we also show the separated qP- and qSV-wave scalar fields
	obtained using the approach proposed by \cite{cheng.fomel:2014}.
Not that in the pseudo-pure-mode qSV-wave fields (see Figure 7g), the incident qP-waves as well as
scattered and converted qP-waves are effectively suppressed.
The spatial filtering appears to remove residual qP-waves and accurately separates qSV-wave data
(including the converted qP-qSV waves) from the pseudo-pure-mode wavefields in this complex model (Figure 7h).
Vertical slices through the scalar fields (Figure 8) provide further proof to evaluate the performance
of the proposed qS-wave propagators.
As we observed, in heterogeneous rough zones with strong variations in tilt angle,
there are differences between the elastic and pseudo-pure-mode qSV-wave fields.
Fortunately, the pseudo-pure-mode qSV-wave equation still captures the shear wave kinematics to a great extent.
For a single time-step, it respectively takes CPU times of 2.71 and 1.22 seconds to extrapolate the elastic
and pseudo-pure-mode qSV-wave fields,
and about 7.50 seconds to separate the qSV-wave fields from both wavefields using low-rank approximate
mixed-domain integral operations based on the qSV-wave's polarization directions \cite[]{cheng.fomel:2014}.

\inputdir{bptti2007}
\multiplot{4}{vp0,epsi,del,the}{width=0.3\textwidth}
{
Partial region of the 2D BP TTI model: (a) vertical qP-wave velocity, Thomsen coefficients
(b) $\epsilon$ and (c) $\delta$, and (d) the tilt angle $\theta$. 
}

\multiplot{8}{Elasticx,Elasticz,ElasticSepP,ElasticSepSV,PseudoPureSVx,PseudoPureSVz,PseudoPureSV,PseudoPureSepSV}{width=0.25\textwidth}
{
Synthesized wavefield snapshots on BP 2007 TTI model using original elastic wave equation and pseudo-pure-mode 
qSV-wave equation respectively: 
(a) x- and (b) z-components synthesized by elastic wave equation; 
(c) scalar qP- and (d) scalar qSV-wave fields separated from the elastic wavefield; 
(e) x- and (f) z-components synthesized by pseudo-pure-mode qSV-wave equation; 
(g) pseudo-pure-mode scalar qSV-wave field and (h) pure-mode scalar qSV-wave field separated from the pseudo-pure-mode qSV-wave field.
}
\multiplot*{3}{ElasticSepSVwave,PseudoPureSVwave,PseudoPureSepSVwave}{width=0.45\textwidth}
{
Vertical slices through the scalar wavefields at $x=50.5$ km in Figure 7: (a) qSV-wave separated from the elastic wavefield;
 (b) pseudo-pure-mode qSV-wave; (c) pure qSV-wave separated from the pseudo-pure-mode wavefield.
}


%\section{Discussion}
As discussed in one of the preceding sections, time-shift 
gathers consist of linear events with slopes corresponding
to the local migration velocity.
In contrast, space-shift gathers consist of events focused 
at $\hh=0$.
Those events can be mapped to the angle-domain using 
transformations \ren{angX} and \ren{angT}, respectively.

In order to understand the angle-domain mapping, 
we consider a simple synthetic in which we model
common-image gathers corresponding to incidence at
a particular angle.
The experiment is depicted in Figure~\ref{fig:ttest}
for time-shift imaging, and in Figure~\ref{fig:htest}
for space-shift imaging.
For this experiment, the sampling parameters are the following:
$\Delta z=0.01$~km,
$\Delta h=0.02$~km, and
$\Delta \tau=0.01$~s.

A reflection event at a single angle of incidence 
maps in common-image gathers as a line of a given slope.
The left panels in Figures~\ref{fig:ttest} and \ref{fig:htest}
show $3$ cases, corresponding to angles of
$0^\circ$, $20^\circ$ and $40^\circ$.
Since we want to analyze how such events map to angle, 
we subsample each line to $5$ selected samples lining-up
at the correct slope.

The middle panels in Figures~\ref{fig:ttest} and \ref{fig:htest}
show the data in the left panels after slant-stacking
in $z-\tt$ or $z-h$ panels, respectively.
Each individual sample from the common-image gathers maps
in a line of a different slope intersecting in a point.
For example, normal incidence in a time-shift gather
maps at the migration velocity $\nu=2$~km/s 
(Figure~\ref{fig:ttest} top row, middle panel), and
normal incidence in a space-shift gather maps at
slant-stack parameter $\tan \t=0$.

The right panels in Figures~\ref{fig:ttest} and \ref{fig:htest}
show the data from the middle panels after mapping to angle
using \reqs{angT} and \ren{angX}, respectively.
All lines from the slant-stack panels map into curves
that intersect at the angle of incidence.

We note that all curves for the time-shift angle-gathers
have zero curvature at normal incidence.
Therefore, the resolution of the time-shift mapping around 
normal incidence is lower than the corresponding
space-shift resolution. 
However, the storage and computational cost of time-shift 
imaging is smaller than the cost of equivalent space-shift 
imaging. 
The choice of the appropriate imaging condition depends on 
the imaging objective and on the trade-off 
between the cost and the desired resolution.

% ------------------------------------------------------------
\inputdir{icomp}
% ------------------------------------------------------------

% ------------------------------------------------------------
\plot{ttest}{width=6.0in}
{Image-gather formation using time-shift imaging.
Each row depicts an event at 
$0 ^\circ$ (top),
$20^\circ$ (middle), and
$40^\circ$ (bottom).
Three columns correspond to
subsampled time-shift gathers (left),
slant-stacked gathers (middle), and
angle-gathers (right).}
% ------------------------------------------------------------

% ------------------------------------------------------------
\plot{htest}{width=6.0in}
{Image-gather formation using space-shift imaging.
Each row depicts an event at 
$0 ^\circ$ (top),
$20^\circ$ (middle), and
$40^\circ$ (bottom).
Three columns correspond to
subsampled space-shift gathers (left),
slant-stacked gathers (middle), and
angle-gathers (right).}
% ------------------------------------------------------------

\section{Discussion}

\subsection{Kinematic and dynamic accuracy}
The similarity transformation to the Christoffel equation preserves the kinematics of the qS-waves,
but inevitably change the phases and amplitudes in their wavefields.
Accordingly, the pseudo-pure-mode wave equations may change the radiation from a point source 
(as demonstrated in the examples),
and even distort the amplitude variation with offset (AVO).
In other word, they do not honor the dynamic elasticity of the waves in real media.
In fact, other simplified forms of the elastic wave equation, such as acoustic or pseudo-acoustic wave equations 
and the pure-mode approximate wave equations, have similar limitations
\cite[]{barnes:2009,operto:2009,cheng.kang:2014,shang:2015}.
For heterogeneous rough media, i.e., when scales for variations in the elastic parameters are small compared with the wavelengths
	of the wavefield, the acoustic approximation is no longer reliable \cite[]{cance:2015}.
The pseudo-pure-mode wave equations have similar limitations for shear-wave modeling in high-contrast TI media. 
However, these limitations are not doomed to be catastrophic, because velocity models 
containing high-wavenumber components are rarely involved at most stages of seismic imaging and inversion for real data.

\subsection{Challenge for anisotropy with lower symmetry}
Unlike the well-behaved qP-wave mode, the qS-wave modes do not consistently polarize as a function of propagation direction,
and thus cannot be designated as SV- and SH-waves, except in isotropic and TI media \cite[]{winterstein:1990}.
To demonstrate the difficulties of extending the methodology in this paper to anisotropic media with symmetry lower than TI,
we first compare the polarization features of qS-waves in typical TI and orthorhombic anisotropic rocks.
Figure 9 shows polarizations of qS1- and qS2-waves in a VTI material - Mesaverde shale \cite[]{thomsen:1986},
which has the parameters $v_{p0}=3.749$ km/s, $v_{s0}=2.621$ km/s, $\epsilon=0.225$, $\delta=0.078$, and $\gamma=0.100$.
The polarization directions are either horizontal or vertical (in the symmetry plane), 
so that we can definitely designate qS-waves as qSV- and SH-wave modes, except at the kiss singularity.
Figure 10 shows polarizations of qS1- and qS2-waves in a ``standard" orthorhombic anisotropic material - vertically fractured shale 
\cite[]{Schoenberg:1997}, which has the parameters $v_{p0}=2.437$, $v_{s0}=1.265$ km/s, $\epsilon_1=0.329$,
$\epsilon_2=0.258$, $\delta_1=0.083$, $\delta_2=-0.078$, $\delta_3=-0.106$, $\gamma_1=0.182$ and $\gamma_2=0.0455$.
The qS-waves polarize in a very complicated way and have point singularities in many propagation directions.

Shear-wave modeling is complicated by the presence of the shear-wave singularities.
As investigated by \cite{crampin.yedlin:1981},
a TI material only has line and kiss singularities, while other anisotropic materials excluding those with triclinic symmetry
(e.g., orthorhombic and monoclinic anisotropic materials) have point singularities in many propagation directions.
Line singularities occur only at a fixed angle from the symmetry axis and cause
no distortion of phase velocity surfaces or polarization phenomena.
For kiss singularities (along the direction of symmetry axis), 
qS-wave polarizations vary rapidly in their vicinity but are well-behaved because there is no distortion in phase-velocity surfaces.
These features facilitate the derivations of pseudo-pure-mode qSV-wave and pure-mode SH-wave equations for TI media.
For directions near point singularities, however, the polarization of plane qS-waves changes very rapidly,
and amplitudes and polarizations of qS-waves with curved wavefronts behave quite anomalously.
Therefore, although pseudo-pure-mode qP-wave equations exist for general anisotropic media, 
it may be more confusing than helpful to extend the proposed pseudo-pure-mode qS-wave equations
to symmetry systems lower than TI.

\inputdir{Matlab}
\multiplot{2}{polar3dvtiS1,polar3dvtiS2}{width=0.5\textwidth}
{
Polarization vectors of 3D qS-waves in a VTI material (Mesaverde shale): (a) qS1-wave; (b) qS2-wave.
}

\multiplot{2}{polar3dortS1,polar3dortS2}{width=0.5\textwidth}
{
Polarization vectors of 3D qS-waves in an orthorhombic anisotropic material: (a) qS1-wave; (b) qS2-wave.
}

%\section{Conclusions}
Different wave-modes in elastic media can be separated by projecting
the vector wavefields onto the polarization vectors of each mode.  For
heterogeneous models, it is necessary to separate wave-modes in the
space domain by non-stationary filtering.  I present a method for
obtaining spatially-varying wave-mode separators for TI models, which
can be used to separate elastic wave-modes in complex media. The
method computes the components of the polarization vectors in the
wavenumber domain and then transforms them to the space domain to
obtain spatially-varying filters. In order for the operators to work
in TI models with non-zero tilt angles, I incorporate one more
parameter---the local tilt angle $\nu$---in addition to the parameters
needed for the VTI operators. This kind of spatial filters can be used
to separate complicated wavefields in TI models with high
heterogeneity and strong anisotropy. I test the separation with
synthetic models that have realistic geologic complexity. The results
support the effectiveness of wave-mode separation with non-stationary
filtering.

I also extend the wave-mode separation to 3D TI models. The P-mode
separators can be constructed by solving the Christoffel equation for
the P-wave eigenvectors with local medium parameters.  The SV and SH
separators are constructed using the mutual orthogonality among P, SV,
and SH modes.  For the three modes, there are a total number of nine
separators, with three components for each mode. The separators vary
according to the medium parameters $V_{P0}$, $V_{S0}$, anisotropy
parameters $\epsilon$ and $\delta$ and tilt $\nu$ and azimuth $\alpha$
of the symmetry axis. The P-wave separators are constructed under no
kinematic assumptions, and amplitudes of P-mode correctly characterize
the plane-wave solution. Shear wave separators are constructed under
kinematic assumptions, and therefore the amplitudes of shear modes are
inaccurate in the singular directions. Nevertheless, the proposed
technique successfully separates fast and slow shear wavefields. The
process of constructing 3D separators and separating wave-modes in 3D
eliminates the step of decomposing the wavefields into symmetry
planes, which only works for models with an invariant symmetry
axis. Spatially-varying 3D separators have potential benefits for
complex models and can be used to separate wave-modes in elastic
reverse time migration (RTM) for TTI models. The spatially-varying 3D
separators imply large computational and storage cost, and therefore, a
more efficient separation method, such as the proposed table look-up
alternative, is necessary for a successful implementation.

\section{Conclusions}
By applying two different similarity transformations to the original Christoffel equation, which aim to project the vector displacement
wavefields onto the isotropic SV- and SH-waves' polarization directions, we have derived the pseudo-pure-mode qSV-wave equation
and the pure-mode SH-wave equation for 2D and 3D heterogeneous TI media, respectively.
These equations are simpler than the original elastic wave equation and involve less material parameters, which reduces computational
cost at least by half if the finite-difference scheme is used in practice.
The theoretical analysis and numerical examples have demonstrated that, the pseudo-pure-mode
qS-wave propagators for TI media have the following features:
First, the qSV-wave equations honor the kinematics for both qP and qSV modes, while the pure-mode SH-wave equation guarantees
the kinematics for the scalar SH-wave.
Second, although qP-waves still remain in the pseudo-pure-mode qSV-wave fields, their horizontal and vertical components have almost
opposite polarities in most propagation directions. As a result, the summation of all components
produces a pseudo-pure-mode scalar qSV-wave field with very weak qP-wave energy. 
Third, the non-SH parts in the pseudo-pure-mode vector SH-wave field have completely opposite polarities,
and thus are thoroughly removed from the scalar SH-wave field once all components are summed.
In addition, 
a filtering step taking into account the polarization deviation can be used to thoroughly remove the residual qP-waves
for pseudo-pure-mode scalar qSV-wave extrapolation.
These features indicate the potential of the proposed qS-wave propagators for developing promising seismic imaging and
inversion algorithms in heterogeneous TI media.
Like the pseudo-acoustic or pseudo-pure-mode qP-wave equations,
the proposed pseudo-pure-mode qS-wave equations take into account ``scalar anisotropy" and may distort the dynamic elasticity
of the real anisotropic media.

%\section{Acknowledgments}
The first author acknowledges support from the National Natural Science Foundation of China (No.41074083).
and would like to thank Xiaolei Song, Jingwei Hu, Siwei Li, Tengfei Wang and
Chenlong Wang for useful discussions.
Constructive comments by Tariq Alkhalifah, Alexey Stovas, and two anonymous reviewers
are much appreciated. We thank SEG, Hess Corporation, and BP for making the
2D VTI and TTI models available.

\section{ACKNOWLEDGMENTS}
Thanks to the support of the National Natural Science Foundation of China (No.41474099)
and Shanghai Natural Science Foundation (No.14ZR1442900). 
We appreciate Sergey Fomel, Paul Fowler, Chenlong Wang and Tengfei Wang
for helpful discussions in the later period of this study.
Constructive comments by Richard Bale, Alexey Stovas, Joe Dellinger and an anonymous reviewer
are much appreciated.
We thank SEG and BP for making the 2D TTI model available,
and the authors and developers of \emph{Madagascar} for providing resources
for reproducible computational experiments.

%\newpage
%\twocolumn
\bibliographystyle{seg}
\bibliography{reference}
