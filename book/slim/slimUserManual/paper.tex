\title{User's manual for SLIM programs in Madagascar}
\email{ghennenfent@eos.ubc.ca}
\author{Gilles Hennenfent}

\footnotetext[1]{\emph{Seismic Laboratory for Imaging and Modeling,
    Dept. of Earth and Ocean Sciences, the University of British
    Columbia, Vancouver, BC, Canada}}
\maketitle

\begin{abstract}
  This guide documents the contributions to Madagascar made by authors
  from the Seismic Laboratory for Imaging and Modeling
  (\href{http://slim.eos.ubc.ca}{SLIM}) at the University of British
  Columbia (\href{http://www.ubc.ca}{UBC}).
\end{abstract}

\section{Copyright}
%
Copyright (c) The University of British Columbia at Vancouver,
2005-2007.

\section{License}
%
\noindent This program is free software; you can redistribute it
and/or modify it under the terms of the GNU General Public License as
published by the Free Software Foundation; either version 2 of the
License, or (at your option) any later version.

\section{Disclaimer}
%
This program is distributed in the hope that it will be useful, but
WITHOUT ANY WARRANTY; without even the implied warranty of
MERCHANTABILITY or FITNESS FOR A PARTICULAR PURPOSE.  See the GNU
General Public License for more details.\\
%
You should have received a copy of the GNU General Public License
along with this program; if not, write to the Free Software
Foundation, Inc., 59 Temple Place, Suite 330, Boston, MA 02111-1307
USA.

\newpage
\section{Utilities}

\noindent\doublebox{\parbox{\textwidth}{
\input{sfthr}
}}

\noindent
Consider the vector $\mathbf{x}:=\{x_i\}_{0\leq i <m}\in\mathbb{R}^m$.
Soft thresholding is defined as
%
\begin{equation}
\label{eq:soft}
\mathcal{S}_\gamma(\mathbf{x}):=\{\mbox{sign}(x_i)\cdot
\max(|x_i|-\gamma,0)\}_{0\leq i <m},
\end{equation}
%
with $\gamma$ a positive threshold level. Hard thresholding is defined
as
%
\begin{equation}
\label{eq:hard}
\mathcal{H}_\gamma(\mathbf{x}):=\{\max(|x_i|-\gamma,0)\cdot x_i\}_{0\leq i <m}.
\end{equation}
%
Finally, non-negative Garrote (nng) thresholding is defined as
%
\begin{equation}
\label{eq:nng}
\mathcal{T}^{nng}_\gamma(\mathbf{x}):=\{\max(|x_i|-\gamma,0)\cdot 
(x_i-\gamma^2/x_i)\}_{0\leq i <m}.
\end{equation}

\inputdir{sfthr}
\multiplot{4}{data,soft,hard,nng}{width=.45\textwidth}{Thresholding
  example. Line whose range is symmetric about the origin (a)
  thresholded using soft (b), hard (c), and NNG (d) methods.}

The extension to positive varying threshold level is straightforward
by replacing $\gamma$ by $\gamma_i$ in
Eq.'s~\eqref{eq:soft},\eqref{eq:hard}, and \eqref{eq:nng}.

In Madagascar, to soft threshold a dataset with a constant, use e.g.
%
\begin{verbatim}
bash$ sfmath n1=100 n2=1 output='1' | sfnoise rep=y >data.rsf
bash$ sfthr <data.rsf thr=2 >res1.rsf
\end{verbatim}
%
or replace the last command by
%
\begin{verbatim}
bash$ sfthr <data.rsf thr=2 method=soft >res2.rsf
\end{verbatim}
%
This is also equivalent to soft thresholding \texttt{data.rsf} with a
vector of same size \texttt{mythr.rsf} whose entries are all set to 2.
%
\begin{verbatim}
bash$ sfmath n1=100 n2=1 output='2' >mythr.rsf
bash$ sfthr <data.rsf fthr=mythr.rsf method=soft >res3.rsf
\end{verbatim}
%
If \texttt{thr=.5} and \texttt{fthr=mythr.rsf} are specified at the
same time, the effective threshold level is 1, obtained by multiplying
\texttt{mythr.rsf} entries by 0.5
%
\begin{verbatim}
bash$ sfthr <data.rsf thr=.5 fthr=mythr.rsf method=soft >res4.rsf
\end{verbatim}

\multiplot{4}{data1,soft1,hard1,nng1}{width=.45\textwidth}{Random
  vector thresholding example.}

Note that thresholding is an element-wise operation. \texttt{sfthr}
can thus deal with arbitrarily large datasets.

\noindent\doublebox{\parbox{\textwidth}{
\input{sfsort}
}}

\noindent
\texttt{sfsort} is useful for sorting Madagascar vectors either in
ascending or descending order with respect to their amplitudes. The
sorting is done using \texttt{qsort} from stdlib.h.  This function is
an implementation of the quicksort algorithm. \texttt{sfsort} has two
modes: 1) in-core if the user-specified memsize is big enough to load
the full dataset in memory and sort it, and else 2) out-of-core. In
the latter case, we implemented a divide and conquer approach. The
large dataset is first divided into pieces that fit in memory. These
pieces are sorted and written to disk in temporary files. The second
step is a merge process of the temporary files.

\inputdir{sfsort}
\multiplot{3}{datasort,incore,outofcore}{width=.3\textwidth}{Sorting
  example.}

\section{Transforms}

\noindent\doublebox{\parbox{\textwidth}{
\input{sffdct}
}}

\inputdir{sffdct}
\multiplot{3}{model,data,res}{width=.3\textwidth}{Data denoising. (a)
  Noise-free data, (b) noisy data, and (c) denoised data using
  sffdct.}

%%% Local Variables: 
%%% mode: latex
%%% TeX-master: t
%%% TeX-master: t
%%% TeX-master: t
%%% TeX-master: t
%%% End: 
