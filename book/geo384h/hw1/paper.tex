\author{Johan Jensen}
%%%%%%%%%%%%%%%%%%%%%
\title{Homework 1}

\begin{abstract}
  This homework has three parts. 
  \begin{enumerate}
  \item Theoretical questions and computations related to digital representation of numbers.
  \item Analyzing digital elevation data from the San Francisco Bay area. You will apply 
  histogram equalization to enhance the image.
  \item Analyzing seismic reflection data. You will apply an amplitude gain 
  correction to enhance the image.
  \end{enumerate}
\end{abstract}

\section{Prerequisites}

Completing the computational part of this homework assignment requires
\begin{itemize}
\item \texttt{Madagascar} software environment available from \\
\url{http://www.ahay.org/}
\item \LaTeX\ environment with \texttt{SEGTeX} available from \\ 
\url{http://www.ahay.org/wiki/SEGTeX}
\end{itemize}
To do the assignment on your personal computer, you need to install
the required environments. Please ask for help if you don't know where
to start.

The homework code is available from the \texttt{Madagascar} repository
by running
\begin{verbatim}
svn co https://github.com/ahay/src/trunk/book/geo384h/hw1 
\end{verbatim}

%\newpage

\section{Digital representation of numbers}

You can either write your answers to theoretical questions on paper or
edit them in the file \verb#hw1/paper.tex#. Please show all the
mathematical derivations that you perform.

\begin{enumerate}

\item UT's official ``burnt orange'' color is expressed by code
  \texttt{\#BF5700}, where each pair of symbols (\texttt{BF},
  \texttt{57}, and \texttt{00}) refers to a hexadecimal (base 16)
  representation of the RGB (red, green, and blue) components. Convert these
  numbers to an octal (base 8) and a decimal (base 10) representations.

\item The C program listed below, when compiled and
  run from the command line, takes a string from the user and prints
  out the string characters. Modify the program to output ASCII
  integer codes for each character in the string. What is the ASCII
  code for the special new line character ``\verb+\n+''?

\lstset{language=c,numbers=left,numberstyle=\tiny,showstringspaces=false}
\lstinputlisting[frame=single,title=string.c]{string.c}

Alternatively, modify the following Python script for the same task.

\lstset{language=python,numbers=left,numberstyle=\tiny,showstringspaces=false}
\lstinputlisting[frame=single,title=string.py]{string.py}

\item In the IEEE double-precision floating-point standard, 64 bits
  (binary digits) are used to represent a real number: 1 bit for the
  sign, 11 bits for the exponent, and 52 bits for the mantissa. A
  double-precision normalized non-zero number $x$ can be written in
  this standard as 
  \[
  x = \pm (1.d_1d_2{\cdots}d_{52})_2 \times 2^{n-1023}\,
  \]
  with $1 \le n \le 2046$, and $0 \le d_k \le 1$ for
  $k=1,2,\ldots,52$. What is the largest number that can be expressed
  in this system?

\item The C program listed below tries to compute the \emph{machine
    epsilon}: the smallest positive number $\epsilon$ such that
  $1+\epsilon > 1$ in double-precision floating-point arithmetic. 

\begin{enumerate}
\item Add the missing part of the program so that, when compiled, it
  runs without an assertion error.
\item Modify the program to find the machine epsilon for single-precision floating-point arithmetic.
\end{enumerate}

\lstset{language=c,numbers=left,numberstyle=\tiny,showstringspaces=false}
\lstinputlisting[frame=single,title=epsilon.c]{epsilon.c}

Alternatively, modify the following Python script for the same task.

\lstset{language=python,numbers=left,numberstyle=\tiny,showstringspaces=false}
\lstinputlisting[frame=single,title=string.py]{epsilon.py}

\end{enumerate}


\section{Histogram equalization}
\inputdir{dem}

\sideplot{byte}{width=\textwidth}{Digital elevation map of the 
San Francisco Bay area.}

Figure~\ref{fig:byte} shows a digital elevation map of the San
Francisco Bay area. Start by reproducing this figure on your screen.
\begin{enumerate}
\item Change directory to \verb#hw1/dem#
\item Run
\begin{verbatim}
scons byte.view
\end{verbatim}
\item Examine the file \texttt{byte.rsf} which refers to the 
byte (unsigned character) numbers which get displayed on the screen.
\begin{enumerate}
\item Open \texttt{byte.rsf} with a text editor to check its contents.
\item Run
\begin{verbatim}
sfin byte.rsf
\end{verbatim}
to check the data size and format.
\item Run
\begin{verbatim}
sfattr < byte.rsf
\end{verbatim}
to check data attributes. What is the maximum and minimum value? What
is the mean value? For an explanation of different attributes,
run \texttt{sfattr} without input.
\end{enumerate}
\end{enumerate}

Each image has a certain distribution of values (a histogram). The
histogram for the west Austin elevation map is shown in
Figure~\ref{fig:hist}. Notice the digitization artifacts. When
different values in a histogram are not uniformly distributed, the
image can have a low contrast. One way of improving the contrast is
\emph{histogram equalization}.

\plot{hist}{width=\textwidth}{Histogram (left)
  and cumulative histogram (right) of the digital elevation
  data.}

Let $f(x,y)$ be the original image. The equalized image will be
$F(x,y)$. Let $h(f)$ be the histogram (probability distribution) of
the original image values. Let $H(F)$ be the histogram of the modified
image. The mapping of probabilities suggests
\begin{equation}
\label{eq:prob}
H(F)\,dF = h(f)\,df
\end{equation}
or, if we want the modified histogram
to be uniform, 
\begin{equation}
\label{eq:dif}
\frac{d F}{d f} = C\,h(f)\,
\end{equation}
where C is a constant. 
Solving equation~\ref{eq:dif}, we obtain the following mapping:
\begin{equation}
\label{eq:int}
F(f) = f_0 + C\,\int\limits_{f_0}^f h(\phi)\,d\phi\;,
\end{equation}
where $f_0$ is the minimum value of $f$.

The algorithm for histogram equalization consists of the following
three steps:
\begin{enumerate}
\item Taking an input image $f(x,y)$, compute its histogram $h(f)$.
\item Compute the cumulative histogram $F(f)$ according to 
equation~(\ref{eq:int}). Choose an appropriate normalization $C$ so that the range
of $F$ is the same as the range of $f$.
\item Map every pixel $f(x,y)$ to the corresponding $F(x,y)$.
\end{enumerate}

\newpage

Your task:
\begin{enumerate}
\item Among the \texttt{Madagascar} programs, find a program that 
implements histogram equalization. \textbf{Hint:} you may find
the \texttt{sfdoc} utility useful.
\item Edit the \texttt{SConstruct} file to add histogram equalization. 
Create a new figure and compare it with Figure~\ref{fig:byte}.
\item Check the effect of equalization by recomputing the histogram 
in Figure~\ref{fig:hist} with equalized data. Run
\begin{verbatim}
scons hist.view
\end{verbatim}
to display the figure on your screen.
\item \textbf{EXTRA CREDIT} for implementing the histogram
equalization algorithm in a different programming language.
\end{enumerate}	

%\newpage

\lstset{language=python,numbers=left,numberstyle=\tiny,showstringspaces=false}
\lstinputlisting[frame=single,title=dem/SConstruct]{dem/SConstruct}

\section{Time-power amplitude-gain correction}
\inputdir{tpow}

Raw seismic reflection data come in the form of shot gathers $S(x,t)$,
where $x$ is the offset (horizontal distance from the receiver to the
source) and $t$ is recording time. Raw data are inconvenient for
analysis because of rapid amplitude decay of seismic waves. The decay
can be compensated by multiplying the data by a gain function. A
commonly used function is a power of time. The gain-compensated gather
is
\begin{equation}
\label{eq:tpow}
S_\alpha(x,t) = t^{\alpha}\,S(x,t)\;.
\end{equation}
The advantage of the time-power gain is its simplicity and the ability
to reverse it by multiplying the data by $t^{-\alpha}$. What value of
$\alpha$ should we use? \cite{iei} argues in favor of $\alpha=2$:
one factor of $t$ comes from geometrical spreading and the other from
scattering attenuation. Your task is to develop an algorithm for finding
a better value of $\alpha$ for a given dataset.

\plot{tpow}{width=0.9\textwidth}{Seismic shot record before and after
  time-power gain correction.}

Figure~\ref{fig:tpow} shows a seismic shot record before and after
applying the time-power gain~(\ref{eq:tpow}) with $\alpha=2$. Start
by reproducing this figure on your screen.

\begin{enumerate}
\item Change directory to \verb#hw1/tpow#
\item Run
\begin{verbatim}
scons tpow.view
\end{verbatim}
\item Edit the \texttt{SConstruct} file. Find where the value of
  $\alpha$ is specified in this file and try changing it to a
  different value. Run \texttt{scons tpow.view} again to check the result.
\item How can we detect if the distribution of amplitudes after
  the gain correction is uniform? Suggest a measure (an objective
  function) that would take $S_\alpha(x,t)$ and produce one number that
  measures uniformity.
\item By modifying the program \texttt{objective.c} (alternatively,
  \texttt{objective.py}), compute your objective function for
  different values of $\alpha$ and display it in a figure.  Does the
  function appear to have a unique minimum or maximum?

\lstset{language=c,numbers=left,numberstyle=\tiny,showstringspaces=false}
\lstinputlisting[frame=single,title=tpow/objective.c]{tpow/objective.c}

\lstset{language=python,numbers=left,numberstyle=\tiny,showstringspaces=false}
\lstinputlisting[frame=single,title=tpow/objective.py]{tpow/objective.py}

\item Suggest an algorithm for finding an optimal value of $\alpha$ by
  minimizing or maximizing the objective function. Your algorithm
  should be able to find the optimal value without scanning all
  possible values. \textbf{Hint:} if the objective function is
  $f(\alpha)=F[S_\alpha(x,t)]$ and 
\begin{equation}
\label{alpha}                      
f(\alpha) \approx f(\alpha_0) + 
  f'(\alpha_0)\,(\alpha-\alpha_0) + \frac{f''(\alpha_0)}{2}\,(\alpha-\alpha_0)^2
\end{equation}
then what is the optimal $\alpha$?
\item \textbf{EXTRA CREDIT} for implementing your algorithm for an automatic 
estimation of $\alpha$ and testing it on the shot gather from
Figure~\ref{fig:tpow}.

\end{enumerate}

\lstset{language=python,numbers=left,numberstyle=\tiny,showstringspaces=false}
\lstinputlisting[frame=single,title=tpow/SConstruct]{tpow/SConstruct}

\newpage

\section{Completing the assignment}

\begin{enumerate}
\item Change directory to \texttt{hw1}.
\item Edit the file \texttt{paper.tex} in your favorite editor and change the
first line to have your name instead of Jensen's.
\item Run
\begin{verbatim}
sftour scons lock
\end{verbatim}
to update all figures.
\item Run
\begin{verbatim}
sftour scons -c
\end{verbatim}
to remove intermediate files.
\item Run
\begin{verbatim}
scons pdf
\end{verbatim}
to create the final document.
\item Submit your result (file \texttt{paper.pdf}) by e-mail.
\end{enumerate}

\bibliographystyle{seg}
\bibliography{hw1}