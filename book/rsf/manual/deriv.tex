\section{First derivative FIR filter (deriv.c)}




\subsection{{sf\_deriv\_init}}
Initializes the derivative calculation of the input trace, that is, it sets the required parameters and allocates the required space.


\subsubsection*{Call}
\begin{verbatim}sf_deriv_init (nt1, n1, c1);\end{verbatim}


\subsubsection*{Definition}
\begin{verbatim}
void sf_deriv_init(int nt1  /* transform length */, 
                   int n1   /* trace length */, 
                   float c1 /* filter parameter */)
{
   ...
}

\end{verbatim}
\subsubsection*{Input parameters}
\begin{desclist}{\tt }{\quad}[\tt ]
   \setlength\itemsep{0pt}
   \item[nt1] length of the transform (derivative) (\texttt{int}). 
   \item[n1]  length of the trace (\texttt{int}). 
   \item[c1]  filter parameter (\texttt{float}).  
\end{desclist}




\subsection{{sf\_deriv\_free}}
Frees the temporary space allocated for the derivative operator.


\subsubsection*{Call}
\begin{verbatim}sf_deriv_free ();\end{verbatim}


\subsubsection*{Definition}
\begin{verbatim}
void sf_deriv_free(void)
{
   ...
}
\end{verbatim}




\subsection{{sf\_deriv}}
Calculates the derivative of the input trace (\texttt{trace}) and outputs it to \texttt{trace2}.


\subsubsection*{Definition}
\begin{verbatim}sf_deriv (trace, trace2);\end{verbatim}


\subsubsection*{Definition}
\begin{verbatim}
void sf_deriv (const float* trace, float* trace2)
/*< derivative operator >*/
{
   ...
}
\end{verbatim}


\subsubsection*{Input parameters}
\begin{desclist}{\tt }{\quad}[\tt ]
   \setlength\itemsep{0pt}
   \item[trace] input trace whose derivative is required (\texttt{float*}).  
   \item[trace2] location where the derivative is to be stored (\texttt{float*}).  
\end{desclist}



