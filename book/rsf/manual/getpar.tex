\section{Parameter handling (getpar.c)}




\subsection{{sf\_stdin}}
Checks whether there is an input in the command line, if not it returns a false. It reads the first character in the file: if it is an \texttt{EOF}, \texttt{false} is returned and if not, then \texttt{true} is the return value. It takes no input parameters and returns a boolean value.

\subsubsection*{Call}
\begin{verbatim}hasinp sf_stdin();\end{verbatim}

\subsubsection*{Definition}
\begin{verbatim}
bool sf_stdin(void)
/*< returns true if there is an input in stdin >*/
{
   ...
}
\end{verbatim}




\subsection{{sf\_init}}\label{sec:sf_init}
Initializes a parameter table which is created using \hyperref[sec:sf_simtab_init]{\texttt{sf\_simtab\_init}}.
The input arguments are same as used in the main function in c when there is some arguments are to be input from the command line.

\subsubsection*{Input parameters}
\begin{desclist}{}{\quad}[\tt argv12]
   \setlength\itemsep{0pt}
   \item[\texttt{argc}] size of the table to be allocated (\texttt{int}). 
   \item[\texttt{argv[]}] pointer to the character array from the command line (\texttt{char*}).
\end{desclist}

\subsubsection*{Call}
\begin{verbatim}sf_init(argc, argv[]);\end{verbatim}

\subsubsection*{Definition}
\begin{verbatim}
void sf_init(int argc,char *argv[]) 
/*< initialize parameter table from command-line arguments >*/
{
   ...
}
\end{verbatim}




\subsection{{sf\_par\_close}}
Frees the allocated space for the table. It uses \hyperref[sec:sf_simtab_close]{\texttt{sf\_simtab\_close}} to close the table. It does not take any input parameters but passes a pointer \texttt{pars} defined by \texttt{sf\_init}.

\subsubsection*{Call}
\begin{verbatim}sf_parclose ();\end{verbatim}

\subsubsection*{Definition}
\begin{verbatim}
void sf_parclose (void)
/*< close parameter table and free space >*/
{
   ...
}
\end{verbatim}




\subsection{{sf\_parout}}
Reads the parameters from the internal table and writes them to a file. It uses \hyperref[sec:sf_simtab_output]{\texttt{sf\_simtab\_output}}. It takes a pointer to the file, in which the parameters are to be written.

\subsubsection*{Call}
\begin{verbatim}sf_parout (file);\end{verbatim}

\subsubsection*{Definition}
\begin{verbatim}
void sf_parout (FILE *file)
/*< write the parameters to a file >*/
{
   ...
}
\end{verbatim}

\subsubsection*{Input parameters}
\begin{desclist}{\tt }{\quad}[\tt parout]
   \setlength\itemsep{0pt}
   \item[parout] the table in which the value has to be entered (\texttt{FILE*}).
\end{desclist}




\subsection{{sf\_getprog}}
Outputs a pointer of \texttt{char} type to the name of the current running program. The pointer it returns is assigned a value in \hyperref[sec:sf_init]{\texttt{sf\_init}}.

\subsubsection*{Call}
\begin{verbatim}
prog = sf_getprog ();
\end{verbatim}

\subsubsection*{Definition}
\begin{verbatim}
char* sf_getprog (void) 
/*< returns name of the running program >*/ 
{
   ...
}
\end{verbatim}

\subsubsection*{Output}
\begin{desclist}{\tt }{\quad}[\tt prog]
   \setlength\itemsep{0pt}
   \item[prog] pointer to an array which contains the current program name (\texttt{char}).
\end{desclist}




\subsection{{sf\_getuser}}
Outputs a pointer of \texttt{char} type to the name of the current user. The pointer it returns is assigned a value in \hyperref[sec:sf_init]{\texttt{sf\_init}}.

\subsubsection*{Call}
\begin{verbatim}user = sf_getuser ();\end{verbatim}

\subsubsection*{Definition}
\begin{verbatim}
char* sf_getuser (void) 
/*< returns user name >*/
{
   ...
}
\end{verbatim}

\subsubsection*{Output}
\begin{desclist}{\tt }{\quad}[\tt user]
   \setlength\itemsep{0pt}
   \item[user] pointer to an array which contains the user name (\texttt{char}).
\end{desclist}




\subsection{{sf\_gethost}}
Outputs a pointer of \texttt{char} type to the name of the current host. The pointer it returns is assigned a value in \hyperref[sec:sf_init]{\texttt{sf\_init}}.

\subsubsection*{Call}
\begin{verbatim}host = sf_gethost ();\end{verbatim}

\subsubsection*{Definition}
\begin{verbatim}
char* sf_gethost (void) 
/*< returns host name >*/
{
   ...
}
\end{verbatim}

\subsubsection*{Output}
\begin{desclist}{\tt }{\quad}[\tt host]
   \setlength\itemsep{0pt}
   \item[host] pointer to an array which contains the host name (\texttt{char}).
\end{desclist}




\subsection{{sf\_getcdir}}
Outputs a pointer of \texttt{char} type to the name of the current working directory. The pointer it returns is assigned a value in \hyperref[sec:sf_init]{\texttt{sf\_init}}.

\subsubsection*{Call}
\begin{verbatim}cdir = sf_getcdir ();\end{verbatim}

\subsubsection*{Definition}
\begin{verbatim}
char* sf_getcdir (void) 
/*< returns current directory >*/
{
   ...
}
\end{verbatim}

\subsubsection*{Output}
\begin{desclist}{\tt }{\quad}[\tt cdir]
   \setlength\itemsep{0pt}
   \item[cdir] pointer to an array which contains the current working directory (\texttt{char}).
\end{desclist}




\subsection{{sf\_getint}}
Extracts an integer from the command line. If the extraction is successful, it returns a \texttt{true}, otherwise a \texttt{false}.  It uses \hyperref[sec:sf_simtab_getint]{\texttt{sf\_simtab\_getint}}.

\subsubsection*{Call}
\begin{verbatim}success = sf_getint (key, par);\end{verbatim}

\subsubsection*{Definition}
\begin{verbatim}
bool sf_getint (const char* key,/*@out@*/ int* par) 
/*< get an int parameter from the command line >*/
{
   ...
}
\end{verbatim}

\subsubsection*{Input parameters}
\begin{desclist}{\tt }{\quad}[\tt key]
   \setlength\itemsep{0pt}
   \item[key] the name of the entry which has to be extracted (\texttt{const char*}).
   \item[par] pointer to the integer variable where the extracted value is to be copied.
\end{desclist}

\subsubsection*{Output}
\begin{desclist}{\tt }{\quad}[\tt success]
   \setlength\itemsep{0pt}
   \item[success] a boolean value. It is \texttt{true}, if the extraction was successful and \texttt{false} otherwise. 
\end{desclist}




\subsection{{sf\_getlargeint}}
Extracts a large integer from the command line. If the extraction is successful, it returns a \texttt{true}, otherwise a \texttt{false}.  
It uses \hyperref[sec:sf_simtab_getlargeint]{\texttt{sf\_simtab\_getlargeint}}.

\subsubsection*{Call}
\begin{verbatim}sf_getlargeint (key, par);\end{verbatim}

\subsubsection*{Definition}
\begin{verbatim}
bool sf_getlargeint (const char* key,/*@out@*/ off_t* par) 
/*< get a large int parameter from the command line >*/
{
   ...
}
\end{verbatim}

\subsubsection*{Input parameters}
\begin{desclist}{\tt }{\quad}[\tt key]
   \setlength\itemsep{0pt}
   \item[key] the name of the entry which has to be extracted (\texttt{const char*}).
   \item[par] pointer to the large integer variable where the extracted value is to be copied. Must be of type \texttt{off\_t} which is defined in the header \texttt{<sys/types.h>}.
\end{desclist}

\subsubsection*{Output}
\begin{desclist}{\tt }{\quad}[\tt success]
   \setlength\itemsep{0pt}
   \item[success]  a boolean value. It is \texttt{true}, if the extraction was successful and \texttt{false} otherwise.
\end{desclist}




\subsection{{sf\_getints}}
Extracts an array of integer values from the command line. If the extraction is successful, it returns a \texttt{true}, otherwise a \texttt{false}. 

\subsubsection*{Call}
\begin{verbatim}success = sf_getints (key, par, n);\end{verbatim}

\subsubsection*{Definition}
\begin{verbatim}
bool sf_getints (const char* key,/*@out@*/ int* par,size_t n) 
/*< get an int array parameter (comma-separated) from the command line >*/
{
   ...
} 
\end{verbatim}

\subsubsection*{Input parameters}
\begin{desclist}{\tt }{\quad}[\tt key]
   \setlength\itemsep{0pt}
   \item[key] the name of the integer array which has to be extracted (\texttt{const char*}).
   \item[par] pointer to the array of integer type value variable where the extracted value id to be copied. 
   \item[n] size of the array to be extracted. Must be of \texttt{size\_t}.
\end{desclist}

\subsubsection*{Output}
\begin{desclist}{\tt }{\quad}[\tt success]
   \setlength\itemsep{0pt}
   \item[success]  a boolean value. It is \texttt{true}, if the extraction was successful and \texttt{false} otherwise. 
\end{desclist}




\subsection{{sf\_getfloat}}
Extracts a float value from the command line. If the extraction is successful, it returns a \texttt{true}, otherwise a \texttt{false}.  

\subsubsection*{Call}
\begin{verbatim}success = sf_getfloat (key, par);\end{verbatim}

\subsubsection*{Definition}
\begin{verbatim}
bool sf_getfloat (const char* key,/*@out@*/ float* par) 
/*< get a float parameter from the command line >*/
{
   ...
}
\end{verbatim}

\subsubsection*{Input parameters}
\begin{desclist}{\tt }{\quad}[\tt key]
   \setlength\itemsep{0pt}
   \item[key] the name of the entry which has to be extracted (\texttt{const char*}).
   \item[par] pointer to the float type value variable where the extracted value is to be copied.
\end{desclist}

\subsubsection*{Output}
\begin{desclist}{\tt }{\quad}[\tt success]
   \setlength\itemsep{0pt}
   \item[success]  a boolean value. It is \texttt{true}, if the extraction was successful and \texttt{false} otherwise. 
\end{desclist}




\subsection{{sf\_getdouble}}
Extracts a double type value from the command line. If the extraction is successful, it returns a \texttt{true}, otherwise a \texttt{false}.  

\subsubsection*{Call}
\begin{verbatim}success = sf_getdouble (key, par);\end{verbatim}

\subsubsection*{Definition}
\begin{verbatim}
bool sf_getdouble (const char* key,/*@out@*/ double* par) 
/*< get a double parameter from the command line >*/
{
   ...
}
\end{verbatim}

\subsubsection*{Input parameters}
\begin{desclist}{\tt }{\quad}[\tt table]
   \setlength\itemsep{0pt}
   \item[table] the table from which the vale has to be extracted. Must be of type \texttt{sf\_simtab}. 
   \item[key]   the name of the entry which has to be extracted (\texttt{const char*}).
   \item[par]   pointer to the double type value variable where the extracted value is to be copied.
\end{desclist}

\subsubsection*{Output}
\begin{desclist}{\tt }{\quad}[\tt success]
   \setlength\itemsep{0pt}
   \item[success]  a boolean value. It is \texttt{true}, if the extraction was successful and \texttt{false} otherwise. 
\end{desclist}




\subsection{{sf\_getfloats}}
Extracts an array of float values from the command line. If the extraction is successful, it returns a \texttt{true}, otherwise a \texttt{false}.  It uses \hyperref[sec:sf_simtab_getfloats]{\texttt{sf\_simtab\_getfloats}}.

\subsubsection*{Call}
\begin{verbatim}success = sf_getfloats (key, n);\end{verbatim}

\subsubsection*{Definition}
\begin{verbatim}
bool sf_getfloats (const char* key,/*@out@*/ float* par,size_t n) 
/*< get a float array parameter from the command line >*/
{
   ... 
}
\end{verbatim}

\subsubsection*{Input parameters}
\begin{desclist}{\tt }{\quad}[\tt key]
   \setlength\itemsep{0pt}
   \item[key] the name of the float array which has to be extracted (\texttt{const char*}).
   \item[par] pointer to the array of float type value variable where the extracted value id to be copied. 
   \item[n] size of the array to be extracted (\texttt{size\_t}).
\end{desclist}

\subsubsection*{Output}
\begin{desclist}{\tt }{\quad}[\tt success]
   \setlength\itemsep{0pt}
   \item[success]  a boolean value. It is \texttt{true}, if the extraction was successful and \texttt{false} otherwise.
\end{desclist}




\subsection{{sf\_getstring}}
Extracts a string pointed by the input key from the command line. If the value is \texttt{NULL} it will return \texttt{NULL}, otherwise it will allocate a new block of memory of \texttt{char} type and copy the memory block from the table to the new block and return a pointer to the newly allocated block of memory. It uses \hyperref[sec:sf_simtab_getstring]{\texttt{sf\_simtab\_getstring}}.

\subsubsection*{Call}
\begin{verbatim}string = sf_getstring (key);\end{verbatim}

\subsubsection*{Definition}
\begin{verbatim}
char* sf_getstring (const char* key) 
/*< get a string parameter from the command line >*/
{
   ...
}
\end{verbatim}

\subsubsection*{Input parameters}
\begin{desclist}{\tt }{\quad}[\tt key]
   \setlength\itemsep{0pt}
   \item[key] the name of the string which has to be extracted (\texttt{const char*}).
\end{desclist}

\subsubsection*{Output}
\begin{desclist}{\tt }{\quad}[\tt string]
   \setlength\itemsep{0pt}
   \item[string] a pointer to allocated block of memory containing a string of characters. 
\end{desclist}




\subsection{{sf\_getstrings}}
Extracts an array of strings from the command line. If the extraction is successful, it returns a \texttt{true}, otherwise a \texttt{false}. 

\subsubsection*{Call}
\begin{verbatim}success = sf_getstrings (key, par, n);\end{verbatim}

\subsubsection*{Definition}
\begin{verbatim}
bool sf_getstrings (const char* key,/*@out@*/ char** par,size_t n) 
/*< get a string array parameter from the command line >*/
{
   ...
}
\end{verbatim}

\subsubsection*{Input parameters}
\begin{desclist}{\tt }{\quad}[\tt key]
   \setlength\itemsep{0pt}
   \item[key] the name of the string array which has to be extracted (\texttt{const char*}).
   \item[par] pointer to the pointer to array of integer type value variable where the extracted value is to be copied. 
   \item[n] size of the array to be extracted. Must be of \texttt{size\_t}.
\end{desclist}

\subsubsection*{Output}
\begin{desclist}{\tt }{\quad}[\tt success]
   \setlength\itemsep{0pt}
   \item[success]  a boolean value. It is \texttt{true}, if the extraction was successful and \texttt{false} otherwise.
\end{desclist}




\subsection{{sf\_getbool}}
Extracts a boolean value from the command line. If the extraction is successful, it returns a \texttt{true}, otherwise a \texttt{false}. 

\subsubsection*{Call}
\begin{verbatim}success = sf_getbool (key, par);\end{verbatim}

\subsubsection*{Definition}
\begin{verbatim}
bool sf_getbool (const char* key,/*@out@*/ bool* par)
/*< get a bool parameter from the command line >*/
{
   ...
}
\end{verbatim}

\subsubsection*{Input parameters}
\begin{desclist}{\tt }{\quad}[\tt key]
   \setlength\itemsep{0pt}
   \item[key] the name of the entry which has to be extracted (\texttt{const char*}).
   \item[par] pointer to the bool variable where the extracted value is to be copied.
\end{desclist}

\subsubsection*{Output}
\begin{desclist}{\tt }{\quad}[\tt success]
   \setlength\itemsep{0pt}
   \item[success]  a boolean value. It is \texttt{true}, if the extraction was successful and \texttt{false} otherwise.
\end{desclist}




\subsection{{sf\_getbools}}
Extracts an array of bool values from the command line. If the extraction is successful, it returns a \texttt{true}, otherwise a \texttt{false}. It uses \hyperref[sec:sf_simtab_getbools]{\texttt{sf\_simtab\_getbools}}.

\subsubsection*{Call}
\begin{verbatim}success = sf_getbools (key, par, n);\end{verbatim}

\subsubsection*{Definition}
\begin{verbatim}
bool sf_getbools (const char* key,/*@out@*/ bool* par,size_t n) 
/*< get a bool array parameter from the command line >*/
{
    return sf_simtab_getbools(pars,key,par,n);
} 
\end{verbatim}

\subsubsection*{Input parameters}
\begin{desclist}{\tt }{\quad}[\tt par]
   \setlength\itemsep{0pt}
   \item[key] the name of the bool array which has to be extracted (\texttt{const char*}).
   \item[par] pointer to the array of bool type value variable where the extracted value id to be copied. 
   \item[n] size of the array to be extracted. Must be of \texttt{size\_t}.
\end{desclist}

\subsubsection*{Output}
\begin{desclist}{\tt }{\quad}[\tt success]
   \setlength\itemsep{0pt}
   \item[success]  a boolean value. It is \texttt{true}, if the extraction was successful and \texttt{false} otherwise. 
\end{desclist}




\subsection{{sf\_getpars}}
This function  returns a pointer to the parameter table which must have been initialized earlier in the program using \hyperref[sec:sf_init]{\texttt{sf\_init}}.

\subsubsection*{Call}
\begin{verbatim}pars = sf_getpars (void);\end{verbatim}

\subsubsection*{Definition}
\begin{verbatim}
sf_simtab sf_getpars (void)
/*< provide access to the parameter table >*/
{
   ...
}
\end{verbatim}

\subsubsection*{Output}
\begin{desclist}{\tt }{\quad}[\tt pars]
   \setlength\itemsep{0pt}
   \item[pars] a pointer to the parameter table. 
\end{desclist}



