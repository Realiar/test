\section{Additional operations with RSF files (files.c)}




\subsection{{sf\_filedims}}\label{sec:sf_filedims}
Returns the dimensions of the file.

\subsubsection*{Call}
\begin{verbatim}dim = sf_filedims (file, n);\end{verbatim}

\subsubsection*{Definition}
\begin{verbatim}
int sf_filedims (sf_file file, /*@out@*/ int *n) 
/*< Find file dimensions.
--- 
Outputs the number of dimensions dim and a dimension array n[dim] >*/
{
   ...
}
\end{verbatim}

\subsubsection*{Input parameters}
\begin{desclist}{\tt}{\quad}[file]
   \setlength\itemsep{0pt}
   \item[file] a pointer to the file structure whose dimensions are required (\texttt{sf\_file}). 
   \item[n]    an array where the dimensions will be stored (\texttt{int}).
\end{desclist}

\subsubsection*{Output}
\begin{desclist}{}{\quad}[nddimm]
   \setlength\itemsep{0pt}
   \item[\texttt{dim}] number of dimensions in the file (\texttt{int}). 
   \item[\texttt{n[dim]}] the array of dimensions (\texttt{int}).
\end{desclist}




\subsection{{sf\_largefiledims}}
Returns the dimensions of the file. It is exactly like \hyperref[sec:sf_filedims]{\texttt{sf\_filedims}} but \texttt{n} in the input is the offset in bytes in the input file (type \texttt{off\_t}) rather than just an integer.

\subsubsection*{Call}
\begin{verbatim}dim = sf_largefiledims (file, n);\end{verbatim}

\subsubsection*{Definition}
\begin{verbatim}
int sf_largefiledims (sf_file file, /*@out@*/ off_t *n) 
/*< Find file dimensions.
--- 
Outputs the number of dimensions dim and a dimension array n[dim] >*/
{
   ...
}
\end{verbatim}

\subsubsection*{Input parameters}
\begin{desclist}{\tt }{\quad}[\tt file]
   \setlength\itemsep{0pt}
   \item[file] a pointer to the file structure whose dimensions are required (\texttt{sf\_file}). 
   \item[n] an array where the dimensions will be stored (\texttt{off\_t}).
\end{desclist}

\subsubsection*{Output}
\begin{desclist}{}{\quad}[\tt nddimm]
   \setlength\itemsep{0pt}
   \item[\texttt{dim}]    number of dimensions in the file (\texttt{off\_t}). 
   \item[\texttt{n[dim]}] the array of dimensions (\texttt{off\_t}).
\end{desclist}




\subsection{{sf\_memsize}}
Returns the memory size defined in the environment variable \texttt{RSFMEMSIZE}. If there is an invalid value the function will print an error message an assign a default value of 100 Mbytes.

\subsubsection*{Call}
\begin{verbatim}memsize = sf_memsize();\end{verbatim}

\subsubsection*{Definition}
\begin{verbatim}
int sf_memsize()
/*< Returns memory size by:
  1. checking RSFMEMSIZE environmental variable
  2. using hard-coded "def" constant
  >*/
{

   ...
    return memsize;
}
\end{verbatim}




\subsection{{sf\_filesize}}
Returns the size of the file, that is, the product of the dimensions. It uses the function \hyperref[sec:sf_leftsize]{\texttt{sf\_leftsize}}.

\subsubsection*{Call}
\begin{verbatim}size = sf_filesize (file);\end{verbatim}

\subsubsection*{Definition}
\begin{verbatim}
off_t sf_filesize (sf_file file) 
/*< Find file size (product of all dimensions) >*/
{    
   ...
}
\end{verbatim}

\subsubsection*{Input parameters}
\begin{desclist}{\tt }{\quad}[\tt ]
   \setlength\itemsep{0pt}
   \item[file] a pointer to the file structure whose size is required (\texttt{sf\_file}).  
\end{desclist}

\subsubsection*{Output}
\begin{desclist}{\tt }{\quad}[\tt ]
   \setlength\itemsep{0pt}
   \item[size] product of dimensions in the file (\texttt{off\_t}).
\end{desclist}




\subsection{{sf\_leftsize}}\label{sec:sf_leftsize}
Returns the size of the file, that is, the product of the dimensions but only for the dimensions greater than the input integer dim. It uses the function \hyperref[sec:sf_leftsize]{\texttt{sf\_leftsize}}.

\subsubsection*{Call}
\begin{verbatim}size = sf_leftsize (file, dim);\end{verbatim}

\subsubsection*{Definition}
\begin{verbatim}
off_t sf_leftsize (sf_file file, int dim) 
/*< Find file size for dimensions greater than dim >*/
{
   ...
}
\end{verbatim}

\subsubsection*{Input parameters}
\begin{desclist}{\tt }{\quad}[\tt file]
   \setlength\itemsep{0pt}
   \item[file] the file whose size is required (\texttt{sf\_file}). 
   \item[dim] a pointer to the file structure whose size is required (\texttt{sf\_file}).  
\end{desclist}

\subsubsection*{Output}
\begin{desclist}{\tt }{\quad}[\tt ]
   \setlength\itemsep{0pt}
   \item[size] product of dimensions greater than dim in the file (\texttt{off\_t}).
\end{desclist}




\subsection{{sf\_cp}}
Copies the input file \texttt{in} to the output file \texttt{out}.

\subsubsection*{Call}
\begin{verbatim}sf_cp (in, out);\end{verbatim}

\subsubsection*{Definition}
\begin{verbatim}
void sf_cp(sf_file in, sf_file out)
/*< Copy file in to file out >*/
{
   ...
}
\end{verbatim}

\subsubsection*{Input parameters}
\begin{desclist}{\tt }{\quad}[\tt ]
   \setlength\itemsep{0pt}
   \item[in] the file which is to be copied (\texttt{sf\_file}). 
   \item[out] the file to which \texttt{in} file is to be copied (\texttt{sf\_file}).  
\end{desclist}




\subsection{{sf\_rm}}
Removes the RSF file. There are options to force removal (files are deleted even if protected), to inquire before removing a file and whether or to require verbose output.

\subsubsection*{Call}
\begin{verbatim}sf_rm (filename, force, verb, inquire);\end{verbatim}

\subsubsection*{Definition}
\begin{verbatim}
void sf_rm(const char* filename, bool force, bool verb, bool inquire)
/*< Remove an RSF file.
---
force, verb, and inquire flags should behave similar to the corresponding flags 
in the Unix "rm" command. >*/
{
   ...
}
\end{verbatim}

\subsubsection*{Input parameters}
\begin{desclist}{\texttt }{\quad}[filename]
   \setlength\itemsep{0pt}
   \item[filename] name of the file which is to be removed (\texttt{sf\_file}). 
   \item[force]    remove forcefully or not (\texttt{sf\_file}). 
   \item[inquire]  ask before removing or not (\texttt{sf\_file}).  
\end{desclist}




\subsection{{sf\_shiftdim}}
Shifts the grid by one dimension after the axis defined in the input parameters (axis). 

\subsubsection*{Call}
\begin{verbatim}n3 = sf_shiftdim(in, out, axis);\end{verbatim}

\subsubsection*{Definition}
\begin{verbatim}
off_t sf_shiftdim(sf_file in, sf_file out, int axis) 
/*< shift grid after axis by one dimension forward >*/
{
   ...
}
\end{verbatim}

\subsubsection*{Input parameters}
\begin{desclist}{\tt }{\quad}[\tt axis]
   \setlength\itemsep{0pt}
   \item[in]   a pointer to the input file structure (\texttt{sf\_file}). 
   \item[out]  a pointer to the output file structure (\texttt{sf\_file}). 
   \item[axis] the axis after which the grid is to be shifted (\texttt{sf\_file}).
\end{desclist}

\subsubsection*{Output}
\begin{desclist}{\tt }{\quad}[\tt ]
   \setlength\itemsep{0pt}
   \item[n3] the file size (product of dimensions) after the shift (\texttt{off\_t}).
\end{desclist}




\subsection{{sf\_unshiftdim}}
Shifts the grid backward by one dimension after the axis defined in the input parameters (axis). 

\subsubsection*{Call}
\begin{verbatim}n3 = sf_unshiftdim (in, out, axis);\end{verbatim}

\subsubsection*{Definition}
\begin{verbatim}
off_t sf_unshiftdim(sf_file in, sf_file out, int axis) 
/*< shift grid after axis by one dimension backward >*/
{
   ...
}
\end{verbatim}

\subsubsection*{Input parameters}
\begin{desclist}{\tt }{\quad}[\tt axis]
   \setlength\itemsep{0pt}
   \item[in]   a pointer to the input file structure (\texttt{sf\_file}). 
   \item[out]  a pointer to the output file structure (\texttt{sf\_file}). 
   \item[axis] the axis after which the grid is to be shifted (\texttt{sf\_file}).
\end{desclist}

\subsubsection*{Output}
\begin{desclist}{\tt }{\quad}[\tt n3]
   \setlength\itemsep{0pt}
   \item[n3] the file size (product of dimensions) after the backward shift (\texttt{off\_t}).
\end{desclist}




\subsection{{sf\_endian}}
Returns true if the machine is little endian. 

\subsubsection*{Definition}
\begin{verbatim}little_endian = sf_endian (); \end{verbatim}

\subsubsection*{Definition}
\begin{verbatim}
bool sf_endian (void)
/*< Endianness test, returns true for little-endian machines >*/
{
   ...    
}
\end{verbatim}

\subsubsection*{Output}
\begin{desclist}{\tt }{\quad}[\tt ]
   \setlength\itemsep{0pt}
   \item[little\_endian] a boolean parameter which is true if the machine is little endian.
\end{desclist}





