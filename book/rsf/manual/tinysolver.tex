\section{Solver function for iterative least-squares optimization (tinysolver.c)}




\subsection{{sf\_tinysolver}}
Performs the linear inversion for equations of the type $Lx=y$ to compute the model $x$.

\subsubsection*{Call}
\begin{verbatim}sf_tinysolver (Fop, stepper, nm, nd, m, m0, d, niter);\end{verbatim}


\subsubsection*{Definition}
\begin{verbatim}
void sf_tinysolver (sf_operator Fop       /* linear operator */, 
                    sf_solverstep stepper /* stepping function */, 
                    int nm                /* size of model */, 
                    int nd                /* size of data */, 
                    float* m              /* estimated model */,
                    const float* m0       /* starting model */,
                    const float* d        /* data */, 
                    int niter             /* iterations */)
/*< Generic linear solver. Solves oper{x} =~ dat >*/
{
   ...
}
\end{verbatim}


\subsubsection*{Input parameters}
\begin{desclist}{\tt }{\quad}[\tt stepper]
   \setlength\itemsep{0pt}
   \item[Fop]     a linear operator applied to the model $x$.  Must be of type \texttt{sf\_operator}. 
   \item[stepper] a stepping function to perform updates on the initial model (\texttt{sf\_solverstep}). 
   \item[nm]      size of the model (\texttt{int}). 
   \item[nd]      size of the data (\texttt{int}). 
   \item[m]       estimated model (\texttt{int}). 
   \item[mo]      initial model (\texttt{const float}). 
   \item[d]       data (\texttt{const float}). 
   \item[niter]   number of iterations (\texttt{int}).
\end{desclist}


