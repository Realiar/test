\section[Identity operator (complex data) (ccopy.c)]{Simple identity (copy) operator for complex data (ccopy.c)}\label{sec:ccopy}
This is the same operator as \hyperref[sec:sf_copy_lop]{\texttt{sf\_copy\_lop}} but for complex data. In particular, the identity operator is defined by
\begin{gather*}
	y = 1x=x,  \qquad\textrm{with}\quad y_t\leftarrow x_t.
\intertext{Its adjoint is}
	x = 1^*y=y,\qquad\textrm{with}\quad x_t\leftarrow y_t.
\end{gather*}




\subsection{{sf\_ccopy\_lop}}

\subsubsection*{Call}
\begin{verbatim}sf_ccopy_lop (adj, add, nx, ny, x, y);\end{verbatim}

\subsubsection*{Definition}
\begin{verbatim}
void sf_ccopy_lop (bool adj, bool add, int nx, int ny, 
                   sf_complex* xx, sf_complex* yy)
/*< linear operator >*/
{
   ...
}
\end{verbatim}

\subsubsection*{Input parameters}
\begin{desclist}{\tt }{\quad}[\tt add]
   \setlength\itemsep{0pt}
   \item[adj] adjoint flag (\texttt{bool}). If \texttt{true}, then the adjoint is computed, i.e.~$x\leftarrow 1^*y$ or $x\leftarrow x+1^*y$. 
   \item[add] addition flag (\texttt{bool}). If \texttt{true}, then $y\leftarrow y+1x$ or $x\leftarrow x+1^*y$.  
   \item[nx]  size of \texttt{x} (\texttt{int}). \texttt{nx} must equal \texttt{ny}.
   \item[ny]  size of \texttt{y} (\texttt{int}). \texttt{ny} must equal \texttt{nx}.
   \item[x]   input data or output (\texttt{sf\_complex*}).
   \item[y]   output or input data (\texttt{sf\_complex*}).
\end{desclist}


