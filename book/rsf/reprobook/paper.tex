\title{Reproducible research: Lessons from the Madagascar project}

\lefthead{Fomel et al.}
\righthead{Implementing
Reproducible Computational Research}

\author{
Sergey Fomel\footnotemark[1],
Paul Sava\footnotemark[2],
Ioan Vlad\footnotemark[3], \\
Yang Liu\footnotemark[4],
Xiaolei Song\footnotemark[1], and
Zhonghuan Chen\footnotemark[5]
}

\address{
\footnotemark[1]Bureau of Economic Geology \\
Jackson School of Geosciences \\
The University of Texas at Austin \\
University Station, Box X \\
Austin, TX 78713-8972 \\
USA \\
\footnotemark[2] Department of Geophysics \\
Colorado School of Mines \\
Golden, CO 80401 \\
USA \\
\footnotemark[3] TGS \\
2500 CityWest Boulevard, Suite 2000 \\
Houston, Texas 77042
USA \\
\footnotemark[4] College of Geo-Exploration Science and Technology \\
Jilin University \\
No.938 Xi Minzhu Street \\
Changchun, Jilin Province 130026
China \\
\footnotemark[5] Department of Automation \\
Tsinghua University \\
FIT 1-115, Tsinghua University \\
Beijing 100084
China
}

\begin{abstract}
PLACEHOLDER: Reproducibility is important. We implement it in Madagascar. Here is how. Here is what we have learned.
\end{abstract}

\section{Introduction}

[Why reproducible]In scientific research over all fields,
everyone needs to "stand on the shoulders of giants" at first.
Currently, there is a steep road to their shoulders.
What we do on reproducible research 
is to help people to climb on "the shoulders of giants".

[How reproducible]The target of reproducible research 
is similar to the open software project.
However, the reproducible research provides not only the software (algorithms)
but also the experiments and documents (paper, slides and reports).
Moreover, the reproducible research makes it clear that
how to organize the experiments to prove the ideas in the documents
and how to design the algorithm software to finish the experiments.
The latter is more significant than the former in reproducible research.

[Who need it]The reproducible research is helpful to the following people:
\begin{description}
\item [Beginner]
\item [Interdisciplinary researcher]
\item [The developer himself]
\end{description}

[About geoscience reproducible research]

[About madagascar]

[Road ahead]For the target of helping people to climb on giants's shoulder,
the open software is not enough, 
providing the experiments, documents is still not enough.
These materials provided by current reproducible research 
can show how to implement the new method
and how to prove the results in the research,
but it is not enough to show the reader
how the author yields the new idea.
More works are ahead to exchange the mind between the authors and the readers.


\section{History of the Madagascar project}

The history of reproducible research in applied geophysics goes back
to the work of Jon Claerbout at Stanford University
\cite[]{cise}. While working on the manuscript of a book
\footnote{published later as \cite[]{pvi}}, Claerbout had what he
described as a ``eureka'' moment \cite[]{Claerbout.sep.67.139}:
\begin{quote}
My third textbook on earth soundings analysis consists of about 300
pages, 120 figures, and listings of 40 terse subroutines. The
subroutines were used in preparing the figures. Each of the figure
captions contains a one-word name for the figure. Behind the visible
book are plot files for each figure as well as main programs and
command scripts (makefiles) that I used to make each figure. I
suddenly realized that I had all the materials needed by any reader or
student to recreate my work. This would give a tremendous boost to
anyone wishing to go beyond me.
\end{quote}
A that moment, he had a vision of computational reproducibility
achieved ... Cake to make ... CiSE article ... CD-Rom to Web...

The work of Claerbout inspired other research groups, most notably the
lab of David Donoho... Claerbout's principle...

However, the computational environment for reproducible experiments
created by Claerbout and his students at the Stanford Exploration
Project (SEP) did not spread to the rest of the geophysical
community. Part of the reason was the complexity of the tools used to
implement it: an accumulation of make rules, Perl scripts, Unix shell
scripts, etc. The Madagascar project started in 2003 as an attempt to
reimplement the functionality of the SEP environment and to make it
accessible to other users in an open-source form.

Madagascar was released under a GPL license and publicly announced in
June 2006 at the workshop ``Open Source E\&P Software – Putting the
Pieces Together'' organized by the European Association of
Geoscientists and Engineers (EAGE). ... version 1.0 ... number of
developers ... number of papers ... 

.... Madagascar schools ...

\section{Madagascar approach to reproducibility}

Describe how scons - scons lock - scons test works. Generating papers
for the website...

\section{Example}

Many appreciative users were familiar with SU or SEPlib,
now they had a new choice -- Madagascar (RSF), another 
open-source software, but there are some different places 
from others open-source softwares. In this section,
we show some simple examples about some geophysics processing 
flows, maybe these can help Madagascar users know
it better.

\subsection{Example1 - Set up 1-D synthetic data}

\begin{enumerate}
\item Command line

For the detailed parameter of every Madagascar function, just type
the function name without parameter in command line.

\begin{verbatim}
sfmake
\end{verbatim} 

The detailed information about ``sfmake'' will be followed:

\begin{verbatim}
NAME
        sfmake
DESCRIPTION
        Simple 2-D synthetics with crossing plane waves.
SYNOPSIS
        sfmake > mod.rsf n1=100 n2=14 n3=1 second=y n=3 p=3 t1=4 t2=4
PARAMETERS
        int     n=3     
        int     n1=100  
        int     n2=14   
        int     n3=1    dimensions 
        int     p=3     
        bool    second=y [y/n]  if n, only one plane wave is modeled 
        int     t1=4    triangle smoother for first wave 
        int     t2=4    triangle smoother for second wave
USED IN
        gee/lal/multiscale
        gee/mda/hole
        gee/pch/signoi
        sep/pwd/hole
        sep/pwd/signoi
        sep/spitz/sign
SOURCE
        user/gee/Mmake.c
VERSION
        Mmake.c 839 2004-10-25 11:54:43Z fomels
\end{verbatim}

{\begin{enumerate}
\item Making 1-D spike data.

\begin{verbatim}
sfspike n1=1000 nsp=2 k1=300,600 mag=1.0,2.0 > spike.rsf
\end{verbatim}
This command generates a 1-D list of 1000 numbers (n1=1000), all zero
except for two spikes (nsp=2) with amplitude 1.0 and 2.0 (mag=1.0,2.0)
at position 300 and 600 (k1=300,600). ps: \texttt{sfspike} need no input.
\texttt{``.rsf''} is Madagascar data file.

The file \texttt{spike.rsf} is a text header. The \emph{actual data}
are stored in the binary file pointed to by the \texttt{in=} parameter
in the header. You can look at the header file directly with
\texttt{more}, or better, examine the file properties with:
\begin{verbatim}
sfin spike.rsf
\end{verbatim}
You can learn more about the contents of \texttt{spike.rsf} with
\begin{verbatim}
sfattr < spike.rsf
\end{verbatim}

\item Generating graphic file.

The following command makes a graphics file from \texttt{spike.rsf}:
\begin{verbatim}
sfgraph title="Hello world!" < spike.rsf > spike.vpl
\end{verbatim}
The frequently-used plot functions are:

\texttt{sfgraph, sfwiggle, sfgrey, sfdots, sfgrey3}.

\texttt{``.vpl''} is Madagascar graphics file.

\item Displaying graphics.
\begin{verbatim}
xtpen < spike.vpl
\end{verbatim}

In addition, if you need save this graphics to other type graphics file,
try the followed commands.
\begin{verbatim}
vplot2eps < spike.vpl > spike.eps
\end{verbatim} 
or
\begin{verbatim}
vplot2gif < spike.vpl > spike.gif
\end{verbatim}

If you are ``Microsoft windows os'' user, another software ``Illustrator''
can be used to edit ``.eps'' vector file. 

\item Making 1-D synthetic trace with 5Hz Ricker wavelet.
\begin{verbatim}
sfricker1 frequency=5 < spike.rsf > wave.rsf
\end{verbatim}

\item Displaying the result.
\begin{verbatim}
sfgraph title="Two Ricker wavelets" < wave.rsf > wave.vpl
xtpen < wave.vpl
\end{verbatim}

\emph{You can \texttt{pipe} Madagascar commands together and do the whole 
thing at once like this:}
\begin{verbatim}
sfspike n1=1000 nsp=2 k1=300,600 mag=1.0,2.0 | sfricker1 frequency=5 | \
sfgraph symbol=y title="Two Ricker wavelets" | xtpen
\end{verbatim}

After that, we will have no any intermediate file except a figure like this:

\inputdir{exam1}
\plot{wave}{width=\textwidth}{Welcome to Madagascar.}

\end{enumerate}}

\item \href{http://rsf.sourceforge.net/SCons}{SCons files}.
%%%%% Lend from Reproducible computational experiments using SCons %%%%%
The main \texttt{SConstruct} commands defined in our reproducible
research environment are collected in Table~\ref{tbl:commands}.

\tabl{commands}{Basic methods of an \texttt{rsfproj} object.}{
\begin{center}
\begin{tabular}{|p{0.9\textwidth}|} \hline
\noindent\texttt{Fetch(data\_file,dir[,ftp\_server\_info])}\\ \ \\
\indent A rule to download \texttt{$<$data\_file$>$} from a specific
directory \texttt{$<$dir$>$} of an FTP server
\texttt{$<$ftp\_server\_info$>$}. \\
\hline 
\noindent\texttt{Flow(target[s],source[s],command[s][,stdin][,stdout])}\\ \ \\
\indent A rule to generate \texttt{$<$target[s]$>$} from
\texttt{$<$source[s]$>$} using \texttt{command[s]} \\
\hline 
\noindent\texttt{Plot(intermediate\_plot[,source],plot\_command)} or\\
\texttt{Plot(intermediate\_plot,intermediate\_plots,combination})\\ \ \\
\indent A rule to generate \texttt{$<$intermediate\_plot$>$} 
in the working directory. \\
\hline 
\noindent\texttt{Result(plot[,source],plot\_command)} or\\ 
\texttt{Result(plot,intermediate\_plots,combination})\\ \ \\
\indent A rule to generate a final \texttt{$<$plot$>$} in
the special \texttt{Fig} folder of the working directory. \\
\hline 
\noindent\texttt{End()} \\ \ \\
\indent A rule to collect default targets. \\
\hline 
\end{tabular}
\end{center}}

These commands are defined in \texttt{\$RSFROOT/lib/rsfproj.py} where
\texttt{RSFROOT} is the environmental variable to the Madagascar
installation directory. The source of this file is in
\href{http://sourceforge.net/p/rsf/code/HEAD/tree/trunk/framework/rsf/proj.py}{python/rsfproj.py}.
%
%%%%%%%%%%%%%%%%%%%%%%%%%%%%%%%%%%%%%%%%%%%%%%%%%%%%%%%%%%%%%%%%%%%%%%%%%

If you have SCons installed, you can use it to automate Madagascar
processing. Here is a simple \texttt{SConstruct} file to make
\texttt{wave.rsf} and
\texttt{wave.vpl}:
\definecolor{frame}{rgb}{0.905,0.905,0.905}
\lstset{language=Python,backgroundcolor=\color{frame},showstringspaces=false,numbers=left,numberstyle=\tiny}
\lstinputlisting[frame=single]{exam1/SConstruct}

Put the file in an empty directory, give it the name
\texttt{SConstruct}, cd to that directory, and issue the command:
\begin{verbatim}
scons
\end{verbatim}
The graphics file is now stored in the Fig subdirectory. You can view it
manually with:
\begin{verbatim}
xtpen Fig/wave.vpl
\end{verbatim}
or you can use
\begin{verbatim}
scons view
\end{verbatim}
When an \texttt{SConstruct} file makes more than one graphics file, the scons
view command will display all of them in sequence. 

When one use \texttt{SConstruct}, all Madagascar commands can be abbreviate by
removing ``sf''.

\end{enumerate}

\subsection{Example2 - Set up 2-D synthetic data and simple processing flow (Scons) }
\inputdir{exam2}

The second example is how to set up a simple 2-D prestack model and
processing flow.  The plan for this experiment is simply to create a
sythetic data, to add some noise and to eliminate it with median
filter.

\lstinputlisting[firstline=5,lastline=5,frame=single]{exam2/SConstruct}
%
is a standard Python command that loads the Madagascar project
management module \texttt{rsfproj.py} which provides our extension to
SCons. Do you want to use ``SU'' project? Try \texttt{``from suproj
import *''}.\\

\lstinputlisting[firstline=11,lastline=22,frame=single]{exam2/SConstruct}
%
instructs SCons to generate a velocity data \texttt{$V=2000+125*t$}. \\

\lstinputlisting[firstline=14,lastline=23,frame=single]{exam2/SConstruct}
%
generates the synthetic prestack multiple layed model. The number of layers 
is 17 \texttt{nsp=17}, the position of layers is controled by k1, 
\texttt{string.join(map(str,range (100,916,48)),',')} is a Python string 
command, the result is string
\texttt{100,148,196,...,868,916}. \texttt{band} and
\texttt{inmo} are used to make hyperbolas with wavelet. One can replace 
\texttt{bandpass flo=4 fhi=20} with \texttt{ricker1 frequency=2}.\\

\lstinputlisting[firstline=29,lastline=30,frame=single]{exam2/SConstruct}
%
displays the synthetic data with grey figure, \texttt{screenratio=1.2} is 
the parameter that can control the length-width ratio of a figure. More
parameters about ploting can be found by using \texttt{sfdoc stdplot} in
command line.

The frequently-used parameters are:

\texttt{screenratio, screenht, title, labeln, yreverse, etc.}

%\plot{synt}{height=0.25\textheight}{Synthetic prestack data.}

\lstinputlisting[firstline=36,lastline=36,frame=single]{exam2/SConstruct}
%
adds white saltpepper noise to synthetic data. \\

\lstinputlisting[firstline=42,lastline=43,frame=single]{exam2/SConstruct}
%
generate the temporary figure file about noisy data with wiggle
figure and final figure file.\\

\lstinputlisting[firstline=49,lastline=54,frame=single]{exam2/SConstruct}
%
executes the simple band-pass filtering and displays with grey figure.\\

\multiplot{3}{synt,noise,band}{width=0.25\textwidth}{Left: Synthetic 
   prestack data. Middle: Noisy data. Right: Result after band-pass filtering.}

\lstinputlisting[firstline=60,lastline=78,frame=single]{exam2/SConstruct}
%
plots the figure with advanced displaying methods. 

The frequently-used parameters of merging in \texttt{Result} are:

\texttt{SideBySideIso, SideBySideAniso, OverUnderIso, OverUnderAniso, TwoRows}.

\plot{sidebyside}{height=0.25\textheight}{Left: grey. Middle: 
                 wiggle. Right: color.}

\lstinputlisting[firstline=84,lastline=85,frame=single]{exam2/SConstruct}
%
executes the simple median filter and generates the corresponding
figure file.\\

\lstinputlisting[firstline=92,lastline=120,frame=single]{exam2/SConstruct}
%
extracts the $64th$ trace from profile (\texttt{window n2=1 f2=64}),
analyzes the corresponding amplitude spectra, and finally displays 
the comparison result.

\plot{compare}{height=0.5\textheight}{Top left: noisy model. Top
right: denoised result with median filter. Bottom left: amplitude spectra 
of the 64th trace in the noisy model. Bottom right: corresponding amplitude
spectra in denoised result.}

\lstinputlisting[firstline=126,lastline=126,frame=single]{exam2/SConstruct}
%
represents the ending symbol.

The whole \texttt{SConstruct} file is as follows:

\definecolor{frame}{rgb}{0.905,0.905,0.905}
\lstset{language=Python,backgroundcolor=\color{frame},showstringspaces=false,numbers=left,numberstyle=\tiny}
\lstinputlisting[frame=single]{exam2/SConstruct}

\subsection{Example3 - Set up 2-D Kirchhoff modeling and secondary processing flow (Scons)}
\inputdir{exam3}

The third example is how to set up a simple 2-D prestack model with
Kirchhoff method and advanced processing flow. The plan for this
experiment is to create a sythetic dataset, to execute NMO and
stacking.

\definecolor{frame}{rgb}{0.905,0.905,0.905}
\lstset{language=Python,backgroundcolor=\color{frame},showstringspaces=false,numbers=left,numberstyle=\tiny}
\lstinputlisting[frame=single]{exam3/SConstruct}

The corresponding figures are displayed as:
\multiplot{2}{modl,modl2}{width=0.45\textwidth}{Left: Model. 
            Right: True model.}
\plot{data}{height=0.4\textheight}{Synthetic data.}
\plot{cdp2500}{height=0.4\textheight}{Left: CDP gather at CDP=2.5km. 
           Middle: Velocity scan. Right: NMO.}
\plot{pick}{height=0.4\textheight}{Stacking velocity.}
\plot{stack}{height=0.4\textheight}{NMO stack.}

\subsection{Example4 - Set up 2-D Time-domain acoustic FD modeling}
\inputdir{exam4}

The fourth example is how to set up a simple 2-D prestack model with acoustic 
finite different method.

\definecolor{frame}{rgb}{0.905,0.905,0.905}
\lstset{language=Python,backgroundcolor=\color{frame},showstringspaces=false,numbers=left,numberstyle=\tiny}
\lstinputlisting[frame=single]{exam4/SConstruct}

\plot{mod2}{width=0.9\textwidth}{Velocity model}
\multiplot{4}{data15,data30,data45,data60}{width=0.45\textwidth}{Shot 
             gather at distance=8.1km,16.2km,24.3km,32.4km}
\multiplot{5}{snap45-1,snap45-2,snap45-3,snap45-4,snap45-5}{width=0.45\textwidth}{Snapshot at different time of shot at distance=21.6km}
\clearpage

\subsection{Example5 - Set up Prestack deep migration processing flow (Scons)}
\inputdir{exam5}

The fifth example will set up a simple 2-D prestack synthetic data
with kirchhoff modeling, and then apply 3-D S/R migration with
extended SSF to get the model image. 8 shots are used to image the
model.

\definecolor{frame}{rgb}{0.905,0.905,0.905}
\lstset{language=Python,backgroundcolor=\color{frame},showstringspaces=false,numbers=left,numberstyle=\tiny}
\lstinputlisting[frame=single]{exam5/SConstruct}

\plot{model}{width=0.6\textwidth}{Velocity model.}
\multiplot{8}{shot0,shot1,shot2,shot3,shot4,shot5,shot6,shot7}{width=0.35\textwidth}{Shot gather at distance=0.5km,1.0km,1.5km,2.0km,2.5km,3.0km,3.5km,4.0km}
\plot{img}{width=0.6\textwidth}{Image.}

\section{Lessons learned}

Over 20 years of experience with reproducible research in geophysical
applications taught us several important lessons. Whether you use
Madagascar in your research or you prefer other tools to implement
computational reproducibility, we hope that you can learn from our
lessons to improve the experience.

\subsection{``The principal beneficiary is the author.''}

\subsection{Each computation is a test.}

\subsection{Reproducibility requires maintenance.}

\subsection{Reproducibility maintenance requires an open community.}

\section{Conclusions}

... again about the importance of reproducibility ... history
... lessons ...

While the goal and purpose for reproducible research in computational
science are clearly defined, the tools used to implement
reproducibility in practice are constantly evolving. The Madagascar
project provides some of the tools that can be used today but awaits
new, more powerful or more convenient tools, that might be available
tomorrow.

In particular ... wish list ...

\bibliographystyle{seg}
\bibliography{mad}
