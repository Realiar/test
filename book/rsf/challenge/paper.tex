\title{Madagascar project: reproducibility challenge}

\author{Sergey Fomel}

\lefthead{Fomel}
\righthead{Reproducibility challenge}

\address {
sergey.fomel@beg.utexas.edu \\
Jackson School of Geosciences \\
The University of Texas at Austin \\ 
University Station, Box X \\
Austin, TX 78713-8924 \\
USA}

\maketitle

\begin{abstract}
This paper presents a challenge for different implementations of
reproducible research. The computational experiment refers to spatial
interpolation of rainfall measurements, a task of the 1997 Spatial
Interpolation Comparison. The reproducibility challenge is to capture
the computational workflow described in this paper in a form that
allows the reader to reproduce the experiment and to extend it by
trying a different algorithm, a different data set, or a different
choice of parameters.
\end{abstract}

\section{Introduction}
\inputdir{rain}

In 1997, the European Communities organized a Spatial Interpolation
Comparison. Many different organizations participated with the results
published in a special issue of the \emph{Journal of Geographic
Information and Decision Analysis} \cite[]{dubois} and a separate
report \cite[]{rain}.

\sideplot{elev}{width=\textwidth}{Digital elevation map of Switzerland.}

The comparison used a dataset from rainfall measurements in
Switzerland on the 8th of May 1986, the day of the Chernobyl disaster.
Figure~\ref{fig:elev} shows the data area: the Digital Elevation Model
of Switzerland with superimposed country's borders.  A total of 467
rainfall measurements were taken that day. A randomly selected subset
of 100 measurements was used as the input data the 1997 Spatial
Interpolation Comparison in order to interpolate other measurements
using different techniques and to compare the results with the known
data. Figure~\ref{fig:raindata} shows the spatial locations of the
selected data samples and the full dataset.

\plot{raindata}{width=\textwidth}{Left: locations of weather stations
  used as input data in the spatial interpolation contest.  Right: all
  weather stations locations.}

Many different spatial interpolation techniques have been tried on
this dataset \cite[]{dubois,Fomel.sepphd.107}. For simplicity, I
present here only one the result of simple Delaunay triangulation with
linear interpolation of rainfall values inside each triangle. The
interpolation result is shown in
Figure~\ref{fig:trian}. Figure~\ref{fig:pred} provides a comparison
between interpolated and known data values. It also indicates the
value of the correlation coefficient.

\sideplot{trian}{width=\textwidth}{Rainfall data interpolated using
  Delaunay triangulation.}

\sideplot{pred}{width=\textwidth}{Correlation between
  interpolated and true data values.}

Add precon?

\section{Reproducibility challenge}

How can the reader... The challenge of this exercise is to...

\subsection{Madagascar solution}

The solution adopted in the Madagascar project consists of...

To reproduce the paper using Madagascar, ...

\subsection{Additional materials}

In addition to the usual Madagascar files (\texttt{paper.tex} for the
source of this paper and \texttt{SConstruct} for describing the
workflow), I provide the following materials:
\begin{itemize}
\item Shell script \texttt{run.sh} for running the workflow (generated with \texttt{scons -n}).
\item IPython notebook \texttt{run.ipynb} that captures the 
      \texttt{SConstruct} file and the output of the experiment.
\end{itemize}

\appendix
\section{SConstruct}

\lstset{language=python,numbers=left,numberstyle=\tiny,showstringspaces=false}
\lstinputlisting[frame=single,title=rain/SConstruct]{rain/SConstruct}

\bibliographystyle{seg}
\bibliography{rain}
