\author{Maurice the Aye-Aye}
%%%%%%%%%%%%%%%%%%%%%%%%%%%%
\title{GEO 365N/384S Seismic Data Processing \\ Computational Assignment 0}

\maketitle

\begin{abstract}
  In this tutorial, you will go through different steps required for writing a research paper with reproducible examples with \texttt{Madagascar}. In particular, you will
  \begin{enumerate}
    \item identify a research problem,
    \item suggest a solution and test it,
    \item write  a report about your work.
  \end{enumerate}
\end{abstract}

\section{Prerequisites}

Completing this tutorial requires
\begin{itemize}
\item \texttt{Madagascar} software environment available from \\
\url{http://www.ahay.org}
\item \LaTeX\ environment with \texttt{SEG}\TeX available from \\ 
\url{http://www.ahay.org/wiki/SEGTeX}
\end{itemize}
To do the assignment on your personal computer, you need to install
the required environments. An Internet connection is required for
access to the data repository.

To setup the Madagascar environment in the JGB 3.216B computer lab, run the following commands:
\begin{verbatim}
$ module load madagascar-devel
$ setenv DATAPATH $HOME/data/
$ setenv RSFBOOK $HOME/data/book
$ setenv RSFFIGS $HOME/data/figs
\end{verbatim}
You can put these commands in your \verb+$HOME/.cshrc+ file to run them automatically at login time. \footnote{All commands above assume you are using the default \texttt{cshell} environment; certain modifications apply to different shell environments.}

To setup the \LaTeX\ environment, run the following commands (you only need to do it once):
\begin{verbatim}
$ cd; git clone https://github.com/SEGTeX/texmf.git
$ cd texmf; texhash
\end{verbatim}

The tutorial itself is available from the \texttt{Madagascar} repository
by running
\begin{verbatim}
$ svn co https://github.com/ahay/src/trunk/book/geo384s/hw0
\end{verbatim}

\section{Generating this document}

At any point of doing this computational assignment, you can
regenerate this document and display it on your screen.

\begin{enumerate}          
\item Change directory to \texttt{hw0}:
\begin{verbatim}
$ cd hw0
\end{verbatim}
\item Run
\begin{verbatim}
$ sftour scons lock
$ scons read &
\end{verbatim}
\end{enumerate}

As the first step, open \texttt{hw0/paper.tex} file in your favorite
editor (\texttt{vi}, \texttt{emacs}, \texttt{nano}, \texttt{gedit}, etc.) and edit the first line to enter your name. Then
run \texttt{scons read} again.

To remove (clean) previously generated files, run \texttt{\$ scons -c}. You can then rerun the program from scratch.


\section{Problem definition}
\inputdir{channel}

\plot{horizon}{width=\textwidth}{Depth slice from 3-D seismic (left) and output of edge detection (right).}

The left plot in Figure~\ref{fig:horizon} shows a depth slice from a 3-D
seismic volume\footnote{Courtesy of Matt Hall (ConocoPhillips Canada
Ltd.)}. You notice a channel structure and decide to extract it using
and edge detection algorithm from the image processing literature
\cite[]{canny}. In a nutshell, Canny's edge detector picks areas of
high gradient that seem to be aligned along an edge. The extracted
edges are shown in the right plot of Figure~\ref{fig:horizon}. The initial
result is not too clear, because it is affected by random
fluctuations in seismic amplitudes. The goal of your research project
is to achieve a better result in automatic channel extraction.

\begin{enumerate}
\item Change directory to the project directory
\begin{verbatim}
$ cd channel
\end{verbatim}
\item Run
\begin{verbatim}
$ scons horizon.view
\end{verbatim}
A number of commands will appear in the shell followed by Figure~\ref{fig:horizon} appearing on your screen. 
\item To understand the commands, examine the script that generated them by opening the \texttt{SConstruct} file in a text editor. Notice that, instead of shell commands, the script contains rules. 
\begin{itemize}
\item The first rule, \texttt{Fetch}, allows the script to download the input data file \texttt{horizon.asc} from the data server. 
\item Other rules have the form \texttt{Flow(target,source,command)} for generating data files or \texttt{Plot} and  \texttt{Result} for 
generating picture files. 
\item \texttt{Fetch}, \texttt{Flow}, \texttt{Plot}, and \texttt{Result} are defined in \texttt{Madagascar}'s \texttt{rsf.proj} package, which extends the functionality of \href{http://www.scons.org}{SCons}
\cite[]{icassp}.
\end{itemize}
\item To better understand how rules translate into commands, run 
\begin{verbatim}
$ scons -c horizon.rsf
\end{verbatim}
The \texttt{-c} flag tells scons to remove the \texttt{horizon.rsf} file and all its dependencies.
\item Next, run
\begin{verbatim}
$ scons -n horizon.rsf
\end{verbatim}
The \texttt{-n} flag tells scons not to run the command but simply to display it on the screen. Identify the lines in the \texttt{SConstruct} file that generate the output you see on the screen.
\item Run
\begin{verbatim}
$ scons horizon.rsf
\end{verbatim}
Examine the file \texttt{horizon.rsf} both by opening it in a text editor and by running
\begin{verbatim}
$ sfin horizon.rsf
\end{verbatim}
How many different \texttt{Madagascar} modules were used to create this file? What are the file dimensions? Where is the actual data stored?

\answer{
% Insert your answer here
}

\item Run
\begin{verbatim}
$ scons smoothed.rsf
\end{verbatim}
Notice that the \texttt{horizon.rsf} file is not being rebuilt.
\item What does the \texttt{sfsmooth} module do? Find it out by running
\begin{verbatim}
$ sfsmooth
\end{verbatim}
without arguments. Has \texttt{sfsmooth} been used in any other \texttt{Madagascar} examples?

\answer{
% Insert your answer here
}

\item What other \texttt{Madagascar} modules perform smoothing? To find out, run
\begin{verbatim}
$ sfdoc -k smooth
\end{verbatim}
\item Notice that Figure~\ref{fig:horizon} does not make a very good use of the color scale. To improve the scale, find the mean value of the data by running
\begin{verbatim}
$ sfattr < horizon.rsf
\end{verbatim}
and insert it as a new value for the \texttt{bias=} parameter in the
\texttt{SConstruct} file. Does smoothing by \texttt{sfsmooth} change
the mean value?

\answer{
% Insert your answer here
}

\item Save the \texttt{SConstruct} file and run 
\begin{verbatim}
$ scons view
\end{verbatim}
to view improved images. Notice that \texttt{horizon.rsf} and \texttt{smoothed.rsf} files are not being rebuilt. SCons is smart enough to know that only the 
part affected by your changes needs to be updated.
\end{enumerate}

As shown in Figure~\ref{fig:smoothed}, smoothing removes random
amplitude fluctuations but at the same time broadens the channel and thus
makes the channel edge detection unreliable. In the next part of this
tutorial, you will try to find a better solution by examining a simple
one-dimensional synthetic example.

\plot{smoothed}{width=\textwidth}{Depth slice from Figure~\ref{fig:horizon} after smoothing (left) and output of edge detection (right).}

\lstset{basicstyle=\small\ttfamily,breaklines=true}
\lstset{
    keywordstyle=\color{magenta},
    stringstyle=\color{blue},
    commentstyle=\color{cyan},
    numberstyle=\color{red},
}
\lstset{language=python,numbers=left,numberstyle=\tiny,showstringspaces=false}
\lstinputlisting[frame=single, title=channel/SConstruct]{channel/SConstruct}

\section{1-D synthetic}
\inputdir{local}

\multiplot{2}{step,smooth}{width=0.45\textwidth}{(a) 1-D synthetic to test edge-preserving smoothing. (b) Output of conventional triangle smoothing.}

To better understand the effect of smoothing, you decide to create a
one-dimensional synthetic example shown in Figure~\ref{fig:step}. The
synthetic contains both sharp edges and random noise.  The output of
conventional triangle smoothing is shown in
Figure~\ref{fig:smooth}. We see an effect similar to the one in the
real data example: random noise gets removed by smoothing at the
expense of blurring the edges. Can you do better?

\multiplot{2}{spray,local}{width=0.45\textwidth}{(a) Input synthetic trace duplicated multiple times. (b) Duplicated traces shifted so that each data 
sample gets surrounded by its neighbors. The original trace is in the middle.}

To better understand what is happening in the process of smoothing,
let us convert 1-D signal into a 2-D signal by first replicating the
trace several times and then shifting the replicated traces with
respect to the original trace (Figure~\ref{fig:spray,local}). This
creates a 2-D dataset, where each sample on the original trace is
surrounded by samples from neighboring traces.

Every local filtering operation can be understood as stacking traces
from Figure~\ref{fig:local} multiplied by weights that correspond to
the filter coefficients.

\begin{enumerate}
\item Change directory to the project directory
\begin{verbatim}
$ cd ../local
\end{verbatim}
\item Verify the claim above by running
\begin{verbatim}
$ scons smooth.view smooth2.view
\end{verbatim}
Open the \texttt{SConstruct} file in a text editor to verify that the first image is computed by \texttt{sfsmooth} and the second 
image is computed by applying triangle weights and stacking. To compare the two images by flipping between them, run
\begin{verbatim}
$ sfpen Fig/smooth.vpl Fig/smooth2.vpl
\end{verbatim}
\item Edit \texttt{SConstruct} to change the weight from triangle
\begin{equation}
\label{eq:triangle}
W_T(x) = 1-\frac{|x|}{x_0}
\end{equation}
to Gaussian
\begin{equation}
\label{eq:gaussian}
W_G(x) = \exp{\left(-\alpha\,\frac{|x|^2}{x_0^2}\right)}
\end{equation}
Repeat the previous computation. Does the result change? What is a good value for $\alpha$? 

\answer{
% Insert your answer here
}

\item Thinking about this problem, you invent an idea\footnote{Actually, you reinvent the idea of \emph{bilateral} or \emph{non-local} filters
\cite[]{tomasi,gilboa}.}. Why not apply non-linear filter weights that would discriminate between points not only based on their distance
from the center point but also on the difference in function values
between the points. That is, instead of filtering by
\begin{equation}
\label{eq:local}
g(x) = \int f(y)\,W(x-y)\,dy\;,
\end{equation}
where $f(x)$ is input, $g(x)$ is output, and $W(x)$ is a linear weight, you decide to filter by
\begin{equation}
\label{eq:nonlocal}
g(x) = \int f(y)\,\hat{W}\left(x-y,f(x)-f(y)\right)\,dy\;,
\end{equation}
where and $\hat{W}(x,z)$ is a non-linear weight. Compare the two weights by running
\begin{verbatim}
$ scons triangle.view similarity.view
\end{verbatim}
The results should look similar to Figure~\ref{fig:triangle,similarity}.
\item The final output is Figure~\ref{fig:nlsmooth}. By examining \texttt{SConstruct}, find how to reproduce this figure.

\clearpage

\item \textbf{EXTRA CREDIT} If you are familiar with programming in C, add 1-D non-local filtering as a new \texttt{Madagascar} module \texttt{sfnonloc}. Ask the instructor for further instructions. 
\end{enumerate}

\multiplot{2}{triangle,similarity}{width=0.45\textwidth}{(a) Linear and stationary triangle weights. (b) Non-linear and non-stationary weights reflecting both distance
between data points and similarity in data values.}

\sideplot{nlsmooth}{width=\textwidth}{Output of non-local smoothing}

Figure~\ref{fig:nlsmooth} shows that non-linear filtering can eliminate random noise while preserving the edges. The problem is solved! Now let us apply the result to our original problem.
 

\lstset{language=c,numbers=left,numberstyle=\tiny,showstringspaces=false}
\lstinputlisting[frame=single, title=Mnonlocal.c]{Mnonlocal.c}

\section{Solution}
\inputdir{channel2}

\begin{enumerate}
\item Change directory to the project directory
\begin{verbatim}
$ cd ../channel2
\end{verbatim}
\item By now, you should know what to do next.
\item Two-dimensional shifts generate a four-dimensional volume. Verify it by running
\begin{verbatim}
$ scons local.rsf
\end{verbatim}
and
\begin{verbatim}
$ sfin local.rsf
\end{verbatim}
View a movie of different shifts by running 
\begin{verbatim}
$ scons local.vpl
\end{verbatim}
\item Modify the filter weights by editing \texttt{SConstruct} in a text editor.
Observe your final result by running
\begin{verbatim}
$ scons smoothed2.view
\end{verbatim}
\item The file $\texttt{norm.rsf}$ contains the non-linear weights stacked over different shifts. Add a \texttt{Result} statement to  \texttt{SConstruct} that would display
the contents of $\texttt{norm.rsf}$ in a figure. Do you notice anything interesting?

\answer{
% Insert your answer here
}

\item Apply the Canny edge detection to your final result and display it in a figure.
\item \textbf{EXTRA CREDIT} Change directory to \verb#../mona# and apply your method to the image of Mona Lisa. Can you extract her smile?
\end{enumerate}

\lstset{language=python,numbers=left,numberstyle=\tiny,showstringspaces=false}
\lstinputlisting[frame=single, title=channel2/SConstruct]{channel2/SConstruct}

\sideplot{smoothed2}{width=0.75\textwidth}{Your final result.}

\inputdir{mona}
\sideplot{mona}{width=0.75\textwidth}{Can you apply your algorithm to Mona Lisa?}

\lstset{language=python,numbers=left,numberstyle=\tiny,showstringspaces=false}
\lstinputlisting[frame=single, title=mona/SConstruct]{mona/SConstruct}

\section{Writing a report}

\begin{enumerate}
\item Change directory to the parent directory
\begin{verbatim}
$ cd ..
\end{verbatim}
This should be the directory that contains \texttt{paper.tex}.
\item Run
\begin{verbatim}
$ sftour scons lock
\end{verbatim}
The \texttt{sftour} command visits all subdirectories and runs \texttt{scons lock}, which copies result files to a different location so that they do not get modified until further notice.
\item You can also run
\begin{verbatim}
$ sftour scons -c
\end{verbatim}
to clean intermediate results.
\item Edit the file \texttt{paper.tex} to include your additional results. If you have not used \LaTeX\ before, no worries. It is a descriptive language. Study the file, and it should become evident by example how to include figures.
\item Run
\begin{verbatim}
$ scons paper.pdf
\end{verbatim}
and open \texttt{paper.pdf} with a PDF viewing program such as \textbf{Acrobat Reader}. 
\item Want to submit your paper to \emph{Geophysics}? Edit \texttt{SConstruct} in the 
paper directory to add \texttt{options=manuscript} to the \texttt{End} statement. Then run
\begin{verbatim}
$ scons paper.pdf
\end{verbatim}
again and display the result.
\item If you have \LaTeX2HTML installed, you can also generate an HTML version of your paper by running
\begin{verbatim}
$ scons html
\end{verbatim}
and opening \verb#paper_html/index.html# in a web browser.
\end{enumerate}

%\lstset{language=python,numbers=left,numberstyle=\tiny,showstringspaces=false}
%\lstinputlisting[frame=single]{SConstruct}

\bibliographystyle{seg}
\bibliography{school}



