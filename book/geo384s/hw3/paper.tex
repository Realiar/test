\author{Team: Longhorns}
%%%%%%%%%%%%%%%%%%%%%%%%
\title{GEO 365N/384S Seismic Data Processing \\ Computational Assignment 3      }

\begin{abstract}
  This assignment is devoted to velocity analysis using NMO (normal
  moveout) and DMO (dip moveout). It has three parts:
  \begin{enumerate}
  \item Applying NMO velocity analysis to generate a stack of 2D marine data.
  \item Applying DMO by a Fourier-domain method.  
  \item Applying NMO velocity analysis to a CMP gather from 3D land data. 
  \end{enumerate}
\end{abstract}

\section{Prerequisites}

Completing the computational part of this homework assignment requires
\begin{itemize}
\item \texttt{Madagascar} software environment available from \\
\url{http://www.ahay.org/}
\item \LaTeX\ environment with \texttt{SEG}\TeX\ available from \\ 
\url{http://www.ahay.org/wiki/SEGTeX}
\end{itemize}
To do the assignment on your personal computer, you need to install
the required environments. 

To setup the Madagascar environment in the JGB 3.216B computer lab, run the following commands:
\begin{verbatim}
module load madagascar-devel
source $RSFROOT/share/madagascar/etc/env.csh
setenv DATAPATH $HOME/data/
setenv RSFBOOK $HOME/data/book
setenv RSFFIGS $HOME/data/figs
\end{verbatim}
You can put these commands in your \verb+$HOME/.cshrc+ file to run them automatically at login time.

To setup the \LaTeX\ environment, run the following commands:
\begin{verbatim}
cd
git clone https://github.com/SEGTeX/texmf.git
texhash
\end{verbatim}
You only need to do it once.

The homework code is available from the class repository
by running
\begin{verbatim}
svn co https://github.com/TCCS-BEG/geo384s/trunk/hw3
\end{verbatim}
You can also download it from your team's private repository.

\section{Generating this document}

At any point of doing this computational assignment, you can
regenerate this document and display it on your screen.

\begin{enumerate}          
\item Change directory to \texttt{hw3}:
\begin{verbatim}
cd hw3
\end{verbatim}
\item Run
\begin{verbatim}
sftour scons lock
scons read &
\end{verbatim}
\end{enumerate}

As the first step, open \texttt{hw3/paper.tex} file in your favorite
editor and edit the first line to enter the name of your team. Then
run \texttt{scons read} again.

\section{Normal moveout}
\inputdir{nmo}

In this part of the assignment, we will apply NMO-based velocity
analysis to the Viking Graben dataset \cite[]{CSI00-00-00010213}. The input data for this
assignment have been preprocessed by applying multiple attenuation
using the parabolic Radon transform (we will discuss this method later
in the class).

\begin{enumerate}          
\item Change directory to \texttt{hw3/nmo}.
\item Run
\begin{verbatim}
scons -c
\end{verbatim}
to remove (clean) previously generated files.
\item Run
\begin{verbatim}
scons cmps.view 
\end{verbatim}
to display the input data (Figure~\ref{fig:cmps}.)

\plot{cmps}{width=0.8\textwidth}{Viking Graben dataset after preprocessing and sorting into CMP gathers.}

\item We will test our velocity analysis first on one selected CMP gather. Run
\begin{verbatim}
scons cmp.view
\end{verbatim}
to display the selected gather (Figure~\ref{fig:cmp}.)

\plot{cmp}{width=0.8\textwidth}{CMP gather selected for velocity analysis.}

\item The process of NMO-based velocity analysis consists of three steps:
\begin{enumerate}
\item Semblance scan.
\item Picking velocities.
\item Applying NMO correction.
\end{enumerate}
The second step is typically done manually and consumes a significant
portion of the processor's effort. To speed up the process, we will do
it using an automatic picking algorithm.

Run
\begin{verbatim}
scons nmo1.view
\end{verbatim}
to observe the result of velocity analysis on the selected CMP gather
using automatic picking (Figure~\ref{fig:nmo1}.)

\item We usually expect seismic velocity to increase with depth. Some of the
low-velocity events picked in the lower part of the gather may not be
primary reflections.

Questions: Why does seismic velocity usually increase with depth? What
can be the physical nature of low-velocity seismic events appearing at
later times?

\answer{% Add your answer here
}

To guide the automatic picker to higher velocities, we can apply a
muting function to the semblance gather. Simple muting by cutting a
lower-left triangle corner from the semblance scan is implemented in
the attached \texttt{mute.c} function.

Run
\begin{verbatim}
scons nmo2.view
\end{verbatim}
to observe the effect of muting (Figure~\ref{fig:nmo1}.)

\multiplot{2}{nmo1,nmo2}{width=0.45\textwidth}{NMO velocity analysis with automatic picking applied to the selected CMP gather: (a) Initial pick, (b) Pick guided by muting.}

\item Once we are happy with our picking parameters, we can apply the procedure to the whole 2D line. 

Run
\begin{verbatim}
scons vstack.view
\end{verbatim}
to run the semblance analysis with automatic picking on every CMP
gather (this may take a while depending on the speed and the number of
cores of your computer). The result is shown in Figure~\ref{fig:vstack}.

\plot{vstack}{width=0.8\textwidth}{NMO stacking velocity picked automatically from the whole line.}

\item Finally, the picked velocity can be applied to NMO and stacking.         

Run
\begin{verbatim}
scons nmo.view
scons stack.view
\end{verbatim}
to display the result (Figures~\ref{fig:nmo} and~\ref{fig:stack}.) 

\plot{nmo}{width=0.8\textwidth}{Viking Graben dataset after the normal-moveout correction.}

\plot{stack}{width=0.8\textwidth}{NMO stack of the Viking Graben dataset.}

\item How well did we do? To examine NMO results by walking through different gathers in a movie, run
\begin{verbatim}
scons nmo.vpl
\end{verbatim}

The last section of the \texttt{SConstruct} file generates velocity
analysis plots for several selected gathers (Figure~\ref{fig:cmp700,cmp1050,cmp1400,cmp1750}.)

\multiplot{4}{cmp700,cmp1050,cmp1400,cmp1750}{width=0.45\textwidth}{NMO velocity analysis with automatic picking displayed for selected CMP gathers.}

\item Your task is to try improving the quality of the stack (Figure~\ref{fig:stack}) by improving the results of automatic picking. To achieve this goal, you have several tools at your disposal:
\begin{itemize}
\item Changing parameters of the picking program \texttt{sfpick}: See \url{http://ahay.org/blog/2012/08/01/program-of-the-month-sfpick/} for more explanation.
\item Changing parameters of the muting program \texttt{mute.c}: \texttt{t1=} and \texttt{v1=}. These parameters define the triangle cut from the semblance scan.
\item Changing the muting program \texttt{mute.c} to provide a better picking gate for the automatic picker.
\item Manual picking.
\end{itemize}

Describe the method you used to improve the results:

\answer{
}

\end{enumerate}

\lstset{language=c,numbers=left,numberstyle=\tiny,showstringspaces=false}
\lstinputlisting[frame=single,title=nmo/mute.c]{nmo/mute.c}

\lstset{language=python,numbers=left,numberstyle=\tiny,showstringspaces=false}
\lstinputlisting[frame=single,title=nmo/SConstruct]{nmo/SConstruct}

\section{DMO}
\inputdir{dmo}

In this section, we will apply an improved method of stacking
\cite[]{Fowler.sepphd.58} to the same dataset.

\begin{enumerate}

\item Change directory to \texttt{hw3/dmo}.
\item Run
\begin{verbatim}
scons -c
\end{verbatim}
to remove (clean) previously generated files.
\item The method starts with generating an ensemble of NMO stacks with constant velocities (Figure~\ref{fig:stacks}.)

\plot{stacks}{width=0.8\textwidth}{Viking Graben dataset after NMO stacking with an ensemble of constant velocities.}

To generate the stacks and display them on your screen, run
\begin{verbatim}
scons stacks.view
\end{verbatim}

\item Fowler's method works by transforming the stack volume into the frequency-wavenumber $\{\omega,k\}$ domain and applying the velocity mapping according to
\begin{equation}
\label{eq:fowler}
v_0 = \displaystyle v\,\left(1+\frac{k^2\,v^2}{4\,\omega^2}\right)^{-1/2}\;,
\end{equation}
where $v$ is NMO stacking velocity (dip-dependent) and $v_0$ is time-migration velocity (dip-independent).

The map defined by equation~(\ref{eq:fowler}) is shown in Figure~\ref{fig:map}. 
To display it, run
\begin{verbatim}
scons map.view
\end{verbatim}

\sideplot{map}{width=\textwidth}{Fourier-domain velocity map used in Fowler's DMO method.}

\item To perform the DMO correction of the constant-velocity stacks (Figure~\ref{fig:dmo}), run
\begin{verbatim}
scons dmo.view
\end{verbatim}
Compare the results by running
\begin{verbatim}
sfpen Fig/dmo.vpl Fig/stacks.vpl
\end{verbatim}
Do you notice any improvements?

\answer{
}

\plot{dmo}{width=0.8\textwidth}{Viking Graben dataset after DMO stacking with an ensemble of constant velocities.}

\item Fowler's method does not generate semblance scans, but it is possible to pick velocities using the power of the stack or the envelope.

To pick the DMO-corrected velocity automatically from the envelope (Figure~\ref{fig:vpick}), run
\begin{verbatim}
scons vpick.view
\end{verbatim}

\plot{vpick}{width=0.8\textwidth}{Migration velocity picked automatically from DMO stacks.}

\item After the velocity is picked, we can generate a DMO stack by slicing through the velocity cube. 

Run
\begin{verbatim}
scons slice.view
\end{verbatim}
to see the result (Figure~\ref{fig:slice}.)

\plot{slice}{width=0.8\textwidth}{DMO stack generated by slicing the constant-velocity stacks.}    

\item To evaluate the velocity picking and to observe the change brought by DMO, we can examine a particular CMP location. Run
\begin{verbatim}
scons envelope.view
\end{verbatim}
to display the result (Figure~\ref{fig:envelope}.)

Modify the \texttt{SConstruct} file to try several other CMP locations and select one that shows the most interesting change.

\sideplot{envelope}{width=\textwidth}{Comparison of velocity analysis before and after DMO at a selected CMP location.}

\item Similarly to the previous section, you creative task is to improve the stack (Figure~\ref{fig:slice}) by improving the result of automatic picking. You can use any tools to find a better slice through the velocity cube. Document your work and include the results in this document.

\item For an \textbf{EXTRA CREDIT}, use the materials of Computational Assignment 2 to perform Stolt migration of the DMO stack without leaving the Fourier transform. Note that you can also pick velocities after migration. Does it improve the migration results?

\end{enumerate}

\lstset{language=python,numbers=left,numberstyle=\tiny,showstringspaces=false}
\lstinputlisting[frame=single,title=dmo/SConstruct]{dmo/SConstruct}

\section{NMO Velocity Analysis with 3D Land Data}
\inputdir{nmo3d}

In the final section, we will apply NMO-based velocity analysis to a CMP gather from the Teapot Dome 3D land dataset.

\begin{enumerate}

\item Change directory to \texttt{hw3/nmo3d}.
\item Run
\begin{verbatim}
scons -c
\end{verbatim}
to remove (clean) previously generated files.
\item The input data for this exercise have been preprocessed to prepare them for prestack migration and velocity analysis. We will start by selecting a CMP gather at in-line 200 and cross-line 100 (Figure~\ref{fig:cmp3}.) You can refer to the acquisition geometry analysis from Computational Assignment 2 to locate this gather inside the survey.

\plot{cmp3}{width=0.8\textwidth}{Selected CMP gather from the Teapot Dome dataset.}

To display the gather in a wiggle plot, run
\begin{verbatim}
scons cmp3.view
\end{verbatim}

\item Despite the preprocessing, the data remain noisy. To examine how the NMO correction may affect seismic events, we can apply it first with an ensemble of constant velocities (Figure~\ref{fig:nmos}.)

To display the result, run
\begin{verbatim}
scons nmos.view
\end{verbatim}

What is the apparent velocity of the event arriving around 1 second?

\answer{
}

\plot{nmos}{width=0.8\textwidth}{Constant-velocity NMO corrections applied to the Teapot Dome CMP gather.}

\item Similarly to the 2D data, we can analyze velocities using semblance scans. To pick velocities, we will use, for a change, manual interactive picking (Figure~\ref{fig:picks}.)

To perform manual picking, uncomment the lines in the \texttt{SConstruct} file, which involve the \texttt{sfipick} program. Then run
\begin{verbatim}
scons picks3.view
\end{verbatim}
to perform your own manual picking.

Run
\begin{verbatim}
scons nmo3.view
\end{verbatim}
to display the NMO correction result (Figure~\ref{fig:nmo3}.)

\plot{picks}{width=0.8\textwidth}{Semblance scan with interactive picks.}

\plot{nmo3}{width=0.8\textwidth}{Velocity analysis using manual picking and NMO applied to the Teapot Dome CMP gather.}

\item Your final task: choose a different CMP gather (possibly the one with the highest CMP fold) and repeat the analysis. 

\end{enumerate}

\lstset{language=python,numbers=left,numberstyle=\tiny,showstringspaces=false}
\lstinputlisting[frame=single,title=nmo3d/SConstruct]{nmo3d/SConstruct}

\section{Saving your work}

It is important to save all files that you edit by hand (such
as \texttt{paper.tex} and \texttt{SConstruct}) in a version-control
system every time you make a revision.  The completed assignment is
due in two weeks and should be submitted through your private GitHub
repository.

\bibliographystyle{seg}
\bibliography{SEG,SEP2}
