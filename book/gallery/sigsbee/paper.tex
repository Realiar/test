\author{Vladimir Bashkardin, Sergey Fomel, Parvaneh Karimi, Alexander Klokov, and Xiaolei Song}
\title{Sigsbee model}

\maketitle

\begin{abstract}
The Subsalt Multiples Attenuation and Reduction Technology Joint
Venture (SMAART JV) publicly released several data sets between
September 2001 and November 2002.  These synthetic data model the
geologic setting found on the Sigsbee escarpment in the deep water
Gulf of Mexico.  Additional information may be found
at \url{http://www.delphi.tudelft.nl/SMAART/}.  The data sets remain
the property of SMAART and are used under the agreement found at the
SMAART site listed above.
\end{abstract}

\section{Model}
\inputdir{model}


\plot{strvel}{width=\textwidth}{``Stratigraphic'' velocity model.} \clearpage
\plot{migvel}{width=\textwidth}{``Migration'' velocity model.} \clearpage

\plot{zodata}{width=\textwidth}{Zero-offset data from
  finite-difference modeling.} \clearpage
\plot{shots}{width=\textwidth}{Shot gathers from
  finite-difference modeling.} \clearpage

\plot{refl}{width=\textwidth}{Exploding reflector.} \clearpage
\plot{sigexp}{width=\textwidth}{High-resolution zero-offset data modeled from exploding-reflector modeling.} \clearpage

\section{Lowrank RTM}
\inputdir{lowrank}

\plot{zomig}{width=\textwidth}{Zero-offset RTM image by the lowrank method.} \clearpage
\plot{zomigs}{width=\textwidth}{Zero-offset RTM image by the lowrank method using high-resolution data.} \clearpage

\section{FFD RTM}
\inputdir{ffd}

\plot{source}{width=\textwidth}{Source wavelet for FFD RTM} \clearpage
\plot{zomig}{width=\textwidth}{Zero-offset RTM image by Fourier finite-differences.} \clearpage
\plot{img-ps}{width=\textwidth}{Full-offset RTM image by Fourier finite-differences.} \clearpage



\section{One-way wave extrapolation}
\inputdir{oway}


\plot{mig}{width=\textwidth}{Zero-offset image by one-way wave-equation migration using extended split-step approximation.} \clearpage
\plot{agc}{width=\textwidth}{PSPI wave equation image.} \clearpage


\section{Multi-arrival Kirchhoff migration}
\inputdir{kirMulti}
\plot{sigs-image}{width=\textwidth}{Prestack multi-arrival Kirchhoff migration.} \clearpage
\plot{fig6150}{width=\textwidth}{Common-image gather in (a) the dip-angle domain and 
(b) the scattering-angle domain from position 6.15~km - sedimentary area.} \clearpage
\plot{fig23000}{width=\textwidth}{Common-image gather in (a) the dip-angle domain and 
(b) the scattering-angle domain from position 23~km - salt area.} \clearpage

\section{Common Reflection Angle Migration}
\inputdir{cram}

\plot{sigsbdcrstk}{width=\textwidth}{Common reflection angle migration.}
\multiplot{2}{sigsbocig0,sigsbdcig0}{width=0.45\textwidth}
{Opening (a) and dip (b) angle gathers at 7 km.}
\multiplot{2}{sigsbocig1,sigsbdcig1}{width=0.45\textwidth}
{Opening (a) and dip (b) angle gathers at 11 km.}
\multiplot{2}{sigsbocig2,sigsbdcig2}{width=0.45\textwidth}
{Opening (a) and dip (b) angle gathers at 15 km.}
\multiplot{2}{sigsbocig3,sigsbdcig3}{width=0.45\textwidth}
{Opening (a) and dip (b) angle gathers at 19 km.}
\multiplot{2}{sigsbocig4,sigsbdcig4}{width=0.45\textwidth}
{Opening (a) and dip (b) angle gathers at 23 km.} \clearpage
