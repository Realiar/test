\section{Moveout analysis}

If we restrict the observation to the immediate vicinity of the
reflection point, which means that we consider the moveout surface in
a small range of lags, we can approximate the typical irregular
wavefront in complex media by a plane, although the shapes of
wavefronts are arbitrary in heterogeneous media. Following the
derivation of \cite{YangSava.geo.mixlag} and using the geometry shown
in \rfg{Reflection}, the source and receiver plane waves are described
by:
%
\bea
\label{eqn:PWS} \ns \cdot \xx &=& v(\theta) t \;, \\
\label{eqn:PWR} \nr \cdot \lp \xx - 2d\nn \rp &=& v(\theta) t \;, 
\eea
%
where $\ns$ and $\nr$ are the unit direction vectors of the source and
receiver plane waves, respectively, \geouline{$\nn$ is the unit vector
  orthogonal to the reflector at the image point, and vector $\xx$
  indicates the image point position}. $v$ is defined as the phase
velocity in the locally homogeneous medium around the reflection
point, and thus it is identical for both wavefields. $\theta$ is half
the scattering angle (reflection angle).

We can also obtain the shifted source and receiver plane waves by
introducing the space- and time-lags
%
\bea
\label{eqn:SPWS} \ns \cdot \lp \xx + \hh \rp &=& v(\theta)  \lp t + \tt \rp \;, \\
\label{eqn:SPWR} \nr \cdot \lp \xx - 2d\nn - \hh \rp &=& v(\theta) \lp t - \tt \rp \;.
\eea
%
Solving the system of equations \ren{SPWS}-\ren{SPWR} leads to the
expression
%
\beq \label{eqn:pmov} 
\lp \ns - \nr \rp \cdot \xx = 2v(\theta) \tt - \lp \ns + \nr \rp \cdot \hh - 2d \nr \cdot \nn \;,
\eeq
%
which characterizes the moveout function \geouline{(surface)}
of space- and time-lags at a
common-image point.

Furthermore, we have the following relations for the reflection
geometry:
%
\bea
\label{eqn:pdif}  \ns - \nr &=& 2 \nn \cosc \;, \\
\label{eqn:psum}  \ns + \nr &=& 2 \vq \sinc \;,
\eea
%
where $\nn$ and $\vq$ are unit vectors normal and parallel to the
reflection plane, \geouline{respectively,} and $\theta$ is the reflection angle. \geouline{Vector
  $\vq$ characterizes the line representing the intersection of the
  reflection and the reflector planes.}  Combining
\reqs{pmov}-\ren{psum}, we obtain the moveout function for
plane waves:
%
\beq \label{eqn:Ztl_plane}
z \lp \hh, \tt \rp = d_0 - \frac{\tanc \lp \vq \cdot \hh \rp}{\nz} + \frac{\vt}{\nz \cosc} \;.
\eeq
%
The quantity $d_0$ is defined as
%
\beq
d_0 = \frac{d - \cn }{\nz} \;,
\eeq
%
and represents the depth of the reflection corresponding to the chosen
\geoulin2{common-image gather} (CIG) location. This quantity is invariant for different plane waves,
thus assumed constant here. \geouline{The vector ${\bf{c}}$ is along the 
Earth's surface given by $(x,y,0)$.}

When incorrect velocity is used for imaging, and thus, an inaccurate
reflection angle is assumed, based on the analysis in the preceding
section, we can obtain the moveout function
%
\beq \label{eqn:Ztl_plane_vm}
z \lp \hh, \tt \rp = d_{0f}
- \frac{\tanc_m \lp \vq_m \cdot \hh \rp}{\nmz} 
+ \frac{\vm(\theta_m) \lp \tt - \td \rp}{\nmz \cosc_m} \;,
\eeq
%
where $d_{0f}$ is the focusing depth of the corresponding reflection
point, $\vm$ is the migration velocity, $\td$ is the focusing error,
$\nnm$ and $\vq_m$ are vectors normal and parallel to the migrated
reflector, respectively. \rEq{Ztl_plane_vm} describes the extended
images moveout for a single seismic experiment and it is essentially
identical to the similar formula obtained by
\cite{YangSava.geo.mixlag} for isotropic media, but for using the phase velocity instead of the isotropic velocity.
