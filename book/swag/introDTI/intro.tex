\section{Introduction}

In recent years and with the increasing emphasis on high resolution and the availability of better computing devices, anisotropic-media treatment of seismic data is becoming part of normal operations rather than the exception. This preference is fueled by the recent observed improvements in\geouline{, for example,} Gulf of Mexico images when anisotropy is included in the process \cite[]{zhou:2347,huang:222}. A special anisotropy, a transversely isotropic medium with a tilt in the axis of symmetry, is especially convenient in approximating the features of the medium in such regions, and provides good imaging results.

Anisotropy characterization comes in many flavors approximating all sorts of phenomena present in the subsurface by introducing directional preferences in the velocity field that accommodates such phenomena. Whether we are dealing with the natural processes of sedimentation and gravity, especially in shales, or the ever localized vertical fractures (some also non-vertical), we can find an anisotropy that will approximate the processes influence and produce wavefields that can accurately represent wave propagation in such media.  The inclusion of anisotropy into the imaging and velocity model-building process evolved through the years. As expected, we started by looking into using the simplest of the anisotropies, that is elliptical anisotropy, to handle depth shortcomings of the isotropic assumption \cite[]{GEO60-05-14951513,ohlsen:1600}, despite its impracticality. However, the size of the consistent \geosout{nonhyperbolcity} \geouline{nonhyperbolicity} forced \geouline{some} \geosout{us} to use a slightly more complex model and yet more practical \cite[]{GEO59-09-14051418}, that is the transversely isotropic \geoulin2{(TI)} media with a vertical symmetry axis (VTI). Though this model proved resilient in many areas of the subsurface \cite[]{GEO62-02-06620675,martinez:149}, the dips of layers near salt flanks seemingly required additional degrees of freedom provided by the tilt of the symmetry axis in the transversely isotropic medium \cite[]{isaac:681}.

The vertical and \geouline{normal moveout} (NMO) velocities, as well as \geouline{the nonhyperbolic dimensionless parameter,} $\eta$, define the anisotropy aspects of the TI model for P-waves, at least to the accuracy required for prestack imaging \cite[]{GEO60-05-15501566}.  If the symmetry axis is vertical no other parameters are needed to define the TI model.  However, since the stratification in the Earth subsurface is not always horizontal, we can expect the symmetry axis to have some deviation from the vertical especially around salt-body flanks.  For TI media with a tilt in the axis of symmetry, two additional parameters that describe the tilt in 3D, are needed to fully characterize acoustic wave propagation.  These two parameters are often estimated by assuming that the tilt direction is normal to the medium structure or in the direction of the velocity gradient \cite[]{SEG-2000-09650968,audebert:P185}.  Setting the tilt normal to the dip direction has been convenient and practical. \cite{audebert:P185} realized through numerical experimentation that constraining the tilt of the symmetry axis to the structure, in what they referred to as structurally conformable TI (STI) media, results in simplifications in the parameter dependency in which the short-spread focusing becomes decoupled from long-spread behavior. In fact, setting the tilt normal to the dip results in simplified equations for data analysis, as we discussed here.

In this paper, we show explicitly that when we constrain the tilt to be normal to the reflector dip, the behavior of plane waves around the scattering point is explicitly represented by closed-form relations. In fact, the reflection angles for the source and the receiver rays are always equal. Thus, the key is to include this constraint as part of the process, whether it is migration or angle-gather development.  As a result, we call the medium dip-constrained TI (DTI) to stress the concept of using this constraint as part of the process as opposed to relating it to the structure of the model. We show that key equations are simplified by this \geouline{constraint} \geosout{constrain}, which will help in boosting the efficiency of such processes.  In fact, the DTI model makes processes such as angle-gather decomposition that depend on development around the scattering point simpler than the VTI model.

