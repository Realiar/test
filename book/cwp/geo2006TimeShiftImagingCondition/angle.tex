\section{Angle transformation in wave-equation imaging}
Using the definitions introduced in the preceding section,
we can make the standard notations for source and 
receiver coordinates:
$\ss = \mm - \hh$ and $\rr = \mm + \hh$.
The traveltime from a source to a receiver is a function
of all spatial coordinates of the seismic experiment
$ t = t \lp \mm,\hh \rp$.
Differentiating $t$ with respect to all
components of the vectors $\mm$ and $\hh$,
and using the standard notations
$ {\bf p}_\alpha = \nabla_\alpha t$, 
where $\alpha=\{\mm,\hh,\ss,\rr\}$, we can write:
\beqa \label{eqn:PmPh}
\pmv &=& \prv + \psv \;, 
\\   \label{eqn:PmPhb}
\phv &=& \prv - \psv \;.
\eeqa
From equations~\ren{PmPh}-\ren{PmPhb}, we can write
\beqa
2 \psv &=& \pmv - \phv \;, \\
2 \prv &=& \pmv + \phv \;.
\eeqa
% ------------------------------------------------------------
\inputdir{XFig}
\plot{vec3}{width=3.0in}
{Geometric relations between ray vectors at a reflection point.}
% ------------------------------------------------------------
By analyzing the geometric relations of various
vectors at an image point (Figure~\ref{fig:vec3}), 
we can write the following trigonometric expressions:
\beqa \label{eqn:cosph}
\mph^2 &=& \mps^2 + \mpr^2 - 2 \mps \mpr \cos(2 \t) \;,
\\   \label{eqn:cospm}
\mpm^2 &=& \mps^2 + \mpr^2 + 2 \mps \mpr \cos(2 \t) \;.
\eeqa
Equations~\ren{cosph}-\ren{cospm} relate wavefield quantities,
$\phv$ and $\pmv$, to a geometric quantity, reflection angle $\t$.
Analysis of these expressions provide sufficient information for
complete decompositions of migrated images in components for
different reflection angles.

\subsection{Space-shift imaging condition}
Defining $\kmv$ and $\khv$ as location and
offset wavenumber vectors, and assuming $\mps=\mpr=s$,
where $s\lp \mm \rp$ is the slowness at image locations,
we can replace 
$\mpm = \mkm/\w$ and
$\mph = \mkh/\w$ in equations~\ren{cosph}-\ren{cospm}:
\beqa \label{eqn:coskh}
\mkh^2 &=& 2(\w\,s)^2 (1 - \cos 2\t ) \;,
\\   \label{eqn:coskm}
\mkm^2 &=& 2(\w\,s)^2 (1 + \cos 2\t ) \;.
\eeqa
Using the trigonometric identity
\beq
\cos(2\t) = \frac{1-\tan^2 \t}
                 {1+\tan^2 \t} \;,
\eeq
we can eliminate from equations~\ren{coskh}-\ren{coskm}
the dependence on frequency and slowness, and obtain an angle 
decomposition formulation after imaging
by expressing $\tan \t$ as a function of position and
offset wavenumbers $(\kmv,\khv)$:
\beq \label{eqn:angX}
\tan \t = \frac{\mkh}{\mkm} \;.
\eeq

We can construct
angle-domain common-image gathers
by transforming prestack migrated images
using \req{angX}
\beq \label{eqn:XSTKa}
\RR \lp \mm, \hh \rp  \Longrightarrow 
\RR \lp \mm, \t  \rp \;.
\eeq
In 2D, this transformation is equivalent with a slant-stack
on migrated offset gathers. For 3D, this transformation
is described in more detail by \cite{Fomel.seg.3dadcig} or
\cite{SavaFomel.pag}.

\subsection{Time-shift imaging condition}
Using the same definitions as the ones introduced in the 
preceding subsection, we can re-write \req{coskm}
as
\beq
\mpm^2 = 4 s^2 \cos^2 \t \;,
\eeq
from which we can derive an expression for angle-transformation
after time-shift prestack imaging:
\beq \label{eqn:angT}
\cos \t = \frac{\mpm}{2s} \;.
\eeq
Relation~\ren{angT} can be interpreted using ray parameter vectors
at image locations (Figure~\ref{fig:img3}).
% ------------------------------------------------------------
\inputdir{XFig}
\plot{img3}{width=3.0in}
{Interpretation of angle-decomposition based on \req{angT}
for time-shift gathers.}
% ------------------------------------------------------------
Angle-domain common-image gathers can be obtained 
by transforming prestack migrated images using \req{angT}:
\beq \label{eqn:TSTKa}
\RR \lp \mm, \tt \rp \Longrightarrow 
\RR \lp \mm, \t  \rp \;.
\eeq
\rEq{angT} can be written as
\beq  \label{eqn:angTcart}
\cos^2 \t 
= \frac{|\nabla_{\mm} 2 \tt |^2}{4 s^2\lp \mm \rp}
= \frac{\tt_x^2 + \tt_y^2 + \tt_z^2}{s^2 \lp x,y,z\rp} \;,
\eeq
where $\tt_x,\tt_y,\tt_z$ are partial derivatives of $\tt$
relative to $x,y,z$.
We can rewrite \req{angTcart} as
\beq \label{eqn:angTalg}
\cos^2 \t = \frac{\tt_z^2}{s^2 \lp x,y,z \rp} \lp 1+z_x^2+z_y^2 \rp \;,
\eeq
where $z_x,z_y$ denote partial derivative of coordinate $z$
relative to coordinates $x$ and $y$, respectively.
\rEq{angTalg} describes an algorithm in two steps
for angle-decomposition after time-shift imaging:
compute $\cos \t$ through a slant-stack in $z-\tt$ panels
(find a change in $\tt$ with respect to $z$),
then apply a correction using the migration slowness $s$ and
a function of the structural dips $\sqrt{1+z_x^2+z_y^2}$.

