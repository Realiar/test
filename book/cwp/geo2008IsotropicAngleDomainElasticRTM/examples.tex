% ------------------------------------------------------------
\section{Examples}
% ------------------------------------------------------------

We test the different imaging conditions discussed in the preceding
sections with data simulated on a modified subset of the Marmousi II
model \cite[]{martin:1979}. The section is chosen to be at the left
side of the entire model which is relatively simple, and therefore it
is easier to examine the quality of the images.
% ------------------------------------------------------------
\inputdir{marm2oneA}

\multiplot[!b]{2}{vp,rx}{width=\textwidth}{(a) P- and S-wave
  velocity models and (b) density model used for isotropic elastic
  wavefield modeling, where $V_P$ ranges from $1.6$ to $3.2$~km/s from
  top to bottom and $V_P/V_S=2$, and density ranges from $1$ to
  $2$~g/cm$^3$ .}
% ------------------------------------------------------------
% ------------------------------------------------------------

\multiplot{2}{de1,de2,df1,df2}{angle=0,width=0.25\textwidth} {Elastic
  data simulated in model \rfn{marm2oneA-vp} and \rfn{marm2oneA-rx}
  with a source at $x=6.75$~km and $z=0.5$~km, and receivers at
  $z=0.5$~km: (a) vertical component, (b) horizontal component, (c)
  scalar potential and (d) vector potential of the elastic
  wavefield. Both vertical and horizontal components, panels (a) and
  (b), contain a mix of P and S modes, as seen by comparison with
  panels (c) and (d).}
% ------------------------------------------------------------
% ------------------------------------------------------------
\multiplot[h]{2}{ieall0,ieall1,ieall2,ieall3}{angle=0,width=0.48\textwidth}
          {Images produced with the displacement components imaging
            condition from \req{EICij}. \geouline{Panels (a), (b), (c)
              and (d) correspond to the cross-correlation of the
              vertical and horizontal components of the source
              wavefield with the vertical and horizontal components of
              the receiver wavefield, respectively. Images (a) to (d)
              are the $zz$, $zx$, $xz$ and $xx$ components,
              respectively.} The image corresponds to one shot at
            position $x=6.75$~km and $z=0.5$~km. Receivers are located
            at all locations at $z=0.5$~km.}
% ------------------------------------------------------------

% ------------------------------------------------------------
\multiplot[h]{2}{jeall0,jeall1,jeall2,jeall3}{angle=0,width=0.48\textwidth}
          {Images produced with the scalar and vector potentials
            imaging condition from \req{PICij}. \geouline{Panels (a),
              (b), (c) and (d) correspond to the cross-correlation of
              the P and S components of the source wavefield with the
              P and S components of the receiver wavefield,
              respectively.  }Images (a) to (d) are the $PP$, $PS$,
            $SP$ and $SS$ components, respectively.  The image
            corresponds to one shot at position $x=6.75$~km and
            $z=0.5$~km.  Receivers are located at all locations at
            $z=0.5$~km. \geouline{Panels (c) and (d) are blank because
              an explosive source was used to generate synthetic
              data.}}
% ------------------------------------------------------------
% ------------------------------------------------------------
\subsection{Imaging with vector displacements}

Consider the images obtained for the model depicted in
\rFgs{marm2oneA-vp} \geosout{\&} \geouline{and}
\rfn{marm2oneA-rx}. \rFg{marm2oneA-vp} depicts the P-wave velocity
(smooth function between $1.6-3.2$~km/s), and \rfg{marm2oneA-rx} shows
the density (variable between $1-2$~g/cm$^3$). The S-wave velocity is
a scaled version of the P-wave velocity with $V_P/V_S=2$.  \geosout{Data
are modeled and migrated in the smooth velocity background to avoid
back-scattering and the migration density is constant throughout the
model.}  \geouline{We use a smooth velocity background for both modeling
and migration.  We use density discontinuities to generate reflections
in modeling, but use a constant density in migration. The smooth
velocity background for both modeling and migration is used to avoid
back-scattering during wavefield reconstruction}.  The elastic data,
\rfgs{marm2oneA-de1} \geosout{\&} \geouline{and} \rfn{marm2oneA-de2}, are
simulated using a space-time staggered-grid finite-difference solution
to the isotropic elastic wave equation
\cite[]{GEO49-11-19331942,GEO51-04-08890901,GEO52-09-12111228,GEO53-06-07500759}.
We simulate data for a source located at position $x=6.75$~km and
$z=0.5$~km. Since we are using an explosive source and the background
velocity is smooth, the simulated wavefield is represented mainly by
P-wave incident energy and the receiver wavefield is represented by a
combination of P- and S-wave reflected energy. The data contain a mix
of P and S modes, as can be seen by comparing the vertical and
horizontal displacement components, shown in \rfgs{marm2oneA-de1}
\geosout{\&} \geouline{and} \rfn{marm2oneA-de2}, with the separated P and S
wave modes, shown in \rfgs{marm2oneA-df1} \geosout{\&} \geouline{and}
\rfn{marm2oneA-df2}.


Imaging the data shown in \rfgs{marm2oneA-de1} \geosout{\&}\geouline{and}
\rfn{marm2oneA-de2} using the imaging condition from \req{EICij}, we obtain
the images depicted in \rfgs{marm2oneA-ieall0} \geosout{\&}\geouline{to}
\rfn{marm2oneA-ieall3}.  \geouline{\rfgs{marm2oneA-ieall0} to
\rfn{marm2oneA-ieall3} correspond to the cross-correlation of the $z$
and $x$ components of the source wavefield with the $z$ and $x$
components of the receiver wavefield, respectively.}  Since the input
data do not represent separated wave modes, the images produced with
the imaging condition based on vector displacements do not separate PP
and PS reflectivity. Thus, the images are hard to interpret, since it
is not clear what incident and reflected wave mode\geouline{s}\geosout{do}
the reflections\geosout{correspond to} \geouline{represent}. In reality,
reflections corresponding to all wave modes are present in all panels.


% ------------------------------------------------------------
\subsection{Imaging with scalar and vector potentials}

Consider the images (\rfgs{marm2oneA-jeall0,jeall1,jeall2,jeall3})
obtained \geosout{for}\geouline{using the} imaging condition\geouline{from}
\geosout{\reqs{PICij} }\geouline{\req{PICij} }applied to the data
(\rfgs{marm2oneA-de1} and \rfn{marm2oneA-de2}) \geosout{used for}
\geouline{from} the preceding example. \geosout{Given the explosive}
\geosout{source used in}\geouline{Because we used an explosive source for}
our simulation, the source wavefield contains mostly P-wave energy,
while the receiver wavefield contains P- and S-wave mode
energy. Helmholtz decomposition after extrapolation but prior to
imaging isolates P and S wavefield components. Therefore, migration
produces images of reflectivity corresponding to PP and PS
reflections, \rfgs{marm2oneA-jeall0} and \rfn{marm2oneA-jeall1}, but
not reflectivity corresponding to SP or SS reflections,
\rfgs{marm2oneA-jeall2} and \rfn{marm2oneA-jeall3}. The illumination
regions are different between PP and PS images, due to different
illumination angles of the two propagation modes for the given
acquisition geometry. The PS image, \rfg{marm2oneA-jeall1}, also shows
the usual polarity reversal for positive and negative angles of
incidence measured relative to the reflector normal.  \geouline{By
comparing \rfgs{marm2oneA-jeall0} and \rfn{marm2oneA-jeall1} with
\rfgs{marm2oneA-ieall0} and \rfn{marm2oneA-ieall1}, it is apparent
that the crosstalk in the images obtained from displacement-based
imaging condition is more prominent than the one obtained from
potential-based imaging conditions, especially in
\rfg{marm2oneA-ieall0}. Furthermore, the polarity in
\rfg{marm2oneA-ieall1}, normally taken as the PS image, does
not reverse polarity at normal incidence, which is not correct
either.}

% ------------------------------------------------------------
\subsection{Angle decomposition}

The images shown in the preceding subsection correspond to the
conventional imaging conditions from \reqs{EICij} and \ren{PICij}.  We can
construct other images using the extended imaging conditions from
\reqs{XICa} and \ren{XICe}, which can be used for angle decomposition
after imaging. Then, we can use \req{PSang_hor} to compute angle
gathers from horizontal space cross-correlation lags.

\rFgs{marm2allA-je-0700-PPcig1} \geosout{\&} \geouline{and}
\rfn{marm2allA-je-0700-PScig1} together with
\rfgs{marm2allA-je-0700-PPang} \geosout{\&} \geouline{and}
\rfn{marm2allA-je-0700-PSang} show, respectively, the PP and PS
horizontal lags and angle gathers for the common image gather (CIG)
location in the middle of the reflectivity model, given a single
source at $x=6.75$~km and $z=0.5$~km. PP and PS horizontal lags are
lines dipping at angles \geosout{related} \geouline{that are equal} to the
incidence angles \geouline{(real incidence angles for PP reflection and
  average of incidence and reflection angles for PS reflection)} at
the CIG location. PP angles are larger than PS angles at all
reflectors, as illustrated on the simple synthetic example shown in
\rfg{cwgat}.

\rFgs{marm2allA-PPcig} \geosout{\&} \geouline{and} \rfn{marm2allA-PScig}
together with \rfgs{marm2allA-PPcig-ang} \geosout{\&} \geouline{and}
\rfn{marm2allA-PScig-ang} show, respectively, the PP and PS
horizontal lags and angle gathers for the same CIG location, given
many sources from $x=5.5$ to $7.5$~km and $z=0.5$~km. The horizontal
space cross-correlation lags are focused around $\hh=\mathbf{0}$,
which justifies the use of conventional imaging condition extracting
the cross-correlation of the source and receiver wavefields at zero
lag in space and time. Thus, the zero lag of the images obtained by
extended imaging condition represent the image at the particular CIG
location. The PP and PS gathers for many sources are flat, since the
migration was done with correct migration velocity. The PS angle
gather, depicted in \rfg{marm2allA-PScig-ang}, shows a polarity
reversal at $\theta=0$\geosout{, which is consistent with the fact that
  PS images change polarity at normal incidence} \geouline{as expected}.

% ------------------------------------------------------------
\inputdir{marm2allA}
% ------------------------------------------------------------

%\multiplot{2}{je-0700-PPcig,je-0700-PPang,je-0700-PScig,je-0700-PSang}
%{width=0.30\textwidth} {Horizontal cross-correlation lags for PP (a)
%and PS (c) reflections for model in \rfgs{marm2oneA-vp} and
%\rfn{marm2oneA-rx}. The source is at $x=6.75$~km, and the CIG is
%located at $x=6.5$~km. Panels (b) and (d) depict PP and PS angle
%gathers decomposed from the horizontal lag gathers in panels (a) and
%(c), respectively. As expected, PS angles are smaller than PP angles
%for a particular reflector due to smaller reflection angles.  }

\multiplot{2}{je-0700-Ecig11,je-0700-Eang11,je-0700-Ecig12,je-0700-Eang12}
{width=0.30\textwidth} {Horizontal cross-correlation lags for (a) PP
and (c) PS reflections for the model in \rfgs{marm2oneA-vp} and
\rfn{marm2oneA-rx}. The source is at $x=6.75$~km, and the CIG is
located at $x=6.5$~km. Panels (b) and (d) depict PP and PS angle
gathers decomposed from the horizontal lag gathers in panels (a) and
(c), respectively. As expected, PS angles are smaller than PP angles
for a particular reflector due to smaller reflection angles.  }

\multiplot{2}{PPcig,PPcig-ang,PScig,PScig-ang} {width=0.30\textwidth}
{Horizontal cross-correlation lags for PP (a) and PS (c) reflections
for \geouline{the} model in \rfgs{marm2oneA-vp} and
\rfn{marm2oneA-rx}. These CIGs correspond to 81 sources from $x=5.5$
to $7.5$~km at $z=0.5$~km. The CIG is located at $x=6.5$~km. Panels
(b) and (d) depict PP and PS angle gathers decomposed from the
horizontal lag gathers in panels (a) and (c), respectively. Since the
velocity used for imaging is correct, the PP and PS gathers are
flat. The PP angle gathers do not change polarity at normal incidence,
but the PS angle gathers change polarity at normal incidence.}

