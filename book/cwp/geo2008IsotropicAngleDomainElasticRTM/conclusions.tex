% ------------------------------------------------------------
\section{Conclusions}
% ------------------------------------------------------------

\geosout{The elastic reverse-time migration imaging condition can be
  formulated using decomposition of extrapolated wavefields in P and S
  wave modes. The resulting images separate reflections corresponding
  to forward-propagating P or S modes and backward propagating P or S
  modes. In contrast, images formed by simple cross-correlation of
  displacement wavefield components mix contributions from P and S
  reflections and are harder to interpret. Artifacts caused by
  back-propagating the recorded data with displacement sources are
  present in both types of images, although they are easier to
  distinguish and attenuate on the images constructed with elastic
  components separated prior to imaging.}

\geosout{Extended imaging conditions can be applied to multi-component
  images constructed using vector potentials. Angle decomposition
  allows for separation of reflectivity function of incidence and
  reflection angles. Angle-dependent reflectivity can be used for
  migration velocity analysis (MVA) and amplitude versus angle (AVA)
  analysis, which are extensions that fall outside the scope of this
  paper. Those techniques can be utilized in processing of land,
  ocean-bottom (OBC) and multi-component VSP data.}


\geouline{ We present a method for reverse-time migration with
  angle-domain imaging formulated for multicomponent elastic data. The
  method is based on the separation of elastic wavefields
  reconstructed in the subsurface into pure wave-modes using
  conventional Helmholtz decomposition. Elastic wavefields from the
  source and receivers are separated into pure compressional and
  transverse wave-modes which are then used for angle-domain
  imaging. The images formed using this procedure are interpretable in
  terms of the subsurface physical properties, for example, by
  analyzing the PP or PS angle-dependent reflectivity.  In contrast,
  images formed by simple cross-correlation of Cartesian components of
  reconstructed elastic wavefields mix contributions from P and S
  reflections and are harder to interpret. Artifacts caused by
  back-propagating the recorded data with displacement sources are
  present in both types of images, although they are easier to
  distinguish and attenuate on the images constructed with pure
  elastic wave-modes separated prior to imaging. }

\geouline{ The methodology is advantageous not only because it forms
  images with clearer physical meaning, but also because it is based
  on more accurate physics of wave propagation in elastic
  materials. For example, this methodology allows for wave-mode
  conversions in the process of wavefield reconstruction. This is in
  contrast with alternative methods for multicomponent imaging which
  separate wave-modes on the surface and then image those
  independently. In addition, elastic images can be formed in the
  angle-domain using extended imaging conditions, which offers the
  potential for migration velocity analysis (MVA) and amplitude versus
  angle (AVA) analysis.}


