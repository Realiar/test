\section{Estimation of interval parameters}

Similarly to effective parameters, interval parameters can also be
regarded as data attributes obtained from local slopes. Unlike the \tx
domain, the \taup domain offers us several alternatives to
conventional \citet{dix:68} processing for retrieving interval values.
In this section, we analyze three different options for extracting
interval parameters in the \taup domain.

\tabl{velocities}{This table lists the inputs required to retrieve
both effective and interval VTI parameters as data attributes obtained
from local slopes. The zero-slope time $\tau_0$ function is always
required to map the attributes to their correct position in
time. Processing data in \taup domain offers two more alternatives
than conventional Dix's processing in \tx domain. Moreover, we observe
that Fowler's is the only set of equations that {does not} require an
explicit use of the curvature.}{
\begin{center}
\large
\begin{tabular}{ cr | c | c | c | c | c | c |}
\cline{3-8}
	\multicolumn{2}{c}{}	  & \multicolumn{1}{|c|}{$\tau_{0}$} & $R$ &  $Q$  & $R_{\tau}$  & $Q_{\tau}$ & $\tau_{0,\tau}$ \\
\hline
\multicolumn{2}{|c|}{Effective $V_N$ $V_H$ $\eta$} & \tick & \tick & \tick  &   &  &\\
\hline
\multicolumn{1}{|c|}{\multirow{3}{*}{{Interval $\hat{V}_N$ $\hat{V}_H$ $\hat{\eta}$}}}
& {{Dix}}        & \tick  & \tick & \tick & \tick  & \tick &   \\
\cline{2-8}
\multicolumn{1}{|c|}{} 
& {{Stripping}} & \tick &         &       & \tick  & \tick &   \\
\cline{2-8}
\multicolumn{1}{|c|}{} 
& {{Fowler}}    & \tick  &         &       &  \tick &           & \tick  \\
\hline
\end{tabular}
\end{center}
}
\subsection{Dix Inversion}

Applying the chain rule, we \old{can} rewrite the Dix inversion formula
\citep{dix:68} as follows:
\begin{equation}
\hat{\mu}=\frac{d}{{d\tau }}\left[ {\tau _{0}(\tau )\mu (\tau )}\right] %
\left[ \frac{d{\tau }_{0}}{{d\tau }}\right] ^{-1}.  \label{eqn:dixchain}
\end{equation}
where $\hat{\mu}$ is a vertically-variable interval parameter, ${\mu}$
is the corresponding effective parameter, and $\tau_{0}(\tau)$ is the
zero-slope time mapping function. Substituting the LHS of
\reqs{2ndmoment} and \ren{4thmoment} as $\mu$ and \req{tau0mapping} as
$\tau_0$, we \old{can} deduce expressions for the interval NMO velocity
$\hat{V}_{N}$, horizontal velocity $\hat{V}_{H}$, and the
anellipticity parameter $\hat{\eta}$. The derivations and the final formulas
are detailed in appendix B. In order to retrieve interval parameters
by slope-based Dix inversion, one needs as inputs the slope $R$ and
\new{the} curvature $Q$ fields \old{and} \new{as well as} their derivatives along the time axis $\tau$
(see Table \ref{tbl:velocities}). This confirms that, even in \taup,
an application of Dix's formula requires the knowledge of the
effective quantities which, in this context, are mathematically
represented by the slope $R$ and curvature $Q$. The $\tau_0$ mapping
field is also needed to map the estimated VTI parameters to the
correct imaging time. The Dix inversion route does not seem very
practical in the \taup domain because the equations (derived in
appendix B) appear cumbersome.
	
%\rEq{dixchain} needs the $\tau$ derivative $R_{\tau}=\partial R / \partial \tau$ and $Q_{\tau}=\partial Q %/ \partial \tau$ of the slope   and curvature fields.




%Employing the Dix inversion approach \citep{dix:68} and defining $R_{\tau
%}=\partial R/\partial \tau $ we can deduce from equations \ref%
%{eqn:tau0mappingISO} and \ref{eqn:vNmapISO} an expression for the interval
%moveout\ velocity $\hat{V}_{N}$ in the isotropic or elliptical anisotropy
%case. Hence interval velocity $\hat{V}_{N}$ \ becomes another attribute we
%can directly extract from the data through slope estimation, as follows:
%
%\begin{equation}
%\hat{V}_{N}=\frac{1}{{p(pR-\tau )}}\frac{{pR(R+\tau R_{\tau })-2R_{\tau
%}\tau ^{2}}}{{{2\tau -p(R+\tau R_{\tau })}}}.  \label{eqn:vNdixISO}
%\end{equation}
%
%This equation is the $\tau -p$ analogous of the equation 15 already
%published in \cite{fomel:2046}. The derivation is in appendix A.
%
%In the case of non-elliptic $(\eta \neq 0) $ VTI anisotropy , although the algebra becomes rather involved, it is still possible to obtain two equations that relate effective $V_{N}^{\text{ \ \ }}$and $\eta $
%parameters to the interval $\hat{V}_{N}$ and $\hat{\eta}$ ones. Again,  starting from the 
%Dix formula and defining $%
%Q_{\tau }=\partial Q/\partial \tau ,N_{\tau }=\partial N/\partial \tau $ and
%$D_{\tau }=\partial D/\partial \tau ,$ we can deduce from equation \ref%
%{eqn:tau0mapping} and \ref{eqn:vNmap} the VTI interval moveout velocity $%
%\hat{V}_{N}:$
%

%
%The formula for interval $\hat{\eta}$ instead is entirely detailed in the
%appendix a. 



%Figure \ref{fig:dataSynthDix} shows the interval moveout $\hat{V}%
%_{N}$ (a) and horizontal $\hat{V}_{N}$ (b) velocity toghether with the
%interval anellipticity $\hat{\eta}$ (c) obtained using the relations \ref%
%{eqn:vNdixVTI} and \ref{eqn:etaDIX} on the synthetic data example shown in
%figure \ref{fig:dataSynth} (a). These profiles comes after the oriented
%mapping in equation \ref{eqn:tau0mapping}. The exact interval profiles (blue
%dotted lines) are recovered perfectly although the resolution worsens with
%respect to the effective profiles shown in figure \ref{fig:dataSynthOriented}%
%. Again, the main reason is the instability of the numerical differentiation
%along the $\tau $ direction. We noticed that the estimates of $\hat{V}_{H}$
%and $\hat{\eta},$the parameters that controls the large offset, are noisier
%than $\hat{V}_{N}$ and this fact agrees with the observation by \cite%
%{ilyabook2006} that states that high-order moveout parameters are in general
%less constrained than the short-spread normal moveout velocity $\hat{V}_{N}$%
%. The field data results in fugure \ref{fig:dataCongoFX} (a) are so much
%noisier such that only $\hat{V}_{N}$ velocity profile is meaningful and
%worthy of being presented.


\subsection{Claerbout's straightedge method}
\cite{Claerbout.sep.14.13} suggested that interval velocity in an
isotropic (\mbox{$\hat{V}_N=\hat{V}_H$}) layered medium could be
estimated with a \textit{pen and a straightedge} by measuring the
offset difference $\Delta x$ between equal slope $p$ points on two
reflection events (\rFg{commonray}a).  The computation of $\Delta x$
is straightforward after a CMP gather has been transformed into the
\taup domain. In fact, the \taup transform naturally aligns seismic
events with equal slope along the same trace
(\rFg{commonray}b). Moreover, local slopes $R(\tau,p)=d\tau/dp$ are
related to the emerging offset $x=-R$  \citep{baan:719} , therefore
Claerbout's inversion formula can be expressed as
\begin{equation}
\hat{V}_N^{2}(\tau,p)=\dfrac{R_{\tau}}{p^2R_{\tau}-p},  \label{eqn:Clarebout}
\end{equation}
where $R_{\tau}=\partial R(\tau,p)/\partial \tau$ and $\hat{V}_N$ is the
NMO interval velocity that we map back to zero-slope time $\tau_0$ using the
isotropic velocity independent \taup NMO, as suggested by \cite{fomel:S139}. The details of the derivation are in appendix C.

The two methods discussed next can be thought of as two alternative
extensions for VTI media of the original Claerbout's straightedge
method.

\subsection{Stripping equations}
The first alternative to the Dix inversion is what we call
\textit{stripping equations} \citep{casasantafomel}. Starting from the integral
\req{taup_int} for \taup reflection moveout and employing the chain
rule (\req{B4}), we first deduce an expression for slope
$R_{\tau}$ (\req{B5}) and curvature $Q_{\tau}$ (\req{B6}) using
$\tau$-derivatives, that now depend on the interval parameters. Then,
solving for $\hat{V}_{N}$ and $\hat{V}_{H}$, we obtain the following
expressions:
\begin{eqnarray}
\hat{V}_{N}^{2}(\tau, p) &=&-\frac{1}{p}\frac{{16R_{{\tau }}^{3}}}{{\hat{N}\hat{D}}},
\label{eqn:strippingVN} \\
\hat{V}_{H}^{2}(\tau, p) &=&\frac{1}{{p^{2}}}\dfrac{{\hat{N}-4R}_{{\tau }}}{{\hat{N}}}%
,  \label{eqn:strippingVH}
\end{eqnarray}
and
\begin{equation}
\hat{\eta}(\tau, p)=\frac{1}{p}\frac{{\hat{N}(4{R}_{{\tau }}-\hat{D})}}{{32\tau {R}_{{%
\tau }}^{3}}},  \label{eqn:strippingETA}
\end{equation}
which provide an estimate for the interval parameters. In the above equations,
$\hat{N}={pQ}_{{\tau }}{+~3R}_{{\tau }}{-3p{R}_{{\tau }}^{2}}$
and 
$\hat{D}=p{Q}_{{\tau }}+3{R}_{{\tau }}+p{{R}_{{\tau }}^{2}},$
which corresponds to the interval values of the numerator $N$ and
denominator $D$ of the square root in equation \ref{eqn:tau0mapping}.
These relations are very similar to those previously derived for the
effective parameters (equations
\ref{eqn:vNmap}--\ref{eqn:etamap}). However, they require the $\tau$
derivative of \new{the} slope and curvature fields (Table
\ref{tbl:velocities}). This result agrees with the discussion above
about layer stripping in $\tau$-$p$. In this domain, layer stripping
\old{is just a matter of} \new{reduces to} computing traveltime differences
(\req{taup_layers}) at each horizontal slowness $p$. Therefore,
differentiating the effective slope $R$ and curvature $Q$ fields in
$\tau$ provides the necessary information to access the interval
parameters directly. This is the power of the \taup domain as opposed
to \tx, where the only practical path to interval parameters is
through Dix inversion that requires the knowledge of effective
parameters. The zero-slope time $\tau_0$ \old{function} is needed to map the
interval parameter estimates to the correct vertical time (Table
\ref{tbl:velocities}).

%Figure \ref{fig:dataSynthStrip} shows the interval NMO velocity $\hat{V}_{N}$
%(a) $,$ interval horizontal velocity $\hat{V}_{H}$ (b) and interval
%anellipticity $\hat{\eta}$ (c) oriented mapping obtained after applying the
%stripping equations to the synthetic data example shown in figure \ref%
%{fig:dataSynth} (a). Again we observe a good match between the exact and the
%recovered \ interval parameters profiles, although $\hat{V}_{H}$ and $\hat{%
%\eta}$ display a loss in resolution mainly due to instabilities of numerical
%differentiation of slopes $R(\tau ,p)$ and curvature $Q(\tau ,p)$ fields.
%Moreover these results confirm the equivalence of all the three inversion
%methods we propose, even though they are derived in different ways.
%Numerical instabilities affected the field data processing even more and the
%method produces a reliable estimate of the NMO\ velocity profile only
%(Figure \ref{fig:dataCongoInterval} (c)).

%\plot{int_Syn}{width=0.8\textwidth}{Oriented inversion for interval parameters %using stripping
%equations: normal moveout (a) and horizontal velocity (b) and anellipticity
%parameter $\eta $ (c). Blue dotted lines represent the exact values
%used for generating the synthetic dataset in figure \ref%
%{fig:dataSynth} (a).\label{fig:dataSynthStrip}}

\subsection{Fowler's equations}

The second alternative to get VTI interval parameters comes from
the integral formulation of the \taup moveout signature in
\req{taup_int}. \new{The derivation is detailed in Appendix C.} We first compute
\mbox{$\tau_{0,\tau}=\partial\tau_{0}/\partial\tau$}  and
then, applying the chain rule, $R_{\tau}$. Solving for $\hat{V}_{N}$
and $\hat{V}_{H}$, we arrive at the following relations:
\begin{eqnarray}
\hat{V}_{N}^{2}(\tau,p) &=&-\frac{[ \tau_{0,\tau}^{2}-1]^{2}}{%
p^{3}\tau_{0,\tau}^{2}~R_{\tau} }, \label{eqn:fowler1} \\ 
\hat{V}_{H}^{2}(\tau,p) &=& \frac{\tau_{0,\tau}^{2}[1-R_{\tau}]+1}{p^{3}\tau_{0,\tau}^{2}~R_{\tau}~},
\label{eqn:fowler2}
\end{eqnarray}
which are equivalent to those proposed previously by
\cite{fowler:3028}. According to \reqs{fowler1} and \ren{fowler2}, the
gradients of offset $x$ and the zero-slope time $\tau_0$ measured at
common slope locations $p$ on two consecutive seismic event return
the VTI interval parameters for the layer bounded by these two events
(\rFg{commonray}a). \cite{fowler:3028} first pick traveltime curves in
\tx domain, and then differentiate those curves in offset to compute
slopes $p$. Finally, for any given $p$ value on each seismic event,
they determine the corresponding $\Delta x$ and $\Delta \tau_0$ values
(\rFg{commonray}a). The main practical limitation in this
inversion scheme is the difficulty of picking seismic events
accurately.

The processing becomes easier if it is accomplished in \taup with
automatic slope estimation. First, \mbox{\taup} transform unveils the
position of equal slope events.  Second, $\tau_{0,\tau}$ and
$R_{\tau}$ are measured automatically (without event picking) on the
\taup transformed CMP gather. The quantity $\tau_{0,\tau}$ can be
estimated as the $\tau$ finite difference of $\tau _{0}$ values
computed according to velocity-independent moveout equation
\ref{eqn:tau0mapping}. The zero-slope time $\tau_0$ function is still
needed to map the interval parameter estimated using \reqs{fowler1}
and \ren{fowler2} to the correct vertical time (Table
\ref{tbl:velocities}).







%In this section we propose an alternative approach to the oriented Dix
%method discussed above. Although the mathematical derivation is different,
%the aim is still to invert data local slopes estimate to obtain interval
%anisotropic parameters $\hat{V}_{N}$ and $\hat{V}_{H}$ (or $\hat{\eta})$. As
%previously stated, we require that the medium is laterally homogeneous, in
%order that the ray parameter $p$ will be preserved along a given ray. %
%\citet{Claerbout.sep.14.13} shows that traveltime and offset difference
%between equal slope $p$ points on two seismic events, can be inverted for
%seismic velocity in the layer bounded by these events. In layered isotropic
%media, he obtained this simple direct inversion formula
%
%\begin{equation}
%\hat{V}^{2}=\frac{\Delta x}{p\Delta t},  \label{eq:Clarebout}
%\end{equation}
%
%where $\hat{V}$ is the interval velocity, $\Delta x$ is the offset
%difference and $\Delta t$ is the traveltime difference, (see figure \ref%
%{fig:commonray} for a graphical interpretation of this quantities\ ) which
%we measure directly on the $\tau -p$ transformed CMP gather .
%
%\inputdir{extFig}
%
%
%
%We can extend this method to estimate interval parameters in VTI\ lateral
%homogeneous media. It is clear that now we need two independent equations to
%describe offset $\Delta x$ and traveltime $\Delta t$ difference as a
%function of the interval $\hat{V}_{N}$ and $\hat{V}_{H}$ velocities$.$
%
%Since local slopes are related to emerging offset $x$ , we start taking the
%first derivative with respect to the integral moveout formula in equation %
%\ref{eqn:taup_int}, namely
%
%\begin{equation}
%x=-R(\tau ,p)=\dint\limits_{0}^{\tau _{0}}\hat{V}_{P0}(\xi )q^{\prime
%}(p,\xi )d\xi ,  \label{eq:emerging_offset}
%\end{equation}
%
%where $\hat{V}_{P0}=\hat{V}_{P0}(\tau _{0})$ is a vertical profile for the
%vertical phase velocity and $q^{\prime }(p,\tau _{0})=\dfrac{dq(p,\tau _{0})%
%}{dp}$ is the first derivative of vertical slowness $q,$ which is as a
%function of interval parameter $\hat{V}_{N}=\hat{V}_{N}(\tau _{0})$ and $%
%\hat{V}_{H}=\hat{V}_{H}(\tau _{0})$. According to \ref{eq:tauptransform} the
%total traveltime $t$ becomes
%
%\begin{equation}
%t=\dint\limits_{0}^{\tau _{0}}\hat{V}_{P0}(\xi )\left( q^{\prime }(p,\xi
%)-pq^{\prime }(p,\xi )\right) d\xi  \label{eq:totaltraveltime}
%\end{equation}
%
%If we consider an arbitrary thin layer, with vertical interval time $\Delta
%\tau _{0},$ in which the medium parameter are assumed constant, we can
%reduce equation \ref{eq:emerging_offset} and \ref{eq:totaltraveltime} into
%their interval versions:
%
%\begin{eqnarray}
%\Delta x &=&-\Delta R(\Delta \tau _{0},p)=-\hat{V}_{P0}\Delta \tau
%_{0}q^{\prime },  \label{eq:DELTAX} \\
%\Delta t &=&\hat{V}_{P0}\Delta \tau _{0}\left( q-pq^{\prime }\right) ,
%\label{eq:DELTAT}
%\end{eqnarray}
%
%where $\Delta x$ and $\Delta R(\Delta \tau _{0},p)$ are the finite
%difference of offset and slope field at the boundaries of the layer, and $%
%\Delta t$ is the total traveltime that a ray with horizontal slowness $p$
%takes to cross the layer.
%
%
%\plot{commonray}{width=0.8\textwidth}{Common ray geometry considered for the inversion procedure. \label{fig:commonray}}
%
%In order to relate vertical $q$ to horizontal $p$ slowness we use the
%approximate relation in equation \ref{eq:qALKHA}. Substituting the first
%derivatives of vertical slowness function given in \ref{eq:qALKHA} we have
%
%\begin{eqnarray}
%\Delta x &=&\frac{\Delta \tau _{0}}{p~a~b^{3}}(b^{2}-a^{2}), \\
%\Delta t &=&\frac{\Delta \tau _{0}}{a~b^{3}}(b^{2}+a^{2}b^{2}-a^{2}),
%\end{eqnarray}
%
%where $a^{2}=1-p^{2}\hat{V}_{H}^{2}$ and $b^{2}=1-p^{2}(\hat{V}_{H}^{2}-\hat{%
%V}_{N}^{2}).$ We have finally obtained two independent equations for the two
%interval VTI parameters. In the case of isotropy or elliptical anisotropy ($%
%b^{2}=1$) they reduce to equation \ref{eq:Clarebout} previously published by
%\cite{Claerbout.sep.14.13}. After solving this system of equations for the
%interval values $\hat{V}_{N}$ and $\hat{V}_{H},$ we obtain the following
%inversion equations
%\begin{eqnarray}
%\hat{V}_{N}^{2} &=&\frac{(\Delta \tau _{0}^{2}-\Delta \tau ^{2})^{2}}{%
%p^{3}\Delta \tau _{0}^{2}~\Delta x~\Delta \tau },  \label{eqn:VNoriented} \\
%\hat{V}_{H}^{2} &=&\frac{\Delta \tau _{0}^{2}(p\Delta x~-\Delta \tau
%)+\Delta \tau ^{3}}{p^{3}\Delta \tau _{0}^{2}~\Delta x~},
%\label{eqn:VHoriented}
%\end{eqnarray}
%
%where $\Delta \tau =\Delta t-p\Delta x.$ These equations have already been
%derived in \cite{fowler:3028} but they were not applied in the context of
%oriented velocity mapping. \cite{fowler:3028} first pick traveltime curves
%in $t-x$ \ domain, and then they differentiate those curves over offset to
%compute slopes $p$. Finally, for any given $p$ value on each seismic event,
%they determine the corresponding $\Delta x$ and $\Delta t$ values (as
%sketched in figure \ref{fig:commonray}). The main practical limitation in
%this inversion scheme is the possibility to pick seismic events accurately
%and have reliable estimates of the event slopes. It's clear that this
%inversion scheme cannot be fully automated and requires time and manual work
%in the picking procedure.
%
%However according to oriented velocity mapping outlined in previous
%sections, we can measure from $\tau -p$ data the local slope field $R(\tau
%,p)$ using the regularized plane-wave destruction (PWD) algorithm. The $\tau
%-p$ transform naturally aligns seismic events with equal slope (figure \ref%
%{fig:taupscheme}) along the same trace. Moreover, since local slopes are
%related to the emerging offset $x=-\dfrac{d\tau }{dp}$ \citep{baan:719}$,$%
%the offset difference $\Delta x~$can be computed as the finite difference of
%the negative slope field $R(\tau ,p)$ along the $\tau $ direction. The
%quantity $\Delta \tau _{0}$ can be estimated as the finite difference of $%
%\tau _{0}$ values computed according to velocity-independent\ moveout
%equation \ref{eqn:tau0mapping} in ${\tau -p}$ domain.
%
%Finally, the equation \ref{eqn:VNoriented} and \ref{eqn:VHoriented} describe
%the interval parameters $\hat{V}_{N}$ and $\hat{V}_{H}$ as data attributes
%that we can extract directly from the data through local slopes estimate.
%Then we can also map these attributes to the appropriate zero slope-time $%
%\tau _{0}$ according to equation \ref{eqn:tau0mapping}. Figure \ref%
%{fig:dataSynthClar} shows the interval NMO velocity $\hat{V}_{N}$ (a) $,$
%interval horizontal velocity $\hat{V}_{H}$ (b) and interval anellipticity $%
%\hat{\eta}$ (c) oriented mapping according to equation \ref{eqn:VNoriented}
%and \ref{eqn:VHoriented} for the synthetic data example shown in figure \ref%
%{fig:dataSynth} (a).We observe a good match between the exact and the
%recovered \ interval parameters profiles, although $\hat{V}_{H}$ and $\hat{%
%\eta}$ display a loss in resolution mainly due to instabilities of numerical
%differentiation of slopes $R(\tau ,p)$ and $\tau _{0}(\tau ,p)$ field.
%Moreover these results are quite similar to those obtained by oriented Dix
%inversion (figure \ref{fig:dataSynthDix}): this confirms the complete
%equivalence of both methods, even though they are derived in different ways.
%Numerical instabilities affected the field data proessing even more and the
%method produces a reliable estimate of the NMO\ velocity profile only
%(Figure \ref{fig:dataCongoInterval} (b)).


%\begin{figure}[tbph]
%\begin{center}
%\subfigure[][]{\includegraphics[height=7.0
%cm]{fig/figSynthSin/oriented_Vnmo_int.pdf}}
%\subfigure[][]{\includegraphics[height=7.0
%cm]{fig/figSynthSin/oriented_Vhor_int.pdf}}
%\subfigure[][]{\includegraphics[height=7.0
%cm]{fig/figSynthSin/oriented_eta_int.pdf}}
%\end{center}
%\caption{Oriented direct inversion to interval normal moveout (a) and
%horizontal velocity (b) and anellipticity parameter $\eta $ (c).
%Blue dotted lines represent the exact values used for generating the
%synthetic dataset in figure \ref{fig:dataSynth} (a).}
%\label{fig:dataSynthClar}
%\end{figure}


\section{Flattening by predictive painting}

As shown in Table \ref{tbl:velocities}, Fowler's is the only set of
equations that do not require an explicit use of the curvature
$Q$. The dependence on the curvature is absorbed by the $\tau_0$
function. The other two sets of equations, Dix and stripping formulas,
as well as the equations for effective parameters, do need
curvature. The curvature computation can be problematic when the data
are contaminated by noise. This makes these three methods (effective,
stripping, and Dix) less practical when applied to real data with poor
SNR. However, Fowler's rules represent a way to circumvent the
problem. In fact, if we can find an algorithm that estimates the $\tau_0$
mapping function directly from the data, all the curvature issues
will get solved.

%So next the question is: is there exist such an algorithm?

\inputdir{flattening}
%\plot{flat-0-all}{width=0.8\textwidth}{Sythetic image form
%\cite{claerbout:2006} (a), local dip estimate (b), flattened image
%(c) using the predicted time of reflecting surfaces (d) obtained
%after painting of a single reference time-trace. This example is
%adapted from \cite{fomel:A25}.}

The desired algorithm exists and is known as seismic image flattening. 
\begin{comment}
Flattening is a common task in geophysical interpretation and is
typically used to identify stratigraphic features within a seismic
image. It attempts to reverse the effects of geologic processes by
removing all the complex structures (\rFg{flat-0-all}a) present in the
image to make these structural features flat (\rFg{flat-0-all}c). A
flattened image is typically created by shifting samples in the
original image up or down. In the recent past we can find in the
literature different approaches and solutions to make this process
fully automated. The methods proposed by \cite{lomask:P13},
\cite{parks:2010} and \cite{fomel:acc} use an estimate of local image
slopes to estimate the vertical shift field at each sample in the
image.
\end{comment}
The idea
of using local slopes for automatic flattening was introduced by
\cite{bienati} and \cite{lomask:P13}. 
Flattening by predictive painting (appendix A)
%\cite[]{fomel:A25}  
uses the local-slope field 
%(\rfg{flat-0-all}b) 
to construct a recursive prediction operator (equation \ref{eq:pred}) that spreads a traveltime reference trace in the image and predicts the reflecting surfaces 
%(\rfg{flat-0-all}d) 
which are then unwrapped until the image is flattened.
%(\rfg{flat-0-all} (c) ).

\inputdir{synth}

We propose bypassing the issue of estimating the zero-slope time $\tau
_{0}$ field by using the predictive painting \old{method}
\new{approach}. Let us discuss how it works on the previously shown
synthetic data in \rFg{dataPsynth}a.
%\citep{fomel:864,fomel:acc}. Predictive painting is a numerical
%algorithm for automatic spreading of information along seismic data,
%according to the local structure of seismic events. Structure is
%commonly characterized by dominant local slopes
%\citep{ClareboutESA}. Thus, the method first finds slope-based
%prediction operators of each trace in the data from its neighbor
%traces.
% (equation 5 in \citep{fomel:864}). 
%In the second step, prediction operators are used to spread
%information along the local data slopes. 
\rFg{dataPsynth}b shows local event slope $R$ measured from the data
using the PWD algorithm. \rFg{dataPsynth}c shows how predictive
painting spreads a zero-slope time $\tau _{0}$ reference trace along
local data slopes to predict the zero-slope time $\tau _{0}$ mapping
field and hence the geometry of the traveltime reflection curves along
\taup CMP gather. Because this procedure does not involve curvature
computations, it represents a much more robust way of obtaining the
$\tau _{0}$ field that is needed by the inversion formulas in
equations \ref{eqn:fowler1} and \ref{eqn:fowler2}. After $\tau _{0}$
has been found, we also have what we need to perform gather flattening
\citep{burnett:WB129,burnett:3710}.
% which is effectively an alternative velocity-independent procedure for accomplishing
%\taup moveout correction.  
Unshifting each trace (\rFg{dataPsynth}d) automatically flattens the
data, thus performing a velocity-independent \taup NMO correction. As
expected, all events are perfectly aligned, and the correction does
not suffer from instabilities of curvature estimation.  Moreover,
predictive painting is automatic and does not require any prior
assumptions about the moveout shape.
%These properties makes the procedure also suitable for other
%tasks, such as the inversion of interval parameters for SV-wave data
%proposed by \cite{stovas:2506}.

\plot*{dataPsynth}{width=\linewidth}{Synthetic  CMP $\tau$-$p$ transformed gather (a), estimated local slopes (b), zero slope time $\tau_0$ obtained by predictive painting (c), and the gather flattened (d). \label{fig:dataSYNTH}}

Now, given the slope field $R$ and its zero-slope time field $\tau
_{0}$, we retrieve interval parameters using equations
\ref{eqn:fowler1} and \ref{eqn:fowler2}. In \rFg{intPmasksynth}, the
estimated NMO (a) horizontal velocities (b) and the anellipticity (c)
parameter are mapped to the appropriate zero-slope time using the
painted zero-slope time $\tau _{0}$ field (\rFg{dataPsynth}c).
%Since the synthetic are noiseless, the results are similar and the
%following observations still hold for the other approaches that employ
%Dix inversion, stripping equation and also Fowler's equation in which
%$\tau_0$ comes from the velocity independent moveout \req{tau0mapping}.
The exact interval profiles (yellow lines) are recovered \new{nearly}
perfectly although the resolution slightly worsens with respect to the
effective profiles (\rFg{eff-Syn}). The main reason is the instability
of the additional numerical differentiation along the $\tau $
direction that all the approaches require.

\inputdir{synth} 
\plot{intPmasksynth}{width=\textwidth}{Fowler's equation based
  inversion to interval normal moveout (a) horizontal velocity (b) and
  anellipticity parameter $\eta $ (c).  The yellow lines represent the
  exact values used for generating the synthetic dataset in figure
  \ref{fig:dataSynth} (a). }
