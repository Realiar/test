\append{Local plane-waves operators}

Local plane-wave operators model seismic data \old{using local plane-waves} \cite[]{FomelPWD} . The mathematical basis is the local plane differential equation
\begin{equation}
  \frac{\partial P}{\partial x} + 
  \sigma\,\frac{\partial P}{\partial t} = 0\;,
  \label{eqn:pde}
\end{equation}
where $P(t,x)$ is the wave field and $\sigma$ the local slope field \cite[]{ClareboutESA}. 
In the case of a constant slope, the solution of equation~\ref{eqn:pde} is a simple plane wave
$ P(t,x) = f(t - \sigma x)$ where $f(t)$ is an arbitrary waveform. Assuming
that the slope $\sigma=\sigma(t,x)$ varies in time and space, one can design
a local operator to propagate each trace to its neighbors. Let $\mathbf{s}$ represent a seismic section as a collection of traces: $\mathbf{s} =
\left[\mathbf{s}_1 \; \mathbf{s}_2 \; \ldots \;
\mathbf{s}_N\right]^T$, where $\mathbf{s}_k$ corresponds to $P(t,x_k)$
for $k=1,2,\ldots$ A plane-wave destruction operator (PWD) effectively
predicts each trace from its neighbor and subtracts the prediction
from the original trace \citep{FomelPWD}.  In linear operator
notation, the plane-wave destruction operation can be defined as
\begin{equation}
  \label{eq:pwd}
  \mathbf{r} = \mathbf{D(\sigma)\,s}\;,
\end{equation}
where $\mathbf{r}$ is the destruction residual, and $\mathbf{D}$ is the
destruction operator defined as
\begin{equation}
  \label{eq:d}
  \mathbf{D}(\sigma) = 
  \left[\begin{array}{ccccc}
      \mathbf{I} & 0 & 0 & \cdots & 0 \\
      - \mathbf{P}_{1,2}(\sigma) & \mathbf{I} & 0 & \cdots & 0 \\
      0 & - \mathbf{P}_{2,3}(\sigma) & \mathbf{I} & \cdots & 0 \\
      \cdots & \cdots & \cdots & \cdots & \cdots \\
      0 & 0 & \cdots & - \mathbf{P}_{N-1,N}(\sigma) & \mathbf{I} \\
    \end{array}\right]\;,
\end{equation}
where $\mathbf{I}$ stands for the identity operator, and
$\mathbf{P}_{i,j}(\sigma)$ describes prediction of trace $j$ from
trace $i$. Prediction of a trace consists of shifting the original
trace along dominant event slopes $\sigma$. The dominant slopes are
estimated by minimizing the prediction residual $\mathbf{r}$ in a
least-squares sense. Since the prediction operators \ref{eq:d} depends
on the slopes themselves, the inverse problem is nonlinear and must
be solved in a iterative fashion by subsequent linearizations. We 
employ shaping regularization \citep{FomelShaping} for controlling the
smoothness of the estimated slope field.

Once the local slope field $\sigma$ has been computed, prediction of a
trace from a distant neighbor can be accomplished by simple
recursion. Predicting trace $k$ from trace $1$ is
\begin{equation}
  \label{eq:pred}
  \mathbf{P}_{1,k} = \mathbf{P}_{k-1,k}\,
  \cdots\,\mathbf{P}_{2,3}\,\mathbf{P}_{1,2}\;.
\end{equation}
If $\mathbf{s}_r$ is a reference trace, then the prediction of trace
  $\mathbf{s}_k$ is $\mathbf{P}_{r,k}\,\mathbf{s}_r$. \cite{fomel:A25}
  called the recursive operator $\mathbf{P}_{r,k}$ predictive
  painting. The elementary prediction operators in equation~\ref{eq:d}
  spread information from a given trace to its neighbors recursively
  by following the local structure of seismic events. \rFg{dataPsynth}
  in the main text illustrates the painting concept.


%%%%%%%%%%%%%%%%%%%%%%%%%%%%%%%%%%%%%%%%%%%%%%%
% Dix Equations
%%%%%%%%%%%%%%%%%%%%%%%%%%%%%%%%%%%%%%%%%%%%%%%
\append{Mathematical derivation of slope-based Dix inversion}

The Dix formula \citep{dix:68} can be written in the differential form
\begin{equation}
\hat{\mu}=\frac{d}{{d\tau _{0}}}\left[ {\tau _{0}\mu (\tau _{0})}\right] ,
\label{eqn:dix}
\end{equation}
where $\hat{\mu}$ is the interval parameter corresponding to
zero-slope time ${\tau _{0}}$ {and }${\mu (\tau _{0})}$ is the
vertically-variable general effective parameter. Using the chain
rule, we rewrite the Dix's formula \ref{eqn:dix} as follows:
\begin{equation}
\hat{\mu}=\frac{d}{{d\tau }}\left[ {\tau _{0}(\tau )\mu (\tau )}\right] %
\left[ \frac{d{\tau }_{0}}{{d\tau }}\right] ^{-1}.  \label{eqn:dixchain2}
\end{equation}

At first, let us consider the VTI NMO velocity as an effective parameter,
hence ${\mu =}V_{N}^{2}$.  Using the expression for ${\tau _{0}(\tau
)}$ (equation \ref{eqn:tau0mapping}) and $V_{N}^{2}{(\tau )}$
(equation \ref{eqn:vNmap}), we obtain, after some algebra, 
\begin{equation}
\frac{{d\tau _{0}}}{{d\tau }}=\frac{1}{{2\tau _{0}}}\frac{{\tau (2ND+\tau
(N_{\tau }D-D_{\tau }N))}}{{D^{2}}},  \label{eqn:dixVNVTI1}
\end{equation}%
\begin{equation}
\frac{{d\left[ {\tau _{0}(\tau )V_{N}^{2}(\tau )}\right] }}{{d\tau }}=
-\frac{{8\tau }^{2}{R^{2}(\tau )\left[ {6\tau DNR_{\tau }-\tau DRN_{\tau }-3\tau
NRD_{\tau }+4DNR}\right] }}{{pD^{3}N\tau _{0}}}, \label{eqn:dixVNVTI2}
\end{equation}
where ${R_{\tau }=}\partial {R/}\partial \tau $~,~ ${Q_{\tau }=}\partial {Q/}\partial \tau $ and
\begin{equation}
N_{\tau }=\partial N/\partial \tau =\left( {3\tau -6pR}\right) R_{\tau
}+p\tau Q_{\tau }+3R+pQ,  \label{eqn:numerator_tau}
\end{equation}%
\begin{equation}
D_{\tau }=\partial D/\partial \tau =\left( {3\tau +2pR}\right) R_{\tau
}+p\tau Q_{\tau }+3R+pQ.  \label{eqn:denominator_tau}
\end{equation}
Inserting equation \ref{eqn:dixVNVTI1} and \ref{eqn:dixVNVTI2}
in \ref{eqn:dixchain2} leads to
\begin{equation}
\hat{V}_{N}=-\frac{{16\tau R^{2}}}{{pDN}}\frac{{\left[ {6\tau DNR_{\tau
}-\tau DRN_{\tau }-3\tau NRD_{\tau }+4DNR}\right] }}{{\left[ {2ND+\tau
N_{\tau }D-\tau D_{\tau }N}\right] }}  \label{eqn:vNdixVTI}
\end{equation}.

To compute the interval $\hat{V}_H$ or $\hat{\eta}$, we employ as
effective value ${\mu =}S$ $V_{N}^{4}$ \ as described in equation
\ref{eqn:4thmoment}. In this case, the modified Dix formula (equation
\ref{eqn:dixchain2}) can be rewritten as follows
\begin{equation}
\hat{S}=\frac{1}{{\hat{V}_{N}^{4}(\tau )}}\frac{d}{{d\tau }}\left[ {\tau
_{0}(\tau )S(\tau )V_{N}^{4}(\tau )}\right] \left[ \frac{d{\tau }_{0}}{{d\tau }}\right] ^{-1},  
\label{eqn:dixchainETA}
\end{equation}
which, after substituting the chain relation for the interval
${\hat{V}_{N}^{2}(\tau )=}\frac{{d\left[ {\tau _{0}(\tau
)V_{N}^{2}(\tau )}\right] }}{{d\tau }}\left[ \frac{d{\tau
}_{0}}{{d\tau }}\right] ^{-1}$ and some simplifications, leads to
the relation	
\begin{equation}
\hat{S}={S(\tau )}\frac{{V_{N}^{2}(\tau )}}{{\hat{V}_{N}^{2}(\tau )}}+\tau
_{0}(\tau )\frac{{V_{N}^{2}(\tau )}}{{\hat{V}_{N}^{4}(\tau )}}\frac{{d}}{{%
d\tau }}\left[ {S(\tau )V_{N}^{2}(\tau )}\right] \left[ \frac{d{\tau }_{0}}{{%
d\tau }}\right] ^{-1},  \label{eqn:etaDIX}
\end{equation}
which involves only the mapping relations for the zero-slope time
$\tau _{0}(\tau)$ (formula \ref{eqn:tau0mapping}) effective NMO\
velocity $V_{N}^{2}(\tau )$ (formula \ref{eqn:vNmap}), and the
anellipticity parameter obtained by equation \ref{eqn:etamap} as
$S(\tau )=1+8\eta (\tau )$. From the interval parameter $\hat{S}$, we
can go back to interval
$\hat{V}_{H}^{2}=\hat{V}_{N}^{2}\,(\hat{S}+3)/4$ and
$\hat{\eta}=(\hat{S}-1)/8$.

In the case of isotropy or elliptical anisotropy ($\hat{V}_{N}=\hat{V}_{H}$), equations \ref{eqn:dixVNVTI1} and
\ref{eqn:dixVNVTI2} simplify to
\begin{equation}
\frac{{d\tau _{0}}}{{d\tau }}=\frac{{2\tau -p(R+\tau R_{\tau })}}{{2\tau _{0}%
}},  \label{eqn:dixISO1}
\end{equation}%
\begin{equation}
\frac{{d\left[ {\tau _{0}(\tau )V_{N}^{2}(\tau )}\right] }}{{d\tau }}=\frac{1%
}{2}\frac{{pR(R+\tau R_{\tau })-2R_{\tau }\tau ^{2}}}{{\tau _{0}p(pR-\tau )}}%
\label{eqn:dixISO2}
\end{equation}

Inserting equations \ref{eqn:dixISO1} and \ref{eqn:dixISO2} in
formula \ref{eqn:dixchain2}, we get
\begin{equation}
\hat{V}_{N}=\frac{1}{{p(pR-\tau )}}\frac{{pR(R+\tau R_{\tau })-2R_{\tau
}\tau ^{2}}}{{{2\tau -p(R+\tau R_{\tau })}}}.  \label{eqn:vNdixISO}
\end{equation}
This equation is the analog of equation 15 in \citep{fomel:2046}.

%%%%%%%%%%%%%%%%%%%%%%%%%%%%%%%%%%%%%%%%%%%%%%%%%%%%%%
% STRIPPING AND FOWLER'S
%%%%%%%%%%%%%%%%%%%%%%%%%%%%%%%%%%%%%%%%%%%%%%%%%%%%%%
\append{Mathematical derivation of stripping equations}

Starting from the integral representation of \taup signature in equation \ref{eqn:taup_int}, we arrive at the expression of
the slope $R$ and the curvature $Q$ fields as the following integrals:
\begin{eqnarray}
R (\tau_{0},p)&=&\int\limits_{0}^{\tau _{0}}\mathcal{F'}(p,\xi)d\xi ,\\
\label{eqn:B1}
Q (\tau_{0},p)&=&\int\limits_{0}^{\tau _{0}}\mathcal{F''}(p,\xi)d\xi ,
\label{eqn:B2}
\end{eqnarray}
where 
\[
\mathcal{F}(\tau_{0},p)=\sqrt{\frac{1-\hat{V}_{H}^{2}(\tau_{0})p^{2}}{1-[\hat{V}_{H}^{2}(\tau_{0} )-\hat{V}_{N}^{2}(\tau_{0} )]p^{2}}}\;,
\]
$\mathcal{F'}=\frac{d\mathcal{F}}{dp}$ and $\mathcal{F''}=\frac{d^2\mathcal{F}}{dp^2}$. According to \req{taup_int}, it descends that
\begin{equation}
\tau_{0,\tau} (\tau ,p) =\left[ \dfrac{\partial \tau_{0}}{\partial \tau}\right] ^{-1} =\dfrac{1}{\mathcal{F}(p,\tau)}.
\label{eqn:B3}
\end{equation}
Therefore, applying the chain rule 
\begin{equation}
\dfrac{\partial }{\partial \tau} = \left[ \dfrac{\partial \tau}{\partial \tau_{0}}\right] ^{-1} \dfrac{\partial }{\partial \tau_{0}}=\dfrac{1}{\mathcal{F}(\tau,p)}\dfrac{\partial}{ \partial \tau_{0}},
\label{eqn:B4}
\end{equation} 
we obtain
\begin{eqnarray}
R_{\tau}(\tau ,p)&=&\dfrac{1}{\mathcal{F}(\tau,p)} \dfrac{\partial R}{\partial \tau_0} =\dfrac{\mathcal{F'}(\tau ,p)}{\mathcal{F}(\tau ,p)}\;,
\label{eqn:B5} \\
Q_{\tau}(\tau ,p)&=&\dfrac{1}{\mathcal{F}(\tau,p)}\dfrac{\partial Q}{\partial \tau_0} =\dfrac{\mathcal{F''}(\tau, p)}{\mathcal{F}(\tau ,p)}\;.
\label{eqn:B6} 
\end{eqnarray} 

Solving equations \ren{B5} and \ren{B6} for $\hat{V}_N$ and $\hat{V}_H$ leads to
the \textit{stripping equations} \ren{strippingVN} and
\ren{strippingVH} in the main text. Alternatively, if we replace
the $\tau$ derivative of the curvature (\req{B4}~) with the squared
derivative of the zero-slope traveltime $\tau^2_{0,\tau}$
(\req{B2}~) and solve again for $\hat{V}_N$ and $\hat{V}_H$, we obtain
\textit{Fowler's equations} \ren{fowler1} and \ren{fowler2} in the
main text.

Note that no approximations were made here, other than Alkhalifah's
\textit{acoustic approximation} in equation~\ref{eqn:taup_int}. In the
case of isotropic or elliptical anisotropy
($\hat{V}_{N}=\hat{V}_{H}$), %since
%$\mathcal{F}(\tau_{0},p)=\tiny{\sqrt{1-\hat{V}_{N}^{2}(\tau_{0})p^{2}}}$,
one can just solve \req{B5} for $\hat{V}_{N}$ obtaining \req{Clarebout}
in the main text.
