\published{Geophysics, 80, V13-V21, (2015)}

\title{Adaptive prediction filtering in $t$-$x$-$y$ domain for random noise attenuation using regularized nonstationary autoregression}

\address{
College of Geo-exploration Science and Technology,\\
Jilin University \\
No.938 Xi minzhu street \\
Changchun, China, 130026}

\author{Yang Liu, Ning Liu, and Cai Liu}

\footer{GEO-2014-0011}
\lefthead{Liu etc.}
\righthead{$t$-$x$-$y$ APF}
\maketitle

\begin{abstract}
Many natural phenomena, including geologic events and geophysical
data, are fundamentally nonstationary. They may exhibit stationarity
on a short timescale but eventually alter their behavior in time and
space. We propose a 2D $t$-$x$ adaptive prediction filter (APF) and
further extend this to a 3D $t$-$x$-$y$ version for random noise
attenuation based on regularized nonstationary autoregression
(RNA). Instead of using patching, a popular method for handling
nonstationarity, we obtain smoothly nonstationary APF coefficients by
solving a global regularized least-squares problem. We use shaping
regularization to control the smoothness of the coefficients of
APF. 3D space-noncausal $t$-$x$-$y$ APF uses neighboring traces around
the target traces in the 3D seismic cube to predict noise-free signal,
so it provides more accurate prediction results than the 2D
version. In comparison with other denoising methods, such as
frequency-space deconvolution, time-space prediction filter, and
frequency-space RNA, we test the feasibility of our method in reducing
seismic random noise on three synthetic datasets. Results of applying
the proposed method to seismic field data demonstrate that
nonstationary $t$-$x$-$y$ APF is effective in practice.

\end{abstract}

\section{Introduction}

Attenuation of noise is a persistent problem in seismic
exploration. Random noise can come from various sources. Although
stacking can at least partly suppress random noise in prestack data,
residual random noise after stacking will decrease the accuracy of
the final data interpretation. In recent years, several authors have
developed effective methods of eliminating random noise. For example,
\cite{Ristau01} compared several adaptive filters, which they applied
in an attempt to reduce random noise in geophysical
data. \cite{Karsli06} applied complex-trace analysis to seismic data
for random-noise suppression, recommending it for low-fold seismic
data. Some transform methods were also used to deal with seismic
random noise, e.g., the discrete cosine transform \cite[]{Lu07}, the
curvelet transform \cite[]{Neelamani08}, and the seislet transform
\cite[]{Fomel10}. 

Seismic reflections are recorded according to special geometry and
always appear lateral continuity, which is used to distinguish events
from the background noise. If events of interest are linear (lines in
2D data and planes in 3D data), one can predict linear events by using
prediction techniques in the frequency-space domain or
the time-space domain \cite[]{Abma95}. The $f$-$x$ prediction
technique was introduced by \cite{Canales84} and further developed by
\cite{Gulunay86}, and is a standard industry 
method known as ``FXDECON''.
\cite{Sacchi01} utilized the autoregressive-moving average
(ARMA) structure of the signal to estimate $f$-$x$ prediction-error
filter (PEF) and the noise sequence. \cite{Liu12} developed $f$-$x$
regularized nonstationary autoregression (RNA) to attenuate random
noise. \cite{Liu13} further extended 2D $f$-$x$ RNA to 3D noncausal
regularized nonstationary autoregression (NRNA) for random noise
elimination.

The prediction process can be also achieved in $t$-$x$ domain
\cite[]{Claerbout92}. \cite{Abma95} discussed $f$-$x$ and $t$-$x$
approaches to predict linear events from random noise. $t$-$x$
prediction always passes less random noise than $f$-$x$ prediction
method because the $f$-$x$ prediction technique, while dividing the
prediction problem into separate problems for each frequency, produces
a filter as long as the data series in time. However, seismic data are
naturally nonstationary, and a standard $t$-$x$ prediction filter can
only be used to process stationary data \cite[]{Claerbout92}. Patching
is a common method to handle nonstationarity \cite[]{Claerbout10},
although it occasionally fails in the assumption of piecewise constant
dips. \cite{Crawley99} proposed smoothly nonstationary
prediction-error filters (PEFs) with ``micropatches'' and radial
smoothing, which typically produces better results than the
rectangular patching approach. \cite{GEO67-06-19461960} developed a
plane-wave destruction (PWD) filter \cite[]{Claerbout92} as an
alternative to $t$-$x$ PEF and applied a PWD operator to represent
nonstationary. However, PWD method depends on the assumption of a
small number of smoothly variable seismic dips. When background random
noise is strong, the dip estimation is a difficult
problem. \cite{Sacchi09} proposed an algorithm to compute time and
space variant prediction filters for noise attenuation, which is
implemented by a recursive scheme where the filter is continuously
adapted to predict the signal.

In this paper, we propose a $t$-$x$ adaptive prediction filter (APF)
to preserve nonstationary signal and attenuate random noise. The key
idea is the use of shaping regularization \cite[]{Fomel07} to
constrain the time and space smoothness of filter coefficients in the
corresponding ill-posed autoregression problem. The structure of
space-causal and space-noncausal APF is also discussed. We test
nonstationary characteristics of APF by using a 2D synthetic curved
model, a 2D synthetic poststack model, and a 3D prestack French
model. Results of applying the proposed method to the field example
demonstrate that regularized APF can be effective in eliminating
random noise.

\section{Theory}

\subsection{$t$-$x$ adaptive prediction filtering using RNA}

Consider a 2D prediction filter (general for stationary PF) with
ten prediction coefficients $B_{i,j}$:
\begin{equation}
   \begin{array}{ccc}
      \cdot         &B_{-2,1}  &B_{-2,2} \\
      \cdot         &B_{-1,1}  &B_{-1,2} \\
      1(t,x)        &B_{0,1}   &B_{0,2}  \\
      \cdot         &B_{1,1}   &B_{1,2}  \\
      \cdot         &B_{2,1}   &B_{2,2}  
  \end{array}
\label{eqn:noilace}
\end{equation}
where $i$ is time shift, $j$ is space shift, the vertical axis is time
axis, and the horizontal axis is space axis. The output position is
under the ``$1(t,x)$'' coefficient on the left side of filter and
``$1(t,x)$'' indicates time- and space-varying samples. The filter is
noncausal along the time axis and causal along the space axis. More
filter structures will be discussed later. The PF has the different
coefficients from PEF, which includes causal time prediction
coefficients.

To obtain stationary PF coefficients, one can solve the
over-determined least-squares problem 
{\setlength\arraycolsep{2pt}
\begin{eqnarray}
  \label{eq:pred1}
\tilde{B}_{i,j} &=& \arg\min_{B_{i,j}}\|S(t,x)-
              \sum_{j=1}^{N}\sum_{i=-M}^{M} B_{i,j}S_{i,j}(t,x)\|_2^2 \;,
\end{eqnarray}}
where $S_{i,j}(t,x)$ represents the translation of linear events
$S(t,x)$ in both time and space directions with time shift $i$ and
space shift $j$. The choice of the filter size depends on the maximum
dip of the plane waves in the data and the number of dips. For
nonlinear events, cutting data into overlapping windows (patching) is
a common method to handle nonstationarity
\cite[]{Claerbout10}, although it occasionally fails in the presence
of variable dips.

For nonstationary situations, we can also assume local linearization
of the data. For estimating APF coefficients, nonstationary
autoregression allows the coefficients $B_{i,j}$ to change with both
$t$ and $x$. The new adaptive filter can be designed as
\begin{equation}
   \begin{array}{ccc} 
      \cdot   &B_{-2,1}(t,x)  &B_{-2,2}(t,x) \\
      \cdot   &B_{-1,1}(t,x)  &B_{-1,2}(t,x) \\
      1(t,x)  &B_{0,1}(t,x)   &B_{0,2}(t,x)  \\
      \cdot   &B_{1,1}(t,x)   &B_{1,2}(t,x)  \\
      \cdot   &B_{2,1}(t,x)   &B_{2,2}(t,x) 
   \end{array}
\label{eqn:ailacefil}
\end{equation}

In the linear notation, prediction coefficients $B_{i,j}(t,x)$ can be
obtained by solving the under-determined least-squares problem
{\setlength\arraycolsep{2pt}
\begin{eqnarray}
  \label{eq:pred2}
\tilde{B}_{i,j}(t,x) &=& \arg\min_{B_{i,j}(t,x)}\|S(t,x)-
  \sum_{j=1}^{N} \sum_{i=-M}^{M} B_{i,j}(t,x)S_{i,j}(t,x)\|_2^2 \nonumber \\
  & & + \epsilon^2\, \sum_{j=1}^{N} \sum_{i=-M}^{M} 
  \|\mathbf{D}[B_{i,j}(t,x)]\|_2^2\;,
\end{eqnarray}}
where $\mathbf{D}$ is the regularization operator and $\epsilon$ is a
scalar regularization parameter. This approach was described by
\cite{Fomel09} as regularized nonstationary autoregression (RNA). Shaping
regularization \cite[]{Fomel07} specified a shaping (smoothing)
operator $\mathbf{R}$ instead of $\mathbf{D}$ and provided better
numerical properties than Tikhonov's regularization
\cite[]{Tikhonov63} in equation~\ref{eq:pred2}. The advantages of 
the shaping regularization include an intuitive selection of
regularization parameters and fast iteration convergence. Coefficients
$B_{i,j}(t,x)$ get constrained by regularization. The required
parameters are the size and shape of the filter, $B_{i,j}(t,x)$, and
the smoothing radius for shaping regularization. The size of APF
controls the range and the number of the predicted dips. Larger filter
parameters, $N$ and $M$, are able to predict more accurate dips,
however, the APFs with the large filter size pass more random noise
and add more computational cost. As the smoothing radius of the APF
increases, the APF removes not only more random noise but also some
structural details. The APF is able to be extended to the adaptive PEF
(APEF), which shows a expected representation of nonstationary signal
and is fit for seismic data interpolation \cite[]{Liu11a} and random
noise attenuation \cite[]{Liu11b}. However, the structure of APEF is
different from that of APF, which excludes the causal time prediction
coefficients and forces only lateral predictions. Meanwhile, in this
paper, we use a two-step method that estimates APF coefficients by
solving an under-determined problem and calculates noise-free signal.
The proposed method is different from the two-step APEF denoising
including APEF estimation for signal and noise plus signal and noise
separation by solving a least-square system as shown in
\cite{Liu11b}.

\subsection{3D space-noncausal adaptive prediction filtering}
Equation~\ref{eq:pred1} and \ref{eq:pred2} show space-forward
prediction equations, which are similar as the causal prediction
filtering equations in $f$-$x$ domain \cite[]{Gulunay00}. Furthermore,
$t$-$x$ prediction filter also includes time-noncausal
coefficients. In the case of both space-forward and space-backward
prediction equations (space-noncausal prediction filter),
equation~\ref{eq:pred2} can be written
\cite[]{GEO56-06-07850794,Naghizadeh09,Liu12,Liu13} with the results
averaged: 
{\setlength\arraycolsep{2pt}
\begin{eqnarray}
  \label{eq:pred3}
\tilde{B}_{i,j}(t,x) &=& \arg\min_{B_{i,j}(t,x)}\|S(t,x)-
  \sum_{j=-N,j\neq0}^{N} \sum_{i=-M}^{M}
   B_{i,j}(t,x)S_{i,j}(t,x)\|_2^2 \nonumber \\
  & & + \epsilon^2\, \sum_{j=-N,j\neq0}^{N} \sum_{i=-M}^{M} 
  \|\mathbf{R}[B_{i,j}(t,x)]\|_2^2\;,
\end{eqnarray}}

Equation~\ref{eq:pred3} shows that one sample in $t$-$x$ domain can be
predicted by the samples in adjacent traces with weight coefficients
$B_{i,j}(t,x)$, which is time- and space-varying. The equation assumes
that the seismic data only consist of plane waves
$\sum_{j=-N,j\neq0}^{N}
\sum_{i=-M}^{M} B_{i,j}(t,x)S_{i,j}(t,x)$ and random noise
that corresponds to a least-squares error.

Figure~\ref{fig:causal2d,noncausal2d}a shows a 2D space-causal APF
structure, which is time-noncausal filter. White grids stand for
prediction samples and the dark-grey grid is the output (or target)
position, while light-grey grids are unused samples. The filter size
of the space-causal APF is $N\times(2M+1)$. Meanwhile, space-noncausal
APF (Figure~\ref{fig:causal2d,noncausal2d}b) has a symmetric structure
along time and space axes. The filter size of the space-noncausal APF
is $2N\times(2M+1)$. The 3D $t$-$x$-$y$ APF also has space-causal or
space-noncausal structure, Figure~\ref{fig:noncausal3d} shows the
noncausal one. In a 3D seismic datacube, the plane events can be
predicted along two different spatial directions. A 2D $t$-$x$ APF
will have difficulty preserving accurate plane waves because it only
uses the information in $x$ or $y$ direction, however, a 3D
$t$-$x$-$y$ APF provides a more natural structure. $t$-$x$-$y$
adaptive prediction filtering for random noise attenuation follows two
steps:

1. Estimating 3D space-noncausal APF coefficients
   $\tilde{B}_{i,j,k}(t,x,y)$ by solving the regularized least-squares
   problem (equation~\ref{eq:pred2} or \ref{eq:pred3} in 2D):
   {\setlength\arraycolsep{2pt}
\begin{eqnarray}
  \label{eq:pred4}
\tilde{B}_{i,j,k}(t,x,y) &=& \arg\min_{B_{i,j,k}(t,x,y)}\|S(t,x,y)-
  \sum_{k=-L,k\neq0}^{L} \sum_{j=-N,j\neq0}^{N} \sum_{i=-M}^{M} 
  B_{i,j,k}(t,x,y)S_{i,j,k}(t,x,y)\|_2^2 \nonumber \\
  & & + \epsilon^2\, \sum_{k=-L,k\neq0}^{L} \sum_{j=-N,j\neq0}^{N} 
  \sum_{i=-M}^{M} 
  \|\mathbf{R}[B_{i,j,k}(t,x,y)]\|_2^2\;,
\end{eqnarray}}

2. Calculating noise-free signal $\tilde{S}(t,x,y)$ according to
{\setlength\arraycolsep{2pt}
\begin{eqnarray}
  \label{eq:signal}
\tilde{S}(t,x,y) &=& \sum_{k=-L,k\neq0}^{L} \sum_{j=-N,j\neq0}^{N}
    \sum_{i=-M}^{M} \tilde{B}_{i,j,k}(t,x,y)S_{i,j,k}(t,x,y) \;,
\end{eqnarray}}

\inputdir{.}
 \multiplot{2}{causal2d,noncausal2d}{width=0.47\textwidth}{Schematic
               illustration of a 2D $t$-$x$ APF. A space-causal filter
               (a) and a space-noncausal filter (b).}
 \plot{noncausal3d}{width=0.47\textwidth}{Schematic
               illustration of a 3D $t$-$x$-$y$ space-noncausal
               APF.}

\section{Synthetic data test}

\subsection{2D curved model}
We start with a synthetic example (Figure~\ref{fig:jcacov,noiz}a)
created by Raymond Abma, which was originally used for testing
nonstationary interpolation \cite[]{Liu11a}. The number of time
samples is 401 and the number of space samples is
240. Figure~\ref{fig:jcacov,noiz}b is the data with
uniformly-distributed random noise added. We compare $t$-$x$ APF with
$f$-$x$ deconvolution \cite[]{Gulunay86}, $t$-$x$ PF \cite[]{Abma95},
and $f$-$x$ RNA \cite[]{Liu12} and test their ability for random noise
attenuation. Figure~\ref{fig:fxpatch,fxdiff,txpatch,txdiff} shows the
denoised results by using stationary methods. Both the $f$-$x$
deconvolution and the $t$-$x$ PF fail in handling nonstationary curved
events. The data was divided into 5 patches with 40\% overlap along
space axis. $f$-$x$ deconvolution eliminates both signal and noise
(Figure~\ref{fig:fxpatch,fxdiff,txpatch,txdiff}a and
\ref{fig:fxpatch,fxdiff,txpatch,txdiff}b) and 
creates some artificial events with weak energy, which are parallel
with the curved event. $t$-$x$ PF preserves signal better and
introduces artifacts fewer than $f$-$x$ deconvolution
(Figure~\ref{fig:fxpatch,fxdiff,txpatch,txdiff}c), but the difference
(Figure~\ref{fig:fxpatch,fxdiff,txpatch,txdiff}d) between
Figure~\ref{fig:jcacov,noiz}b and
\ref{fig:fxpatch,fxdiff,txpatch,txdiff}c also shows obvious signal.

Another approach is to apply nonstationary filters. The denoised
results by using $f$-$x$ RNA and $t$-$x$ APF are shown in
Figure~\ref{fig:fxrna,fxnoiz,aspred,asnoiz}a and
\ref{fig:fxrna,fxnoiz,aspred,asnoiz}c, respectively. The filter length
of $f$-$x$ RNA is 8 and it has a 10-sample (frequency) and 20-sample
(space) smoothing radius. $f$-$x$ RNA
(Figure~\ref{fig:fxrna,fxnoiz,aspred,asnoiz}a) has a better result
than stationary methods, e.g., $f$-$x$ deconvolution
(Figure~\ref{fig:fxpatch,fxdiff,txpatch,txdiff}a) and $t$-$x$ PF
(Figure~\ref{fig:fxpatch,fxdiff,txpatch,txdiff}c), however, there is
still signal trend in the noise section
(Figure~\ref{fig:fxpatch,fxdiff,txpatch,txdiff}b) and artificial
events appear that are similar to those from $f$-$x$
deconvolution. For the $t$-$x$ APF, the choice of the filter length in
space is similar to that in $f$-$x$ RNA. We tend to use a 12-sample
filter in space, and the filter length in time for the $t$-$x$ APF is
selected to five samples.  As the time-length of the $t$-$x$ APF
increases, the $t$-$x$ APF passes more random noise. We use the
shaping regularization with a 60-sample (time) and 20-sample (space)
smoothing radius to constrain the APF coefficient space. The denoised
result and removed noise are shown in
Figure~\ref{fig:fxrna,fxnoiz,aspred,asnoiz}c and
\ref{fig:fxrna,fxnoiz,aspred,asnoiz}d, respectively. $t$-$x$ APF also
introduces a few artifacts, but the artifacts show a random-trend
distribution
(Figure~\ref{fig:fxrna,fxnoiz,aspred,asnoiz}c). Meanwhile, the $t$-$x$
APF, shown in Figure~\ref{fig:fxrna,fxnoiz,aspred,asnoiz}d, preserves
signal better than the $f$-$x$ RNA.

\inputdir{curve}
\multiplot{2}{jcacov,noiz}{width=0.55\columnwidth}{Curved
              model (a) and noisy data (b).}

\multiplot{4}{fxpatch,fxdiff,txpatch,txdiff}{width=0.55\columnwidth}{Comparison
              of stationary methods. The denoised result by $f$-$x$
              deconvolution (a), the noise removed by $f$-$x$
              deconvolution (b), the denoised result by $t$-$x$ PF
              (c), and the noise removed by $t$-$x$ PF (d).}

\multiplot{4}{fxrna,fxnoiz,aspred,asnoiz}{width=0.55\columnwidth}{Comparison of
              nonstationary methods. The denoised result by $f$-$x$
              RNA (a), the noise removed by $f$-$x$ RNA (b), the
              denoised result by $t$-$x$ APF (c), and the noise
              removed by $t$-$x$ APF (d).}

For further discussion, we added extra spike noise to
Figure~\ref{fig:jcacov,noiz}b, the new noisy model with a wiggle display
is shown in Figure~\ref{fig:noiz1,txpatch1,fxrna1,aspred1}a. When
comparing with the $t$-$x$ PF with patching
(Figure~\ref{fig:noiz1,txpatch1,fxrna1,aspred1}b) and the $f$-$x$ RNA
(Figure~\ref{fig:noiz1,txpatch1,fxrna1,aspred1}c), the $t$-$x$ APF
shows better signal-protection ability, however, the quality of the
denoised result gets worse than
Figure~\ref{fig:fxrna,fxnoiz,aspred,asnoiz}c because of the spikes
(Figure~\ref{fig:noiz1,txpatch1,fxrna1,aspred1}d). Larger smoothing
radius can reduce the artifacts at the cost of attenuating part of the
signals.

\multiplot{4}{noiz1,txpatch1,fxrna1,aspred1}{width=0.55\columnwidth}{Tests
              of hybrid noise model by using different methods. Data
              with hybrid noise (a), $t$-$x$ PF (b), $f$-$x$ RNA (c),
              and $t$-$x$ APF (d).}

\subsection{2D poststack model}
The second example is shown in Figure~\ref{fig:sigmoid,noise}a.  The
input data are borrowed from \cite{Claerbout09}: a synthetic seismic
image containing dipping beds, an unconformity and a
fault. Figure~\ref{fig:sigmoid,noise}b shows the same image with
Gaussian noise added. The challenge in this example is to account for
both nonstationary and event
truncations. Figure~\ref{fig:sfxrna,sfxnoiz,apfs,apfn}a shows the
denoised result using the $f$-$x$ RNA, which was implemented in each
frequency slice. Note that the $f$-$x$ RNA can eliminate part of
noise, but the result shows some artificial events, which are parallel
with the events. The difference
(Figure~\ref{fig:sfxrna,sfxnoiz,apfs,apfn}b) between
Figure~\ref{fig:sigmoid,noise}b and
Figure~\ref{fig:sfxrna,sfxnoiz,apfs,apfn}a also shows the
corresponding artifacts. The denoised result and the noise removed
using the $t$-$x$ APF are shown in
Figure~\ref{fig:sfxrna,sfxnoiz,apfs,apfn}c and
Figure~\ref{fig:sfxrna,sfxnoiz,apfs,apfn}d, respectively. The APF has
10 (time) $\times$ 10 (space) coefficients for each sample and a
20-sample (time) $\times$ 20-sample (space) smoothing radius. The
proposed method eliminates most of random noise and preserves the
truncations well.

\inputdir{sigmoid}
\multiplot{2}{sigmoid,noise}{width=0.47\columnwidth}{2D poststack
              model (a) and noisy data (b).}

\multiplot{4}{sfxrna,sfxnoiz,apfs,apfn}{width=0.47\columnwidth}{Comparison of
              nonstationary methods. The denoised result by $f$-$x$
              RNA (a), the noise removed by $f$-$x$ RNA (b), the
              denoised result by $t$-$x$ APF (c), and the noise
              removed by $t$-$x$ APF (d).}

\subsection{3D prestack French model}

We created a 3D example by Kirchhoff modeling
(Figure~\ref{fig:fdata,fnoise}a), the corresponding velocity model is
a slice out of the benchmark French model
\cite[]{French74,Liu10c}. Three surfaces of
Figure~\ref{fig:fdata,fnoise}a illustrate the corresponding slices at
time=0.6s, midpoint=1.0km, and
half-offset=0.2km. Figure~\ref{fig:fdata,fnoise}b is the noisy
data. The challenge in this example is to account for multi-dimension,
nonstationarity, and conflicting dips. For comparison, we use 3D
$f$-$x$-$y$ NRNA \cite[]{Liu13} to attenuate random noise
({Figure~\ref{fig:tpre,diff3d,fpred3,ferr3}a}). The 3D NRNA method
produces a reasonable result. In
Figure~\ref{fig:tpre,diff3d,fpred3,ferr3}b, 3D $f$-$x$-$y$ NRNA shows
a better ability of signal preservation than 2D version
\cite[]{Liu12}. However, 3D NRNA still produces artificial events parallel
with curved events. We design a 3D $t$-$x$-$y$ APF with 5 (time)
$\times$ 4 (midpoint) $\times$ 4 (half offset) coefficients for each
sample and 15-sample (time), 10-sample (midpoint), and 10-sample (half
offset) smoothing radius to further handle the variability of
event. Figure~\ref{fig:tpre,diff3d,fpred3,ferr3}c and
\ref{fig:tpre,diff3d,fpred3,ferr3}d show the denoised result and the
difference between noisy data (Figure~\ref{fig:fdata,fnoise}b) and the
denoised result (Figure~\ref{fig:tpre,diff3d,fpred3,ferr3}c) plotted
at the same clip value, respectively. The proposed method succeeds in
the sense that it is hard to distinguish the curved and conflicting
events in the removed noise. Meanwhile, the 3D $t$-$x$-$y$ APF has
fewer artificial events, which are more visible for 3D $f$-$x$-$y$
NRNA (Figure~\ref{fig:tpre,diff3d,fpred3,ferr3}a).

\inputdir{french}
\multiplot{2}{fdata,fnoise}{width=0.47\columnwidth}{3D
              synthetic data (a) and the corresponding noisy data
              (b).}

\multiplot{4}{tpre,diff3d,fpred3,ferr3}{width=0.47\columnwidth}{The 
              denoised result by 3D $f$-$x$-$y$ NRNA (a), the noise
              removed by 3D $f$-$x$-$y$ NRNA (b), the denoised result
              by 3D $t$-$x$-$y$ APF (c), and the noise removed by 3D
              $t$-$x$-$y$ APF (d).}

\section{Field data test}
For the field data test, we use a time-migrated seismic image from
\cite{Liu13}. The input is shown in Figure~\ref{fig:data}. 
The three sections in Figure~\ref{fig:data} show the time slice at
time position of 0.34 s (top section), X line section at Y space
position of 7.62 km (bottom left section), and Y line section at X
space position of 1.41 km (bottom right section). The datacube
displays simple plane layers above 1.0 s and complex structure below
1.0 s.  Noise is mainly strong random noise caused by the surface
conditions in this area. For comparison, we apply $f$-$x$-$y$ NRNA to
remove the random noise. We use a total of 24 neighboring traces
around each output trace after applying a Fourier transform along time
axis. The denoised result is shown in Figure~\ref{fig:tpre1}, which
gives a much clearer lateral continuity than original field
data. $f$-$x$-$y$ NRNA improves the both shallow plane events and deep
dipping events. Figure~\ref{fig:diff3d} shows the difference section,
in which the processed data using $f$-$x$-$y$ NRNA have been
subtracted from the original data. Some horizontal events are shown in
Figure~\ref{fig:diff3d}, especially at locations from 0.3 s to 1.0 s
in Y line section (bottom right section). Figure~\ref{fig:ppred3}
shows that the proposed $t$-$x$-$y$ APF method also produces
reasonable result, where continuity of events and geology structure
are enhanced, and there is little noise left. The $t$-$x$-$y$ APF
parameters correspond to 5 (time) $\times$ 4 (X) $\times$ 4 (Y)
coefficients for each sample ($M$=2, $N$=2, and $L$=2 in
Equation~\ref{eq:pred4} and
\ref{eq:signal}) and 20-sample (time), 10-sample (X), and 10-sample
(Y) smoothing radius for regularization operator $\mathbf{R}$. After
carefully comparing with the filtering result of 3D $f$-$x$-$y$ NRNA
(Figure~\ref{fig:tpre1}), 3D $t$-$x$-$y$ APF (Figure~\ref{fig:ppred3})
can be seen to preserve more detailed structure because $t$-$x$-$y$
APF has extra nonstationarity along time axis while $f$-$x$-$y$ NRNA
only consider nonstationarity along the two space axis. Splitting into
windows can partly help $f$-$x$-$y$ NRNA to improve the result, but it
still cannot provide a naturally nonstationary domain. Comparing with
Figure~\ref{fig:diff3d}, the difference (Figure~\ref{fig:diff3})
between Figure~\ref{fig:data} and Figure~\ref{fig:ppred3} shows no
obvious horizontal events and random noise is more
uniformly-distributed.

\inputdir{real3d}
\plot{data}{width=0.9\columnwidth}{3D field data.}
\plot{tpre1}{width=0.9\columnwidth}{The denoised result by using 
             3D $f$-$x$-$y$ NRNA.}
\plot{ppred3}{width=0.9\columnwidth}{The denoised result by using 
             3D $t$-$x$-$y$ APF.}
\plot{diff3d}{width=0.9\columnwidth}{The difference between the noisy data 
             (Figure~\ref{fig:data}) and the denoised result by using 3D
             $f$-$x$-$y$ NRNA (Figure~\ref{fig:tpre1}).}
\plot{diff3}{width=0.9\columnwidth}{The difference between the noisy data 
             (Figure~\ref{fig:data}) and the denoised result by using 3D
             $t$-$x$-$y$ APF (Figure~\ref{fig:ppred3}).}

\section{Conclusions}
We have introduced a new approach to adaptive prediction filter (APF)
for seismic random noise attenuation in $t$-$x$-$y$ domain. Our
approach uses regularized nonstationary autoregression (RNA) to handle
time-space variation of nonstationary seismic data. These properties
are useful for application such as random noise attenuation. The
predicted signal provides a noise-free estimation of local plane
events. Compared with the $f$-$x$-$y$ NRNA method, $t$-$x$-$y$ APF can
capture more detailed signal and avoid most artifacts, which occur
more in frequency domain methods. However, the $f$-$x$-$y$ NRNA method
uses fewer prediction coefficients (no time prediction) to save
storage space and can be applied in parallel to different frequency
slices. Therefore, $f$-$x$-$y$ NRNA is appropriate for mild complex
structure and fast computation while $t$-$x$-$y$ APF is more
appropriate for very complex structures.  Experiments with synthetic
examples and field data tests show that the proposed filters are able
to depict variations in the nonstationary signal and provide a
accurate estimation of complex wavefields even in the presence of
strongly curved and conflicting events.

\section{Acknowledgments}
We thank Jeffrey Shragge, Kris Innanen, and three 
anonymous reviewers for helpful suggestions, which improved the
quality of the paper. This work is partly supported by National
Natural Science Foundation of China (Grant No. 41274119, 41430322),
863 Program of China (Grant No. 2012AA09A2010), and 973 Program
of China (Grant No. 2013CB429805). All results are reproducible in the
Madagascar open-source software environment \cite[]{m8r}.

\bibliographystyle{seg}
\bibliography{SEG,SEP2,paper}

