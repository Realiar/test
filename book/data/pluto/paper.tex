\title{Pluto Model}
\author{Trevor Irons}
\shortpaper
\maketitle
\lstset{language=python,numbers=left,numberstyle=\tiny,showstringspaces=false}

\textbf {Data Type:} \emph{Synthetic}\\
\textbf {Source:} \emph{SMAART Consortium}\\
\textbf {Location:} \emph{http://www.delphi.tudelft.nl/SMAART/pluto15.htm}\\
\textbf {Format:} \emph{SEGY and Native} \\
\textbf{Date of origin:} \emph{Publicly released November 2000}\\

\section{Introduction} 
The Pluto dataset is one of several that The Subsalt Multiples Attenuation and Reduction Technology Joint Venture (SMAART JV) 
publicly released between September 2001 and November 2002.  Additional information may be found at:
\\ \emph{http://www.delphi.tudelft.nl/SMAART/}.  The data remain the property of SMAART and are used under the agreement found 
at the aforementioned web address.

The Pluto 1.5 dataset is a 2D elastic dataset released in November 2000, designed to emulate deep water subsalt prospects as 
found in the Gulf of Mexico. It contains realistic free surface and internal multiples over a structure that is relatively 
easy to image. Table \ref{tbl:FILES} shows the files contained within the Pluto repository of Madagascar.  
 
\tabl{FILES}{A list of all files contained in the \emph{Pluto} repository}
{
\tiny
\lstinputlisting[frame=single]{FILES}
\normalsize
}

\section{Velocity Models}
The Pluto model was designed to offer a complex environment to test multiple attenuation algorithms.  The model is 32 km 
(105,000 ft) long and 9.14 km (30,000 ft) in deep.  

The velocity model file \textit{int\_depth\_vp.sgy} has 1201 datapoints in the vertical direction and 6960 datums in 
the horizontal direction. The actual synthetic surveys were conducted on a padded model which contains constant velocity 
cells outside of the model boundaries.   

To assure the proper geometry Pluto velocity model headers should be formatted as shown in table \ref{tbl:modelHeader}.  
Values are listed for both metric and standard units.  This article will display metric units exclusively.

\tabl{modelHeader}{Header information for Pluto velocity models}
{
\begin{tabular}[t]{|llllll|}
	\hline
	\textbf{Standard} &           &              &               &          &       \\
	n1=1201  &   n2=6960 & 	d1=0.025    &  	 d2=0.025   &	o1=0   & o2=-34.875     \\ 
	\textbf{Metric}   &           &              &               &          &       \\
	n1=1201  &   n2=6960 &   d1=.0076  &    d2=.0076  &     o1=0 & o2=-10.629  \\
	\textbf{Padded} & & & & & \\
	n1=1401 &    n2=6960 &	d1=.025 or .0076 & d2=.025 or .0076 & 	o1=0 & o2=-34.875 or -10.629   \\
	\hline
\end{tabular}
}


The\emph{SConstruct} file found within \emph{rsf/book/data/pluto} is shown in table \ref{tbl:velSConstruct}.  
This \emph{SConstruct} file produces both metric and standard plots of the velocity model.  However, only the metric one is 
presented here in figure \ref{fig:velocityProfileMetric}.  Additionally, the padded model found in file \emph{P15VPint\_25f\_padded.SEGY}, 
is displayed in figure \ref{fig:velocityProfilePadded} for reference.     

\tabl{velSConstruct}{\emph{SConstruct} script generating the velocity model images}
{
\tiny
\lstinputlisting[frame=single]{model/SConstruct}
\normalsize
}        

Typing command \ref{eq:SCvel} within the \emph{pluto} directory runs the script.
\begin{equation}\label{eq:SCvel} \texttt{bash-3.1\$\ scons\ view} \end{equation}

\inputdir{model}
\plot{velocityProfileMetric}{width=\textwidth}{Pluto P-wave velocity model in metric units}
\plot{velocityProfilePadded}{width=\textwidth}{Padded velocity model that surveys were conducted on}

\section{Shot Records}
BP performed a fourth order finite differencing modeling code on the padded velocity model.  Madagascar can easily be used to 
display and manipulate the data.  The script \emph{pluto/shot/SConstruct} presented in table \ref{tbl:shotSConstruct} fetches 
the dataset and constructs the \emph{RSF} formatted dataset \emph{plutoShots.rsf}.  

As written this script outputs two images; figure \ref{fig:zero} shows the Pluto zero offset shot gather while figure 
\ref{fig:shot30} shows shot 30.  

\tabl{shotSConstruct}{\emph{Scons} script that generates \emph{RSF} formatted pluto shot data}
{
\tiny
\lstinputlisting[frame=single]{shot/SConstruct}
\normalsize
}


\inputdir{shot}
\plot{zero}{width=\textwidth}{Zero offset data for Pluto synthetic dataset}
\plot{shot30}{width=\textwidth}{Shot 30 of Pluto dataset}

Shot data should be formatted as shown in table \ref{tbl:shotHeader}.  Again both metric and standard units are shown.  

\tabl{shotHeader}{Header information for Pluto velocity models}
{
\begin{tabular}[t]{|lllll|}
	\hline
	\textbf{Standard} &     &                       &                       &                \\
	n1=1126      &	    d1=.008	&   o1=0	&	label1=Z Depth	&	unit1=s  \\
	n2=350       &	    d2=75	&   o2=0	&	label2=X	&	unit2=ft \\
	n3=694       &	    d3=150	&   o3=0	&	label3=Shot 	&                \\
	\textbf{Metric}   &    		&               &                       &                \\
	    n1=1126  &      d1=0.008    &   o1=0        &	label1="Depth"  &	unit1=s  \\
	    n2=350   &      d2=0.02286  &   o2=0        &  	label2="Position" &	unit2=km \\
	    n3=694   &      d3=0.0457   &   o3=0        &  	label3="Shot"	&		 \\
	\hline
\end{tabular}
}

%\section{Multiples}
%The Pluto model was specifically created to test multiple suppression algorithms.  The file mult.H contained within the repository 
%contains the data produced by these multiples.   


